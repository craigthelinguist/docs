\documentclass{sig-alternate}
  \pdfpagewidth=8.5truein
  \pdfpageheight=11truein

\usepackage{epstopdf}
\usepackage{bussproofs}
\usepackage[usenames,dvipsnames]{color} % Required for specifying custom colors and referring to colors by name
\usepackage{listings}
\usepackage{xcolor}
\usepackage{verbatim}
\usepackage{multirow}
\usepackage{array}
\usepackage{parcolumns}
\usepackage{stackengine}
\EnableBpAbbreviations
\newcommand{\code}[1]{\texttt{\footnotesize #1}}
\newcommand{\todo}[1]{{\bf \{TODO: {#1}\}}}

\newcommand\BeraMonottfamily{%
  \def\fvm@Scale{0.85}% scales the font down
  \fontfamily{fvm}\selectfont% selects the Bera Mono font
}

\lstdefinestyle{wyvern}{
%backgroundcolor=\color{highlight}, % Set the background color for the snippet - useful for highlighting
basicstyle=\scriptsize\BeraMonottfamily, % The default font size and style of the code
breakatwhitespace=false, % If true, only allows line breaks at white space
breaklines=true, % Automatic line breaking (prevents code from protruding outside the box)
captionpos=b, % Sets the caption position: b for bottom; t for top
morecomment=[s]{(*}{*)},
commentstyle=\fontshape{it}\color{Gray}\selectfont, % Style of comments within the code - dark green courier font
deletekeywords={}, % If you want to delete any keywords from the current language separate them by commas
%escapeinside={\%}, % This allows you to escape to LaTeX using the character in the bracket
firstnumber=1, % Line numbers begin at line 1
frame=lines, % Frame around the code box, value can be: none, leftline, topline, bottomline, lines, single, shadowbox
frameround=tttt, % Rounds the corners of the frame for the top left, top right, bottom left and bottom right positions
keywords=[1]{new, objtype, type, casetype, val, def, metadata, syntax, of, fn, with, let},
keywordstyle={[1]\bfseries},
keywordstyle={[3]\color{red!80!orange}},
morekeywords={}, % Add any functions no included by default here separated by commas
numbers=left, % Location of line numbers, can take the values of: none, left, right
numbersep=8pt, % Distance of line numbers from the code box
numberstyle=\tiny\color{Gray}, % Style used for line numbers
rulecolor=\color{black}, % Frame border color
showstringspaces=false, % Don't put marks in string spaces
showtabs=false, % Display tabs in the code as lines
stepnumber=1, % The step distance between line numbers, i.e. how often will lines be numbered
tabsize=4, % Number of spaces per tab in the code
}

\lstdefinestyle{tempwyvern}{
basicstyle=\scriptsize\BeraMonottfamily, % The default font size and style of the code
breakatwhitespace=false, % If true, only allows line breaks at white space
breaklines=true, % Automatic line breaking (prevents code from protruding outside the box)
captionpos=b, % Sets the caption position: b for bottom; t for top
morecomment=[s]{(*}{*)},
commentstyle=\fontshape{it}\color{Gray}\selectfont, % Style of comments within the code - dark green courier font
deletekeywords={}, % If you want to delete any keywords from the current language separate them by commas
%escapeinside={\%}, % This allows you to escape to LaTeX using the character in the bracket
firstnumber=1, % Line numbers begin at line 1
frame=lines, % Frame around the code box, value can be: none, leftline, topline, bottomline, lines, single, shadowbox
frameround=tttt, % Rounds the corners of the frame for the top left, top right, bottom left and bottom right positions
keywords=[1]{new, objtype, type, casetype, val, def, metadata, expkw, of, fn, with, typekw, let},
keywordstyle={[1]\bfseries},
keywordstyle={[3]\color{red!80!orange}},
morekeywords={}, % Add any functions no included by default here separated by commas
numbers=left, % Location of line numbers, can take the values of: none, left, right
numbersep=8pt, % Distance of line numbers from the code box
numberstyle=\tiny\color{Gray}, % Style used for line numbers
rulecolor=\color{black}, % Frame border color
showstringspaces=false, % Don't put marks in string spaces
showtabs=false, % Display tabs in the code as lines
tabsize=4, % Number of spaces per tab in the code
}
\lstset{basicstyle=\footnotesize,breaklines=true}
\lstset{escapeinside={@}{@}}
\newcommand{\htmlcolor}[1]{\textcolor[HTML]{339933}{#1}}
\newcommand{\expkwparsercolor}[1]{\textcolor[HTML]{336699}{#1}}
\newcommand{\typekwparsercolor}[1]{\textcolor[HTML]{7C803E}{#1}}
\newcommand{\urlcolor}[1]{\textcolor[HTML]{FFCC33}{#1}}
\newcommand{\expcolor}[1]{\textcolor[HTML]{FF0033}{#1}}
\newcommand{\membercolor}[1]{\textcolor[HTML]{FF6600}{#1}}
\newcommand{\typecolor}[1]{\textcolor[HTML]{660066}{#1}}
\newcommand{\dbcolor}[1]{\textcolor[HTML]{FF47FF}{#1}}
\newcommand{\hastslcolor}[1]{\textcolor[HTML]{002FC9}{#1}}
\newcommand{\simpleHTMLcolor}[1]{\textcolor[HTML]{7D5100}{#1}}
\newcommand{\boolIfcolor}[1]{\textcolor[HTML]{5E0C0C}{#1}}
\newcommand{\dbshcemacolor}[1]{\textcolor[HTML]{5AC3D1}{#1}}

\newcommand{\flyingbox}[1]{\begin{flushleft}\fbox{{#1}}\end{flushleft}}
\newcommand{\myvdash}{\vdash_{\Theta}^{\Delta;\Psi}}
\newcommand{\textcd}[1]{\textbf{\scriptsize\BeraMonottfamily{#1}}}
\newcommand{\textsp}[1]{\text{\footnotesize\BeraMonottfamily{#1}}}
\newcommand{\mycaption}[1]{\vspace{-4px}\caption{#1}\vspace{-2px}}
\newcommand{\tabularspace}{~~~~~~~~}

\newtheorem{theorem}{Theorem}
\newtheorem{lemma}{Lemma}
\newtheorem{definition}{Definition}
\newtheorem{property}{Property}

\begin{document}

%
% --- Author Metadata here ---
\conferenceinfo{XXX}{XXX}
\CopyrightYear{XXXX} % Allows default copyright year (2002) to be over-ridden - IF NEED BE.
\crdata{X-XXXXX-XX-X/XX/XX}  % Allows default copyright data (X-XXXXX-XX-X/XX/XX) to be over-ridden.
% --- End of Author Metadata ---

\title{Composable and Hygienic Typed Syntax Macros}
%
% You need the command \numberofauthors to handle the 'placement
% and alignment' of the authors beneath the title.
%
% For aesthetic reasons, we recommend 'three authors at a time'
% i.e. three 'name/affiliation blocks' be placed beneath the title.
%
% NOTE: You are NOT restricted in how many 'rows' of
% "name/affiliations" may appear. We just ask that you restrict
% the number of 'columns' to three.
%
% Because of the available 'opening page real-estate'
% we ask you to refrain from putting more than six authors
% (two rows with three columns) beneath the article title.
% More than six makes the first-page appear very cluttered indeed.
%
% Use the \alignauthor commands to handle the names
% and affiliations for an 'aesthetic maximum' of six authors.
% Add names, affiliations, addresses for
% the seventh etc. author(s) as the argument for the
% \additionalauthors command.
% These 'additional authors' will be output/set for you
% without further effort on your part as the last section in
% the body of your article BEFORE References or any Appendices.

\numberofauthors{1} %  in this sample file, there are a *total*
% of EIGHT authors. SIX appear on the 'first-page' (for formatting
% reasons) and the remaining two appear in the \additionalauthors section.
%
\author{
% You can go ahead and credit any number of authors here,
% e.g. one 'row of three' or two rows (consisting of one row of three
% and a second row of one, two or three).
%
% The command \alignauthor (no curly braces needed) should
% precede each author name, affiliation/snail-mail address and
% e-mail address. Additionally, tag each line of
% affiliation/address with \affaddr, and tag the
% e-mail address with \email.
%
% 1st. author
\alignauthor
Chenglong Wang ~~~~~~~~ Cyrus Omar ~~~~~~~~ Jonathan Aldrich \\ Carnegie Mellon University \\ \email{\{stwong, comar, aldrich\}@cs.cmu.edu}
% 2nd. author
}
%\and  % use '\and' if you need 'another row' of author names

% There's nothing stopping you putting the seventh, eighth, etc.
% author on the opening page (as the 'third row') but we ask,
% for aesthetic reasons that you place these 'additional authors'
% in the \additional authors block, viz.
\date{30 July 1999}
% Just remember to make sure that the TOTAL number of authors
% is the number that will appear on the first page PLUS the
% number that will appear in the \additionalauthors section.

\maketitle
\begin{abstract}
Syntax extension mechanisms can be powerful, but ensuring that extensions are individually well-behaved and that they can be unambiguously composed is difficult. Recent work on \emph{type-specific languages (TSLs)} addressed these problems in the specific setting of literal forms. We supplement TSLs with \emph{typed syntax macros (TSMs)}: explicitly invoked delimited syntax extensions at both the level of terms and types. To maintain a strong typing discipline, we describe two flavors of term-level TSMs: synthetic TSMs specify the type of term that they elaborate to, while analytic TSMs can elaborate to arbitrary type, but can only be used in positions where the type is known (like TSLs). Type-level TSMs generate both a type declaration and its corresponding TSL, so the two mechanisms can operate in concert. To support conventional syntactic idioms, we supplement the previous set of delimiters with a new \emph{multipart} delimited form. We specify TSMs by extending  the bidirectionally typed elaboration semantics previously given for TSLs, leveraging the same hygiene mechanism and internal language. Taken together, TSLs and TSMs provide significant expressive power without compromising composability, hygiene and typing.
\end{abstract}

% A category with the (minimum) three required fields
%\category{H.4}{Information Systems Applications}{Miscellaneous}
%A category including the fourth, optional field follows...
%\category{D.2.8}{Software Engineering}{Metrics}[complexity measures, performance measures]
%\terms{Delphi theory}
%\keywords{ACM proceedings, \LaTeX, text tagging}

\section{Introduction}
\label{sec-intro}
One way programming languages evolve is by introducing \emph{syntactic sugar} that captures common idioms more concisely and naturally. In most contemporary languages, this is the  responsibility of the language designers. Unfortunately, the designers of general-purpose languages do not have strong incentives to capture idioms that arise only situationally, motivating research into mechanisms that allow the users of a language to extend it with new syntactic sugar themselves.%, freeing designers from this responsibility.%, to varying degrees. 

Designing a useful syntax extension mechanism is non-trivial, however, because the designer can no longer  directly ensure that no parsing ambiguities have arisen and that the desugaring logic is semantically well-behaved. Instead, the mechanism must make several important guarantees:

\noindent
\textbf{Composability} The mechanism cannot simply allow the base language's syntax to  be modified arbitrarily due to the potential for parsing ambiguities, both with the base language and, critically, with one another (e.g. extensions adding support for XML and HTML).% To avoid this issue, extensions must be kept delimited from the base language and from one another. 

\noindent
\textbf{Hygiene} The logic specifying how newly introduced forms are elaborated must be constrained to ensure that the meaning of a program cannot change simply because some of the variables have been uniformly renamed (manually, or by a refactoring tool). It should also be straightforward to identify the binding site of a variable, even with intervening uses of sugar. These two situations correspond to inadvertent variable capture and shadowing by the desugaring logic. 

\noindent
\textbf{Typing Discipline} In a statically typed setting, which will be our focus in this work, a \emph{typing discipline} is also desirable: determining the type a term will have should be possible without requiring that the desugaring be performed, to aid both the programmer and tools like editors. 

Most prior approaches to syntax extension, discussed in Sec. \ref{related}, fail to provide all of these guarantees. Recent work on \emph{type-specific languages  (TSLs)} makes these guarantees, but only in a limited setting: library providers can define new syntax only for introducing values of a type (i.e. \emph{literal forms}), associating it as metadata with the type when it is declared \cite{TSLs}. Local type inference, specified as a bidirectional type system \cite{Pierce:2000:LTI:345099.345100}, controls which TSL is used to parse delimited pieces of syntax, so TSLs are composable and maintain the typing discipline. The semantics also guarantees that hygiene is maintained. We will review TSLs in Sec. \ref{background}. 

While many forms of syntactic sugar can be implemented as TSLs, there remain situations where TSLs do not suffice: 1) only a single TSL can be associated with a type, and only when it is declared, so alternative syntax for a type or syntax for a type not under a user's control cannot be defined; 2) idioms other than those that arise when introducing a value of a type (e.g. those related to control flow or API protocols) cannot be captured; and 3) types cannot themselves be declared using specialized syntax. In this paper, we introduce \emph{typed syntax macros (TSMs)}, which supplement TSLs to support these scenarios while maintaining the strong, and we believe crucial, guarantees above.% A TSM is invoked explicitly but otherwise benefits from the same mechanisms developed for TSLs. 

We introduce TSMs first at the term level in Sec. \ref{tsms-term}.  To maintain a typing discipline, there are two flavors of term-level TSMs: \emph{synthetic TSMs} can be used anywhere, while \emph{analytic TSMs} can only be used where the expected type of the term is otherwise known. Both TSLs and TSMs leverage a common set of  ligthweight delimited forms to separate syntax extensions from the host language. To support ``multi-argument'' TSMs (and TSLs), we supplement those previously defined with \emph{multipart delimited forms}. 

We next turn to the level of type declarations  in Sec. \ref{tsm-type}. Type-level TSMs generate not just a type declaration but also the TSL associated with it, allowing TSLs and TSMs to operate in concert, as we will demonstrate. 

In Sec. \ref{theory}, we give a minimal type-theoretic account of these mechanisms by extending the bidirectionally typed elaboration semantics for TSLs given previously by Omar et al., leveraging the same underlying hygiene mechanism and internal language.% (emphasizing the cohesion of these mechanisms). %   mechanism to support efforts to  decentralize control over the concrete syntax of a typed programming language. 

Taken together, TSLs and TSMs represent what we see as a new ``high water mark'' in expressive power, particularly within the space of systems providing the strong guarantees described above. We more specifically compare our work to related work in Sec. \ref{related}. 

\section{Background}\label{background}
\subsection{Wyvern}

\begin{figure}[t!]
\begin{lstlisting}[style=wyvern]
type @\htmlcolor{HTML}@ = casetype 
  Empty
  Seq of HTML * HTML 
  Text of String
  BodyElement of Attributes * HTML
  H1Element of Attributes * HTML
  StyleElement  of Attributes * CSS
  (* ... *)
  metadata = new : HasTSL
    val parser = ~
      @\expkwparsercolor{start <- '<body' attributes '>' start '</body>'}@
        fn atts, child => '@\expcolor{BodyElement((\$}@atts@\expcolor{, \$}@child@\expcolor{))}@'
      @\expkwparsercolor{start <- '<\{' EXP '\}>'}@
        fn e => e
      (* ... *)

let heading : HTML = H1Element({}, Text("My Heading"))
serve(~) (* serve : HTML -> Unit *)
  @\htmlcolor{<body id="doc1">}@
    @\htmlcolor{<\{}@heading@\htmlcolor{\}>}@
    @\htmlcolor{<p>My first paragraph.</p>}@
  @\htmlcolor{</body>}@
\end{lstlisting}
\mycaption{A case type with an associated TSL.}
\label{f-htmltype}
\end{figure}
We will present TSMs in the context of a simplified variant of the Wyvern programming language introduced previously to describe TSLs  \cite{TSLs}, making only minor changes that we will note as they come up. Wyvern is a statically typed  language with features from both the functional and object-oriented traditions and has a layout-sensitive concrete syntax. 

An example of a type encoding the tree structure of HTML is declared in Figure \ref{f-htmltype}. The named type \verb|HTML| is a \emph{case type}, with cases for each HTML tag and additional cases for an empty document, a sequence of nodes and a text node. Case types are similar to datatypes in an ML-like language (in type-theoretic terms, recursive labeled sum types). 
We can introduce a value of type \verb|HTML| by naming a case and providing data of the type the case declares, seen on line 12.

Type declarations are generative: \verb|HTML| is identified by name and so is distinct from any other type, including other case types with the same cases. Unlike in ML and the previously described variant of Wyvern, we do not combine type generativity and case types into a single construct. Instead, a named type is declared by deferring semantically to any other type underneath. For the purposes of this paper, this may be a  case type, a tuple types, e.g., \verb|HTML * HTML|, a function type, e.g., \verb|HTML -> Unit|, or an object type. We also assume that parameterized types can be defined, written e.g. \verb|List(T)|, where \verb|T| is another type. 
We assume that the definitions of standard  types like strings, lists and options are ambiently available via the \emph{prelude}, a collection of type declarations loaded before all others. 

Object types in Wyvern are structural (not class-based) and can declare fields (via \texttt{val}) and methods (via \texttt{def}). Two object types included in the prelude are shown in Figure \ref{exp-prelude}, described further below. The introductory form for an object type is \verb|new|, and it can only be used in an \emph{analytic position}, i.e. where the expected type is known (e.g. via an explicit type ascription or as an argument to a function). The keyword \verb|new| serves as a syntactic \emph{forward reference}: it can appear once per line, and the next indented block must give values for all the fields and implementations for all the methods specified by its type. We will see an example below.


All named types can be equipped with \emph{metadata}: a value constructed at compile-time and available for use by the language itself (in particular, the TSL mechanism) as well as other tools. Metadata is analagous to class annotations in Java, or class attributes in .NET languages, but unlike in these languages, it can be any Wyvern value (see below). % Here, we will use metadata to associate a TSL with \verb|HTML|. 

\subsection{Type-Specific Languages (TSLs)}


\begin{figure}[t!]
\begin{lstlisting}[style=wyvern]
type HasTSL = objtype
  val parser : Parser(Exp)

type @\expkwparsercolor{Parser}@(T) = objtype
  def parse(ps : ParseStream) : Result(T)
  metadata = new : HasTSL
    val parser = (* ... parser generator ... *)

type Result(T) = casetype
  OK of T
  Error of String * Location

type @\expcolor{Exp}@ = casetype
  Var of ID
  Lam of ID * Exp
  Ap of Exp * Exp
  Ascription of Exp * Type
  CaseIntro of ID * Exp
  (* ... *)
  metadata = new : HasTSL
    val parser = (* ... exp quasiquotes ... *)

type @\typecolor{Type}@ = casetype
  Named of ID
  Objtype of List(MemberDecl)
  Casetype of List(CaseDecl)
  Arrow of Type * Type
  metadata : HasTSL = new
    val parser = (* ... type quasiquotes ... *)
\end{lstlisting}
\mycaption{A portion of the Wyvern prelude relevant to TSLs and TSMs.}
\label{exp-prelude}
\end{figure}

Introducing a value of a type like \verb|HTML| using general-purpose syntax like that shown on line 12 can be tedious. Moreover, there is standard concrete syntax for HTML that might be preferable for reasons of familiarity or compatibility. To allow for this, we associate a \emph{type-specific language} with the \verb|HTML| type by setting the metadata to a value of type \verb|HasTSL|, an object type with a field \verb|parser| of type \verb|Parser(Exp)| (Figure \ref{exp-prelude}). 
 %We omit the implementation of the \verb|HTML| type's TSL \verb|parse| method here for concision. 

We see the TSL for \verb|HTML| being used on lines 13-15 of Figure \ref{f-htmltype}. On line 13, we wish to call a function \verb|serve| of type \verb|HTML -> Unit|. Rather than explicitly constructing a term of type \verb|HTML| as the argument, however, we use the \emph{forward referenced literal form} \lstinline[style=wyvern]{~}. The \emph{body} of the literal consists of the text in the indented block beginning on the next line, stripped of the leading indentation. In effect, whitespace is serving as a delimiter for the literal. We could equivalently have used other \emph{inline delimiters}, e.g. curly braces or single quotes, which restrict what can appear inside them, as described in Figure \ref{f-delimited}. For example, we could have written line 12 equivalently as:
\begin{lstlisting}[style=wyvern, numbers=none, frame=none]
  val heading : HTML = '@\htmlcolor{<h1>My Heading</h1>}@'
\end{lstlisting}

\begin{figure}[t]
\begin{lstlisting}[style=tempwyvern]
'@\htmlcolor{body here, '{}'inner single quotes'{}' must be doubled}@'
[@\htmlcolor{body here, [inner braces] must be balanced}@]
~ (* can appear at any expression position *)
  @\htmlcolor{forward referenced body here, leading indent stripped}@
{when body is a single base term, curly braces
can be useful because forward references propagate out}
[@\htmlcolor{adjacent}@] {@\htmlcolor{delimited forms}@} @\htmlcolor{or}@ [@\htmlcolor{those}@] @\htmlcolor{separated}@ ~ @\htmlcolor{form}@
  @\htmlcolor{by identifiers create a single multipart delimited}@
\end{lstlisting}
\mycaption{Available delimited forms. The curly brace delimited and multipart delimited forms are novel and shown being used in Sec. \ref{tsms-term}. }
\label{f-delimited}
\end{figure}


When the type system encounters literal forms like this, it defers to the parser defining the TSL of the type the literal is being analyzed against, here \verb|HTML| because it is the argument type of \verb|serve|. As suggested by the declarations in Figure \ref{exp-prelude}, the TSL is responsible for transforming a \verb|ParseStream| based on the body to a \verb|Result(Exp)|, which is either an \verb|Exp| or a parse error. The case type \verb|Exp| simply encodes the abstract syntax of Wyvern, so the TSL is defined as a desugaring. Note that the prelude types \verb|Parser| and \verb|Exp| each have TSLs associated with them, which provide a \emph{grammar-based parser generator} (based on Adams grammars \cite{Adams:2013:PPI:2429069.2429129}, see below) and \emph{quasiquote} facilities, respectively. We see them being used in Figure \ref{f-htmltype}. We refer the reader to \cite{TSLs} for further details on the TSL mechanism and this application  of them.

The \verb|parse| method can request that some portion of the parse stream be treated as a \emph{spliced} host language term. For example, the TSL for \verb|HTML| specifies the delimiters \verb|<{| and \verb|}>| to mean ``insert the enclosed term, of type \verb|HTML|, here''. The parser generator reserves the non-terminal \verb|EXP| for such spliced host language terms. The hygiene mechanism for TSLs, which we will use essentially as-is, ensures that only spliced terms can refer to variables in the surrounding scope.

For clarity in this paper, we will color host language terms, including those spliced in this way, black. Portions of TSL (and TSM) bodies that are not spliced host language terms will be colored a unique color corresponding to the TSL or TSM being used, identified when declared. 

\section{Term-Level TSM\lowercase{s}}\label{tsms-term}
In this section, we will give examples of term-level typed syntax macros in Wyvern to illustrate how they are defined and can be used in situations where TSLs are not suitable. We follow up with a more formal treatment in Sec. \ref{theory}. 

%Term-level TSMs come in two flavors: \emph{synthetic TSMs} specify the type of term they will elaborate to, meaning they can be used anywhere, while \emph{analytic TSMs} can elaborate to a term of any type, but to maintain the typing discipline, they can only be used in positions where the type can otherwise be determined. 

\subsection{Synthetic TSMs}
TSMs are defined using the \verb|syntax| keyword. Figure \ref{f-simplehtml} shows a synthetic TSM, \verb|simpleHTML|, being defined and used. The annotation on the first line indicates that valid uses of the TSM will always elaborate to a term that synthesizes the type \verb|HTML|. Like defining a TSL, defining a TSM requires defining a parser, which is a statically-evaluated value of type \verb|Parser(Exp)| (for the purposes of exposition, we include type annotations that are not strictly necessary in comments throughout the paper). 

We again define the parser by using the TSL for \verb|Parser|. Here, we are defining an alternative layout-sensitive syntax for HTML that is more concise than the conventional one by way of an \emph{Adams grammar}, which supports declarative specifications of layout-sensitive grammars by using \emph{layout constraints} within productions \cite{Adams:2013:PPI:2429069.2429129}. Here, \verb|=| indicates that the left-most column (on any line) occupied by the annotated terminal or non-terminal must occur at the same column as the parent production and \verb|>| indicates that it must be further indented. More detail on Adams' grammars and this example syntax for HTML can be found in \cite{TSLs}. 

A TSM is invoked by naming it and following it with a delimited form (Figure \ref{f-delimited}). The body of the delimited form is parsed according to the definition of the TSM. Notice here that on line 7, we no longer need a type annotation on \verb|heading| because \verb|simpleHTML| is synthetic. On lines 8-11, we use the same forward referenced delimited form introduced in the work on TSLs to avoid syntactic clashes between explicit delimiters and the extended syntax. The only difference here is the addition of the \verb|simpleHTML| ``keyword'', which indicates to the type system that the TSM should be used rather than the TSL for \verb|HTML|. Note that both can straightforwardly be used in the same program, so synthetic TSMs address the issue of defining more than one possible syntax for a type that either has a TSL defined for it already, or a type which a user cannot modify (because it appears in an external library). We do not here address namespacing issues, as standard techniques can be used to ensure that different TSMs have globally unique names. 

%An expression keyword is a keyword associated with a parser to transform DSL literals into a Wyvern expression. Depending on whether a return type is provided in the keyword declaration, expression keywords can be further divided into black-box keyword and white-box keyword (This terminology is borrowed from Scala's macro system). 


\begin{figure}[t]
\begin{lstlisting}[style=wyvern]
syntax @\simpleHTMLcolor{simpleHTML}@ : HTML = ~ (* : Parser(Exp) *)
  @\expkwparsercolor{start <- '>body'= attributes> start>}@
    fn atts, child => '@\expcolor{BodyElement((\$}@atts@\expcolor{, \$}@child@\expcolor{))}@'
  @\expkwparsercolor{start <- '<'= EXP>}@
    fn e => e
  (* ... *)
let heading = simpleHTML '@\simpleHTMLcolor{>h1 My Heading}@'
serve(simpleHTML ~)
  @\simpleHTMLcolor{>body[id="doc1"]}@
    @\simpleHTMLcolor{<}@ heading
    @\simpleHTMLcolor{>p My first paragraph}@
\end{lstlisting}
\mycaption{A synthetic TSM providing alternative syntax for the \texttt{HTML} type in Figure \ref{f-htmltype}. The programs are semantically identical.}
\label{f-simplehtml}
\end{figure}
\begin{figure}[t]
\begin{lstlisting}[style=wyvern]
type Bool = casetype 
  True
  False
syntax @\boolIfcolor{if}@ = ~ (* : Parser(Exp) *)
  @\expkwparsercolor{EXP BOUNDARY EXP BOUNDARY `else' BOUNDARY EXP}@
    fn guard, branch1, branch2 => ~ (* : Exp *)
      @\expcolor{case \$}@guard
        @\expcolor{True => \$}@branch1
        @\expcolor{False => \$}@branch2
def testIf(ok : Bool) : HTML
  if [ok] {simpleHTML ~} @\boolIfcolor{else }@{simpleHTML '@\simpleHTMLcolor{>h1 Not OK!}@'}
    @\simpleHTMLcolor{>h1 Everything is OK!}@
\end{lstlisting}
\mycaption{An analytic TSM providing a conventional syntax for \texttt{if} based on case analysis. Lines 11-12 demonstrate multipart delimited forms.}
\label{if-example}
\end{figure}

\subsection{Analytic TSMs}
Some idioms may be valid at many types. As perhaps the simplest example, consider a case type encoding booleans, \verb|Bool|, shown in Figure \ref{if-example}. Using explicit case analysis on booleans is often unnecessarily verbose, so we wish to introduce the more idiomatic and concise \verb|if| construct. Rather than having to build this in to the language, however, we can implement it as an analytic TSM. These are distinguished from synthetic TSMs by the absence of a type annotation, because the type of an \verb|if| expression is determined by its branches. 

We see \verb|if| being used on lines 10-12 of Figure \ref{if-example} with a \emph{multipart delimited form}. Each \emph{part} can be either a delimited form (e.g. the guard and the two branches) or an intervening keyword (e.g. \verb|else|). There are implicit boundaries between each part which parse to  a special token called \verb|BOUNDARY| by our parser generator (line 5). For the branches, we use curly brace delimiters, which have evolved since our previous work on TSLs to be specialized for situations where the body consists of a single spliced term. Because the parser can assume this, forward references can be identified prior to typechecking and thus be allowed to escape, as we see in the ``then'' branch in our example: the body is on the next line. Were, for example, square brackets used, then we would need to write the example as follows:

\begin{lstlisting}[style=wyvern]
def testIf2(ok : Bool) : HTML
  if [ok] [simpleHTML ~
    @\simpleHTMLcolor{>h1 Everything is OK!}@
  ] else [simpleHTML '@\simpleHTMLcolor{>h1 Not OK!}@']
\end{lstlisting}

An analytic TSM can only be used in a position where the type is otherwise known, e.g. due to the return type annotation on line 10. This is to maintain the typing discipline: we do not need to expand the TSM to know what type it will have, as with synthetic TSMs and TSLs. Although we believe this trade-off is worthwhile, another point in the design space is to permit a special signifier that can be used to allow analytic TSMs to be used in synthetic positions. For example, we might permit a post-fix asterisk on the TSM to indicate that the elaboration is expected to synthesize a type. The specific type that is synthesized requires a deeper understanding of the TSM in question (e.g. by knowing how it elaborates, or based on a ``derived'' typing rule that the authors of \verb|if| assert, or prove \cite{conf/icfp/LorenzenE13}, is always admissible):

\begin{lstlisting}[style=wyvern]
def testIf3(ok : Bool) (* no return type annotation *)
  if* [ok] {simpleHTML ~} else {simpleHTML '@\simpleHTMLcolor{>h1 Not OK!}@'}
    @\simpleHTMLcolor{>h1 Everything is OK!}@
\end{lstlisting}

The most permissive point in the design space is to simply allow such TSMs in synthetic positions without an explicit signifier. We note that would be the syntax macro analogy of \emph{white-box (term rewriting) macros} in Scala \cite{ScalaMacros2013}, which are not  available by default in the upcoming Scala 2.12. Synthetic TSMs are analagous to black-box macros. 

\section{Type-Level TSMs}
Besides using expression keywords and TSLs to extend the host language with new syntax in expressions, constructing a type with external syntax can be useful in object-relational mapping (ORM). Although data schema already exists, declaring a type with fields to represent the data structure still require users to create them manually in Wyvern syntax. Type keywords make this process easier by allowing users to generate type declarations directly using the data schema syntax, and furthermore, a metadata will be generated for initializing the value of that type. The following example shows how we construct a type using database schema.



In figure \ref{f-tykwexample}, we present an example of constructing the type \verb|EmployeeDB| using the type keyword \verb|DBSchema|. By defining the type \verb|EmployeeDB| using a type keyword, the type \verb|EmployeeDB| will be provided with the following fields/methods after elaboration (line 1-13 in figure \ref{typekw-example-2}):
\begin{itemize}\setlength{\itemsep}{0pt}
\item Fields and method declarations based on the data schema. (e.g. \verb|employee|, \verb|getByName|)
\item Common fields and method declarations provided for all types using the type keyword, and they don't depend on a certain schema. (e.g. \verb|connection| and \verb|stmts|)
\item A TSL metadata for value initialization using the DSL syntax.
\end{itemize}

Line 1-3 in figure \ref{f-tykwexample} shows the declaration of the type \verb|EmployeeDB| using database schema syntax. And a value \verb|db| (line 5-10) is defined using the syntax provided by the generated TSL metadata. 

The elaborated version of the type \verb|Employee| and the value \verb|db| is presented in figure \ref{typekw-example-2}: a type member \verb|Entry| is declared in the type to represent a data table entry, and methods like \verb|getById| are generated for database access. The elaborated value \verb|db| (line 15) is initialized with fields and methods.

The definition of the type keyword \verb|DBSchema| can be referred to figure \ref{typekw-example-1}. A type keyword itself is a parser for DSL literals: it is a value of type \verb|TypeParser|, which takes in a parsestream and returns a parsing result (of type \verb|Result|, which is a casetype defined in figure \ref{exp-prelude}). When there is no parsing error, a tuple \verb|(t:Type, e:Exp, k:List(KwMember))| will be returned for type construction: the type structure stored in \verb|t|, metadata in \verb|e|, and expression keywords defined by \verb|k|. \verb|DBSchema| will construct an object type with fields specified in line 5-18, and it will provide a TSL metadata for value initialization (line 20-35). The type of the metadata is provided on the first line of keyword declaration with keyword \verb|with metadata|, which is used by the type checker to analyze the metadata type.

% \begin{figure}
% \begin{lstlisting}[style=wyvern]
% type @\typekwparsercolor{TypeParser}@ = objtype
%   def parse(ps : ParseStream) : Result(Type * Exp * List(KwMember))
%   metadata : HasTSL = new 
%     val parser = (* parser generator *)

% type @\typecolor{Type}@ = casetype
%   Named of ID
%   Objtype of List(MemberDecl)
%   Casetype of List(CaseDecl)
%   Arrow of Type * Type
%   metadata : HasTSL = new
%     val parser = (* type quasiquotes *)

% type KwMember = casetype
%   Whitebox of Label * ExpKw
%   Blackbox of Label * ExpKw * Type
% \end{lstlisting}
% \mycaption{Wyvern prelude for type keywords}
% \label{type-prelude}
% \end{figure}
\begin{figure}[t]
\begin{lstlisting}[style=wyvern]
type @\dbcolor{EmployeesDB}@ = DBSchema ~
  @\dbshcemacolor{*ID   int}@
  @\dbshcemacolor{Name  varchar}@

val db : EmployeeDB = ~
  @\dbcolor{connect to}@ ~
    @\urlcolor{mysql://localhost:3306}@
  @\dbcolor{username: }@"user1"
  @\dbcolor{password: }@"001"
  @\dbcolor{table: Employees}@
\end{lstlisting}
\mycaption{The usage of the type keyword ``DBTable''}
\label{f-tykwexample}
\end{figure}

\begin{figure}[t]
\begin{lstlisting}[style=wyvern]
type EmployeesDB = objtype
  type Entry = objtype
    val ID : Int
    val Name : String 
  val connection : URL
  val username : String
  val password : String
  ...
  def getById (x:Int) : Option(Entry)
  def getByName (x : String) : List(Entry)
  metadata = new : HasTSL
    val parser = ~
      ... (* TSL parser generated by type constructor *)

val db : EmployeesDB = new
  val connection = new
    val domain = "localhost"
    ...
  val username = "user1"
  val password = "001"
  ... (* Other fields *)
  def getByID(x:Int) : List(Entry)
    ...
\end{lstlisting}
\mycaption{The elaborated type declaration}
\label{typekw-example-2}
\end{figure}

\begin{figure}[t]
\begin{lstlisting}[style=wyvern]
typekw @\dbshcemacolor{DBSchema}@ with metadata:HasTSL = ~ (*:TypeParser*)
  @\typekwparsercolor{start <- pairs}@
    fn pairs =>
      let ty : Type = ~
        @\typecolor{objtype}@
          @\typecolor{type Entry = objtype}@
            @\typecolor{\{}@map(pairs, Nil, 
              fn ((p, lbl, ty), l) => Cons(~, l))
                @\membercolor{val \$}@lbl @\membercolor{:  \$}@ty
            @\typecolor{\}}@
          @\typecolor{val connection : URL}@
          @\typecolor{val username : String}@
          @\typecolor{val password : String}@
          @\typecolor{...}@
          @\typecolor{\{}@map(pairs, Nil, 
            (fn ((p, lbl, ty), l) => Cons(~, l)))
              @\membercolor{def getBy\$}@lbl @\membercolor{(\$}@ty@\membercolor{)}@
          @\typecolor{\}}@
      let md : HasTSL = new 
        val parser = ~
          @\hastslcolor{start <- ("connect to "= EXP>}@
                    @\hastslcolor{"username:"= EXP>}@
                    @\hastslcolor{"password:"= EXP>}@
                    @\hastslcolor{"table:" EXP>)}@
            fn url, un, pw, table => ~
              @\expcolor{new}@ 
                @\expcolor{val connection = \$}@url
                @\expcolor{val username = \$}@un
                @\expcolor{val password = \$}@pw
                @\expcolor{...}@
                @\expcolor{\{}@map(pairs, Nil, 
                  (fn ((p, lbl, ty), l) => Cons(~, l)))
                    @\membercolor{def getBy}@$lbl @\membercolor{(x)}@ 
                      @\membercolor{...}@(* implementation *)
                @\expcolor{\}}@
      (ty, md, Nil)
  @\typekwparsercolor{pairs <- ()}@
    fn () => Nil 
  @\typekwparsercolor{pairs <- pair= pairs=}@
    fn hd, tl => Cons(hd, tl)
  @\typekwparsercolor{pair <- "*"? ID ID}@
    fn is_primary, colid, tyid => (
      is_primary, colid, ty_from_sqlty(tyid))
\end{lstlisting}
\mycaption{The declaration the type keyword ``DBTable''}
\label{typekw-example-1}
\end{figure}

% IS THIS DELETED?
\begin{comment}
<<<<<<< HEAD
In figure \ref{f-tykwexample}, we present an example of constructing the type \verb|EmployeeDB| using the type keyword \verb|DBSchema|. By defining the type \verb|EmployeeDB| using a type keyword, the type \verb|EmployeeDB| will be provided with the following fields/methods after elaboration (line 1-13 in figure \ref{typekw-example-2}):
\begin{itemize}\setlength{\itemsep}{0pt}
\item Fields and method declarations based on the data schema. (e.g. \verb|employee|, \verb|getByName|)
\item Common fields and method declarations provided for all types using the type keyword, and they don't depend on a certain schema. (e.g. \verb|connection| and \verb|stmts|)
\item A TSL metadata for value initialization using the DSL syntax.
\end{itemize}

Line 1-3 in figure \ref{f-tykwexample} shows the declaration of the type \verb|EmployeeDB| using database schema syntax. And a value \verb|db| (line 5-10) is defined using the syntax provided by the generated TSL metadata. 

The elaborated version of the type \verb|Employee| and the value \verb|db| is presented in figure \ref{typekw-example-2}: a type member \verb|Entry| is declared in the type to represent a data table entry, and methods like \verb|getById| are generated for database access. The elaborated value \verb|db| (line 15) is initialized with fields and methods.

The definition of the type keyword \verb|DBSchema| can be referred to figure \ref{typekw-example-1}. A type keyword itself is a parser for DSL literals: it is a value of type \verb|TypeParser|, which takes in a parsestream and returns a parsing result (of type \verb|Result|, which is a casetype defined in figure \ref{exp-prelude}). When there is no parsing error, a tuple \verb|(t:Type, e:Exp, k:List(KwMember))| will be returned for type construction: the type structure stored in \verb|t|, metadata in \verb|e|, and expression keywords defined by \verb|k|. \verb|DBSchema| will construct an object type with fields specified in line 5-18, and it will provide a TSL metadata for value initialization (line 20-35). The type of the metadata is provided on the first line of keyword declaration with keyword \verb|with metadata|, which is used by the type checker to analyze the metadata type.

%\begin{figure}
%\begin{lstlisting}[style=wyvern]
%type @\typekwparsercolor{TypeParser}@ = objtype
%  def parse(ps : ParseStream) : Result(Type * Exp * List(KwMember))
%  metadata : HasTSL = new 
%    val parser = (* parser generator *)

%type @\typecolor{Type}@ = casetype
%  Named of ID
%  Objtype of List(MemberDecl)
%  Casetype of List(CaseDecl)
%  Arrow of Type * Type
%  metadata : HasTSL = new
%    val parser = (* type quasiquotes *)

%type KwMember = casetype
%  Whitebox of Label * ExpKw
%  Blackbox of Label * ExpKw * Type
%\end{lstlisting}
%\mycaption{Wyvern prelude for type keywords}
%\label{type-prelude}
%\end{figure}

\begin{figure}
\begin{lstlisting}[style=wyvern]
type @\typekwparsercolor{TypeParser}@ = objtype
  def parse(ps : ParseStream) : Result(Type * Exp)
  metadata : HasTSL = new 
    val parser = (* parser generator *)

type @\typecolor{Type}@ = casetype
  Named of ID
  Objtype of List(MemberDecl)
  Casetype of List(CaseDecl)
  Arrow of Type * Type
  metadata : HasTSL = new
    val parser = (* type quasiquotes *)
\end{lstlisting}
\mycaption{Wyvern prelude for type keywords}
\label{type-prelude}
\end{figure}

\section{Syntax}
We start our formal presentation by introducing the abstract syntax together with their concrete forms built upon pure functional Wyvern. For simplicity consideration, we omit some of the syntax not directly related to composable syntax macros, which can be referred to \cite{TSLs}.
=======
\end{comment}

\section{Syntax}\label{theory}
We start our formal presentation by introducing the abstract syntax together with their concrete forms built upon pure functional Wyvern. For simplicity consideration, we omit some of the syntax not directly related to composable syntax macros, which can be referred to \todo{TSL paper}.
%>>>>>>> ea4d2199ced57f3233d65f0288f777fe918f4ad7

A Wyvern program ($\rho$) consists of four parts: type keyword declarations ($\eta$), named-type declarations ($\theta$), expression keyword declarations ($\kappa$) and an expression ($e$) representing the program body.

\begin{figure}[ht]
  \[
  \begin{array}{ll}
      \textbf{Abstract Forms}   & \textbf{Concrete Forms}\\
      \text{Program}\\
      \rho~::=~\eta;\theta;\kappa;e  &\\
      \text{Type Keywords}\\
      \eta~::=~\emptyset      &\\
      \tabularspace\eta;k[\mathbf{ty},e,\tau] & \textcd{typekw}~k~\textcd{with metadata:}\tau~\textcd{=}~e\\
      \text{Expression Keywords}\\
      \kappa~::=~\emptyset\\
      \tabularspace\kappa;k[\mathbf{bk}(\tau),e]    & \textcd{expkw}~k : \tau~\textcd{=}~e\\
      \tabularspace\kappa;k[\mathbf{wk},e]          & \textcd{expkw}~k~\textcd{=}~e\\
      \text{Type Declarations}\\
      \theta~::=~\emptyset\\
      \tabularspace\theta; T[\mathbf{explicit},\tau, e]  & \textcd{type}~T~\textcd{=}~\tau\\
                                                            & ~~~\textcd{metadata = }e\\
      \tabularspace\theta; T[\mathbf{tykw},k, body, e]   & \textcd{type}~T~\textcd{=}~k~delims\\
                                                            & ~~~\textcd{metadata = }e\\
      \text{Types}\\
      \tau~::=~\mathbf{named}[T]              & T\\
      \tabularspace\mathbf{objtype}[\omega]       & \textcd{objtype}~\omega \\
      \tabularspace\mathbf{casetype}[\chi]        & \textcd{casetype}~\chi\\
      \tabularspace\mathbf{arrow}[\tau, \tau]     & \tau \textcd{->}~\tau\\
      \text{Object Fields}\\
      \omega~::=~\emptyset                      \\
      \tabularspace l[\mathbf{val},\tau];\omega                 & \textcd{val}~l : \tau\\
      \tabularspace l[\mathbf{def},\tau];\omega                 & \textcd{def}~l : \tau\\
      \text{Casetype Cases}\\
      \chi~::=~\emptyset                      \\                 
      \tabularspace C[\tau];\chi                   & C~\textcd{of}~\tau\\
      \text{External Terms}\\
       e~::=~...                              & \\
      \tabularspace\mathbf{lit}[body]             & delims\\
      \tabularspace\mathbf{ekey}[k,body]       & k~delims\\
      \text{Translational Terms}\\
      \hat{e}~::=~...                              & \\
      \tabularspace\mathbf{spliced}[e]            & \\
      \text{Internal Terms}\\
      i~::=~...
  \end{array}
  \]
\mycaption{Abstract and Concrete Forms}
\label{formal-syntax}
\end{figure}


\begin{comment}
\begin{figure}[ht]
\hspace{-5px}\begin{tabular}{ l l l l l }
 \multicolumn{1}{l}{\textbf{Abstract Forms}} & \multicolumn{1}{l}{\textbf{Concrete Forms}}\\
 \multicolumn{3}{l}{Programs}\\
$\rho$~::=~$\eta;\theta;e$\\
\multicolumn{3}{l}{Type Keywords}\\
$\eta$~::=~$\emptyset$\\
\tabularspace$\eta;k[e,\tau]$ & \textcd{typekw} $k$ \textcd{with metadata:}$\tau$ \textcd{=} $e$ \\
\multicolumn{3}{l}{Expression Keywords}\\
$\kappa$~::=~$\emptyset$                        & \\
\tabularspace$\kappa;k[\mathbf{bk}(\tau),e]$    & \textcd{expkw} $k$ : $\tau$ \textcd{=} $e$\\
\tabularspace$\kappa;k[\mathbf{wk},e]$          & \textcd{expkw} $k$ \textcd{=} $e$\\
\multicolumn{3}{l}{Type Declarations}\\
$\theta$~::=~$\emptyset$                        & \\
\tabularspace$\theta; T[\mathbf{explicit},\tau, e, \kappa]$  & \textcd{type} $T$ \textcd{=} $\tau$\\
&~~~\textcd{metadata = }$e$\\
&~~~$\kappa$\\
\tabularspace$\theta; T[\mathbf{tykw},k, body, e, \kappa]$   & \textcd{type} $T$ \textcd{=} $k$ $delims$\\
                                                          & ~~~\textcd{metadata = }$e$\\
                                                          & ~~~$\kappa$\\
\multicolumn{3}{l}{Types}\\
$\tau$~::=~$\mathbf{named}[T]$              & $T$\\
\tabularspace$\mathbf{objtype}[\omega]$       & \textcd{objtype} $\omega$ \\
\tabularspace$\mathbf{casetype}[\chi]$        & \textcd{casetype} $\chi$\\
\tabularspace$\mathbf{arrow}[\tau, \tau]$     & $\tau$ \textcd{->} $\tau$\\
\multicolumn{3}{l}{Object Fields}\\       
$\omega$~::=~$\emptyset$                      \\
\tabularspace$l[\mathbf{val},\tau];\omega$                 & \textcd{val} $l$ : $\tau$\\
\tabularspace$l[\mathbf{def},\tau];\omega$                 & \textcd{def} $l$ : $\tau$\\
\multicolumn{3}{l}{Casetype Cases}\\
$\chi$~::=~$\emptyset$                      \\                 
\tabularspace$C[\tau];\chi$                   & $C$~\textcd{of}~$\tau$\\
\multicolumn{3}{l}{External Term}\\
 $e$~::=~...                              & \\
\tabularspace$\mathbf{lit}[body]$             & $delims$\\
\tabularspace$\mathbf{ekey}[k,body](e)$       & $e.k$ $delims$\\
\multicolumn{3}{l}{Translational Terms}\\
$\hat{e}$~::=~...                              & \\
\tabularspace$\mathbf{spliced}[e]$            & \\
\multicolumn{3}{l}{Internal Terms}\\
$i$~::=~...                                                     
\end{tabular}
\mycaption{Abstract and Concrete Forms}
\label{formal-syntax}
\end{figure}
\end{comment}


\paragraph{Type-keyword Declarations}
Users define type keywords in the type-keyword declaration section ($\eta$). The declaration of a keyword includes an expression $e$ and a type $\tau$: the expression $e$ is the parser to parse the DSL literals, and the type $\tau$ is the type of the generated metadata.

\paragraph{Expression-keyword Declarations}
Similar to type-keyword declarations, expression-keywords are declared in $\kappa$. As we presented in section \ref{sec-example}, black box keywords ($k[\mathbf{bk}(\tau),e]$) are defined with its return type $\tau$ while white-box keywords ($k[\mathbf{wk}, e]$) are not. The expression ($e$) in both cases represents the parser, which is supposed to be of type $ExpParser$.

\paragraph{Named-type declarations}
Named-type declarations are the declaration of the types $\tau$ which are associated with a name $T$.

Depending on the definition method, there are two kinds of type declarations: an explicit type declaration with its structure and metadata explicitly declared, or a keyword-defined type using a type keyword invocation, whose structure and metadata will be generated by the elaborated DSL literals presented in the keyword invocation body.

An explicit type declaration ($T[\mathbf{explicit},\tau,e]$) consists of three parts: $T$ is the type name, $\tau$ is the type structure, which can be one of $\mathbf{objtype}$, $\mathbf{casetype}$, $\mathbf{arrowtype}$ or named type $\mathbf{named}[T']$,  and $e$ is the metadata associated with the type.

The second kind of type declaration ($T[\mathbf{tykw},k,body,e]$) is a type declared using DSL syntax specified in a type keyword $k$. DSL literals are presented as $body$ in the declaration and the type metadata as $e$. Different from those in the first kind, metadata $e$ here serve as a metadata extender to extend the metadata generated by the keyword parser.

\paragraph{External Expressions}
By naming expressions presented in a Wyvern program ``external expressions'' ($e$), we distinguish them from ``internal expressions'' ($i$), which are pure Wyvern expressions without DSL literals. 

Wyvern expression syntax is extended with a keyword invocation expression and several expressions for run-time parser access (omitted here, can be referred to \todo{TR}) to support expression level syntax macros. The keyword invocation expression $\mathbf{ekey}[k,body]$ specifies an invocation of the keyword $k$ and DSLs presented in delimited forms as $body$, its concrete form can be referred to figure \ref{formal-syntax}.

An external expression with DSL literals will elaborate to internal Wyvern expressions with the help of the DSL parsers. During elaboration, translational expressions  
are defined to support ``spliced expression'' with internal Wyvern terms and external terms. The syntax for translational terms mirrors that of external terms, except that literal forms are removed and and translational state $\mathbf{spliced}[e]$ is added, which represents an external term $e$ spliced into a literal body.

\section{Bidirectional Typechecking and Elaboration}
Bidirectional typechecking is used here as type can be clearly specified as input or output during typechecking process. In bidirectional type system, a type judgment is written as $\Gamma\vdash_{\Theta} e\Leftarrow(\Rightarrow)\tau$ instead of $\Gamma\vdash_{\Theta} e:\tau$ in a traditional type system, where $\Gamma$ is the typing context, $\Theta$ is the declaration context, and the type $\tau$ is specified as input ($\Leftarrow$) or output ($\Rightarrow$) according to the arrow direction. 

Syntax macros contains DSL literals and they belongs to external Wyvern language, and an elaboration phase is required to transform them into internal Wyvern representations for execution. The elaboration of an external expression is part of type checking process, written as $\Gamma\vdash_{\Theta} e\rightsquigarrow i \Leftarrow(\Rightarrow) \tau$. It indicates that an external expression elaborates to an internal expression and synthesize to (or analyze against) the type $\tau$. The arrow $\rightsquigarrow$ is used to represent the elaboration process, including literal elimination and hygiene process.

\subsection{Contexts}
The program contexts (figure \ref{typechecking-environment}) include type keywords context ($\Delta$), expression keyword context ($\Psi$), named type declarations ($\Theta$) and typing context ($\Gamma$). 

\begin{figure}[ht]
\begin{center}
\begin{tabular}{r r c l}
Type Keyword Context & $\Delta$ & ::= & $\emptyset$\\
              &                 &  |  & $\Delta;k[\mathbf{ty},i,\tau]$\\
Expression Keyword Context & $\Psi$    & ::= & $\emptyset$\\
            &                 &   |  & $\Psi;k[\mathbf{bk}(\tau),i]$\\
            &                 &   |  & $\Psi;k[\mathbf{wk},i]$\\
Named Type Context  & $\Theta$        & ::= & $\emptyset$\\
              &                 &  |  & $\Theta,T[\delta,\mu]$\\
              & $\delta$        & ::= & $?$ ~ | ~ $\tau$\\
   & $\mu$           & ::= & $?$ ~ | ~ $i:\tau$\\
Typing Context & $\Gamma$ &   ::=  & $\emptyset$\\
                 &          &     |  & $\Gamma,x:\tau$
\end{tabular}
\end{center}
\mycaption{Definition of type context}
\label{typechecking-environment}
\end{figure}

Type keyword context ($\Delta$) is the environment for user defined type keywords. Type keywords defined in a Wyvern program goes into context in the form of $k[\mathbf{ty},i,\tau]$: the keyword $k$ is a type keyword with parser $i$ and return metadata type $\tau$. Similarly, expression keyword context ($\Psi$) is the context for expression keywords, a black-box expression keyword is $k[\mathbf{bk}(\tau),i]$, representing a keyword $k$ with parser $i$ and return type $\tau$, while a white-box keyword in the context is represented as $k[\mathbf{wk},i]$, which is a keyword $k$ with parser $i$.

Named typed context ($\Theta$) is the environment to keep elaborated type declarations. A declared type (either explicitly declared or type keyword defined) has the form of $T[\delta,\mu]$, which is interpreted as a type of name $T$ with structure $\delta$ and metadata $\mu$. We wrap the type structure $\tau$ in $\delta$ and the metadata $i:\tau$ in $\mu$ in order to represent their ``unknown state'' using ``$?$'' in type checking.

$\Gamma$ is the typing context, elements in $\Gamma$ are bindings of variables and their types ($x:\tau$).

\subsection{Checking rules for keywords}
A type keyword declaration $k[\mathbf{ty},e,\tau]$ is well type iff 1) the keyword has no name conflicts with those defined in previous context, 2) the parser $e$ elaborates to an internal term $i$ of type $\mathbf{named}[TypeKw]$ and 3) the metadata type $\tau$ is well formed within the given named type context $\Theta$. The named type context we used for checking type keywords is $\Theta_0$ (in the rule $\mathsf{compile}$), thus the type keyword is supposed to be well formed only under Wyvern prelude.

While for expression keywords, we use the context $\Theta_0\Theta$ for type checking, which allows expression keywords to use both prelude types and user defined types. The checking rules of expression keywords can be referred to expkw-wk and expkw-bk, and the checking process is similar to that for type keywords: it will check that no name conflicts in keyword declarations, the parser should elaborate to an internal term $i$ of type $\mathbf{named}[ExpKw]$, and the type defined in a black box keyword is well formed under named type context $\Theta$.

\subsection{External Type Literals}
Declaring a type using a type keyword involves the use of external DSL literals, and the rule ctx-typekw-type is used to transform it into internal Wyvern type representations with concrete type structure and metadata.

The elaboration process of a type $T[\mathbf{tykw},k,body,e_m]$ under context $\Theta,\Delta,\Psi$ contains the following steps:
\begin{enumerate}\setlength{\itemsep}{0pt}
\item Check that the Wyvern prelude is contained in the named type context, as parser types are defined in Wyvern preludes. ($\Theta_0\subset \Theta$)
\item Elaborate the previous named type context ($\theta'\sim\Theta'$), and check that the type name $T$ has no name conflicts with those in previous context ($T\notin dom(\Theta\Theta')$).
\item Look up the keyword $k$ in $\Delta$. ($k[\mathbf{ty},i_k,\tau_m]\in\Delta$)
\item Read in the literals ($body$) as an expression $i$, and apply the parser $i_k$ to the DSL body, which will generate a tuple: $(i_{type}, i_m)$. $i_{type}$ is an expression of type $\mathbf{named}[Type]$ and $i_m$ is the generated metadata term. ($\mathbf{iap}(\mathbf{iprj}[parse](i_k);i_{ps})\Downarrow(i_{type},i_{m})$)
\item Reificate the term $i_{type}$ to an Wyvern type $\tau$, and check the formation of the type $\tau$ under context $\Theta\Theta',T[?,?]$, which allows self recursion. ($i_{type}\uparrow\tau$ and $\vdash_{\Theta\Theta',T[?,?]}\tau$) Reification rules for types can be referred to \todo{cite TR}.
\item Elaborate the metadata defined in the type $e_m$ to an internal term $i'_m$ of type $\mathbf{arrow}[\tau_m,\tau'_m]$, which is an metadata extender to extend the metadata generated by the type keyword. ($\emptyset\vdash_{\Theta\Theta',T[\tau,?]}^{\Delta;\Psi}e_m\rightsquigarrow i_m' \Rightarrow \mathbf{arrow}[\tau_m, \tau'_m]$)
\item Apply the keyword extender ($i'_m$) to the generated metadata ($i_m$) and generate the metadata of the type $T$. ($\mathbf{iap}(i'_m,i_m)\Downarrow i''_m$)  
\end{enumerate}
After elaboration, the type declaration $T[\mathbf{tykw},k,body,e_m]$ elaborates to $T[\tau,i''_m:\tau'_m]$ and is added into environment for further reference.

%!TEX root = sac15.tex
\begin{figure*}[ht]
\flyingbox{$d\sim (\Psi;\Theta)$}
\vspace{-15px}\begin{center}
\AXC{}\RightLabel{(D-empty)}
\UIC{$\emptyset \sim (\emptyset;\emptyset)$}
\DP
\end{center}

\begin{center}
\AXC{
	%\stackanchor
	$d\sim(\Psi;\Theta)$ ~~~~ $s \notin \text{dom}(\Psi)$ ~~~~ $\emptyset\vdash_{\Theta_0\Theta} \tau::\star$ ~~~~ $\emptyset;\emptyset\vdash_{\Theta_0\Theta}^{\Psi} e_{tsm}\rightsquigarrow i_{tsm}\Leftarrow \mathtt{Parser}(\mathtt{Exp})$
}\RightLabel{(D-syntsm)}
\UIC{$ d;\mathbf{syntsm}(s,\tau,e_{tsm})\sim (\Psi,s[\mathbf{syn}(\tau,i_{tsm})];\Theta)$}
\DP
\end{center}

\begin{center}
%\AXC{$\Theta_0\subset \Theta$ ~~~~ $\myvdash  d'\rightsquigarrow\Theta'$ ~~~~ $k\notin dom(\Theta\Theta')$ ~~~~ $\emptyset\vdash_{\Theta\Theta'} e_{k}\rightsquigarrow i_{k}\Leftarrow \mathbf{named}[ExpKw]$}								\RightLabel{(expkw-wk)}
\AXC{
	{$d\sim(\Psi;\Theta)$ ~~~~ $s \notin \text{dom}(\Psi)$ ~~~~ $\emptyset;\emptyset\vdash_{\Theta_0\Theta}^{\Psi} e_{tsm}\rightsquigarrow i_{tsm}\Leftarrow \mathtt{Parser}(\mathtt{Exp})$}
}\RightLabel{(D-anatsm)}
\UIC{$ d;\mathbf{anatsm}(s, e_{tsm})\sim (\Psi, s[\mathbf{ana}(i_{tsm})];\Theta)$}
\DP
\end{center}

\begin{center}
\AXC{$d\sim(\Psi;\Theta)$ ~~~~ $s \notin \text{dom}(\Psi)$ ~~~~  $\emptyset\vdash_{\Theta_0\Theta}\tau_{md}::
\star$ ~~~~ $\emptyset; \emptyset \vdash_{\Theta_0\Theta}^{\Psi}e_{tsm}\rightsquigarrow i_{tsm}\Leftarrow \mathtt{Parser}(\mathtt{Type} \times \tau_{md})$}
\RightLabel{(D-tytsm)}
\UIC{$d;\mathbf{tytsm}(s,\kappa,\tau_{md}, e_{tsm}) \sim (\Psi,s[\mathbf{ty}(\kappa, \tau_{md}, i_{tsm})]; \Theta)$}
\DP
\end{center}

\begin{center}
\AXC
{
	$d \sim (\Psi;\Theta)$ ~~~~ $\mathtt{T}\notin \text{dom}(\Theta_0\Theta)$ ~~~~ $\emptyset\vdash_{\Theta_0\Theta}\tau::\kappa \rightarrow \kappa$ ~~~~ $\emptyset;\emptyset\vdash_{\Theta_0\Theta,T[\tau(T) :: \kappa,-]}^{\Psi}e_{md}\rightsquigarrow i_{md}\Rightarrow\tau_{md}$
}\RightLabel{(D-tydecl)}
\UIC{$d;\mathbf{tydecl}(\mathtt{T},\tau,e) \sim (\Psi; \Theta,\mathtt{T}[\tau(\mathtt{T}) :: \kappa,i_{md}:\tau_{md}])$}
\DP
\end{center}

\begin{center}
\AXC{
	\stackanchor{
		\stackanchor{$d \sim (\Psi;\Theta)$ ~~~~ $\mathtt{T}\notin \text{dom}(\Theta_0\Theta)$ ~~~~ $s[\mathbf{ty}(\kappa,\tau_{md},i_{tsm})]\in\Psi$}
		{$\mathsf{parsestream}(body)=i_{ps}$ ~~~~ $i_{tsm}.parse(i_{ps}) \Downarrow OK((i_{type},i_{md}))$ ~~~~ $i_{type}\uparrow\hat\tau$ ~~~~ $\Delta;\emptyset\vdash_{\Theta\Theta'}\hat\tau\rightsquigarrow\tau::\kappa \rightarrow \kappa$}
	}
	{$\emptyset;\emptyset\vdash_{\Theta_0\Theta,\mathtt{T}[\tau(\mathtt{T}) :: \kappa,-]}^{\Psi}e_{mdx}\rightsquigarrow i_{mdx} \Rightarrow \tau_{md}\rightarrow\tau'_{md}$ ~~~~ $i_{mdx}(i_{md})\Downarrow i'_{md}$}
}
\RightLabel{(D-aptsm)}
\UIC{$d;\mathbf{tyaptsm}(\mathtt{T},s,body,e_{mdx}) \sim (\Psi; \Theta,\mathtt{T}[\tau(\mathtt{T}) :: \kappa,i'_{md}:\tau'_{md}])$}
\DP
\end{center}
\caption{Declaration checking and elaboration of type-level TSMs.}
\label{typechecking-elaboration}
\end{figure*}




\subsection{Expression Keywords Literals}
Expression keyword invocations contains the DSL literals which will be transformed to internal Wyvern expressions. Elaboration rules are presented in figure~\ref{expkw-kwstatics}: T-wk for white-box invocation and T-bk for black-box keyword invocation.

For white-box keywords, we have the following elaboration steps:
\begin{enumerate}\setlength{\itemsep}{0pt}
\item Check that the prelude is in the named type context. ($\Theta_0\subset\Theta$)
\item Look up the keyword in $\Psi$. ($k[\mathbf{wk},i_k]\in\Psi$)
\item Read in the DSL literals into an expression $i_{ps}$ of type $\mathbf{named}[ParseStream]$ and parse them by invoking the parser $i_k$. The parsing result is a value of type $\mathbf{named}[Result]$, and an AST term $i_{ast}$ is provided. ($\mathbf{iap}(\mathbf{iprj}[parse](i_k); i_{ps});\Downarrow \mathbf{iinj}[OK]((i_{ast}, i'_{ps}))$)
\item Reificate the term $i_{ast}$ to an translational term $\hat{e}$ with spliced DSL body, then elaborate the translational term $\hat{e}$ to an internal Wyvern term and check it against the type $\tau$. ($i_{ast}\uparrow \hat{e}$ and $\Gamma;\emptyset\myvdash \hat{e} \rightsquigarrow i \Leftarrow \tau$)
\end{enumerate}
The term $\mathbf{ekey}[k,body]$ is transformed into $i$ of type $\tau$, which is an internal term can be used for evaluation. The process for black-box keyword elaboration is similar, except that the type of the elaborated expression comes from the keyword definition. 

In fact, for white-box keyword, besides using type analysis, we may also allow type synthesis in the 4th step by replacing the rule to  $\Gamma;\emptyset\myvdash \hat{e} \rightsquigarrow i \Rightarrow \tau$. We only allow the use of type analysis for strong typing property, as type synthesis can only inference the type after elaboration.
%!TEX root = sac15.tex
\begin{figure*}
\flyingbox{$\Delta;\Gamma\vdash_{\Theta}^{\Psi} e\rightsquigarrow i\Rightarrow \tau$}
\vspace{-18px}
\begin{center}
\AXC{
	\stackanchor
	{$s[\mathbf{syn}(\tau,i_{tsm})]\in\Psi$ ~~~~ $\mathsf{parsestream}(body)=i_{ps}$}
	{$i_{tsm}.parse(i_{ps}) \Downarrow OK(i_{exp})$ ~~~~ $i_{exp}\uparrow \hat{e}$ ~~~~ $\Delta;\emptyset;\Gamma;\emptyset\myvdash \hat{e} \rightsquigarrow i \Leftarrow \tau$}
} \RightLabel{(T-syn)}
\UIC{$\Delta;\Gamma\myvdash\mathbf{eaptsm}[s,body] \rightsquigarrow i\Rightarrow \tau$}
\DP
\end{center}
%%%
\begin{center}
\AXC{
    \stackanchor
    {$s[\mathbf{ana}(i_{tsm})]\in\Psi$ ~~~~ $\mathsf{parsestream}(body)=i_{ps}$}
    {$i_{tsm}.parse(i_{ps}) \Downarrow OK(i_{exp})$ ~~~~ $i_{exp}\uparrow \hat{e}$ ~~~~ $\Delta;\emptyset;\Gamma;\emptyset\myvdash \hat{e} \rightsquigarrow i \Leftarrow \tau$}
}\RightLabel{(T-ana)}
\UIC{$\Delta;\Gamma\myvdash\mathbf{eaptsm}[s,body] \rightsquigarrow i \Leftarrow \tau$}
\DP
\end{center}
\caption{Statics for Keyword Invocation}
\label{expkw-kwstatics}
\end{figure*}

\subsection{Hygiene}
\todo{Where are we supposed to discuss Hygiene?}

\subsection{Metatheory}
\paragraph{Reification and Dereification}
For each type and expression, we have the following properties to support converting a type to an expression of type \textbf{named}[$Type$] and an expression to an expression of type \textbf{named}[$Exp$] (Property 1, 2). And reversely, converting an internal term to an Wyvern expression $i$ of type $\mathbf{named}[Exp]$ (Property 3).  
\begin{property}If $\emptyset\vdash_{\Theta} i:\mathbf{named}[Type]$ then there exists a corresponding $\tau$, s.t. $i\uparrow\tau$. 
\end{property}
\begin{property}
If $\emptyset\vdash_{\Theta} i:\mathbf{named}[Exp]$ then there exists an translational term $\hat{e}$, s.t. $i\uparrow\hat{e}$.
\end{property}

\begin{property}
For any $i$, these exists an internal term $i'$, s.t. $i\downarrow i'$ and $\emptyset\vdash_{\Theta_0} i':\mathbf{named}[Exp]$.
\end{property}

\paragraph{Type Safety and Preservation}
Extending Wyvern semantics with TSMs still constitutes a type safe language. We will outline the key theorems and lemmas here presenting the properties a type safe language has: 1) internal type safety 2) type preservation. 

To formalize the type safety property, we defined the judgments for context formation, type formation and bidirectional typing judgment. \todo{cite TR}

\begin{theorem}[Internal Type Safety]
If $\vdash\Theta$, and $\emptyset\vdash_{\Theta}i\Leftarrow\tau$ or $\vdash_{\Theta}i\Rightarrow\tau$, then either $i~\texttt{val}$ or $i\mapsto i'$ such that $\emptyset\vdash_{\Theta}i'\Leftarrow\tau$.
\end{theorem}
\begin{proof}
As the keyword extension on TSL framework does not extend internal Wyvern expressions ($i$), and this is the same as the proof in TSL. 
\end{proof}

\begin{theorem}[External Type Preservation]
If ~$\vdash_{\Theta_0}\Delta$, $\vdash\Theta$, $\vdash_{\Theta_0\Theta}\Psi$, $\vdash_{\Theta_0\Theta}\Gamma$, and $\Gamma\vdash_{\Theta_0\Theta}^{\Delta;\Psi} e\rightsquigarrow i\Leftarrow\tau$ or $\Gamma\vdash_{\Theta_0\Theta}^{\Delta;\Psi} e\rightsquigarrow i\Rightarrow\tau$ then $\Gamma\vdash_{\Theta} i\Leftarrow\tau$.
\end{theorem}
\begin{proof}
%Based on TSL proof of the external type preservation, we need to proof the following cases on keywords extension (Expressions for parser access is presented in the \todo{TR}, thus the corresponding proofs of the case for the expression is omitted here. ):
%\begin{itemize}
%\item $e=\mathbf{ekey}[k,body](e)$. According to the rule T-bk and T-wk, $\Gamma\vdash_{\Theta}\mathbf{ekey}[k,body](e_0) \rightsquigarrow i \Rightarrow \tau \Longrightarrow \Gamma;\emptyset\myvdash \hat{e} \rightsquigarrow i \Rightarrow \tau$. According to Lemma 1, $\Gamma;\emptyset\myvdash \hat{e} \rightsquigarrow i \Rightarrow \tau \Longrightarrow \Gamma;\emptyset\vdash_{\Theta} i\Rightarrow\tau$.
%\end{itemize}

To prove the theorem, we only need to consider the keyword invocation term, as other cases have similar proofs to those in TSL paper~\cite{TSLs}.
\begin{itemize}
\item $e=\mathbf{ekey}[k,body]$. 
\end{itemize} 
\end{proof}

\begin{lemma}[Translational Type Preservation]
If \\$\vdash_{\Theta_0}\Delta$, $\vdash_{\Theta_0}\Theta$, $\vdash_{\Theta_0\Theta}\Psi$ and $\vdash_{\Theta_0\Theta}\Gamma_{out}$, $\vdash_{\Theta_0\Theta}\Gamma$, $dom(\Gamma_{out})\cap dom(\Gamma)=\emptyset$ and $\Gamma_{out};\Gamma\vdash_{\Theta\Theta_0}^{\Delta;\Psi}\hat{e}\rightsquigarrow i\Leftarrow\tau$ or $\Gamma_{out};\Gamma\vdash_{\Theta\Theta_0}^{\Delta;\Psi}\hat{e}\rightsquigarrow i\Rightarrow \tau$ then $\Gamma_{out}\Gamma\vdash_{\Theta}i\Leftarrow \tau$.
\end{lemma}

\begin{theorem}[Compilation]
If ~$\rho\sim\Delta\sim\Theta\sim\Psi\rightsquigarrow i:\tau$~ then $\vdash_{\Theta_0}\Delta$,\ $\vdash_{\Theta_0}\Theta$, $\vdash_{\Theta_0\Theta}\Psi$ and $\emptyset\vdash_{\Theta_0\Theta} i\Leftarrow\tau$.
\end{theorem}
\begin{proof}
This theorem can be proved with the following two lemmas for the formation of $\Delta$ and $\Theta$.
\end{proof}

\begin{lemma}[Type Keyword Declarations] 
If $\vdash_{\Theta_0}\eta\sim\Delta$, then $\vdash_{\Theta_0}\Delta$.
\end{lemma}
\begin{proof}
The proof is simple a induction on $\vdash_{\Theta_0}\Delta$ and using the External Type Preservation Theorem. (Not shown)
\end{proof}

\begin{lemma}[Expression Keyword Declarations]\  \\
If $\vdash_{\Theta}\kappa\sim\Psi$, then $\vdash_{\Theta}\Psi$.
\end{lemma}

\begin{lemma}[Type Declaration]
If $\vdash_{\Theta_0}^{\Delta}\theta\sim\Theta$ then $\vdash\Theta_0\Theta$.
\end{lemma}
\begin{proof}
By induction on the formation of $\Theta$, we have the following three cases:
\begin{itemize}
\item $\vdash_{\Theta}\emptyset\sim\emptyset \Longrightarrow \vdash{\emptyset}$.
\item ${\vdash^{\Delta}_{\Theta}} \theta';T[\mathbf{explicit},\tau,e_m,\kappa] \sim \Theta',T[\tau,i_m:\tau_m,\dot\kappa] \Longrightarrow \vdash_{\Theta}\Theta',T[\tau,i_m:\tau_m,\dot{\kappa}]$. 

By induction, we have $\vdash_{\Theta}\Theta'$. And by the rule (ctx-explicit-type), we have $\vdash_{\Theta\Theta',T[?,?,?]}^{\Delta}\tau, \myvdash\dot\kappa$ and $\emptyset\vdash_{\Theta\Theta',T[\tau,?,?]}^{\Delta}e\rightsquigarrow i_m\Rightarrow\tau_m$. With External Type Preservation Lemma, we have $\vdash_{\Theta}\Theta',T[\tau,i_m:\tau_m,\dot{\kappa}]$.
\item $\myvdash \theta';T[\mathbf{tykw},k,body,e_m,\kappa] \sim \Theta',T[\tau,i''_m:\tau'_m,\dot{\kappa}\dot{\kappa}'] \Longrightarrow \vdash_{\Theta}\Theta',T[\tau,i''_m:\tau'_m,\dot{\kappa}\dot{\kappa}'].$

By induction, we have $\myvdash\Theta'$. And by rule (ctx-typekw-type), we have (1) $\vdash_{\Theta\Theta',T[?,?,?]}^{\Delta}\tau$. (2).$\dot\kappa\dot\kappa'$ is well formed the the fact that $\myvdash \dot\kappa$~~~~ $\vdash^{\Delta}_{\Theta\Theta',T[\tau,i'_m:\tau'_m,?]}\kappa\rightsquigarrow\dot{\kappa}'$ ~~~~ $dom(\dot{\kappa})\cap dom(\dot{\kappa}')=\emptyset$. (3) $i''_m:\tau'_m$ is well formed by the formation of $i'_m$ and the application of $i'_m$ on $i_m$. Thus we have the formation of the \textbf{tykw} construction.
\end{itemize}
With the three cases proved, we have the formation of type declaration proved.
\end{proof}

%\begin{figure*}
\flyingbox{$\vdash_{\Theta} \Delta_\kappa$}
\vspace{-25px}
\begin{center}
\AXC{}
\UIC{$\vdash_{\Theta} \emptyset$}
%%% the next rule
\AXC{$\Theta\subset\Theta_0$ ~~~~ $\vdash_{\Theta} \Delta_{\kappa}$ ~~~~ $k\notin dom(\Delta_{\kappa})$ ~~~~ $\vdash_{\Theta}i\Leftarrow TypeKw$}
\UIC{$\vdash_{\Theta}\Delta_{\kappa};k[i]$}
\noLine
\BIC{}
\DP
\end{center}

\flyingbox{$\vdash_{\Theta}\Theta$}
\vspace{-25px}
\begin{center}
\AXC{}
\UIC{$\vdash_{\Theta} \emptyset$}
%%% the next rule
\AXC{$\vdash_{\Theta} \Theta'$ ~~~~ $T\notin dom(\Theta')$ ~~~~ $\vdash_{\Theta,T[?,?,?]}\delta~ok$ ~~~~ $\vdash_{\Theta,T[\delta,?,?]}\mu~ok$ ~~~~ $\vdash_{\Theta,T[\delta,\mu,?]}\zeta~ok$}
\UIC{$\vdash_{\Theta}\Theta',T[\delta,\mu,\zeta]$}
\noLine
\BIC{}
\DP
\end{center}

\flyingbox{$\vdash_{\Theta} \delta~ok$}
\vspace{-25px}
\begin{center}
\AXC{}
\UIC{$\vdash_{\Theta} ?~ok$}
%%% the next rule
\AXC{$\vdash_{\Theta} \tau$}
\UIC{$\vdash_{\Theta} \tau~ok$}
\noLine
\BIC{}
\DP
\end{center}

\flyingbox{$\vdash_{\Theta} \mu~ok$}
\vspace{-25px}
\begin{center}
\AXC{}
\UIC{$\vdash_{\Theta} ?~ok$}
%%% the next rule
\AXC{$\emptyset\vdash_{\Theta} i\Leftarrow \tau$}
\UIC{$\vdash_{\Theta} i:\tau~ok$}
\noLine
\BIC{}
\DP
\end{center}

\flyingbox{$\vdash_{\Theta} \zeta~ok$}
\vspace{-25px}
\begin{center}
\AXC{}
\UIC{$\vdash_{\Theta} ?~ok$}
%%% the next rule
\AXC{$\emptyset\vdash_{\Theta} \dot\kappa$}
\UIC{$\vdash_{\Theta} \dot\kappa~ok$}
\noLine
\BIC{}
\DP
\end{center}

\flyingbox{$\vdash_{\Theta} \Gamma$}
\vspace{-25px}
\begin{center}
\AXC{}
\UIC{$\vdash_{\Theta} \emptyset~ok$}
%%% the next rule
\AXC{$\vdash_{\Theta} \Gamma$ ~~~~ $\emptyset\vdash_{\Theta} \tau$}
\UIC{$\vdash_{\Theta} \Gamma,x:\tau$}
\noLine
\BIC{}
\DP
\end{center}

\flyingbox{$\vdash_{\Theta}{\tau}$}
\vspace{-25px}
\begin{center}
\AXC{$\vdash_{\Theta} \omega$}
\UIC{$\vdash_{\Theta} \mathbf{objtype}[\omega]$}
%%% the next rule
\AXC{$\vdash_{\Theta} \chi$}
\UIC{$\vdash_{\Theta} \mathbf{casetype}[\chi]$}
%%% the next rule
\AXC{$\vdash_{\Theta} \tau_1$ ~~~~ $\vdash_{\Theta}\tau_2$}
\UIC{$\vdash_{\Theta} \mathbf{arrow}[\tau_1,\tau_2]$}
%%% the next rule
\AXC{$T[\tau,\mu,\zeta]\in\Theta$}
\UIC{$\vdash_{\Theta} \mathbf{named}[T]$}
\noLine
\QIC{}
\DP
\end{center}

\flyingbox{$\vdash_{\Theta} \omega$}
\vspace{-25px}
\begin{center}
\AXC{}
\UIC{$\vdash_{\Theta} \emptyset$}
%%% the next rule
\AXC{$l\notin dom(\omega)$ ~~~~ $\vdash_{\Theta}\tau$ ~~~~ $\vdash_{\Theta}\omega$}
\UIC{$\vdash_{\Theta} l[\tau];\omega$}
\noLine
\BIC{}
\DP
\end{center}

\flyingbox{$\vdash_{\Theta} \chi$}
\vspace{-25px}
\begin{center}
\AXC{}
\UIC{$\vdash_{\Theta} \emptyset$}
%%% the next rule
\AXC{$C\notin dom(\chi)$ ~~~~ $\vdash_{\Theta}\tau$ ~~~~ $\vdash_{\Theta}\chi$}
\UIC{$\vdash_{\Theta} C[\tau];\chi$}
\noLine
\BIC{}
\DP
\end{center}
\label{kw-contextformation}
\caption{Context formation}
\end{figure*}

% end the environment with {table*}, NOTE not {table}!
\section{Related Work}\label{related}
\section{Conclusions}
This paragraph will end the body of this sample document.
Remember that you might still have Acknowledgments or
Appendices; brief samples of these
follow.  There is still the Bibliography to deal with; and
we will make a disclaimer about that here: with the exception
of the reference to the \LaTeX\ book, the citations in
this paper are to articles which have nothing to
do with the present subject and are used as
examples only.
%\end{document}  % This is where a 'short' article might terminate

%ACKNOWLEDGMENTS are optional
\section{Acknowledgments}
This section is optional; it is a location for you
to acknowledge grants, funding, editing assistance and
what have you.  In the present case, for example, the
authors would like to thank Gerald Murray of ACM for
his help in codifying this \textit{Author's Guide}
and the \textbf{.cls} and \textbf{.tex} files that it describes.

%
% The following two commands are all you need in the
% initial runs of your .tex file to
% produce the bibliography for the citations in your paper.
\bibliographystyle{abbrv}
\bibliography{../ecoop14/research}  % sigproc.bib is the name of the Bibliography in this case
% You must have a proper ".bib" file
%  and remember to run:
% latex bibtex latex latex
% to resolve all references
%
% ACM needs 'a single self-contained file'!
%
%APPENDICES are optional
%\balancecolumns
\end{document}