

% This is "sig-alternate.tex" V1.9 April 2009
% This file should be compiled with V2.4 of "sig-alternate.cls" April 2009
%
% This example file demonstrates the use of the 'sig-alternate.cls'
% V2.4 LaTeX2e document class file. It is for those submitting
% articles to ACM Conference Proceedings WHO DO NOT WISH TO
% STRICTLY ADHERE TO THE SIGS (PUBS-BOARD-ENDORSED) STYLE.
% The 'sig-alternate.cls' file will produce a similar-looking,
% albeit, 'tighter' paper resulting in, invariably, fewer pages.
%
% ----------------------------------------------------------------------------------------------------------------
% This .tex file (and associated .cls V2.4) produces:
%       1) The Permission Statement
%       2) The Conference (location) Info information
%       3) The Copyright Line with ACM data
%       4) NO page numbers
%
% as against the acm_proc_article-sp.cls file which
% DOES NOT produce 1) thru' 3) above.
%
% Using 'sig-alternate.cls' you have control, however, from within
% the source .tex file, over both the CopyrightYear
% (defaulted to 200X) and the ACM Copyright Data
% (defaulted to X-XXXXX-XX-X/XX/XX).
% e.g.
% \CopyrightYear{2007} will cause 2007 to appear in the copyright line.
% \crdata{0-12345-67-8/90/12} will cause 0-12345-67-8/90/12 to appear in the copyright line.
%
% ---------------------------------------------------------------------------------------------------------------
% This .tex source is an example which *does* use
% the .bib file (from which the .bbl file % is produced).
% REMEMBER HOWEVER: After having produced the .bbl file,
% and prior to final submission, you *NEED* to 'insert'
% your .bbl file into your source .tex file so as to provide
% ONE 'self-contained' source file.
%
% ================= IF YOU HAVE QUESTIONS =======================
% Questions regarding the SIGS styles, SIGS policies and
% procedures, Conferences etc. should be sent to
% Adrienne Griscti (griscti@acm.org)
%
% Technical questions _only_ to
% Gerald Murray (murray@hq.acm.org)
% ===============================================================
%
% For tracking purposes - this is V1.9 - April 2009

\documentclass{sig-alternate}
  \pdfpagewidth=8.5truein
  \pdfpageheight=11truein

\usepackage{epstopdf}
\usepackage{bussproofs}
\EnableBpAbbreviations
\newcommand{\code}[1]{\texttt{\footnotesize #1}}
\newcommand{\todo}[1]{{\bf \{TODO: {#1}\}}}
\begin{document}
%
% --- Author Metadata here ---
\conferenceinfo{SAC'15}{April 13-17, 2015, Salamanca, Spain.}
\CopyrightYear{2015} % Allows default copyright year (2002) to be over-ridden - IF NEED BE.
\crdata{978-1-4503-3196-8/15/04}  % Allows default copyright data (X-XXXXX-XX-X/XX/XX) to be over-ridden.
% --- End of Author Metadata ---

\title{Alternate {\ttlit ACM} SIG Proceedings Paper in LaTeX
Format\titlenote{(Produces the permission block, and
copyright information). For use with
SIG-ALTERNATE.CLS. Supported by ACM.}}
\subtitle{[Extended Abstract]
\titlenote{A full version of this paper is available as
\textit{Author's Guide to Preparing ACM SIG Proceedings Using
\LaTeX$2_\epsilon$\ and BibTeX} at
\texttt{www.acm.org/eaddress.htm}}}
%
% You need the command \numberofauthors to handle the 'placement
% and alignment' of the authors beneath the title.
%
% For aesthetic reasons, we recommend 'three authors at a time'
% i.e. three 'name/affiliation blocks' be placed beneath the title.
%
% NOTE: You are NOT restricted in how many 'rows' of
% "name/affiliations" may appear. We just ask that you restrict
% the number of 'columns' to three.
%
% Because of the available 'opening page real-estate'
% we ask you to refrain from putting more than six authors
% (two rows with three columns) beneath the article title.
% More than six makes the first-page appear very cluttered indeed.
%
% Use the \alignauthor commands to handle the names
% and affiliations for an 'aesthetic maximum' of six authors.
% Add names, affiliations, addresses for
% the seventh etc. author(s) as the argument for the
% \additionalauthors command.
% These 'additional authors' will be output/set for you
% without further effort on your part as the last section in
% the body of your article BEFORE References or any Appendices.

\numberofauthors{8} %  in this sample file, there are a *total*
% of EIGHT authors. SIX appear on the 'first-page' (for formatting
% reasons) and the remaining two appear in the \additionalauthors section.
%
\author{
% You can go ahead and credit any number of authors here,
% e.g. one 'row of three' or two rows (consisting of one row of three
% and a second row of one, two or three).
%
% The command \alignauthor (no curly braces needed) should
% precede each author name, affiliation/snail-mail address and
% e-mail address. Additionally, tag each line of
% affiliation/address with \affaddr, and tag the
% e-mail address with \email.
%
% 1st. author
\alignauthor
Ben Trovato\titlenote{Dr.~Trovato insisted his name be first.}\\
       \affaddr{Institute for Clarity in Documentation}\\
       \affaddr{1932 Wallamaloo Lane}\\
       \affaddr{Wallamaloo, New Zealand}\\
       \email{trovato@corporation.com}
% 2nd. author
\alignauthor G.K.M. Tobin\titlenote{The secretary disavows
any knowledge of this author's actions.}\\
       \affaddr{Institute for Clarity in Documentation}\\
       \affaddr{P.O. Box 1212}\\
       \affaddr{Dublin, Ohio 43017-6221}\\
       \email{webmaster@marysville-ohio.com}
% 3rd. author
\alignauthor Lars Th{\o}rv{\"a}ld\titlenote{This author is the
one who did all the really hard work.}\\
       \affaddr{The Th{\o}rv{\"a}ld Group}\\
       \affaddr{1 Th{\o}rv{\"a}ld Circle}\\
       \affaddr{Hekla, Iceland}\\
       \email{larst@affiliation.org}
\and  % use '\and' if you need 'another row' of author names
% 4th. author
\alignauthor Lawrence P. Leipuner\\
       \affaddr{Brookhaven Laboratories}\\
       \affaddr{Brookhaven National Lab}\\
       \affaddr{P.O. Box 5000}\\
       \email{lleipuner@researchlabs.org}
% 5th. author
\alignauthor Sean Fogarty\\
       \affaddr{NASA Ames Research Center}\\
       \affaddr{Moffett Field}\\
       \affaddr{California 94035}\\
       \email{fogartys@amesres.org}
% 6th. author
\alignauthor Charles Palmer\\
       \affaddr{Palmer Research Laboratories}\\
       \affaddr{8600 Datapoint Drive}\\
       \affaddr{San Antonio, Texas 78229}\\
       \email{cpalmer@prl.com}
}
% There's nothing stopping you putting the seventh, eighth, etc.
% author on the opening page (as the 'third row') but we ask,
% for aesthetic reasons that you place these 'additional authors'
% in the \additional authors block, viz.
\additionalauthors{Additional authors: John Smith (The
Th{\o}rv{\"a}ld Group, email: {\texttt{jsmith@affiliation.org}})
and Julius P.~Kumquat (The Kumquat Consortium, email:
{\texttt{jpkumquat@consortium.net}}).}
\date{30 July 1999}
% Just remember to make sure that the TOTAL number of authors
% is the number that will appear on the first page PLUS the
% number that will appear in the \additionalauthors section.

\maketitle
\begin{abstract}
This is abstract.
\end{abstract}

% A category with the (minimum) three required fields
\category{H.4}{Information Systems Applications}{Miscellaneous}
%A category including the fourth, optional field follows...
\category{D.2.8}{Software Engineering}{Metrics}[complexity measures, performance measures]

\terms{Delphi theory}

\keywords{ACM proceedings, \LaTeX, text tagging}

\section{Introduction}

\section{Motivation}
Typically, the body of a paper is organized
into a hierarchical structure, with numbered or unnumbered
headings for sections, subsections, sub-subsections, and even
smaller sections.
\section{Formalization}
\subsection{Abstract Syntax}
\begin{figure*}[htb!]
\begin{align*}
      \rho        &::=~ \theta;e\\
      \theta      &::=~ \emptyset ~ | ~ \theta; T[\tau,e,\kappa]\\
      \tau        &::=~ \mathbf{named}[T] ~ | ~ \mathbf{objtype}[\omega] \\
                  &\quad | ~ \mathbf{casetype}[\chi] ~ | ~ \mathbf{arrow}[\tau,\tau]\\
      \omega      &::=~\emptyset ~|~ l[\tau];\omega\\
      \chi        &::=~\emptyset ~|~ C[\tau];\chi\\
      \kappa      &::=~ \emptyset \\
                  &\quad | ~ \kappa;k[\mathbf{bk}(\tau),e] \\
                  &\quad | ~ \kappa;k[\mathbf{wk},e] \\
      e           &::=~ ...\\
                  &\quad |~ \mathbf{ekey}[k,body](e) \\
                  &\quad |~ \mathbf{ekeydef}[T,k]\\
      \hat{e}     &::=~ ...\\
      \dot{e}     &::=~ ...
\end{align*}
\caption{Abstract Syntax for Expression Keyword}
\label{formal-syntax}
\end{figure*}

\subsection{Typechecking and Elabaration}
\begin{figure*}[htb!]
\begin{align*}
      \Theta &::=~ \emptyset ~~ | ~~ \Theta,T[\delta,\mu,\zeta] \\
      \delta &::=~ ? ~~ | ~~ \tau\\
      \mu      &::=~ ? ~~ | ~~ i:\tau\\
            \zeta  &::=~ ? ~~ | ~~ \gamma\\
      \gamma &::=~ ~ \emptyset \\
                  &\quad | ~~~\gamma;k[\mathbf{bk}(\tau),i]\\
                  &\quad | ~~~\gamma;k[\mathbf{wk},i]
\end{align*}
\caption{Definition of Environment}
\label{typechecking-environment}
\end{figure*}
\begin{figure*}
\begin{flushleft}\fbox{$\vdash_{\Theta}\theta\sim \Theta$}\end{flushleft}
\begin{center}
\AXC{$\vdash_{\Theta_{0}}\theta \sim_{names}\Theta_{names} ~~~~~ \vdash_{\Theta_{0}\Theta_{names}}\theta \sim_{defs}\Theta_{defs} ~~~~~ \vdash_{\Theta_{0}\Theta_{defs}}\theta \sim_{exts}\Theta $} \RightLabel{~~~ (rec-decls)}
\UIC{$\vdash_{\Theta_{0}} \theta \sim \Theta$} 
\DP
\end{center}

\begin{flushleft}\fbox{$\vdash_{\Theta}\theta\sim_{names} \Theta$}\end{flushleft}
\begin{center}
\AXC{} \RightLabel{~~~ (empty-names)}
\UIC{$\vdash_{\Theta} \emptyset \sim_{names} \emptyset$}
\AXC{$\vdash_{\Theta}\theta' \sim_{names} \Theta' ~~~~~~ T\notin {dom}(\Theta) ~~~~~~ T\notin dom(\Theta')$} \RightLabel{~~~ (type-names)}
\UIC{$\vdash_{\Theta} \theta';T[\tau,e_m,\kappa] \sim_{names} \Theta',T[?,?,?]$}
\noLine
\BIC{}
\DP
\end{center}

\begin{flushleft}\fbox{$\vdash_{\Theta}\theta\sim_{defs} \Theta$}\end{flushleft}
\begin{center}
\AXC{} \RightLabel{~~~ (empty-defs)}
\UIC{$\vdash_{\Theta} \emptyset \sim_{defs} \emptyset$}
\AXC{$\vdash_{\Theta}\theta' \sim_{defs} \Theta' ~~~~~ \vdash_{\Theta}\tau $}\RightLabel{~~~ (type-defs)}
\UIC{$\vdash_{\Theta}\theta';T[\tau,e_m,\kappa] \sim_{defs} \Theta',T[\tau,?,?]$}
\noLine
\BIC{}
\DP
\end{center}


\begin{flushleft}\fbox{$\vdash_{\Theta}\theta\sim_{exts} \Theta$}\end{flushleft}
\begin{center}
\AXC{} \RightLabel{~~~ (empty-exts)}
\UIC{$\vdash_{\Theta} \emptyset \sim_{exts} \emptyset$}
\DP
\end{center}

\begin{center}
\AXC{$\vdash_{\Theta_{0},T[\tau,?,?],\Theta_1}\theta'\sim_{exts}\Theta' $ ~~~~~ $\emptyset\vdash_{\Theta_0,T[\tau,?,?],\Theta_1}e_m\rightsquigarrow i_m\Rightarrow \tau_m$ ~~~~~ $\emptyset\vdash_{\Theta_0,T[\tau,?,?],\Theta_1}\kappa \sim \gamma$}             \RightLabel{~~~ (type-exts)}
\UIC{$\vdash_{\Theta_{0},T[\tau,?,?],\Theta_1}\theta';T[\tau,e_m,\kappa] \sim_{exts} \Theta',T[\tau,i_m:\tau_m,\gamma]$}
\DP
\end{center}
\label{typechecking-declaration}
\caption{Type declaration checking}
\end{figure*}
\begin{figure*}[htb!]
\begin{flushleft}\fbox{$\vdash\Theta$}\end{flushleft}
\begin{center}
\AXC{}      \RightLabel{~~~(Th-empty)}
\UIC{$\vdash \emptyset$}
\AXC{$\vdash\Theta$ ~~~~ $T\notin dom(\Theta)$ ~~~~ $\vdash_{\Theta,T[?,?,?]}\delta~\mathit{ok}$ ~~~~ $\vdash_{\Theta,T[\delta,?,?]}\mu~\mathit{ok}$ ~~~~~ $\vdash_{\Theta;T[\delta,\mu,?]}\zeta~\mathit{ok}$}\RightLabel{~~~(Th-extend)}
\UIC{$\vdash \Theta,T[\delta,\mu,\zeta]$}
\noLine
\BIC{}
\DP
\end{center}

\begin{flushleft}\fbox{$\vdash_{\Theta}\delta~\mathit{ok}$}\end{flushleft}
\begin{center}
\AXC{}      \RightLabel{~~~(def-unknown-ok)}
\UIC{$\vdash_{\Theta}~?~\mathit{ok}$}
\AXC{$\vdash_{\Theta}\tau$}   \RightLabel{~~~(def-type-ok)}
\UIC{$\vdash_{\Theta}\tau~\mathit{ok}$}
\noLine
\BIC{}
\DP
\end{center}

\begin{flushleft}\fbox{$\vdash_{\Theta}\tau$}\end{flushleft}
\begin{center}
\AXC{$\vdash_{\Theta}\omega$}\RightLabel{~~~(Ty-ot)}
\UIC{$\vdash_{\Theta}\mathbf{objtype}[\omega]$}
\AXC{$\vdash_{\Theta}\chi$}\RightLabel{~~~(Ty-ct)}
\UIC{$\vdash_{\Theta}\mathbf{casetype}[\chi]$}
\AXC{$\vdash_{\Theta}\tau_1$ ~~~~ $\vdash_{\Theta}\tau_2$}  \RightLabel{~~~(Ty-arrow)}
\UIC{$\vdash_{\Theta}\mathbf{arrow}[\tau_1, \tau_2]$} 
\AXC{$T[\tau,\mu,\zeta]\in\Theta$}\RightLabel{~~~(Ty-named)}
\UIC{$\vdash_{\Theta}\mathbf{named}[T]$}
\noLine
\QIC{}
\DP
\end{center}

\begin{flushleft}\fbox{$\vdash_{\Theta}\mu~\mathit{ok}$}\end{flushleft}
\begin{center}
\AXC{}      \RightLabel{~~~(metadata-unknown)}
\UIC{$\vdash_{\Theta}~?~\mathit{ok}$}
\AXC{$\emptyset\vdash_{\Theta} i\Leftarrow \tau$}\RightLabel{~~~(metadata)}
\UIC{$\vdash_{\Theta}i:\tau~\mathit{ok}$}
\noLine
\BIC{}
\DP
\end{center}

\begin{flushleft}\fbox{$\vdash_{\Theta}\Gamma$}\end{flushleft}
\begin{center}
\AXC{}      \RightLabel{~~~(G-empty)}
\UIC{$\vdash_{\Theta} \emptyset$}
\AXC{$\vdash_{\Theta} \tau$}\RightLabel{~~~(G-extend)}
\UIC{$\vdash_{\Theta}\Gamma,x:\tau$}
\noLine
\BIC{}
\DP
\end{center}

\begin{flushleft}\fbox{$\vdash_{\Theta}\omega$}\end{flushleft}
\begin{center}
\AXC{$\mathit{l}\notin dom(\omega)$ ~~~~ $\vdash_{\Theta}\tau$ ~~~~ $\vdash_{\Theta}\omega$}    \RightLabel{~~~(M-decl)}
\UIC{$\vdash_{\Theta}\mathit{l}[\tau];\omega$}
\DP
\end{center}

\begin{flushleft}\fbox{$\vdash_{\Theta}\chi$}\end{flushleft}
\begin{center}

\AXC{$C\notin dom(\chi)$ ~~~~ $\vdash_{\Theta}\tau$ ~~~~ $\vdash_{\Theta}\chi$}     \RightLabel{~~~(C-decl)}
\UIC{$\vdash_{\Theta}C[\tau];\chi$}
\DP
\end{center}

\begin{flushleft}\fbox{$\vdash_{\Theta}\zeta~\mathit{ok}$}\end{flushleft}
\begin{center}
      \AXC{}      \RightLabel{~~~(kw-unknown)}
      \UIC{$\vdash_{\Theta}~?~\mathit{ok}$}
      \AXC{$\vdash_{\Theta}\gamma$} \RightLabel{~~~(kw-ok)}
      \UIC{$\vdash_{\Theta}\gamma~\mathit{ok}$}
      \noLine
      \BIC{}
      \DP
\end{center}

\begin{flushleft}\fbox{$\vdash_{\Theta}\gamma$}\end{flushleft}
\begin{center}
      \AXC{}      \RightLabel{~~~(kw-empty)}
      \UIC{$\vdash_{\Theta}\emptyset$}
      \AXC{$\Theta_0 \subset \Theta$ ~~~~ $\vdash_{\Theta}\gamma'$ ~~~~ $\vdash_{\Theta}k\notin dom(\gamma') ~~~~ \vdash_{\Theta}e\rightsquigarrow i\Leftarrow Keyword$}
      \UIC{$\vdash_{\Theta} \gamma';k[\mathbf{wk},e]$}
      \noLine
      \BIC{}
      \DP
\end{center}
\begin{center}
      \AXC{$\Theta_0 \subset \Theta$ ~~~~ $\vdash_{\Theta}\gamma'$ ~~~~ $\vdash_{\Theta}k\notin dom(\gamma') ~~~~ \vdash_{\Theta}\tau ~~~~\vdash_{\Theta}e\rightsquigarrow i\Leftarrow Keyword$}
      \UIC{$\vdash_{\Theta} \gamma';k[\mathbf{bk(\tau)},e]$}
      \DP
\end{center}


\begin{flushleft}\fbox{$\vdash_{\Theta}\kappa\rightsquigarrow\gamma$}\end{flushleft}
\begin{center}
      \AXC{$\Theta_0 \subset \Theta$ ~~~~ $\vdash_{\Theta}\kappa\rightsquigarrow\gamma$ ~~~~ $k\notin dom(\gamma)$ ~~~~ $\vdash_{\Theta} e_k\rightsquigarrow i_k \Leftarrow Keyword$} \RightLabel{~~~(white-box-keyword)}
      \UIC{$\vdash_{\Theta} \kappa;k[\mathbf{wk},e_{k}] \rightsquigarrow \gamma;k[\mathbf{wk}, i_k]$}
      \DP
\end{center}

\begin{center}
      \AXC{$\Theta_0 \subset \Theta$ ~~~~ $\vdash_{\Theta}\kappa\rightsquigarrow\gamma$ ~~~~ $k\notin dom(\gamma)$ ~~~~ $\vdash_{\Theta}\tau$ ~~~~ $\vdash_{\Theta}e_k\rightsquigarrow i_k \Leftarrow Keyword$} \RightLabel{~~~(black-box-keyword)}
      \UIC{$\vdash_{\Theta}\kappa;k[\mathbf{bk}(\tau),e_{k}] \rightsquigarrow \gamma;k[\mathbf{bk}(\tau),i_k]$}
      \DP
\end{center}
\label{typechecking-elaboration}
\caption{Typechecking and elaboration of programs}
\end{figure*}
\begin{figure*}[htb!]
\vspace{3pt}
\begin{center}
\AXC{$\Theta_0\in\Theta ~~~~ T[\delta, \mu, \gamma]\in\Theta ~~~~ k[\_, i] \in \gamma$}   \RightLabel{~~~(T-Keyworddef)}
\UIC{$\Gamma\vdash_{\Theta} \mathbf{ekeydef}[T,k] \rightsquigarrow i\Rightarrow Keyword$}
\DP
\end{center}


\vspace{3pt}
\begin{center}
\AXC{$\Theta_0 \subset \Theta$ ~~~~ $\Gamma\vdash_{\Theta} e_0\rightsquigarrow i_0 \Rightarrow T$ ~~~~~ $T[\delta,\mu,\gamma]\in\Theta$ ~~~~~ $k[\mathbf{bk}(\tau),i_k]\in\gamma$ }
\noLine
\UIC{$\texttt{parsestream}(body)=i_{ps}$ ~~~~~ $\mathbf{iap}(\mathbf{iprj}[parse](\mathbf{iprj}[parser](i_k)); i_{ps})\Downarrow \mathbf{iinj}[OK]((i_{ast}, i'_{ps}))$}
\noLine
\UIC{~~~~~~~~~~~~~~~~~~~~~~~~~~~~~~~~~~~~~~~~$i_{ast}\uparrow \hat{e}$ ~~~~~~ $\Gamma;\emptyset\vdash_{\Theta} \hat{e} \rightsquigarrow i \Leftarrow \tau$~~~~~~~~~~~~~~~~~~~~~~~~~~~~~~~~~~~~~~~~}               \RightLabel{(T-bk)}
\UIC{$\Gamma\vdash_{\Theta}\mathbf{ekey}[k,body](e_0) \rightsquigarrow i\Rightarrow \tau$}  
\DP
\end{center}

\vspace{3pt}
\begin{center}
\AXC{$\Theta_0 \subset \Theta$ ~~~~ $\Gamma\vdash_{\Theta} e_0\rightsquigarrow i_0 \Rightarrow T$ ~~~~~ $T[\delta,\mu,\gamma]\in\Theta$ ~~~~~ $k[\mathbf{wk},i_k]\in\gamma$ }
\noLine
\UIC{$\texttt{parsestream}(body)=i_{ps}$ ~~~~~ $\mathbf{iap}(\mathbf{iprj}[parse](\mathbf{iprj}[parser](i_k)); i_{ps})\Downarrow \mathbf{iinj}[OK]((i_{ast}, i'_{ps}))$}
\noLine
\UIC{~~~~~~~~~~~~~~~~~~~~~~~~~~~~~~~~~~~~~~~~$i_{ast}\uparrow \hat{e}$ ~~~~~~ $\Gamma;\emptyset\vdash_{\Theta} \hat{e} \rightsquigarrow i \Rightarrow \tau$~~~~~~~~~~~~~~~~~~~~~~~~~~~~~~~~~~~~~~~~}              \RightLabel{(T-wk-syn)}
\UIC{$\Gamma\vdash_{\Theta}\mathbf{ekey}[k,body](e_0) \rightsquigarrow i\Rightarrow \tau$}  
\DP
\end{center}

\vspace{3pt}
\begin{center}
\AXC{$\Theta_0 \subset \Theta$ ~~~~ $\Gamma\vdash_{\Theta} e_0\rightsquigarrow i_0 \Rightarrow T$ ~~~~~ $T[\delta,\mu,\gamma]\in\Theta$ ~~~~~ $k[\mathbf{wk},i_k]\in\gamma$ }
\noLine
\UIC{$\texttt{parsestream}(body)=i_{ps}$ ~~~~~ $\mathbf{iap}(\mathbf{iprj}[parse](\mathbf{iprj}[parser](i_k)); i_{ps})\Downarrow \mathbf{iinj}[OK]((i_{ast}, i'_{ps}))$}
\noLine
\UIC{~~~~~~~~~~~~~~~~~~~~~~~~~~~~~~~~~~~~~~~~$i_{ast}\uparrow \hat{e}$ ~~~~~~ $\Gamma;\emptyset\vdash_{\Theta} \hat{e} \rightsquigarrow i \Leftarrow \tau$~~~~~~~~~~~~~~~~~~~~~~~~~~~~~~~~~~~~~~~~}               \RightLabel{(T-wk-ana)}
\UIC{$\Gamma\vdash_{\Theta}\mathbf{ekey}[k,body](e_0) \rightsquigarrow i\Leftarrow \tau$}  
\DP
\end{center}
\label{expkw-kwstatics}
\caption{Statics for Keyword Invocation}
\end{figure*}

% end the environment with {table*}, NOTE not {table}!

\section{Conclusions}
This paragraph will end the body of this sample document.
Remember that you might still have Acknowledgments or
Appendices; brief samples of these
follow.  There is still the Bibliography to deal with; and
we will make a disclaimer about that here: with the exception
of the reference to the \LaTeX\ book, the citations in
this paper are to articles which have nothing to
do with the present subject and are used as
examples only.
%\end{document}  % This is where a 'short' article might terminate

%ACKNOWLEDGMENTS are optional
\section{Acknowledgments}
This section is optional; it is a location for you
to acknowledge grants, funding, editing assistance and
what have you.  In the present case, for example, the
authors would like to thank Gerald Murray of ACM for
his help in codifying this \textit{Author's Guide}
and the \textbf{.cls} and \textbf{.tex} files that it describes.

%
% The following two commands are all you need in the
% initial runs of your .tex file to
% produce the bibliography for the citations in your paper.
\bibliographystyle{abbrv}
\bibliography{sigproc}  % sigproc.bib is the name of the Bibliography in this case
% You must have a proper ".bib" file
%  and remember to run:
% latex bibtex latex latex
% to resolve all references
%
% ACM needs 'a single self-contained file'!
%
%APPENDICES are optional
%\balancecolumns
\appendix
%Appendix A
\section{Headings in Appendices}
The rules about hierarchical headings discussed above for
the body of the article are different in the appendices.
In the \textbf{appendix} environment, the command
\textbf{section} is used to
indicate the start of each Appendix, with alphabetic order
designation (i.e. the first is A, the second B, etc.) and
a title (if you include one).  So, if you need
hierarchical structure
\textit{within} an Appendix, start with \textbf{subsection} as the
highest level. Here is an outline of the body of this
document in Appendix-appropriate form:
% This next section command marks the start of
% Appendix B, and does not continue the present hierarchy
\section{More Help for the Hardy}
The sig-alternate.cls file itself is chock-full of succinct
and helpful comments.  If you consider yourself a moderately
experienced to expert user of \LaTeX, you may find reading
it useful but please remember not to change it.
%\balancecolumns % GM June 2007
% That's all folks!
\end{document}