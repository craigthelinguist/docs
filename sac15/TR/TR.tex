\documentclass[letterpaper, notitlepage]{article}
\usepackage{bussproofs}
\usepackage{amssymb}
\usepackage{latexsym}
\usepackage{fancyhdr}
\usepackage{amsthm}
\usepackage{hyperref}
\usepackage{amsmath}
\usepackage{stackengine}
\usepackage[usenames,dvipsnames]{color} % Required for specifying custom colors and referring to colors by name
\usepackage{listings}
\usepackage[top=1in, bottom=1in, left=1.25in, right=1.25in]{geometry}
\usepackage{xcolor}
\usepackage{bera}% optional; just for the example

\definecolor{DarkGreen}{rgb}{0.0,0.4,0.0} % Comment color
\definecolor{highlight}{RGB}{255,251,204} % Code highlight color
% This is the "centered" symbol
\def\fCenter{{\mbox{\Large$\rightarrow$}}}

% Optional to turn on the short abbreviations
\EnableBpAbbreviations

% This is the "centered" symbol
\def\fCenter{{\mbox{\Large$\rightarrow$}}}

\newcommand{\blue}[1] {\textcolor{blue}{#1}}
\newcommand{\flyingbox}[1]{\begin{flushleft}\fbox{{#1}}\end{flushleft}}
\newcommand{\doublebox}[2]{\begin{flushleft}\fbox{{#1}}\ \fbox{{#2}}\end{flushleft}}
\newcommand{\myvdash}{\vdash_{\Theta}^{\Delta}}
\newcommand{\todo}[1]{{\bf \{TODO: {#1}\}}}

\newtheorem{theorem}{Theorem}
\newtheorem{lemma}{Lemma}
\newtheorem{definition}{Definition}
\newtheorem{property}{Property}

% Optional to turn on the short abbreviations
\EnableBpAbbreviations

% \alwaysRootAtTop  % makes proofs upside down
% \alwaysRootAtBottom % -- this is the default setting

\lstdefinestyle{wyvern}{ % Define a style for your code snippet, multiple definitions can be made if, for example, you wish to insert multiple code snippets using different programming languages into one document
%backgroundcolor=\color{highlight}, % Set the background color for the snippet - useful for highlighting
basicstyle=\footnotesize\ttfamily, % The default font size and style of the code
breakatwhitespace=false, % If true, only allows line breaks at white space
breaklines=true, % Automatic line breaking (prevents code from protruding outside the box)
captionpos=b, % Sets the caption position: b for bottom; t for top
morecomment=[s]{(*}{*)},
commentstyle=\fontshape{it}\color{DarkGreen}\selectfont, % Style of comments within the code - dark green courier font
deletekeywords={}, % If you want to delete any keywords from the current language separate them by commas
%escapeinside={\%}, % This allows you to escape to LaTeX using the character in the bracket
firstnumber=1, % Line numbers begin at line 1
frame=lines, % Frame around the code box, value can be: none, leftline, topline, bottomline, lines, single, shadowbox
frameround=tttt, % Rounds the corners of the frame for the top left, top right, bottom left and bottom right positions
keywords=[1]{new, objtype, type, casetype, val, def, metadata, keyword, of},
keywordstyle={[1]\ttfamily\color{blue!90!black}},
keywordstyle={[3]\ttfamily\color{red!80!orange}},
morekeywords={}, % Add any functions no included by default here separated by commas
numbers=left, % Location of line numbers, can take the values of: none, left, right
numbersep=10pt, % Distance of line numbers from the code box
numberstyle=\tiny\color{Gray}, % Style used for line numbers
rulecolor=\color{black}, % Frame border color
showstringspaces=false, % Don't put marks in string spaces
showtabs=false, % Display tabs in the code as lines
stepnumber=5, % The step distance between line numbers, i.e. how often will lines be numbered
tabsize=4, % Number of spaces per tab in the code
}


\begin{document}

\title{Composable and Hygienic Typed Syntax Macros}
\maketitle

\section{Wyvern Prelude}
\begin{lstlisting}[style=wyvern]
type TypeParser = objtype
	def parse(ps : ParseStream) : Result(Type * Exp)
	metadata : HasTSL = new 
		val parser = (* parser generator *)

type Type = casetype
	Named of ID
	Objtype of List(MemberDecl)
	Casetype of List(CaseDecl)
	Arrow of Type * Type
	metadata : HasTSL = new
		val parser = (* type quasiquotes *)

type MemberList = casetype
	Nil  of Unit
	Cons of Label * Type * MemberList

type CaseList = casetype
	Nil  of Unit
	Cons of Label * Type * CaseList

type ExpParser = objtype
	def parse(ps : ParseStream) : Result(Exp)
	metadata : HasTSL = new 
		val parser = (* parser generator *)

type Exp = casetype
	Var of ID
	Lam of ID * Exp
	Ap of Exp * Exp
	(* ... *)
	metadata : HasTSL = new
		val parser = (* exp quasiquotes *)

type Result(T) = casetype
	OK of Exp * ParseStream
	Error of String * Location
\end{lstlisting}

\section{Syntax \& Type Checking}
\subsection{Syntax}
\[
\begin{array}{rlrlrl}
	\rho		~::=&~ \theta;e\\				
	\theta		~::=&~ \emptyset								&\tau 		~::=&~ \mathbf{named}[T]	& \omega 		~::=&~ 	\emptyset\\
				| ~	&~ \theta; T[\mathbf{explicit}, \tau, e]	&|~	& ~ \mathbf{objtype}[\omega]		& |~&  	~l[\mathbf{val}, \tau];\omega\\
				| ~	&~ \theta; T[\mathbf{tykw}, k, body, e]		&|~	& ~ \mathbf{casetype}[\chi]			& |~&	~l[\mathbf{def}, \tau];\omega\\
				| ~	&~ \theta;k[\mathbf{ty},e,\tau]				&|~ & ~ \mathbf{arrow}[\tau, \tau]		& \chi 			~::=&~	\emptyset ~ \\
				| ~ &~ \theta;k[\mathbf{bk}(\tau),e]			&	&									& |~&	~C[\tau];\omega\\
				| ~ &~ \theta;k[\mathbf{wk},e]\\
	e 			~::=&~ x 								&\hat{e}	~::=&~ 	x 										& i 		~::=&~ 	x\\
				| ~ &~ \mathbf{easc}[\tau](e)			& 		 	| ~ &~ 	\mathbf{hasc}[\tau](\hat{e})			& 		 	| ~ &~	\mathbf{iasc}[\tau](\dot{e})\\
				| ~ &~ \mathbf{elet}(e; x.e)     		& 		 	| ~ &~ 	\mathbf{hlet}(\hat{e}; x.\hat{e})		& 		 	| ~ &~	\mathbf{ilet}(i;x.i)\\
				| ~ &~ \mathbf{elam}(x.e)     			& 		 	| ~ &~ 	\mathbf{hlam}(x.\hat{e})				& 		 	| ~ &~	\mathbf{ilam}(x.i)\\
				| ~ &~ \mathbf{eap}(e;e)     			& 		 	| ~ &~ 	\mathbf{hap}(\hat{e};\hat{e})			& 		 	| ~ &~	\mathbf{iap}(i;i)\\
				| ~ &~ \mathbf{enew}(m)     			& 		 	| ~ &~	\mathbf{hnew}(\hat{m})					& 		 	| ~ &~	\mathbf{inew}(\dot{m})\\
				| ~ &~ \mathbf{eprj}[l](e)     			& 		 	| ~ &~	\mathbf{hprj}[l](\hat{e})				& 		 	| ~ &~	\mathbf{iprj}[l](i)\\
				| ~ &~ \mathbf{einj}[C](e)     			& 		 	| ~ &~	\mathbf{hinj}[C](\hat{e})				& 		 	| ~ &~	\mathbf{iinj}[C](i)\\
				| ~ &~ \mathbf{ecase}[e]\{r\}     		& 		 	| ~ &~	\mathbf{hcase}[\hat{e}]\{\hat{r}\}		& 		 	| ~ &~	\mathbf{icase}[i]\{\dot{r}\}\\
				| ~ &~ \mathbf{etoast}(e)     			& 		 	| ~ &~	\mathbf{htoast}(\hat{e})				& 		 	| ~ &~	\mathbf{itoast}[i]\\
				| ~ &~ \mathbf{emetadata}[T]     		& 		 	| ~ &~	\mathbf{hmetadata}[T]\\
				| ~ &~ \mathbf{etermtsmdef}[k]     		&			| ~ &~ 	\mathbf{htermtsmdef}[k]\\
				| ~ &~ \mathbf{etypetsmdef}[k]			& 			| ~ &~ 	\mathbf{htypetsmdef}[k]\\
				| ~ &~ \mathbf{lit}[T]					& 		 	| ~ &~ 	\mathbf{spliced}[e]\\
				| ~ &~ \mathbf{etsmap}[k,body]\\
	m 			~::=&~ \emptyset						&\hat{m}	~::=&~ \emptyset								&\dot{m}	~::=&~ \emptyset\\
				| ~ &~ \mathbf{eval}[l](e);m 			&			| ~ &~ \mathbf{hval}[l](\hat{e});\hat{m} 		&			| ~ &~ \mathbf{ival}[l](i);\dot{m}\\
				| ~ &~ \mathbf{edef}[l](x.e);m 			&			| ~ &~ \mathbf{hdef}[l](x.\hat{e});\hat{m}		&			| ~ &~ \mathbf{idef}[l](x.i);\dot{m}\\
	r 			~::=&~ \emptyset 						&\hat{r} 	~::=&~ \emptyset 								&\dot{r} 	~::=&~ \emptyset\\
				| ~ &~ \mathbf{erule}[C](x.e);r 		& 			| ~ &~ \mathbf{hrule}[C](x.\hat{e});\hat{r} 	&			| ~ &~ \mathbf{irule}[C](x.i);\dot{r}
\end{array}
\]
\subsection{Context Definition}
\[
\begin{array}{rrl}
\text{Keyword Context}	&	\Delta 	~::=&~ 	\emptyset\\
						&			| ~ &~ 	\Delta;k[\mathbf{ty}(\gamma,\delta)]\\
						&			| ~ &~ 	\Delta;k[\mathbf{syn}(\delta,\gamma)]\\
						&			| ~ &~ 	\Delta;k[\mathbf{ana}(\gamma)]\\
\text{Type Context}		&	\Theta 	~::=&~ \emptyset\\
						&			| ~ &~ \Theta,T[\delta,\mu] \\
						&	\delta 	~::=&~ ? ~~ | ~~ \tau\\
						&	\mu	   	~::=&~ ? ~~ | ~~ i:\tau\\
						& 	\gamma  ~::=&~ ? ~~ | ~~ i\\
\text{Typing Context}	&	\Gamma 	~::=&~ \emptyset\\
						&			| ~ &~ \Gamma,x:\tau\\
\end{array}
\]

\subsection{Type Checking and Elaboration}
\begin{definition}[Context Override]Given two type contexts $\Theta_1$ and $\Theta_2$, we define $\Theta_1\Theta_2$ as a new context $\Theta'$, which satisfies:
\[
	\Theta_1\Theta_2 = \Theta' = \{~T[\delta,\mu]~|~ (T\notin dom(\Theta_2)\land T[\delta,\mu]\in\Theta_1)\lor(T[\delta,\mu]\in\Theta_2) ~\}
\]
Similarly, we define $\Delta_1\Delta_2$ as a new keyword context $\Delta'$, which satisfies:
\[
	\begin{array}{rl}
	\Delta_1\Delta_2=\Delta'=&\{~k[\mathbf{tytsm}(\gamma,\delta)]~|~ (k\notin dom(\Delta_2) \land k[\mathbf{tytsm}(\gamma,\delta)]\in\Delta_1)\lor(k[\mathbf{tytsm}(\gamma,\delta)]\in\Delta_2)  ~\}\\
							 &\cup~\{~k[\mathbf{syntsm}(\delta,\gamma)]~|~ (k\notin dom(\Delta_2) \land k[\mathbf{syntsm}(\delta,\gamma)]\in\Delta_1) \lor(k[\mathbf{syntsm}(\delta,\gamma)]\in\Delta_2)  ~\}\\
							 &\cup~\{~k[\mathbf{anatsm}(\gamma)]~|~ (k\notin dom(\Delta_2) \land k[\mathbf{anatsm}(\gamma)]\in\Delta_1) \lor(k[\mathbf{anatsm}(\gamma)]\in\Delta_2)  ~\}
	\end{array}
\]
\end{definition}
By this definition, we are able to present the `override' of $\Theta_1$ (or $\Delta_1$) by a new context $\Theta_2$ (or $\Delta_2$), which can update the old context with new elements from the new context.

\flyingbox{$\rho \sim (\Theta;\Delta)\rightsquigarrow i:\tau$}
\begin{center}
\AXC{$\vdash_{\Theta_{0}}^{\emptyset}\theta\sim(\Theta;\Delta)$ ~~~~ $\emptyset\vdash_{\Theta_0\Theta}^{\Delta}e\rightsquigarrow i\Rightarrow \tau$}      \RightLabel{(compile)}
\UIC{$\theta;e\sim(\Theta;\Delta)\rightsquigarrow i:\tau$}
\DP
\end{center}

\flyingbox{${\myvdash}\theta\sim (\Theta;\Delta)$}
\begin{center}
\AXC{
	\stackanchor
	{$\vdash_{\Theta_{0}}^{\emptyset}\theta \sim_{names}(\Theta_{names};\Delta_{names})$}
	{$\vdash_{\Theta_{0}\Theta_{names}}^{\Delta_{names}}\theta \sim_{defs}(\Theta_{defs};\Delta_{defs})$ ~~~~ $\vdash^{\Delta_{defs}}_{\Theta_{0}\Theta_{defs}}\theta \sim_{exts}(\Theta;\Delta) $}
} \RightLabel{(rec-decls)}
\UIC{$\vdash_{\Theta_{0}}^{\Delta} \theta \sim (\Theta;\Delta)$} 
\DP
\end{center}

\flyingbox{$\vdash_{\Theta}^{\Delta}\theta\sim_{names} (\Theta;\Delta)$}
\begin{center}
\AXC{} \RightLabel{(empty-names)}
\UIC{${\vdash_{\Theta}^{\Delta}} \emptyset \sim_{names} (\emptyset;\emptyset)$}
\DP
\end{center}
\begin{center}
\AXC{${\vdash_{\Theta}^{\Delta}}\theta' \sim_{names} (\Theta';\Delta') ~~~~~~ T\notin {dom}(\Theta) ~~~~~~ T\notin dom(\Theta')$} \RightLabel{(type-names)}
\UIC{${\vdash_{\Theta}^{\Delta}} \theta';T[\mathbf{explicit}(\tau,e_m)] \sim_{names} (\Theta',T[?,?];\Delta')$}
\DP
\end{center}

\begin{center}
\AXC{${\vdash_{\Theta}^{\Delta}}\theta' \sim_{names} (\Theta';\Delta') ~~~~~~ T\notin {dom}(\Theta) ~~~~~~ T\notin dom(\Theta')$} \RightLabel{(type-names-2)}
\UIC{${\vdash_{\Theta}^{\Delta}} \theta';T[\mathbf{tsmap}(k,body,e_m)] \sim_{names} (\Theta',T[?,?];\Delta')$}
\DP
\end{center}

\begin{center}
\AXC{${\vdash_{\Theta}^{\Delta}}\theta' \sim_{names} (\Theta';\Delta') ~~~~~~ \nexists~k.(k[\mathbf{ty}(\_,\_)]\in \Delta\Delta')$} \RightLabel{(typetsm-names)}
\UIC{${\vdash_{\Theta}} \theta';k[\mathbf{tytsm}(e,\tau)] \sim_{names} (\Theta';\Delta',k[\mathbf{ty}(?,?)])$}
\DP
\end{center}

\begin{center}
\AXC{${\vdash_{\Theta}^{\Delta}}\theta' \sim_{names} (\Theta';\Delta') ~~~~~~ \nexists~k.(k[\mathbf{syn}(\_,\_)]\in \Delta\Delta' \lor k[\mathbf{ana}(\_)]\in \Delta\Delta')$} \RightLabel{(syntsm-names)}
\UIC{${\vdash_{\Theta}} \theta';k[\mathbf{syntsm}(\tau,e)] \sim_{names} (\Theta';\Delta',k[\mathbf{syn}(?,?)])$}
\DP
\end{center}

\begin{center}
\AXC{${\vdash_{\Theta}^{\Delta}}\theta' \sim_{names} (\Theta';\Delta') ~~~~~~ \nexists~k.(k[\mathbf{syn}(\_,\_)]\in \Delta\Delta' \lor k[\mathbf{ana}(\_)]\in \Delta\Delta')$} \RightLabel{(anatsm-names)}
\UIC{${\vdash_{\Theta}} \theta';k[\mathbf{anatsm}(e)] \sim_{names} (\Theta';\Delta',k[\mathbf{ana}(?)])$}
\DP
\end{center}


\flyingbox{${\myvdash}\theta\sim_{defs} (\Theta;\Delta)$}
\begin{center}
\AXC{} \RightLabel{~~~ (empty-defs)}
\UIC{${\myvdash} \emptyset \sim_{defs} (\emptyset;\emptyset)$}
\DP
\end{center}

\begin{center}
\AXC{${\vdash_{\Theta}}\theta' \sim_{defs} (\Theta';\Delta')$ ~~~~ ${\vdash_{\Theta\Theta'}}\tau$}\RightLabel{(type-defs)}
\UIC{${\myvdash}\theta';T[\mathbf{explicit}(\tau,e_m)] \sim_{defs} (\Theta',T[\tau,?];\Delta)$}
\DP
\end{center}

\begin{center}
\AXC{
	\stackanchor
	{\stackanchor
		{\stackanchor
			{$\Theta_0\subset\Theta$ ~~~~ $\myvdash\theta'\sim_{defs}(\Theta';\Delta')$ ~~~~ $k[\mathbf{ty}(i,\tau_m)]\in\Delta'$}
			{$\texttt{parsestream}(body)=i_{ps}$}
		}
		{$\mathbf{iap}(\mathbf{iprj}[parse](\mathbf{iprj}[parser](i_{k}));i_{ps})\Downarrow(i_{type},i_m)$}
	}
	{$\vdash_{\Theta\Theta'}i_{type}\uparrow \tau$ ~~~~ $\vdash_{\Theta\Theta'} \tau$ ~~~~ $\vdash_{\Theta\Theta'} i_m \Leftarrow \tau_m$}
}\RightLabel{(type-defs-2)}
\UIC{${\myvdash} \theta';T[\mathbf{tsmap}(k,body,e_m)] \sim_{defs} \Theta',T[\tau,i_m:\tau_m]$}
\DP
\end{center}

\begin{center}
\AXC{$\myvdash\theta'\sim_{defs}(\Theta';\Delta')$ ~~~~ $\vdash_{\Theta\Theta'}\tau$ ~~~~ $\vdash_{\Theta\Theta'}^{\Delta\Delta'}e\rightsquigarrow i\Leftarrow\mathbf{named}[TypeParser]$} \RightLabel{(tytsm-defs)}
\UIC{$\myvdash\theta';k[\mathbf{tytsm}(e,\tau)]\sim_{defs}(\Theta';\Delta',k[\mathbf{ty}(i,\tau)])$}
\DP
\end{center}

\begin{center}
\AXC{$\myvdash\theta'\sim_{defs}(\Theta';\Delta')$ ~~~~ $\vdash_{\Theta\Theta'}\tau$} \RightLabel{(syntsm-defs)}
\UIC{$\myvdash\theta';k[\mathbf{syn}(\tau,e)]\sim_{defs}(\Theta';\Delta',k[\mathbf{syn}(\tau,?)])$}
\DP
\end{center}

\begin{center}
\AXC{$\myvdash\theta'\sim_{defs}(\Theta';\Delta')$} \RightLabel{(anatsm-defs)}
\UIC{$\myvdash\theta';k[\mathbf{ana}(e)]\sim_{defs}(\Theta';\Delta',k[\mathbf{ana}(?)])$}
\DP
\end{center}


\flyingbox{${\myvdash}\theta\sim_{exts} (\Theta;\Delta)$}
\begin{center}
\AXC{} \RightLabel{(empty-exts)}
\UIC{${\myvdash} \emptyset \sim_{exts} (\emptyset;\emptyset)$}
\DP
\end{center}

\begin{center}
\AXC{$\vdash^{\Delta}_{\Theta}\theta'\sim_{exts}(\Theta';\Delta')$ ~~~~ $T[\tau,?]\in\Theta$ ~~~~ $\emptyset\vdash^{\Delta\Delta'}_{\Theta\Theta'}e_m\rightsquigarrow i_m\Rightarrow \tau_m$}			\RightLabel{(type-exts)}
\UIC{$\vdash^{\Delta}_{\Theta}\theta';T[\mathbf{explicit}(\tau,e_m)] \sim_{exts} (\Theta',T[\tau,i_m:\tau_m];\Delta')$}
\DP
\end{center}

\begin{center}
\AXC{
	\stackanchor
	{$\vdash^{\Delta}_{\Theta}\theta'\sim_{exts}(\Theta';\Delta')$ ~~~~ $T[\tau,i_m:\tau_m]\in\Theta$}
	{$\vdash_{\Theta\Theta'}^{\Delta\Delta'} e_m\rightsquigarrow i'_m\Leftarrow\mathbf{arrow}(\tau_m,\tau''_m)$ ~~~~ $\mathbf{iap}(i'_m; i_m) \Downarrow i''_m$}
}\RightLabel{(type-exts-2)}
\UIC{$\vdash_{\Theta}\theta';T[\mathbf{tsmap}(k,body,e_m)] \sim_{exts} (\Theta',T[\tau,i'_m:\tau'_m];\Delta')$}
\DP
\end{center}

\begin{center}
\AXC{$\vdash_{\Theta}^{\Delta}\theta'\sim_{exts}(\Theta';\Delta')$ ~~~~ $k[\mathbf{syn}(\tau,?)]\in \Delta$ ~~~~ $\emptyset\vdash_{\Theta\Theta'}^{\Delta\Delta'}e\rightsquigarrow i\Leftarrow \mathbf{named}[TypeParser]$} \RightLabel{(tytsm-exts)}
\UIC{$\vdash_{\Theta}^{\Delta} \theta';k[\mathbf{ana}(e)]\sim_{exts}(\Theta;\Delta',k[\mathbf{syn}(\tau,i)])$}
\DP
\end{center}

\begin{center}
\AXC{$\vdash_{\Theta}^{\Delta}\theta'\sim_{exts}(\Theta';\Delta')$ ~~~~ $\emptyset\vdash_{\Theta\Theta'}^{\Delta\Delta'}e\rightsquigarrow i\Leftarrow \mathbf{named}[TypeParser]$}	\RightLabel{(syntsm-exts)}
\UIC{$\vdash_{\Theta}^{\Delta} \theta';k[\mathbf{ana}(e)]\sim_{exts}(\Theta;\Delta',k[\mathbf{ana}(i)])$}
\DP
\end{center}

\begin{center}
\AXC{$\vdash_{\Theta}^{\Delta}\theta'\sim_{exts}(\Theta';\Delta')$ ~~~~ $k[\mathbf{ty}(i,\tau)]\in\Delta$}	\RightLabel{(anatsm-exts)}
\UIC{$\vdash_{\Theta}^{\Delta} \theta';k[\mathbf{tytsm}(e,\tau)]\sim_{exts}(\Theta;\Delta',k[\mathbf{ty}(i,\tau)])$}
\DP
\end{center}

\doublebox{$\vdash_{\Theta} i\uparrow \tau$}{$\vdash_{\Theta}\tau\downarrow i$}
\begin{center}
\AXC{$\vdash_{\Theta} i\uparrow T$}
\UIC{$\vdash_{\Theta} \mathbf{iinj}[Named](i)\uparrow \mathbf{named}[T]$}
%%% the next rule
\AXC{$T\downarrow i_{id}$}
\UIC{$\mathbf{named}[T]\downarrow \mathbf{iinj}[Named](i_{id})$}
\noLine
\BIC{}
\DP
\end{center}

\begin{center}
\AXC{$\vdash_{\Theta} i\uparrow \omega$}
\UIC{$\vdash_{\Theta} \mathbf{iinj}[Objtype](i)\uparrow \mathbf{objtype}[\omega]$}
%%% the next rule
\AXC{$\omega\downarrow i$}
\UIC{$\mathbf{objtype}[\omega]\downarrow\mathbf{iinj}[Objtype](i)$}
\noLine
\BIC{}
\DP
\end{center}

\begin{center}
\AXC{$\vdash_{\Theta} i\uparrow \chi$}
\UIC{$\vdash_{\Theta} \mathbf{iinj}[Casetype](i)\uparrow \mathbf{casetype}[\chi]$}
%%% the next rule
\AXC{$\chi\downarrow i$}
\UIC{$\mathbf{casetype}[\omega]\downarrow\mathbf{iinj}[Casetype](i)$}
\noLine
\BIC{}
\DP
\end{center}

\begin{center}
\AXC{$\vdash_{\Theta} i_1\uparrow \tau_1$ ~~~~ $\vdash_{\Theta} i_2\uparrow \tau_2$}
\UIC{$\vdash_{\Theta} \mathbf{iinj}[Arrow]((i_1, i_2))\uparrow \mathbf{arrow}(\tau_1,\tau_2)$}
%%%
\AXC{$\tau_1\downarrow i_1$ ~~~~ $\tau_2\downarrow i_2$}
\UIC{$\mathbf{arrow}(\tau_1,\tau_2) \downarrow\mathbf{iinj}[Arrow]((i_1, i_2))$}
\noLine
\BIC{}
\DP
\end{center}

\begin{center}
\AXC{$\vdash_{\Theta} i\Leftarrow MemberList$}
\UIC{$\vdash_{\Theta}\mathbf{iinj}[Nil](i)\uparrow \emptyset$}
%%%
\AXC{}
\UIC{$\emptyset \downarrow\mathbf{iinj}[Nil](i)$}
\noLine
\BIC{}
\DP
\end{center}

\begin{center}
\AXC{
	\stackanchor
	{$\vdash_{\Theta} i\Leftarrow MemberList$ ~~~~ $i_l\uparrow l$}
	{$\vdash_{\Theta} i_t\uparrow \tau$ ~~~~ $\vdash_{\Theta} i_c\uparrow \omega$}
}
\UIC{$\vdash_{\Theta}\mathbf{iinj}[Cons]((i_l, i_t, i_c))\uparrow l[\tau],\omega$}
%%%
\AXC{$l\downarrow i_l$ ~~~~ $\tau\downarrow i_t$ ~~~~ $\omega\downarrow i_c$}
\UIC{$l[\tau],\omega \downarrow \mathbf{iinj}[Cons]((i_l,i_t,i_c))$}
\noLine
\BIC{}
\DP
\end{center}

\begin{center}
\AXC{$\vdash_{\Theta} i\Leftarrow CaseList$}
\UIC{$\vdash_{\Theta}\mathbf{iinj}[Nil](i)\uparrow \emptyset$}
%%%
\AXC{}
\UIC{$\emptyset \downarrow\mathbf{iinj}[Nil](i)$}
\noLine
\BIC{}
\DP
\end{center}

\begin{center}
\AXC{
	\stackanchor
	{$\vdash_{\Theta} i\Leftarrow CaseList$ ~~~~ $i_l\uparrow C$}
 	{$\vdash_{\Theta} i_t\uparrow \tau$ ~~~~ $\vdash_{\Theta} i_c\uparrow \chi$}
 }
\UIC{$\vdash_{\Theta}\mathbf{iinj}[Cons]((i_l, i_t, i_c))\uparrow C[\tau],\chi$}
%%%
\AXC{$C\downarrow i_l$ ~~~~ $\tau\downarrow i_t$ ~~~~ $\chi\downarrow i_c$}
\UIC{$C[\tau],\chi \downarrow \mathbf{iinj}[Cons]((i_l,i_t,i_c))$}
\noLine
\BIC{}
\DP
\end{center}

\doublebox{$\vdash_{\Theta}i\uparrow\hat{e}$}{$i\downarrow i$}
\begin{center}
\AXC{$i_{id}\uparrow x$}
\UIC{$\vdash_{\Theta}\mathbf{iinj}[Var](i_{id})\uparrow x$}
%%%
\AXC{$x\downarrow i_{id}$}
\UIC{$x\downarrow \mathbf{iinj}[Var](i_{id})$}
\noLine
\BIC{}
\DP
\end{center}

\begin{center}
\AXC{$\vdash_{\Theta} i_{1}\uparrow \tau$ ~~~~ $\vdash_{\Theta} i_2\uparrow\hat{e}$}
\UIC{$\vdash_{\Theta}\mathbf{iinj}[Asc]((i_1,i_2))\uparrow \mathbf{hasc}[\tau](\hat{e})$}
%%%%
\AXC{$\tau\downarrow i_1$ ~~~~ $i\downarrow i_2$}
\UIC{$\mathbf{iasc}[\tau](i)\downarrow \mathbf{iinj}{Asc}((i_1,i_2))$}
\noLine
\BIC{}
\DP
\end{center}

\begin{center}
\AXC{$i_{id}\uparrow x$ ~~~~ $\vdash_{\Theta}i\uparrow \hat{e}$}
\UIC{$\vdash_{\Theta}\mathbf{iinj}[Lam]((i_{id},i))\uparrow \mathbf{hlam}(x.\hat{e})$}
%%%
\AXC{$x\downarrow i_{id}$ ~~~~ $i\downarrow i'$}
\UIC{$\mathbf{ilam}(x.i)\downarrow \mathbf{iinj}[Lam]((i_{id},i'))$}
\noLine
\BIC{}
\DP
\end{center}

\begin{center}
\AXC{$\vdash_{\Theta}i_1\uparrow\hat{e}_1$ ~~~~ $\vdash_{\Theta}i_2\uparrow\hat{e}_2$}
\UIC{$\vdash_{\Theta}\mathbf{iinj}[Ap]((i_1,i_2))\uparrow \mathbf{hap}(\hat{e}_1,\hat{e}_2)$}
%%%
\AXC{$i_1\downarrow i'_1$ ~~~~ $i_2\downarrow i'_2$}
\UIC{$\mathbf{iap}(i_1;i_2)\downarrow \mathbf{iinj}[Ap]((i'_1,i'_2))$}
\noLine
\BIC{}
\DP
\end{center}

\begin{center}
... ...
\end{center}

\begin{center}
\AXC{$\mathsf{body}(i_{ps})=body$ ~~~~ $\mathsf{eparse}(body)=e$}
\UIC{$\vdash_{\Theta}\mathbf{iinj}[Spliced](i_{ps})\uparrow \mathbf{spliced}[e]$}
\DP
\end{center}


\subsection{Context Formation}

\flyingbox{$\vdash_{\Theta} \Delta$}
\begin{center}
\AXC{}
\UIC{$\vdash_{\Theta} \emptyset$}
%%% the next rule
\AXC{$\Theta_{0}\subset\Theta$ ~~~~ $\vdash_{\Theta} \Delta$ ~~~~ $\nexists~k.(k[\mathbf{ty}(\_,\_)]\in\Delta)$ ~~~~ $\vdash_{\Theta}\delta$ ~~~~ $\vdash_{\Theta}\gamma:\mathbf{named}[TypeParser]$}
\UIC{$\vdash_{\Theta}\Delta;k[\mathbf{ty}(\gamma,\delta)]$}
\noLine
\BIC{}
\DP
\end{center}

\begin{center}
\AXC{$\Theta_{0}\subset\Theta$ ~~~~ $\vdash_{\Theta} \Delta$ ~~~~ $\nexists~k.(k[\mathbf{syn}(\_,\_)]\in \Delta\lor k[\mathbf{ana}(\_)]\in \Delta)$ ~~~~ $\vdash_{\Theta}\gamma:\mathbf{named}[ExpParser]$}
\UIC{$\vdash_{\Theta}\Delta;k[\mathbf{ana}(\gamma)]$}
\DP
\end{center}

\begin{center}
\AXC{
	{$\Theta_{0}\subset\Theta$ ~~~~ $\vdash_{\Theta} \Delta$ ~~~~ $\nexists~k.(k[\mathbf{syn}(\_,\_)]\in \Delta\lor k[\mathbf{ana}(\_)]\in \Delta)$} ~~~~
	{$\vdash_{\Theta}\delta$~~~~ $\vdash_{\Theta}\gamma:\mathbf{named}[ExpParser]$}
}
\UIC{$\vdash_{\Theta}\Delta;k[\mathbf{bk}(\delta), \gamma]$}
\DP
\end{center}

\flyingbox{$\vdash \Theta$}
\begin{center}
\AXC{}
\UIC{$\vdash \emptyset$}
%%% the next rule
\AXC{$\vdash \Theta$ ~~~~ $T\notin dom(\Theta)$ ~~~~ $\vdash_{\Theta,T[?,?]}\delta~ok$ ~~~~ $\vdash_{\Theta,T[\delta,?]}\mu~ok$}
\UIC{$\vdash\Theta,T[\delta,\mu]$}
\noLine
\BIC{}
\DP
\end{center}

\flyingbox{$\vdash_{\Theta}{\tau}$}
\begin{center}
\AXC{$\vdash_{\Theta} \omega$}
\UIC{$\vdash_{\Theta} \mathbf{objtype}[\omega]$}
%%% the next rule
\AXC{$\vdash_{\Theta} \chi$}
\UIC{$\vdash_{\Theta} \mathbf{casetype}[\chi]$}
%%% the next rule
\AXC{$\vdash_{\Theta} \tau_1$ ~~~~ $\vdash_{\Theta}\tau_2$}
\UIC{$\vdash_{\Theta} \mathbf{arrow}[\tau_1,\tau_2]$}
%%% the next rule
\AXC{$T[\tau,\mu]\in\Theta$}
\UIC{$\vdash_{\Theta} \mathbf{named}[T]$}
\noLine
\QIC{}
\DP
\end{center}

\flyingbox{$\vdash_{\Theta} \omega$}
\begin{center}
\AXC{}
\UIC{$\vdash_{\Theta} \emptyset$}
%%% the next rule
\AXC{$l\notin dom(\omega)$ ~~~~ $\vdash_{\Theta}\tau$ ~~~~ $\vdash_{\Theta}\omega$}
\UIC{$\vdash_{\Theta} l[\mathbf{def},\tau];\omega$}
%%%
\AXC{$l\notin dom(\omega)$ ~~~~ $\vdash_{\Theta}\tau$ ~~~~ $\vdash_{\Theta}\omega$}
\UIC{$\vdash_{\Theta} l[\mathbf{val},\tau];\omega$}
\noLine
\TIC{}
\DP
\end{center}

\flyingbox{$\vdash_{\Theta} \chi$}
\begin{center}
\AXC{}
\UIC{$\vdash_{\Theta} \emptyset$}
%%% the next rule
\AXC{$C\notin dom(\chi)$ ~~~~ $\vdash_{\Theta}\tau$ ~~~~ $\vdash_{\Theta}\chi$}
\UIC{$\vdash_{\Theta} C[\tau];\chi$}
\noLine
\BIC{}
\DP
\end{center}

\flyingbox{$\vdash_{\Theta} \delta$}
\begin{center}
\AXC{}
\UIC{$\vdash_{\Theta} ?$}
%%% the next rule
\AXC{$\vdash_{\Theta} \tau$}
\UIC{$\vdash_{\Theta} \tau$}
\noLine
\BIC{}
\DP
\end{center}

\flyingbox{$\vdash_{\Theta} \mu$}
\begin{center}
\AXC{}
\UIC{$\vdash_{\Theta} ?$}
%%% the next rule
\AXC{$\vdash_{\Theta}\tau$ ~~~~ $\emptyset\vdash_{\Theta} i\Leftarrow \tau$}
\UIC{$\vdash_{\Theta} i:\tau$}
\noLine
\BIC{}
\DP
\end{center}


\flyingbox{$\vdash_{\Theta} \gamma:\tau$}
\begin{center}
\AXC{$\vdash_{\Theta}\tau$}
\UIC{$\vdash_{\Theta} ?:\tau$}
%%% the next rule
\AXC{$\vdash_{\Theta}\tau$ ~~~~ $\emptyset\vdash_{\Theta} i\Leftarrow \tau$}
\UIC{$\vdash_{\Theta} i:\tau$}
\noLine
\BIC{}
\DP
\end{center}

\flyingbox{$\vdash_{\Theta} \Gamma$}
\begin{center}
\AXC{}
\UIC{$\vdash_{\Theta} \emptyset$}
%%% the next rule
\AXC{$\vdash_{\Theta} \Gamma$ ~~~~ $\emptyset\vdash_{\Theta} \tau$}
\UIC{$\vdash_{\Theta} \Gamma,x:\tau$}
\noLine
\BIC{}
\DP
\end{center}

\subsection{Statics for externel terms}

\begin{center}
\AXC{
	\stackanchor{
		\stackanchor
		{$\Theta_0\subset\Theta$ ~~~~ $T[\delta,i_m:\mathbf{named}[HasTSL]]\in\Theta$ ~~~~ $\mathsf{parsestream}(body)=i_{ps}$}
		{$\mathbf{iap}(\mathbf{iprj}[parse](\mathbf{iprj}[parser](i_m));i_{ps})\Downarrow \mathbf{iinj}[OK]((i_{ast},i'_{ps}))$}
	}
	{$i_{ast}\uparrow \hat{e}$ ~~~~ $\Gamma;\emptyset\vdash_{\Theta}\hat{e}\rightsquigarrow i\Leftarrow \mathbf{named}[T]$}
}	\RightLabel{(T-lit)}
\UIC{$\Gamma\vdash_{\Theta}\mathbf{lit}[body]\rightsquigarrow i\Leftarrow \mathbf{named}[T]$}
\DP
\end{center}


\begin{center}
\AXC{$\Theta_0\subset\Theta$ ~~~~ $k[\mathbf{syn}(\tau,i)] \in \Delta$}\RightLabel{(T-syntsmdef)}
\UIC{$\Gamma\myvdash \mathbf{esyntsmdef}[k] \rightsquigarrow i\Rightarrow \mathbf{named}[ExpParser]$}
\DP
\end{center}

\begin{center}
\AXC{$\Theta_0\subset\Theta$ ~~~~ $k[\mathbf{ana}(i)] \in \Delta$}\RightLabel{(T-anatsmdef)}
\UIC{$\Gamma\myvdash \mathbf{eanatsmdef}[k] \rightsquigarrow i\Rightarrow \mathbf{named}[ExpParser]$}
\DP
\end{center}

\begin{center}
\AXC{$\Theta_0\subset\Theta$ ~~~~ $k[\mathbf{ty}(i,\tau)]\in\Delta$} \RightLabel{(T-typetsmdef)}
\UIC{$\Gamma\myvdash \mathbf{etypetsmdef}[k]\rightsquigarrow i \Rightarrow \mathbf{named}[TypeParser]$}
\DP
\end{center}

\begin{center}
\AXC{
	\stackanchor
	{$\Theta_0 \subset \Theta$ ~~~~ $k[\mathbf{ana}(i_k)]\in\Theta$ ~~~~ $\mathsf{parsestream}(body)=i_{ps}$}
	{$\mathbf{iap}(\mathbf{iprj}[parse](i_k); i_{ps});\Downarrow \mathbf{iinj}[OK]((i_{ast}, i'_{ps}))$ ~~~~ $i_{ast}\uparrow \hat{e}$ ~~~~ $\Gamma;\emptyset\myvdash \hat{e} \rightsquigarrow i \Leftarrow \tau$}
}\RightLabel{(T-ana)}
\UIC{$\Gamma\myvdash\mathbf{etsmap}[k,body] \rightsquigarrow i \Leftarrow \tau$}  
\DP
\end{center}

\begin{center}
\AXC{
	\stackanchor
	{$\Theta_0 \subset \Theta$ ~~~~ $k[\mathbf{syn}(\tau,i_k)]\in\Theta$ ~~~~ $\mathsf{parsestream}(body)=i_{ps}$}
	{$\mathbf{iap}(\mathbf{iprj}[parse](i_k); i_{ps})\Downarrow \mathbf{iinj}[OK]((i_{ast}, i'_{ps}))$ ~~~~ $i_{ast}\uparrow \hat{e}$ ~~~~ $\Gamma;\emptyset\myvdash \hat{e} \rightsquigarrow i \Leftarrow \tau$}
} \RightLabel{(T-syn)}
\UIC{$\Gamma\myvdash\mathbf{etsmap}[k,body] \rightsquigarrow i\Rightarrow \tau$}  
\DP
\end{center}

\section{Metatheory}
\begin{theorem}[Internal Type Safety]
If $\vdash\Theta$, and $\emptyset\vdash_{\Theta}i\Leftarrow\tau$ or $\vdash_{\Theta}i\Rightarrow\tau$, then either $i~\texttt{val}$ or $i\mapsto i'$ such that $\emptyset\vdash_{\Theta}i'\Leftarrow\tau$.
\end{theorem}
\begin{proof}
As the keyword extension on TSL framework does not extend internal Wyvern expressions ($i$), and this is the same as the proof in TSL. 
\end{proof}

\begin{theorem}[External Type Preservation]
If $\vdash\Theta$, $\vdash_{\Theta_0\Theta}\Delta$ $\vdash_{\Theta_0\Theta}\Gamma$, and $\Gamma\vdash_{\Theta_0\Theta}^{\Delta} e\rightsquigarrow i\Leftarrow\tau$ or $\Gamma\vdash_{\Theta_0\Theta}^{\Delta} e\rightsquigarrow i\Rightarrow\tau$ then $\Gamma\vdash_{\Theta_0\Theta} i\Leftarrow\tau$.
\end{theorem}
\begin{proof}
Base on the proof of core Wyvern external terms in TSL, we only need to present the proofs for the new cases extended in external Wyvern:
\begin{itemize}
\item $\mathbf{etsmap}[k,body]$, in this case, according to the rule T-syn, there exists a translational term $\hat{e}$, s.t. $\Gamma;\emptyset\vdash_{\Theta}^{\Delta}\hat{e}\rightsquigarrow i\Leftarrow \tau$. For the typing context, we have $\vdash\emptyset$ (which is $\Gamma_{out}$). And by the conditions in the theorem $\vdash_{\Theta}\Gamma$, $\vdash\Theta$ and $\vdash_{\Theta_0\Theta}\Delta$, by Lemma~1, we have $\emptyset;\Gamma\vdash_{\Theta_0\Theta}i\Leftarrow\tau$. 
\item $\mathbf{esyntsmdef}[k]$, by induction, we have the formation of the internal term $i$ in $k[\mathbf{syn}(\tau,i)]$, i.e. $\vdash_{\Theta_0\Theta}i\Leftarrow \mathbf{named}[ExpParser]$ as $\Theta_0\Theta$ is well formed. Then the case is proved.
\item $\mathbf{eanatsmdef}[k]$, similar to the case $\mathbf{eanatsmdef}[k]$.
\item $\mathbf{etypetsmdef}[k]$, similar to the case $\mathbf{eanatsmdef}[k]$.
\end{itemize}
With these cases proved, we have the property holds for all external terms.
\end{proof}


\begin{lemma}[Translational Type Preservation]
If $\vdash\Theta$, $\vdash_{\Theta_0\Theta} \Delta$, $\vdash_{\Theta_0\Theta}\Gamma_{out}$, $\vdash_{\Theta_0\Theta}\Gamma$, $dom(\Gamma_{out})\cap dom(\Gamma)=\emptyset$ and $\Gamma_{out};\Gamma\vdash_{\Theta\Theta_0}^{\Delta}\hat{e}\rightsquigarrow i\Leftarrow\tau$ or $\Gamma_{out};\Gamma\vdash_{\Theta\Theta_0}^{\Delta}\hat{e}\rightsquigarrow i\Rightarrow \tau$ then $\Gamma_{out};\Gamma\vdash_{\Theta_0\Theta}i\Leftarrow \tau$.
\end{lemma}
\begin{proof}
The proof by induction over the typing derivation follows the same outline as
above for all the shared cases. The outer context is threaded through opaquely when
applying the inductive hypothesis. The only rules of note are the two for the spliced external
terms, which require applying the external type preservation theorem recursively.
This is well-founded by a metric measuring the size of the spliced external term, written
in concrete syntax, since we know it was derived from a portion of the literal body.
\end{proof}

\begin{theorem}[Compilation]
If ~$\rho\sim(\Theta;\Delta)\rightsquigarrow i:\tau$ then $\vdash\Theta$, $\vdash_{\Theta_0\Theta}\Delta$ and $\emptyset\vdash_{\Theta_0\Theta} i\Leftarrow\tau$.
\end{theorem}
\begin{proof}
The proof contains two parts: the formation of the contexts ($\Theta$, $\Delta$) and the formation of the term $i$.

The proof for the contexts formation requires the following lemma. And the proof for the term $i$ is obvious: According to the rule `compile', we have that $\emptyset\vdash_{\Theta_0\Theta}^{\Delta}e\rightsquigarrow i\Rightarrow \tau$, and for the context: $\vdash\emptyset$ by the checking rule for $\Gamma$, $\vdash_{\Theta_0\Theta}\Delta$, $\vdash\Theta_0\Theta$ by the proof of the first part. Then by Theorem~1, we have $\vdash_{\Theta_0\Theta}i\Leftarrow\tau$, which completes the proof.
\end{proof}

\begin{lemma} If $\vdash_{\Theta}^{\Delta}\theta\sim(\Theta';\Delta')$, then we have $\vdash\Theta\Theta'$ and $\vdash_{\Theta\Theta'}\Delta\Delta'$.
\end{lemma}
\begin{proof}
The proof of the lemma can be further separated into three cases for different phases, i.e. the elaboration for names, defs and exts. With the following lemma proved, the lemma is proved.
\end{proof}

\begin{lemma}
If $\vdash_{\Theta}^{\Delta}\theta\sim_{names}(\Theta';\Delta')$, then we have $\vdash\Theta\Theta'$ and $\vdash_{\Theta\Theta'}\Delta\Delta'$.
\end{lemma}
\begin{proof}
By induction on the elaboration steps of $\theta$ for all of the three phases (i.e. names, defs, exts). We only need to discuss the last step of the elaboration rule, which contains the following cases:
\begin{itemize}
\item $\vdash_{\Theta}^{\Delta}\theta_1,T[\mathbf{explicit}(\tau,e_m)] \sim (\Theta_1,T[?,?];\Delta_1)$. For the names phase, we will check that $T\notin dom(\Theta\Theta')$. By the formation rule for $\Theta$, we have $\vdash ?$ (for $\delta$), and $\vdash ?$(for $\mu$). And by induction, we have $\vdash\Theta\Theta_1$. Then the case is well formed for $\vdash\Theta\Theta_1,T[?,?]$. As this case does not change the TSM contexts, $\vdash_{\Theta\Theta_1}\Delta_1$ remains true.
\item$\vdash_{\Theta}^{\Delta}\theta_1,T[\mathbf{tsmap}(k,body)] \sim (\Theta_1,T[?,?];\Delta_1)$. Similar to the proof of the first case.
\item $\vdash_{\Theta}^{\Delta}\theta_1,k[\mathbf{tytsm}(e,\tau)] \sim (\Theta_1;\Delta_1,k[\mathbf{ty}(?,?)])$. As the named type context is not modified, by induction, we have $\vdash\Theta\Theta_1$. And by the rule typetsm-names, it checks that there is no type TSM name conflicts, which satisfies the formation rule for $\Delta$. And thus we have $\vdash_{\Theta\Theta_1}\Delta\Delta_1,k[\mathbf{ty}(?,?)]$, and the case is proved.
\item $\vdash_{\Theta}^{\Delta}\theta_1,k[\mathbf{syntsm}(\tau,e)] \sim (\Theta_1;\Delta_1,k[\mathbf{syn}(?,?)])$.
\item $\vdash_{\Theta}^{\Delta}\theta_1,k[\mathbf{tytsm}(e)] \sim (\Theta_1;\Delta_1,k[\mathbf{ana}(?)])$. These two cases are similar to that of the type-TSM case.
\end{itemize}
With these cases proved, we have the lemma proved.
\end{proof}

\begin{lemma}
If $\vdash_{\Theta}^{\Delta}\theta\sim_{defs}(\Theta';\Delta')$, then we have $\vdash\Theta\Theta'$ and $\vdash_{\Theta\Theta'}\Delta\Delta'$.
\end{lemma}
\begin{proof}
The proof of the lemma is similar to that of the last case. By induction on the elaboration steps, again, we have five cases to prove.
\begin{itemize}
	\item $\vdash_{\Theta}^{\Delta}\theta_1,T[\mathbf{explicit}(\tau,e_m)] \sim_{defs} (\Theta_1,T[\tau,?];\Delta_1)$, in this case we check the formation of the type under the context $\Theta\Theta_1$ by the elaboration rule type-defs, which modifies the formation of $\vdash{\Theta\Theta_1}T[?,?]$ to $\vdash{\Theta\Theta_1}T[\tau,?]$. Thus it is also well formed.
	\item  $\vdash_{\Theta}^{\Delta}\theta_1,T[\mathbf{tsmap}(k,body)] \sim_{defs} (\Theta_1,T[\tau,i_m:\tau_m];\Delta_1)$. Besides the checking rule for type in the last case, the case is also checked with the elaborated internal term $i_m$ against its type $\tau_m$. And in the rule type-defs-2, an judgment $\vdash_{\Theta\Theta'}i_m\Leftarrow\tau_m$ is checked, thus the term is well formed in the context. And by the checking rule of the context and by induction, the named type context is well formed.
	\item  $\vdash_{\Theta}^{\Delta}\theta_1,k[\mathbf{tytsm}(e,\tau)] \sim_{defs} (\Theta_1;\Delta_1,k[\mathbf{ty}(i,\tau)])$. In the case, the rule tytsm-defs check the formation of the type $\tau$ under the context $\Theta\Theta_1$, and the term $e$ elaborates to an internal term $i$ of type $\mathbf{named}[TypeParser]$, by Theorem~2, we have the formation of the term $i$ under the context $\Theta\Theta_1$. Thus the environment is well formed.
	\item $\vdash_{\Theta}^{\Delta}\theta_1,k[\mathbf{syntsm}(\tau,e)] \sim (\Theta_1;\Delta_1,k[\mathbf{syn}(\tau,?)])$.
	\item $\vdash_{\Theta}^{\Delta}\theta_1,k[\mathbf{tytsm}(e)] \sim (\Theta_1;\Delta_1,k[\mathbf{ana}(?)])$. These two cases are similar to that of the type-TSM case as the type is checked.
\end{itemize}
With these cases proved, we have the lemma proved.
\end{proof}

\begin{lemma}
If $\vdash_{\Theta}^{\Delta}\theta\sim_{exts}(\Theta';\Delta')$, then we have $\vdash\Theta\Theta'$ and $\vdash_{\Theta\Theta'}\Delta\Delta'$.
\end{lemma}
\begin{proof}
The proof of this case is obvious following the proof of the previous two lemmas: in this lemma, the parser elaborates to an internal term $i$ and by Theorem~1, we can prove the formation of the term $i$ according to its elaboration process. For TSMs, the prove is similar.
\end{proof}

\end{document}
