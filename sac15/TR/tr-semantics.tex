%!TEX root=sac15-tr.tex
\section{Bidirectional Typechecking and Elaboration}
\label{tr-semantics}

\subsection{Contexts}
Before turning to the elaboration rules, we will first give the definition of contexts (Fig.~\ref{typechecking-environment}). After declarations elaboration, TSM declarations and named type declarations will go into the contexts, $\Psi$ and $\Theta$ respectively, for further reference in the program body elaboration phase. We use $s[\mathbf{ty}(\kappa,i)]$ to denote the mapping between the TSM name $s$ to its body, which is a type-level TSM of kind $\kappa$ and with a parser $i$ (in the internal form after elaboration).

Similarly, named type contexts maps named types $\mathtt{T}$ to their structure $\tau$ of kind $\kappa$ together with a TSL parser $i$. After elaboration, the difference between explicit declared types and types generated defined by TSM application no longer exists, and they will both go into the contexts with the form $\mathtt{T}[\tau :: \kappa,i]$. Besides, a special form $\T[-::\kappa,-]$ is used in the named type context to represent an intermediate state in elaboration: a type $\T$ only maps to its kind $\kappa$, with its structure and syntax parser to be elaborated later, which will be discussed more later in elaboration rules.

\begin{figure}[ht]
\[
\begin{array}{l}
      \text{TSM Contexts}\\
      \mytab\Psi ::= \emptyset ~|~ \Psi, s[\mathbf{ty}(\kappa, i)] ~|~ \Psi, s[\mathbf{syn}(\tau,i)] ~|~ \Psi, s[\mathbf{ana}(i)]\\
      \text{Named Type Contexts}\\
      \mytab\Theta ::= \emptyset~|~ \Theta, \T[-::\kappa,-] ~|~ \Theta, \T[\tau :: \kappa,i]\\
      \text{Typing Contexts}\\
      \mytab\Gamma ::= \emptyset ~|~ \Gamma, x : \tau\\
      \text{Kinding Contexts}\\
      \mytab\Delta ::= \emptyset ~|~ \Delta,t::\kappa
\end{array}
\]
\mycaption{Syntax for contexts.}\label{typechecking-environment}
\end{figure}


\subsection{Compilation}
The top-level judgment in the system is the compilation judgment:
$$\AXC{$d\sim(\Psi;\Theta)$ ~~~~ $\emptyset;\emptyset\vdash_{\Theta_0\Theta}^{\Psi}e\rightsquigarrow i\Rightarrow \tau$}      \RightLabel{(compile)}
\UIC{$ d;e\sim(\Psi;\Theta)\rightsquigarrow i:\tau$}
\DP$$

The compilation judgment will compile an external Wyvern program to internal Wyvern program by 1) parsing declarations, $d$, and 2) transforming the external term $e$, representing the Wyvern program body, into an internal term $i$.

The judgment $d \sim (\Psi, \Theta)$, defined in Figure \ref{typechecking-elaboration}, generates a corresponding TSM context, $\Psi$, and named type context, $\Theta$, from $d$. We refer to the type declarations in the prelude, some of which were shown in Fig. \ref{exp-prelude} below, by using the named type context $\Theta_0$ in our rules.

%!TEX root=sac15-tr.tex
\begin{figure}[t!]
\begin{lstlisting}[style=wyvern]
objtype Parser(T)
  def parse(ParseStream) : Result(T)
  syntax = (* ... parser generator syntax ... *)

casetype Result(T)
  OK of T
  Error of String * Location

casetype Exp
  Var of ID
  Ascription of Exp * Type
  Let of Exp * ID * Exp
  Fn of ID * Exp
  TypeFn of Type * Kind * Exp
  Ap of Exp * Exp
  New of List(ID * Exp)
  Objfield of ID * Exp
  CaseIntro of ID * Exp
  ToAST of Exp
  SyntaxParser of ID
  Spliced of ParseStream
  ApTSM of ID * ParserStream
  syntax = (* ... exp quasiquotes ... *)

casetype Type
  Declared of ID
  Objtype of List(MemberDecl)
  Casetype of List(CaseDecl)
  Arrow of Type * Type
  TyVar of ID 
  TyFn of ID * Type
  TyAp of Type * Type
  Spliced of ParseStream 
  syntax = (* ... type quasiquotes ... *)

objtype Pair(T, T)
  val first : T
  val second : T
\end{lstlisting}
\mycaption{A portion of the Wyvern prelude relevant to TSLs and TSMs. Wyvern prelude contains all built-in types used for the core Wyvern language.}
\label{exp-prelude}
\end{figure}


Judgments $\Delta; \Gamma \vdash_\Theta^\Psi e \leadsto i \Rightarrow \tau$ and  $\Delta; \Gamma \vdash_\Theta^\Psi e \leadsto i \Leftarrow \tau$ can be read ``under kinding context $\Delta$, typing context $\Gamma$, named type context $\Theta$ and TSM context $\Psi$, external term $e$ elaborates to internal term $i$ and (synthesizes / analyzes against) type $\tau$''. This is a \emph{bidirectionally typed elaboration semantics}, used to elaborate TSL literals and TSM applications, which appear only in the external language, to an internal language, which includes only the core operations in the language. The semantics follows that given in \cite{TSLs}, so we do not repeat the rules for the core operations or TSL literals. 

In the next sections, we will introduce the elaboration rules for mutually recursive type declarations, TSM declarations and the the rules for the new form in the external terms, $\textbf{eaptsm}[s, body]$.

\subsection{Declarations}
We will now introduce the elaboration rules for declarations, where named type context $\Theta$ and TSM context $\Psi$ will be generated from declarations $d$ through the judgment $d \sim (\Psi, \Theta)$. 

\subsubsection{Term-level TSMs}
The rules D-syntsm and D-anatsm presents how term-level TSMs are elaborated. To elaborate a synthetic TSM $\mathbf{syntsm}(s,\tau,e_{tsm})$ to its corresponding context form $s[\mathbf{syn}(\tau,i_{tsm})]$, we need the following premises:
\begin{itemize}\itemsep0pt
\item $d\sim(\Theta,\Psi)$ will first elaborate the the context before the declaration $d$ into contexts $\Theta$ and $\Psi$.
\item $s\notin\text{dom}(\Psi)$ checks that the TSM $s$ was not yet declared before (in $\Psi$).
\item $\emptyset\vdash_{\Theta_0\Theta}\tau::\mathbf{Ty}$ will check that if the type $\tau$ is of kind $\mathbf{Ty}$, as the type $\tau$ is the type for the term generated by the TSM application.
\item $\emptyset;\emptyset\vdash_{\Theta_0\Theta}^{\Psi}e_{tsm}\rightsquigarrow i_{tsm} \Leftarrow \mathtt{Parser(Exp)}$, this is the elaboration judgment of the parser defined with the $s$. The parser term $e$ will be elaborated under the TSM context $\Psi$ and named type context $\Theta_0\Theta$, where $\Psi$ and $\Theta$ are declarations appeared before $s$ and $\Theta_0$ is the context with types declared in the Wyvern prelude (Fig.~\ref{exp-prelude}). The term $e$ will be checked against the type $\mathtt{Parser(Exp)}$ when elaborating to internal form $i_{tsm}$, as it should be an expression generator.
\end{itemize}

The rule for analytical TSMs is similar to that of synthetic TSMs, except that no return type $\tau$ will be checked, as the type of a term generated from an analytical TSM will be determined by the type provided in TSM application term $\mathbf{eaptsm}(s,body)$.

\subsubsection{Type-level TSMs}
The elaboration rule for type-level TSMs (D-tytsm) is also similar to that of synthetic TSMs except that 1) the parser is checked against the type $\mathtt{Parser(Pair(Type,Parser(Exp)))}$, where the type argument $\mathtt{Pair(Type,Parser(Exp))}$ specifies the return type of the parser, and 2) the kind $\kappa$ is provided, which is the kind for a type generated by the application of the TSM.

\subsubsection{Types}
To support mutually recursive type declarations, as presented in the rule D-rec, the elaboration of type declarations is separated into two phases: 1) type names and kinds checking (specified as $\vdash_{\Theta}^{\Psi}\theta\sim_{\mathrm{I}}\Theta_1$) and 2) type structure and parser elaboration ($\vdash_{\Theta}^{\Psi}\theta\sim_{\mathrm{II}}\Theta_2$). 

\paragraph{Phase-I elaboration} The rule T-I will promote the elaboration of mutually defined types $\theta_1 \mathsf{and} \theta_2$ to its components, namely $\theta_1$ and $\theta_2$, independently.  In both rules T-tydecl-I and T-aptsm-I, the type name $\T$ will be checked to be free of conflicts in the previous context and then store the type with its kind $\kappa$ into the context.

\paragraph{Phase-II elaboration}
Following the first elaboration phase, the type structure, together with its syntax parser, will be added into the contexts to finish type elaboration. The rule T-II is again a promotion to its components. 

The rule T-tydecl-II, will 1) check the structure of the type $\tau::\kappa$ under the named type context $\Theta$ and 2) elaborate the syntax parser to an internal term of type $\mathtt{Parser}(Exp)$, which is a parser that will transforming literals to an expression. We should note that the names of recursively defined types are in the context $\Theta$ according to the rule D-rec, and this how mutually recursively defined types are supported.

The rule T-aptsm-II will deal with type structure and the syntax parser elaboration for the types generated by TSM application. The elaboration contains the following premises:
\begin{enumerate}
\item Look up the TSM $s$ in the TSM context $\Psi$. ($s[\mathbf{tytsm}(\kappa,i_{tsm})]\in\Psi$)
\item Parse the literal stream $i_{ps}$ use the TSM parser $i_{tsm}$, resulting in an expression pair with the type expression $i_{type}$ and the syntax parser $i$. ($i_{tsm}(i_{ps})\Downarrow (i_{type},i)$)
\item Dereificate the type $\hat\tau$ from the term-represented form $i_{type}$, and then elaborate the translational type to normal type form $\tau::\kappa$ ($i_{type}\uparrow\hat\tau$ and $\emptyset;\emptyset\vdash^{\Psi}_{\Theta}\hat\tau\leadsto\tau:\kappa$)
\end{enumerate}
With these premises, the type $\T$ will go into the context in the elaborated form: $\T[\tau:\kappa,i]$, with its kind, structure and parser specified.

% Before going to the rule, we will first introduce the annotation we used for intermediate type representation in the rule: we use $\mathring\theta$ to present the type structure before parser elaboration. 
% \[
%       \mathring\theta~::=~\T[\tau::\kappa,e] ~| ~\T[\tau::\kappa,i]
% \]
% The first form $\T[\tau::\kappa,e]$ represents the form for named type declaration, with the parser in an external form $e$, while the second form $\T[\tau:\kappa,i]$ is for the intermediate state of type-level TSM application, where $i$ is the parser generated from literal parsing. (It is already an internal Wyvern term as it is the result of parser application.) 

% \paragraph{D-rec}Now we turn to the elaboration rules for type declarations, and how the three phase elaboration is used in the rule (D-rec):
% \begin{itemize}\itemsep0pt
% \item We will first elaborate the previous context with the judgment $d\sim(\Theta,\Psi)$.
% \item Name check: the premise $\neg\exists~u,v\in\{1,...,n\}.(u\neq v\land\T_u=\T_v)$ and $\forall~k.(\T_k\notin\text{dom}(\Theta))$ will make sure that no name conflicts exist.
% \item Kinds check: In the premise $\forall~k\in\{1,...,n\}.(\vdash_{\Theta_0\Theta}^{\Psi}\theta_k\leadsto \mathring\theta_k\land \mathsf{name}(\mathring\theta_k)=\T_k \land \emptyset\vdash\mathsf{type}(\mathring\theta_k)::\kappa_1\rightarrow...\rightarrow\kappa_n\rightarrow\kappa_k)$, we will check the kinds for each type. In the mutually recursive types declarations, we do not directly check the type structure with names declared in the body (namely $\T_1,...,\T_n$), instead, we first treat each type as an type argument $\T$, and the declared names will be input of the type in next rule. Thus each type $\T_k$ should have kind $\kappa_1\rightarrow ... \rightarrow \kappa_n\rightarrow\kappa_k$, by applying the abstraction form to the arguments $\T_1,...,\T_n$, we are able to generate the type $\T_k$ with kind $\kappa_k$, and in this way, we allow types refer to all names declared together recursively.
% \item Parser elaboration: $\forall~k.(\vdash_{\Theta_0\Theta,\T_1[\tau_1(\T_1,...,\T_n)::\kappa_1,-],...,\T_n[\tau_n(\T_1,...,\T_n)::\kappa_n,-]}\mathring\theta_k\leadsto \T_k[\tau_k::\kappa_k,i_k])$ is the elaboration rules for parsers. The definition of the judgment $\vdash_{\Theta}^{\Psi}\mathring\theta\leadsto\T[\tau_k:\kappa_k,i_k]$ will be presented later. In this rule, we use the named type context $\Theta_0\Theta,\T_1[\tau_1(\T_1,...,\T_n)::\kappa_1,-],...,\T_n[\tau_n(\T_1,...,\T_n)::\kappa_n,-]$ with 1) Types in the Wyvern prelude $\Theta_0$, 2) Types declared in the previous context $\Theta$ and 3) Types appeared in the mutually recursive declaration body  $\T_n[\tau_n(\T_1,...,\T_n)::\kappa_n]$. This enables the parsers to visit types declared together in the declarations.
% \end{itemize}

% For each type $\theta$, the elaboration rules are presented in (D-tydecl-*) and (D-aptsm-*), where we check the type structure and elaborate the TSL parser.
% \paragraph{D-tydecl-*} For regularly declared types, the elaboration rule is straightforward: we will check its kind in the rule (D-tydecl-1) and then elaborate the TSL parser in (D-tydecl-2), where the parser will be checked against type $\mathtt{Parser(Exp)}$.

% \paragraph{D-aptsm-*} The rule (D-aptsm-1) will first apply the TSM parser to user defined literals to generate to type structure and its TSL parser, and then in the rule (D-aptsm-2) and (D-aptsm-2'), the TSL parser may be substituted with the one defined with TSM $s$ depending on its definition. 

% In the rule (D-aptsm-1), to transform a type declaration $\theta$ to its intermediate form $\mathring\theta$, we need the following premises:
% \begin{itemize}\itemsep0pt
% \item Look up the TSM $s[\mathbf{ty}(\kappa,i_{tsm})]$ from the context $\Psi$. ($s[\mathbf{ty}(\kappa,i_{tsm})]$)
% \item Parse the delimited literals $body$ into an expression represented form $i_{ps}$. ($\mathsf{parsestream}(body)=i_{ps}$)
% \item Apply the TSM parser $i_{tsm}$ to the parse stream $i_{ps}$, the result of which is an expression of type $\mathtt{Result(Type \times Parser(Exp))}$, with the type structure $i_{type}$ and the TSL parser $i_{syn}$.
% \item Dereificate the type expression $i_{type}$ to a translational type form $\hat\tau$.
% \item Elaborate the type $\hat\tau$ into internal form $\tau$ of kind $\kappa$ under the kinding context $\Delta;\emptyset$. Two contexts are used in this situation for hygiene purpose: the $\mathbf{spliced}(\tau)$ defined in the translational types in Fig.~\ref{syntax-types} will refer to the inner context $\Delta_{in}=\emptyset$ and other translational type terms will use the outer context $\Delta_{out}=\Delta$.
% \end{itemize}
% The elaboration result will be a type declaration $\T[\tau::\kappa,e_{synx},i_{syn}]$, where $e_{synx}$ is the parser declared with the type level TSM, allowing users to use $e_{synx}$ to override the generated TSL parser $i_{syn}$ if $e_{edx}$ is defined.

% We then elaborate the parser in the rule (D-aptsm-2) and (D-aptsm-2'). Under the elaboration context $\Psi$ and $\Theta$, we will try to elaborate the parser term $e_{synx}$ to an internal term. If the parser is defined, then according to the rule (D-aptsm-2), the parser will elaborates to $i_{syn}$ of type $\mathtt{Parser(Exp)}$ and substitute the parser generated before. If not, according to the (D-aptsm-2'), the generated parser will be the TSL parser for the type.


% here is v2
%!TEX root = sac15-tr.tex
\begin{figure}
\flyingbox{$d\sim (\Psi;\Theta)$}
\begin{center}
\AXC{} \RightLabel{(D-empty)}
\UIC{$\emptyset \sim (\emptyset;\emptyset)$} 
\DP
\end{center}

\begin{center}

\AXC{
	\stackanchor{
	$d\sim(\Psi;\Theta)$ ~~~~ $s\notin\text{dom}(\Psi)$ ~~~~ $\emptyset\vdash_{\Theta_0\Theta}\tau::\mathtt{Ty}$}{$\emptyset;\emptyset\vdash_{\Theta_0\Theta}^{\Psi}e_{tsm}\rightsquigarrow i_{tsm} \Leftarrow \mathtt{Parser(Exp)}$}
}\RightLabel{(D-syntsm)}
\UIC{$d;\mathbf{syntsm}(s,\tau,e_{tsm})\sim (\Psi,s[\mathbf{syn}(\tau,i_{tsm})];\Theta)$}
\DP
\end{center}

\begin{center}
\AXC{
	$d\sim(\Psi;\Theta)$ ~~~~ $s\notin\text{dom}(\Psi)$ ~~~~ $\emptyset;\emptyset\vdash_{\Theta_0\Theta}^{\Psi}e_{tsm}\rightsquigarrow i_{tsm} \Leftarrow \mathtt{Parser(Exp)}$
}\RightLabel{(D-anatsm)}
\UIC{$d;\mathbf{anatsm}(s,e_{tsm})\sim (\Psi,s[\mathbf{ana}(i_{tsm})];\Theta)$}
\DP
\end{center}

\begin{center}
\AXC{
	\stackanchor
	{$d\sim(\Psi;\Theta)$ ~~~~ $s\notin \text{dom}(\Psi)$}
	{$\emptyset;\emptyset\vdash_{\Theta_0\Theta}e_{tsm}\rightsquigarrow i_{tsm}\Leftarrow \mathtt{Parser}(\mathtt{Pair}(\mathtt{Type},\mathtt{Parser(Exp)}))$}
}\RightLabel{(D-tytsm)}
\UIC{$d;\mathbf{tytsm}(s,\kappa,e_{tsm})\sim(\Psi,s[\mathbf{ty}(\kappa,i_{tsm})];\Theta)$}
\DP
\end{center}

%\begin{center}
%\AXC{
%	\stackanchor
%	{$d\sim(\Psi;\Theta)$ ~~~~ $\T\notin\text{dom}(\Theta_0\Theta)$ ~~~~ $\emptyset\vdash_{\Theta_0\Theta}\tau::\kappa\rightarrow\kappa$}
%	{$\emptyset;\emptyset\vdash_{\Theta_0\Theta,\T[\tau(\T)::\kappa,-]}^{\Psi}e_{md}\rightsquigarrow i_{md}\Rightarrow\tau_{md}$}
%}\RightLabel{(D-tydecl)}
%\UIC{$d;\mathbf{typedecl}(\T,\tau,e)\sim (\Psi;\Theta,\T[\tau(\T)::\kappa,i_{md}::\tau_{md}])$}
%\DP
%\end{center}
%
%\begin{center}
%\AXC{
%	\stackanchor{
%		\stackanchor{$d \sim (\Psi;\Theta)$ ~~~~ $\mathtt{T}\notin \text{dom}(\Theta_0\Theta)$ ~~~~ $s[\mathbf{ty}(\kappa,\tau_{md},i_{tsm})]\in\Psi$}
%		{
%			\stackanchor{$\mathsf{parsestream}(body)=i_{ps}$ ~~~~ $i_{tsm}.parse(i_{ps}) \Downarrow OK((i_{type},i_{md}))$}{
%				$i_{type}\uparrow\hat\tau$ ~~~~ $\Delta;\emptyset\vdash_{\Theta\Theta'}\hat\tau\rightsquigarrow\tau::\kappa \rightarrow \kappa$
%			}
%		}
%	}
%	{$\emptyset;\emptyset\vdash_{\Theta_0\Theta,\mathtt{T}[\tau(\mathtt{T}) :: \kappa,-]}^{\Psi}e_{mdx}\rightsquigarrow i_{mdx} \Rightarrow \tau_{md}\rightarrow\tau'_{md}$ ~~~~ $i_{mdx}(i_{md})\Downarrow i'_{md}$}
%}
%\RightLabel{(D-aptsm)}
%\UIC{$d;\mathbf{tyaptsm}(\mathtt{T},s,body,e_{mdx}) \sim (\Psi; \Theta,\mathtt{T}[\tau(\mathtt{T}) :: \kappa,i'_{md}:\tau'_{md}])$}
%\DP
%\end{center}

% \begin{center}
% \AXC{
% 	\stackanchor{
% 	\stackanchor{$d\sim(\Psi;\Theta)$ ~~~~ $\neg\exists~u,v\in\{1,...,n\}.(u\neq v\land\T_u=\T_v)$ ~~~~ $\forall~k.(\T_k\notin\text{dom}(\Theta))$}
% 	{$\forall~k\in\{1,...,n\}.(\vdash_{\Theta_0\Theta}^{\Psi}\theta_k\leadsto \mathring\theta_k\land \mathsf{name}(\mathring\theta_k)=\T_k \land \emptyset\vdash\mathsf{type}(\mathring\theta_k)::\kappa_1\rightarrow...\rightarrow\kappa_n\rightarrow\kappa_k)$}}
% 	{$\forall~k.(\vdash_{\Theta_0\Theta,\T_1[\tau_1(\T_1,...,\T_n)::\kappa_1,-],...,\T_n[\tau_n(\T_1,...,\T_n)::\kappa_n,-]}\mathring\theta_k\leadsto \T_k[\tau_k::\kappa_k,i_k])$}
% 	}\RightLabel{(D-rec)}
% \UIC{$d;\mathbf{mrectydecl}(\theta_1,...,\theta_n)\sim(\Psi;\Theta,\T_1[\tau_1::\kappa_1,i_1],...,\T_n[\tau_n::\kappa_n,i_n])$}
% \DP
% \end{center}



\begin{center}
\AXC{$d\sim(\Theta,\Psi)$~~~~$\vdash_{\Theta}^{\Psi}\theta \sim_{\mathrm{I}} \Theta_{1}$~~~~$\vdash^{\Psi}_{\Theta\Theta_{1}}\theta\sim_{\mathrm{II}}\Theta_{2}$}\RightLabel{(D-rec)}
\UIC{$d;\theta \sim (\Theta\Theta_2;\Psi)$}
\DP
\end{center}

% \flyingbox{$\vdash_{\Theta}^{\Psi}\theta\leadsto\mathring\theta$}
% \begin{center}
% \AXC{
% 	{$\emptyset\vdash_{\Theta_0\Theta}\tau::\kappa$}
% }\RightLabel{(D-tydecl-1)}
% \UIC{$\vdash_{\Theta}^{\Psi}\mathbf{typedecl}(\T,\tau,e)\leadsto \T[\tau::\kappa,e]$}
% \DP
% \end{center}

% \begin{center}
% \AXC{
% 		\stackanchor{$s[\mathbf{ty}(\kappa,i_{tsm})]\in\Psi$ ~~~~ $\mathsf{parsestream}(body)=i_{ps}$}
% 		{
% 			{$i_{tsm}.parse(i_{ps}) \Downarrow OK((i_{type},i_{syn}))$ ~~~~	$i_{type}\uparrow\hat\tau$ ~~~~ $\Delta;\emptyset\vdash_{\Theta\Theta'}\hat\tau\rightsquigarrow\tau::\kappa$
% 			}
% 		}
% }\RightLabel{(D-aptsm-1)}
% \UIC{$\vdash_{\Theta}^{\Psi}\mathbf{tyaptsm}(\mathtt{T},s,body) \leadsto \mathtt{T}[\tau::\kappa,i_{syn}]$}
% \DP
% \end{center}

% \flyingbox{$\vdash^{\Psi}_{\Theta}\mathring\theta\leadsto\T[\tau::\kappa,i]$}
% \begin{center}
% \AXC{
% 	$\emptyset;\emptyset\vdash_{\Theta}^{\Psi}e_{syn}\rightsquigarrow i_{syn}\Leftarrow\mathtt{Parser(Exp)}$
% }\RightLabel{(D-tydecl-2)}
% \UIC{$\vdash_{\Theta}^{\Psi}\T[\tau::\kappa,e_{syn}]\leadsto\T[\tau::\kappa,i_{syn}]$}
% \DP
% \end{center}

% \begin{center}
% \AXC{~}\RightLabel{(D-aptsm-2)}
% \UIC{$\vdash_{\Theta}^{\Psi}\T[\tau::\kappa,i_{syn}] \leadsto\mathtt{T}[\tau::\kappa,i_{syn}]$}
% \DP
% \end{center}

\flyingbox{$\vdash_{\Theta}^{\Psi}\theta\sim_{\mathrm{I}}\Theta$}

\begin{center}
\AXC{$\vdash_{\Theta}^{\Psi}\theta_1\sim_{\mathrm{I}}\Theta_1$~~~~$\vdash_{\Theta\Theta_1}^{\Psi}\theta_2\sim_{\mathrm{I}}\Theta_2$}\RightLabel{(T-I)}
\UIC{$\vdash_{\Theta}^{\Psi}\theta_1~\mathsf{and}~\theta_2 \sim_{\mathrm{I}} \Theta_1\Theta_2$}
\DP
\end{center}

\begin{center}
\AXC{$\T\notin \text{dom}(\Theta)$}\RightLabel{(T-tydecl-I)}
\UIC{$\vdash_{\Theta}^{\Psi}\mathbf{tydecl}(\T,\tau,\kappa,e)\sim_{\mathrm{I}}\T[-::\kappa,-]$}
\DP
\end{center}

\begin{center}
\AXC{$\T\notin\text{dom}(\Theta)$~~~~$s[\mathbf{tytsm}(\kappa,i_{tsm})]\in\Psi$}\RightLabel{(T-aptsm-I)}
\UIC{$\vdash_{\Theta}^{\Psi}\mathbf{tyaptsm}(\mathtt{T},s,body)\sim_{\mathrm{I}}\T[-::\kappa,-]$}
\DP
\end{center}

\flyingbox{$\vdash_{\Theta}^{\Psi}\theta\sim_{\mathrm{II}}\Theta$}
\begin{center}
\AXC{$\vdash_{\Theta}^{\Psi}\theta_1\sim_{\mathrm{II}}\Theta_1$~~~~$\vdash_{\Theta}^{\Psi}\theta_2\sim_{\mathrm{II}}\Theta_2$}\RightLabel{(T-II)}
\UIC{$\vdash_{\Theta}^{\Psi}\theta_1~\mathsf{and}~\theta_2 \sim_{\mathrm{II}} \Theta_1\Theta_2$}
\DP
\end{center}

\begin{center}
\AXC{$\emptyset\vdash_{\Theta}^{\Psi}\tau::\kappa$~~~~$\emptyset;\emptyset\vdash_{\Theta}^{\Psi}e\leadsto i\Leftarrow \mathtt{Parser}(\mathtt{Exp})$}
\RightLabel{(T-tydecl-II)}
\UIC{$\vdash_{\Theta}^{\Psi}\mathbf{tydecl}(\T,\tau,\kappa,e)\sim_{\mathrm{II}}\T[\tau:\kappa,i]$}
\DP
\end{center}

\begin{center}
\AXC{
\stackanchor
	{$s[\mathbf{tytsm}(\kappa,i_{tsm})]\in\Psi$~~~~$\mathsf{parsestream}(body)=i_{ps}$}
	{$i_{tsm}(i_{ps})\Downarrow(i_{type},i)$~~~~$i_{type}\uparrow \hat\tau$~~~~$\emptyset;\emptyset\vdash_{\Theta}^{\Psi} \hat\tau\leadsto\tau::\kappa$}
}\RightLabel{(T-aptsm-II)}
\UIC{$\vdash_{\Theta_1}^{\Psi}\mathbf{tyaptsm}(\T,s,body)\sim_{\mathrm{II}}\T[\tau::\kappa,i]$}
\DP
\end{center}


\caption{Declaration checking and elaboration of type-level TSMs.}
\label{typechecking-elaboration}
\end{figure}






\subsection{Terms}
\subsubsection{Term-level TSM Application}
Term-level TSM application is captured by the abstract form $\textbf{eaptsm}[s, body]$, where $s$ is the name of the TSM and $body$ is the body of the delimited form, per Fig.~\ref{f-delimited}. The rule (T-syn) shows how typing and elaboration for a synthetic TSM proceeds:
\begin{enumerate}\itemsep0pt
\item The definition of $s$ is extracted from $\Psi$. 
\item A parse stream is constructed on the basis of $body$. We assume the relation $\mathsf{parsestream}(body)=i_{ps}$ is defined such that $i_{ps}$ is a closed term of type $\mathtt{ParseStream}$. 
\item The $parse$ method of the TSM implementation is invoked with the parse stream. The judgement $i \Downarrow i'$ captures evaluation of $i$ to a value, $i'$. (The evaluation rules for internal Wyvern terms can be referred to those in TSL \cite{TSLs}, as they are not directly related to TSM mechanisms)
\end{enumerate}
As suggested by the declarations in Figure \ref{exp-prelude}, the result is either a parse error or a valid parse, written $OK(i_{exp})$, where $i_{exp}$ is a value of type $\mathtt{Exp}$. Here, we simply leave the error case undefined -- the typing judgement cannot be derived if there is a parse error.% -- though in practice, the compiler would report errors that occurred. 

The \emph{dereification judgement} $i \uparrow \hat{e}$, defined in \cite{TSLs}, takes a value of type $\mathtt{Exp}$ to a corresponding \emph{translational term}, $\hat{e}$. 
Translational terms mirror external terms but include an additional form, $\textbf{spliced}[e]$, which captures portions of the parse stream parsed as a spliced term. The case $\mathtt{Spliced}$ of case type $\mathtt{Exp}$, which takes a (portion of) a parse stream, dereifies to this form. This permits us to ensure that only spliced portions of parse streams can refer to variables in the surrounding scope (and no others), ensuring that hygiene is maintained. 

This is technically accomplished by the judgements 
\[\Delta_{out}; \Gamma_{out}; \Delta; \Gamma \vdash_{\Theta}^{\Psi} \hat{e} \leadsto i {\Rightarrow}{(\Leftarrow)} \tau\]  
which can be read ``under outer typing and kinding contexts $\Delta_{out}$ and $\Gamma_{out}$ and inner typing and kinding contexts $\Delta$ and $\Gamma$,  $\hat{e}$ elaborates to internal term $i$ and (synthesizes/analyzes against) type $\tau$''. These judgements behave identically to the corresponding judgements for external terms, using the inner typing and kinding contexts, until a term of the form $\textbf{spliced}[e]$ is encountered. The outer contexts are then used. The relevant rule can be found in \cite{TSLs}. 

In the rules for TSLs and TSMs, (T-syn) and (T-ana), the inner contexts begin empty so only variables inside spliced terms can refer to outer variables. A $\mathtt{parse}$ function that generated, for example, $\mathtt{Var('x')}$, would not typecheck, because it captures a variable that it cannot know exists. An occurrence of a variable $\mathtt{x}$ inside a spliced portion (e.g. between $\mathtt{<\{}$ and $\mathtt{\}>}$ when using the $\mathtt{HTML}$ TSL) would be checked in the outer context and thus be acceptable. Parse streams cannot be created manually, so this guarantee is strict. %The details of this mechanism at the term level are essentially identical (up to the orthogonal addition of parameterized types) to that in \cite{TSLs}, so we omit the rules.

In the rule (T-syn), the type that $\hat{e}$ is being analyzed against is determined by the definition of the TSM. The rule (T-ana) is essentially identical, but the type is determined by the type that the whole application is being analyzed against instead.

\subsubsection{Parser Access}
Users can also access the parsers defined within the TSMs at runtime with the term $\mathbf{etsmdef}[s]$. Depending on whether the TSM $s$ is a type-level TSM or a term-level TSM, one of the rules (T-syntsmdef), (T-anatsmdef) and (T-typetsmdef) in Fig.~\ref{tsm-statics} will be used. The elaboration is straightforward: simply look up $s$ in the TSM context $\Psi$ and access the parser defined within. If the TSM is a type-level TSM, the parser will synthesize to a term $i$ of type $\mathtt{Parser(Type)}$ and for a term-level TSM, its type will be $\mathtt{Parser(Exp)}$.

%!TEX root=sac15-tr.tex
\begin{figure}
\begin{center}
\AXC{
	\stackanchor
	{$\Theta_0 \subset \Theta$ ~~~~ $s[\mathbf{ana}(i_{tsm})]\in\Theta$ ~~~~ $\mathsf{parsestream}(body)=i_{ps}$}
	{$i_{tsm}.parse(i_{ps});\Downarrow OK((i_{ast}, i'_{ps}))$ ~~~~ $i_{ast}\uparrow \hat{e}$ ~~~~ $\Delta;\emptyset;\Gamma;\emptyset\myvdash \hat{e} \rightsquigarrow i \Leftarrow \tau$}
}\RightLabel{(T-ana)}
\UIC{$\Delta;\Gamma\myvdash\mathbf{eaptsm}[s,body] \rightsquigarrow i \Leftarrow \tau$}  
\DP
\end{center}

\begin{center}
\AXC{
	\stackanchor
	{$\Theta_0 \subset \Theta$ ~~~~ $s[\mathbf{syn}(\tau,i_{tsm})]\in\Theta$ ~~~~ $\mathsf{parsestream}(body)=i_{ps}$}
	{$i_{tsm}.parse(i_{ps})\Downarrow OK((i_{ast}, i'_{ps}))$ ~~~~ $i_{ast}\uparrow \hat{e}$ ~~~~ $\Delta;\emptyset;\Gamma;\emptyset\myvdash \hat{e} \rightsquigarrow i \Leftarrow \tau$}
} \RightLabel{(T-syn)}
\UIC{$\Delta;\Gamma\myvdash\mathbf{eaptsm}[s,body] \rightsquigarrow i\Rightarrow \tau$}  
\DP
\end{center}

\begin{center}
\AXC{$\Theta_0\subset\Theta$ ~~~~ $s[\mathbf{syn}(\tau,i)]\in\Psi$}\RightLabel{(T-syntsmdef)}
\UIC{$\Delta;\Gamma\myvdash\mathbf{etsmdef}[s] \rightsquigarrow i\Rightarrow\mathtt{Parser}(\mathtt{Exp})$}
\DP
\end{center}

\begin{center}
\AXC{$\Theta_0\subset\Theta$ ~~~~ $s[\mathbf{ana}(i)] \in \Psi$}\RightLabel{(T-anatsmdef)}
\UIC{$\Delta;\Gamma\myvdash \mathbf{etsmdef}[s] \rightsquigarrow i\Rightarrow\mathtt{Parser}(\mathtt{Exp})$}
\DP
\end{center}

\begin{center}
\AXC{$\Theta_0\subset\Theta$ ~~~~ $s[\mathbf{ty}(\kappa,i)]\in\Psi$} \RightLabel{(T-typetsmdef)}
\UIC{$\Delta;\Gamma\myvdash \mathbf{etsmdef}[s]\rightsquigarrow i\Rightarrow\mathtt{Parser}(\mathtt{Type})$}
\DP
\end{center}
\caption{Statics rules for TSMs applications}
\label{tsm-statics}
\end{figure}
