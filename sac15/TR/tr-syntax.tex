%!TEX root=sac15-tr.tex
\section{Syntax}


\begin{figure}[ht]
\hspace{-5px}$\begin{array}{rrlll}
      &\multicolumn{3}{l}{\textbf{Abstract Forms}}   & \textbf{Concrete Forms}\\
      \text{Programs}		&	\rho & ::= & d;e\\
      \text{Declarations}	&	d &::=&\emptyset\\
                          &&    | & {d}; \mathbf{syntsm}(s,\tau,e) & \textcd{syntax}~s~\textcd{for}~\tau~\textcd{=}~e\\
                          &&    | & {d}; \mathbf{anatsm}(s,e) & \textcd{syntax}~s~\textcd{=}~e\\
                          &&    | & {d}; \mathbf{tytsm}(s,\kappa,e) & \textcd{syntax}~s::\kappa~\textcd{=}\,e\\
                          &&    | & {d}; \theta \\
      \text{Type Declaration}&\theta  &::=& \theta_1~\mathbf{and}~\theta_2 & \theta_1~\textcd{and}~\theta_2\\
                          &&| & \mathbf{tydecl}(\T,\tau,\kappa,e) & \textcd{type}~\mathtt{T}~\textcd{=}~\tau::\kappa\\
                          && &&~~~~~~\textcd{syntax = }e\\
               &&|&\mathbf{tyaptsm}(\T,s,body) & \textcd{type}~\mathtt{T}~\textcd{=}~\mathit{s~dform}\\
%      \text{Object Fields}\\
%      \omega~::=~\emptyset                      \\
%      \tabularspace l[\mathbf{val},\tau];\omega                 & \textcd{val}~l : \tau\\
%      \tabularspace l[\mathbf{def},\tau];\omega                 & \textcd{def}~l : \tau\\
%      \text{Casetype Cases}\\
%      \chi~::=~\emptyset                      \\                 
%      \tabularspace C[\tau];\chi                   & C~\textcd{of}~\tau\\
      \text{External Terms}	&	e &::=& ...    & \\
                            && | &	\mathbf{lit}[body]             & \mathit{dform}\\
      && | &	\mathbf{eaptsm}[s,body]       & \mathit{s~dform}\\
      \text{Translational Terms}	& \hat{e}&::=&... \\
      								&& | & \mathbf{spliced}[e] \\
      \text{Internal Terms}			& i &::=& ...
  \end{array}$
\mycaption{Abstract and concrete forms for declarations and terms. Metavariable $s$ ranges over TSM names, $\mathtt{T}$ over type names, $\mathit{dform}$ over delimited forms per Fig.~\ref{f-delimited} , and $body$ over their bodies. Translational and internal terms are used in the semantics only. Elided forms are given in \cite{TSLs}.}
\label{formal-syntax}
\end{figure}

\begin{figure}[t]
\begin{lstlisting}[style=wyvern,numbers=none,xleftmargin=0px,framexleftmargin=0px]
'@\htmlcolor{body here, '{}'inner single quotes'{}' must be doubled}@'
[@\htmlcolor{body here, [inner braces] must be balanced}@]
{@\htmlcolor{body here, \{inner curly braces\} must be balanced}@}
~ (* can appear at any expression position *)
  @\htmlcolor{forward referenced body here, leading indent stripped}@
(parentheses_delimitv_spliced_terms_in_TSM_arguments(~)) 
  @\htmlcolor{so forward references can propagate out}@
[@\htmlcolor{adjacent}@] {@\htmlcolor{delimited forms}@} @\htmlcolor{or}@ [@\htmlcolor{those}@] @\htmlcolor{separated}@ ~ @\htmlcolor{form}@
  @\htmlcolor{by identifiers create a single multipart delimited}@
\end{lstlisting}
\mycaption{Available delimited forms in Wyvern.}
\label{f-delimited}
\end{figure}

In the pure functional Wyvern definition, a Wyvern program consists two parts: 1) declarations $d$, consisting of both TSM declarations and type declarations and 2) an external term $e$ representing the program body.

\paragraph{Declarations}
The first part of the declarations $d$ is TSM declarations, containing analytical term-level TSM $\mathbf{anatsm}(s,e)$, synthetic term-level TSM $\mathbf{syntsm}(s,\tau,e)$ and type-level TSM $\mathbf{tytsm}(s,\kappa,e)$. The analytical TSM declaration $\mathbf{anatsm}(s,e)$ contains its name $s$ and its associated parser $e$ (which is presented in the form of an Wyvern term). A synthetic TSM declaration is similar, except that the type $\tau$ of the TSM is explicitly declared, while a type-level TSM contains the kind information $\kappa$ for the type along with its macro name $s$ and parser $e$.

Besides TSM declarations, there are also type declarations. To support mutually recursive type declarations, we use the Ocaml style type declarations: mutually recursive types are declared by connecting to each other using the keyword $\textcd{and}$. The atomic forms of type declarations can be one of the namely explicit type declaration $\mathbf{tydecl}(\mathtt{T},\tau,\kappa,e)$ and type-level TSM application $\mathbf{tyaptsm}(\T,s,body)$. An explicit type declaration $\mathbf{tydecl}(\mathtt{T},\tau,e)$ contains the type name $\mathtt{T}$, the type structure $\tau$ together with its kind $\kappa$, and an optional TSL parser $e$ if the type is declared with a TSL. To define a type using a type-level TSM, one should replace the type structure $\tau$, defined in Fig.~\ref{syntax-types}, with a TSM application term: $s~\mathit{dform}$, by applying the type level TSM $s$ to the literals in delimited forms $\mathit{dform}$, the type structure will then be generated in the elaboration phase, which will be presented in the Sec.~\ref{tr-semantics}.

\paragraph{Terms} The external terms are the syntax users used in programming with the Wyvern language, and they will elaborate to internal Wyvern terms for execution, which contain neither literals nor the form for accessing the TSM or TSL parsers of a named type or macro explicitly, in the elaboration phase (which will be discussed later in the next section). The translational terms $\hat{e}$ and internal terms $i$ are exactly the same as those defined in \cite{TSLs}. For external terms, besides those in \cite{TSLs}, we supplement an expression $\mathbf{eaptsm}[s,body]$ to support explicitly TSM parser access in terms. 

\paragraph{Types and Kinds}
Wyvern types and kinds are described below in Fig.~\ref{syntax-types}. Type definitions are quite standard: containing named type $\T$, arrow type $\tau\rightarrow\tau$, object type $\mathbf{objtype}[\omega]$ (similar to record types in ML), case type $\mathbf{casetype}[\chi]$ (similar to ML sum type), type variable $t$, type abstraction $\lambda[\kappa](t.\tau)$ and type application form $\tau(\tau)$. Besides types definition, we defined translational types $\hat\tau$, which are similar to translational terms: they mirror the types and are used for hygiene purpose when deriving types from type-level TSM application in the elaboration phase.

\begin{figure}[ht]
\[
	\begin{array}{llcl}
      \text{Kinds}\\
      \mytab	\kappa ::=  \mathbf{Ty}  ~|~ \kappa \rightarrow \kappa\\
      \text{Types}\\
      \mytab	 \tau  ::=  \mathtt{T} ~|~ \tau \rightarrow \tau ~|~ \mathbf{objtype}[\omega] ~|~ \mathbf{casetype}[\chi] ~|~  t ~|~ \lambda[\kappa](t.\tau) ~|~ \tau(\tau) \\
      \text{Translational Types}\\
      \mytab\hat{\tau} ::= \mathtt{T} ~|~ \hat\tau \rightarrow \hat\tau ~|~ \mathbf{objtype}[\hat\omega] ~|~ \mathbf{casetype}[\hat\chi] ~|~  t ~|~ \lambda[\kappa](t.\hat\tau) ~|~ \hat\tau(\hat\tau) ~|~ \mathbf{spliced}[\tau]\\
      \text{Objtype}~~~~~~~~~~~~~~~~~~~~~~~~~~~~~~~~~~~~~~~~~~~~~~~~~~~~~~~~~~~~~\text{Casetype}\\
      \mytab\omega ::= \emptyset ~|~ l[\mathbf{val},\tau];\omega ~|~ l[\mathbf{def},\tau];\omega ~~~~~ \mytab\chi ::= \emptyset ~|~ C[\tau];\chi\\
      %\text{Translational Objtype Declaration}\\
      %\text{Casetype Declaration}\\
      \mytab\hat\omega ::= \emptyset ~|~ l[\mathbf{val},\hat\tau];\hat\omega ~|~ l[\mathbf{def},\hat\tau];\hat\omega ~~~~~ \mytab\hat\chi ::= \emptyset ~|~ C[\hat\tau];\hat\chi
      %\text{Translational Casetype Declaration}\\
  \end{array}
\]
\mycaption{Syntax for types and kinds. Metavariable $t$ ranges over type variables.
%Metavariable $T$ ranges over type names, $l$ over member labels, $C$ over case names, $x$ over term variables, $X$ over type variables.
}\label{syntax-types}
\end{figure}