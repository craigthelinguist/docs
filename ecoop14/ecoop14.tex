\documentclass[runningheads]{llncs}
\usepackage{amsmath}
\usepackage{latexsym}
\usepackage{verbatim}
\usepackage[T1]{fontenc}
%\usepackage[defaultmono]{droidmono}
\usepackage{proof,amssymb,enumerate}
\usepackage{math-cmds}
\usepackage{listings}
%\setcounter{tocdepth}{3}
%\renewcommand*\ttdefault{txtt}
\usepackage[scaled]{beramono}
\usepackage[usenames,dvipsnames]{xcolor}
\usepackage{graphicx}
\usepackage{url}
\usepackage{mathtools}
\newcommand{\keywords}[1]{\par\addvspace\baselineskip
\noindent\keywordname\enspace\ignorespaces#1}

\usepackage{fancyvrb}
\renewcommand{\theFancyVerbLine}{%
\color{gray}\ttfamily{\scriptsize
\arabic{FancyVerbLine}}}

\def\implies{\Rightarrow}
\newcommand{\todo}[1]{\textbf{[TODO: #1]}}
\newcommand{\keyw}[1]{\textbf{#1}}

\newtheorem{thm}{Theorem}
\newtheorem{dfn}{Definition}

\lstset{tabsize=2, 
basicstyle=\ttfamily\fontsize{8pt}{1em}\selectfont, 
commentstyle=\itshape\rmfamily, 
numbers=left, numberstyle=\scriptsize\color{gray}\ttfamily, language=java,moredelim=[il][\sffamily]{?},mathescape=true,showspaces=false,showstringspaces=false,xleftmargin=15pt,escapechar=@, morekeywords=[1]{let,fn,val,def,casetype,objtype,metadata,of,*},deletekeywords={for,double},classoffset=0,belowskip=\smallskipamount,
moredelim=**[is][\color{cyan}]{SSTR}{ESTR},
moredelim=**[is][\color{OliveGreen}]{SHTML}{EHTML},
moredelim=**[is][\color{purple}]{SCSS}{ECSS},
moredelim=**[is][\color{brown}]{SSQL}{ESQL},
moredelim=**[is][\color{orange}]{SCOLOR}{ECOLOR},
moredelim=**[is][\color{magenta}]{SPCT}{EPCT}, 
moredelim=**[is][\color{gray}]{SNAT}{ENDNAT}, 
moredelim=**[is][\color{blue}]{SURL}{EURL},
moredelim=**[is][\color{SeaGreen}]{SQT}{EQT},
moredelim=**[is][\color{Periwinkle}]{SGRM}{EGRM},
moredelim=**[is][\color{YellowGreen}]{SID}{EID}
}
\lstloadlanguages{Java,VBScript,XML,HTML}

\let\li\lstinline

%% Save the class definition of \subparagraph
\let\llncssubparagraph\subparagraph
%% Provide a definition to \subparagraph to keep titlesec happy
\let\subparagraph\paragraph
%% Load titlesec
\usepackage[compact]{titlesec}
%% Revert \subparagraph to the llncs definition
\let\subparagraph\llncssubparagraph


\begin{document}

\title{Safely Composable Type-Specific Languages}
\author{~}
\institute{~}
%\author{Cyrus Omar \and Darya Kurilova \and Ligia Nistor \and Benjamin Chung \and\\
%Alex Potanin$^{1}$ \and Jonathan Aldrich}
%\institute{Carnegie Mellon University}\\
%\texttt{\scriptsize \{comar, darya, lnistor, bwchung, aldrich\}@cs.cmu.edu}
%and
%$^{1}$\texttt{\scriptsize alex@ecs.vuw.ac.nz}}

\maketitle

\begin{abstract}
%The abstract should summarize the contents of the paper and should
%contain at least 70 and at most 150 words. It should be written using the
%\emph{abstract} environment.
Programming languages often include specialized notation for common datatypes (e.g. lists) and some also build in support for specific specialized datatypes (e.g. regular expressions), but user-defined types must use general-purpose notations. Frustration with this causes developers to use strings, rather than structured representations, with alarming frequency, leading to correctness, performance, security and usability  issues.
Allowing developers to modularly extend a language with new notations could help address these issues. Unfortunately, prior mechanisms either limit expressiveness or are not safely composable: individually-unambiguous extensions can cause ambiguities when used together. We introduce \emph{type-specific languages} (TSLs):  logic associated with a type that determines how \emph{generic literal forms}, able to contain arbitrary syntax, are parsed and expanded, hygienically, into general-purpose syntax. The TSL for a type is invoked only when a literal appears where a term of that type is expected, guaranteeing non-interference. We give evidence supporting the applicability of  this approach and specify it with a bidirectional type system for an emerging language, Wyvern.

%
%Domain-specific languages can improve ease-of-use, expressiveness and verifiability, but defining and using different DSLs within a single application remains difficult.
%
%We introduce an approach for embedding DSLs in a common host language where the type of a piece of domain-specific code can specify which grammar governs it. Because this grammar is type-specific, but the block is delimited by the host language, we can guarantee that link-time conflicts cannot arise. These grammars can recursively include top-level expressions using special entry tokens that guarantee that the composition of the type-specific language and the host language is also sound. We argue that this approach occupies a previously-unexplored sweet spot providing high expressiveness and ease-of-use while guaranteeing safety. We introduce the design, provide examples, sketch the safety theorems and describe an ongoing implementation of this strategy in the Wyvern programming language.
%
%Domain-specific languages improve ease-of-use, expressiveness and
%verifiability, 
%but defining and using different 
%DSLs within a single application remains difficult.  
%We introduce an approach for embedded DSLs where 1) whitespace delimits DSL-governed blocks, and 2) the parsing and type checking phases occur in tandem so that the expected type of the block determines which domain-specific parser governs that block.
%We argue that this approach occupies a sweet spot, providing   
%high expressiveness and ease-of-use while maintaining safe composability. We introduce the design, provide examples and describe an ongoing implementation of this strategy in the Wyvern programming language. We also discuss how a more conventional keyword-directed strategy for parsing of DSLs can arise as a special case of this type-directed strategy.
%
\keywords{extensible languages; parsing; bidirectional typechecking}
%\newcommand{\vkeyw}[1]{\texttt{\textbf{da
\end{abstract}
% !TEX root = ecoop14.tex
\section{Motivation}
\label{s:intro}
By using a general-purpose abstraction mechanism to encode a data structure, one immediately benefits from a body of established reasoning principles and primitive operations. For example, inductive datatypes can be used to express data structures like lists: intuitively, a list can either be empty, or be broken down into a value (its \emph{head}) and another list (its \emph{tail}). In an ML-like language, this concept is conventionally written:
\begin{Verbatim}[commandchars=\\\{\},numbers=left,formatcom=\fontsize{8pt}{1em}\selectfont]
  \keyw{datatype} 'a list = Nil | Cons of 'a * 'a list
\end{Verbatim}
By encoding lists in this way, we can reason about them by structural induction, construct them by choosing the appropriate case and inspect them by pattern matching. In object-oriented languages, one can encode lists similarly as singly-linked cells, reason about them using a variety of program analysis techniques, construct them using \li{new} and inspect them by traversing the links iteratively. In each case, the programmer only needs to provide an encoding of the structure of lists; the semantics are inherited.

While inheriting semantics can be quite useful, inheriting associated general-purpose syntax can sometimes be a liability. For example, few would claim that writing a list of numbers as a sequence of \li{Cons} cells is convenient:
\begin{lstlisting}
Cons(1, Cons(2, Cons(3, Cons(4, Nil))))
\end{lstlisting}
General-purpose object-oriented notation is similarly inconvenient:
\begin{lstlisting}
new Cons<int>(1, new Cons<int>(2, new Cons<int>(3, new Cons<int>(4))))
\end{lstlisting}
Because lists are a common data structure, many languages provide specialized notation for constructing them, e.g. \li{[1, 2, 3, 4]}. This notation is semantically equivalent to the general-purpose notation shown above, but brings cognitive benefits by drawing attention to the content of the list, rather than the nature of the encoding. More specifically, it is more \emph{terse}, \emph{visible} and \emph{maps more closely} to the intuitive notion of a list, to use terminology from the literature on the cognitive dimensions of notations \cite{green1996usability}.

Although number, string and list literals are nearly ubiquitous features of modern languages, some languages also  provide specialized notation for other common data structures, like maps and sets, data formats, like XML and JSON, query languages, like regular expressions and SQL, and markup languages, like HTML. For example, a language with built-in syntax for HTML and SQL, with type-safe interpolation of host language terms via curly braces, might allow:
\begin{lstlisting}
let webpage : HTML = <html><body><h1>Results for {keyword}</h1>
  <ul id="results">{to_list_items(query(db, 
    SELECT title, snippet FROM products WHERE {keyword} in title)}
</ul></body></html>
\end{lstlisting}
to mean \todo{colors}:
\begin{lstlisting}
let webpage : HTML = TSL1 HTMLElement(Dict(), [BodyElement(Dict(),
  [H1Element(Dict(), [TextNode("Results for " + keyword)]), 
  ULElement(Dict().add("id", "results"), to_list_items(query(db, 
    SelectStmt(["title", "snippet"], "products", 
      [WhereClause(InPredicate(StringLit(keyword), "title"))]))))]]] TSL1
\end{lstlisting}

When a specialized notation is not available, and equivalent general-purpose notation is too cognitively demanding for comfort, developers typically turn to run-time mechanisms to make constructing data structures more convenient. Among the most common strategies in these situations, as we will discuss in Sec. \ref{s:study}, is to simply use a string representation that is parsed at run-time. Developers are frequently  tempted to write the example above as:
\begin{lstlisting}
let webpage : HTML = parse_html("<html><body><h1>Results for "+keyword+"</h1>
  <ul id=\"results\">" + to_string(to_list_items(query(db, parse_sql(
  	"SELECT title, snippet FROM products WHERE '"+keyword+"' in title")))) + 
"</ul></body></html>")
\end{lstlisting}

Though recovering much of the notational convenience of the literal version, it is still more awkward to write, requiring explicit conversions to and from structured representations and escaping when the syntax of the language clashes with the syntax of string literals (line 2). But code like this also causes a number of more serious problems beyond cognitive load. Because parsing occurs at run-time, syntax errors will not be discovered statically, causing potential problems in production scenarios. Run-time parsing also incurs performance overhead, particularly relevant when code like this is executed often (as on a heavily-trafficked website, or in a loop). But the most serious issue with this code is that it is fundamentally insecure: it is vulnerable to cross-site scripting attacks (line 1) and SQL injection attacks (line 3). For example, if a user provided the keyword \li{'; DROP TABLE products --}, the entire product database could be erased. These attack vectors are considered to be two of the most serious security threats on the web today \cite{owasp2013}. Although developers are cautioned to sanitize their input, it can be difficult to verify that this was done correctly throughout a codebase. The most straightforward way to avoid these problems is to use structured representations throughout the codebase, aided by specialized notation like that above \cite{Bravenboer:2007:PIA:1289971.1289975}.

%For example, in languages without regular expression literals, it is quite tedious to write out a regular expression in a structured manner. A simple regular expression like \verb!(\d\d):(\d\d)\w?((am)|(pm))! representing times might be written:
%\begin{lstlisting}
%Seq(Group(Seq(Digit, Digit), Seq(Char(":"), Seq(Group(Seq(Digit, Digit)), 
%  Seq(Optional(Whitespace), Group(Or(Group(Seq(Char("a"), Char("m"))), 
%  Group(Seq(Char("p"), Char("m"))))))))))
%\end{lstlisting}
%Among the most common strategies in these situations, as we will discuss in Sec. \ref{s:motivation}, is to simply use a string representation that is parsed at run-time. 
%\begin{lstlisting}
%rx_from_str("(\\d\\d):(\\d\\d)\\w?((am)|(pm))")
%\end{lstlisting}
%
%For example, in languages without SQL literals, developers can implement a builder pattern:
%\begin{lstlisting}
%new SQLQuery().SELECT("*").FROM("table").WHERE("username").Eq(username)
%\end{lstlisting}
As we will discuss further in Sec. \ref{s:study}, situations like this, where specialized notation is  necessary to maintain strong correctness, performance and security guarantees while avoiding unacceptable cognitive overhead, are quite common. 
Today, implementing new notations within an existing programming language requires the cooperation of the language designer. The primary technical reason for this is that, with conventional parsing strategies, not all notations can safely coexist, so a designer is needed to make choices about which syntactic forms are available and what their semantics are. For example, conventional notations for sets and dictionaries are both delimited by curly braces. When Python introduced set literals, it chose to distinguish them based on whether the literal contained only values, or key-value pairs. But this causes an ambiguity with the syntactic form \verb|{ }| -- should it mean an empty set or an empty dictionary? The designers of Python chose the latter interpretation for backwards compatibility.

Languages that allow users to introduce new syntax from within libraries hold promise, but because there is no longer a designer making decisions about such ambiguities, the burden of resolving them falls to the users of extensions. For example, SugarJ \cite{Erdweg:2011:SLL:2048147.2048199} and other extensible languages generated by Sugar* \cite{Erdweg:2013:FEL:2517208.2517210} allow users to extend the base syntax of the host language (e.g. Java) with new forms, like set and dictionary literals. New forms are imported transitively throughout a program. To resolve syntactic ambiguities that arise, users must manually augment the composed grammar with new rules that allow them to choose the correct interpretation explicitly. This is both difficult to do, requiring an understanding of the underlying parser technology (in Sugar*, GLR parsing using SDF) and increases the cognitive load of using the conflicting notations (e.g. both sets and dictionaries) in the same file. These kinds of conflicts occur in a variety of circumstances: HTML and XML, different variants of SQL, JSON literals and dictionaries, or simply different implementations (``desugarings'') of the same specialized syntax (e.g. two regular expression engines) cause problems.

In this paper, we describe an alternative parsing strategy that avoids these problems by shifting responsibility for parsing certain \emph{generic literal forms} into the typechecker. The typechecker, in turn, defers responsibility to user-defined types, by treating the body of the literal as a term of the   \emph{type-specific language (TSL)} associated with the type it is being checked against. The TSL rewrites this term to use only general-purpose notation. TSLs can contain expressions in the host language. This strategy avoids the problem of ambiguous syntax, because neither the base language nor TSLs are ever extended directly. It also avoids ambiguous semantics -- the meaning of a form like \verb|{ }| can differ depending  on its type, so it is safe to use it for empty sets, dictionaries and other data structures, like JSON literals. This also frees notation from being tied to the variant of a  data structure built into the standard library, which sometimes does not provide the exact semantics that a programmer needs (for example, Python dictionaries do not preserve order, while JSON does):
\begin{lstlisting}
let empty_set : Set = { }
let empty_dict : Dict = { }
let empty_json : JSON = { }
\end{lstlisting}

We develop our work as a variant of an emerging programming language called Wyvern \cite{Nistor:2013:WST:2489828.2489830}. To allow us to focus on the essence of our proposal, the variant of Wyvern we develop here is simpler than the variant previously described: it is purely functional (there are no effects or mutable state) and it does not enforce a uniform access principle for objects (fields can be accessed directly). It also adds inductive datatypes, which we call \emph{case types}, similar to those found in ML. One can refer to our version of the language as \emph{TSL Wyvern} when the variant being discussed is not clear. Our work extends and makes concrete a mechanism sketched out in an earlier short workshop paper \cite{Omar:2013:TWP:2489812.2489815}. We make the following novel contributions:
\begin{enumerate}
\item We specify a more complete layout-sensitive concrete syntax and show how it can be written without the need for a context-sensitive lexer or parser. It now includes a variety of inline literals, provides a full specification for the whitespace-delimited literal form introduced by a forward reference, \li{~}, and provides other forms of forward-referenced blocks.
\item We develop a general mechanism for associating metadata with a type. A TSL is then implemented by associating a parser (of type \li{Parser}) with a type. The parser is responsible for rewriting tokenstreams (of type \li{Tokenstream}) into Wyvern ASTs (of type \li{Exp}). These types are defined in the standard library.
\item This lower-level mechanism is general, but writing a hand-written parser and manipulating syntax trees manually is cognitively demanding. We observe that \emph{grammars} and \emph{quasiquotes} can both be seen as TSLs for parsers and ASTs respectively and discuss how to implement them as such.
\item A na\"ive rewriting strategy would be \emph{unhygienic} -- it could allow for the inadvertent capture of local variables. We show a novel mechanism that ensures hygiene by requiring that the generated AST is closed except for subtrees derived from portions of the user's tokenstream that are interpreted as nested Wyvern expressions. We also show how to explicitly refer to local values available in the parser definition (e.g. helper functions) in a safe way. 
\item We formalize the static semantics and literal parsing rules of TSL Wyvern as a bidirectional type system. By distinguishing locations where an expression synthesizes a type from locations where an expression is being analyzed against a previously synthesized type, we can precisely state where generic literals can appear. This also formalizes the hygiene mechanism.
\item We provide several examples of TSLs throughout the paper, but to examine how broadly applicable the technique is, we conduct a simple corpus analysis, finding that string languages are used ubiquitously.
\end{enumerate}
%\todo{Mention that the examples above look like this in Wyvern?}
%\begin{lstlisting}
%val webpage : HTML = ~
%  <html><body><h1>Results for {keyword}</h1><ul id="results">{
%  	to_list_items(query(db, ~))
%      SELECT title, snippet FROM products WHERE {keyword} in title
%  }</ul></body></html>
%\end{lstlisting}
%
%\begin{lstlisting}
%astOf(e) : ExpAST
%
%casetype ExpAST { 
%  Var of ID
%  Lam of ID * TypeAST * ExpAST
%| ...
%  ParseAsExp(ts, x)
%}
%
%casetype TyAST {
%  Var of ID
%  Arrow of TyAST * TyAST
%  ParseAsTy(ts, x)
%}
%\end{lstlisting}
%https://en.wikipedia.org/wiki/Cognitive_dimensions_of_notations
%Domain-specific languages (DSLs) \cite{fowler2010domain} % have been widely-studied because they 
%allow developers to work with   
%specialized abstractions in a natural manner, and allow for specialized 
%verification and compilation strategies that can improve verifiability and performance. However, for DSLs to reach their full
%potential, it must be simple to define a new DSL, invoke it when needed, and to use multiple
%DSLs within a host general-purpose language (GPL), such that pieces of DSL code
% can interoperate to form a complete application. These intuitions are captured by the following core design criteria that govern our work:
%
%\begin{itemize}
%
%\item \emph {Composability}: It should be possible to use multiple DSLs and a GPL
%%, in addition to a GPL,
%within a single program unit.  %Here 
%Within the file-based paradigm used by most contemporary languages, this 
%means including multiple DSLs within a single file.
% Moreover, it should be possible to embed code written in one DSL
%  within another DSL when appropriate, without requiring them to have specific knowledge of each other. This should be possible without interference between DSLs used in any combination: DSLs should be \emph{safely composable}.
%
%\item \emph{Interoperability}: It should be
%  possible to pass around and operate on values 
%%such as functions or data structures
%  that were defined in foreign DSLs in a reasonably natural manner (that is, without requiring large amounts of ``glue code''). Additional requirements, such as the ability to do so with the safety guarantees provided in the foreign DSL, may also be relevant in many settings. 
%  
%%Moreover,
%  %Minimally, DSLs should be able to define a function or data structure that satisfies an interface specified in
%  %a common interface description language (such as the type system of
%  %a host GPL), code written in another DSL should be able to use those
%  %values according to that interface, without requiring that the client DSL
%  %have knowledge of the details of the provider DSL or \emph{vice versa}.
%\end{itemize}
%
%
%%\item type system: one DSL depends on types defined by another DSL,
%%  can use objects of that type in special ways [this goes beyond the
%%    scope - save for a later paper]
%
%In addition to these fundamental criteria, we believe that to be most useful, a system supporting DSLs should satisfy the following related design criteria:
%
%\begin{itemize}
%\item \emph{Flexibility}: Support a variety of notations and new language mechanisms, with minimal bias;
%\item \emph{Modularity}: Support defining DSLs as combinations of reusable components distributed directly within  libraries;
%\item \emph{Identifiability}: Make it easy for programmers to identify which code is written in which DSL and what it means;
%%\item \emph{Consistency}: Encourage DSLs written in similar styles whenever possible in order to enhance readability and learnability of each DSL;
%\item \emph{Simplicity}: Keep the complexity and cost of both defining and invoking a DSL as low as possible.
%%\item Share conventions between DSLs and a host language, making each DSL easier for programmers to learn, helping programmers to identify which code is written in which DSL, and avoiding unintended conflicts between DSLs. \todo{Avoid conflicts (visual and real), enhance learnability}
%%\item Reuse low-level mechanisms and design decisions from a host language, thereby reducing the cost of defining DSLs.  \todo{Easy to define DSLs}
%\end{itemize}
%
%We are developing a comprehensive language design, \emph{Wyvern}, that we believe can satisfy these design criteria well, and that specifically considers language-internal extensibility from the start. In Wyvern, DSL developers define the run-time semantics of DSL constructs via translation
%into a common host language, as in many other DSL frameworks. The novelty of the proposed extensibility mechanism lies in the ways in which we delimit and determine the \emph{scope} of DSL code:
%\begin{itemize}
%%\item The host language and its DSLs share a tokenization and lexing 
%%  strategy, standardizing conventions for identifiers, operators,
%%  constants, and comments.  This avoids the cost of defining lexing
%%  within each DSL, avoids many kinds of low-level clashes between
%%  languages, and makes the composed language more readable. Note that this does not limit the ability of a DSL to define new keywords and other constructs.
%% 
%\item Wyvern is a \emph{whitespace-delimited} language. Source code that is governed by a DSL, rather than the GPL, occurs in whitespace-delimited blocks and must be indented further than the GPL line introducing it. A decrease in indentation relative to the baseline of the DSL block signals its end. This scheme delimits the scope of each DSL in a clear manner, both to 
%  the programmer and the top-level parser, supporting the principle of identifiability.  It also allows Wyvern to avoid restrictions on a DSL's use of delimiters internally. Because the GPL grammar is not extended in a global manner, it also guarantees that syntactic 
%  conflicts cannot arise at link-time.
%
%%There is also a flexible mechanism for explicit delimiters, for small DSLs, that we will discuss briefly later in the paper.
%
% % \emph{whitespace-delimited} at the top level, according to a particular strategy that we will describe.  Various forms of parentheses 
%  %can also serve as delimiters.  Thus, indentation levels or
%  %parenthesized expressions clearly delimit blocks that are governed by a particular 
%  %language.  This makes the boundaries of each DSL clear to
%  %the programmer and the compiler, enhancing usability and guaranteeing that 
%  %ubtle conflicts cannot arise.
%  
%%\end{itemize}
%
%\item Within this basic syntactic framework, we then propose a novel
%type-directed dispatch mechanism: the \emph{expected type} of an expression, rather than an explicit keyword, 
%%, rather than a keyword, 
%determines which DSL grammar should parse the delimited block that generates that expression. That is, \emph{grammars are associated with types}. We will show that the  more common keyword-directed strategy arises as a special case of this strategy. 
%\end{itemize}
%
%This mechanism allows us to satisfy many of the criteria above, including safe composability, while still being quite expressive, as we will show with examples in the next section. We will continue by describing our approach in more detail (\S\ref{s:approach}), discuss ongoing research directions (\S\ref{s:discussion}), and conclude with related work (\S\ref{s:related}).
%
%

% keep discussion of type-based parsing brief - active typing for parsing (only)
% avoids conflict with Cyrus' paper

%The rest of the paper is organized as follows.  The next section
%illustrates our approach by example, discussing the components of the
%solution in more detail. Section~\ref{s:approach} outlines our
%approach, shows a wider variety of examples and discusses variations
%of our approach.  Our in-progress implementation of the proposal is
%described in Section~\ref{s:implementation}.  Section~\ref{s:related}
%compares to related work, and Section~\ref{s:conclusion} concludes.

% with a discussion of future work.

% !TEX root = ecoop14.tex
\section{Motivating Examples}
\label{s:motivation}
\todo{motivation - Darya}

\subsection{Empirical Study}
\todo{empirical study - Darya}
%
%\begin{figure}
%  \centering
%  \begin{lstlisting}
%val dashboardArchitecture : Architecture = ~
%    external component twitter : Feed
%        location www.twitter.com
%    external component client : Browser
%        connects to servlet
%    component servlet : DashServlet
%        connects to productDB, twitter
%        location intranet.nameless.com
%    component productDB : Database
%        location db.nameless.com
%    policy mainPolicy = ~
%        must salt servlet.login.password
%        connect * -> servlet with HTTPS
%        connect servlet -> productDB with TLS
%  \end{lstlisting}
%  \caption{Wyvern DSL: Architecture Specification}
%  \label{f:dsl-arch}
%\end{figure}
%
%We start with a few examples to illustrate the expressiveness of our approach and the breadth of DSLs we plan for it to support.  The examples are presented in the proposed syntax for Wyvern, a new language being developed by our group that is targeted toward building secure web and mobile applications. We will informally describe each of these examples here, and further explain how such code is parsed in Section \ref{s:approach}.
%
%The first example, shown in Figure~\ref{f:dsl-arch}, describes the overall architecture of a ``hot product dashboard'' application.  The variable \lstinline{dashboardArchitecture} is explicitly ascribed type \lstinline{Architecture}. Rather than explicitly providing a value of this type, we instead use a DSL that makes specifying the component architecture of the application more concise and readable. This DSL code appears in the subsequent whitespace-delimited block and is introduced by a tilde (\lstinline{~}). The example architecture declares several components, some of which are declared \keyw{external} to indicate that they are used by this application but are not part of it directly. Component types are declared after a colon and attributes like connectivity location, are declared after the type (formatted in an indented block for readability). 
%The \keyw{policy} keyword (line 11) introduces a security policy, which constrains the communication protocols that can be used and 
%enforces the secure handling of passwords. A separate type, \lstinline{Policy}, is associated with such policies. Although we could instantiate this type explicitly using a Wyvern expression, we use a DSL for defining policies instead, again within a whitespace-delimited block introduced by a tilde.
%
%\begin{figure}
%  \centering
%  \begin{lstlisting}
%val newProds = productDB.query(~)
%    select twHandle 
%    where introduced - today < 3 months
%val prodTwt = new Feed(newProds)
%return prodTwt.query(~)
%    select *
%    group by followed
%    where count > 1000
%  \end{lstlisting}
%  \caption{Wyvern DSL: Queries}
%  \label{f:dsl-query}
%\end{figure}
%
%Figure~\ref{f:dsl-query} shows how a DSL for database queries can be used from within ordinary Wyvern code.  The example shows code for computing a feed that is derived from tweets about a company's new products.  In this example, the use of a querying DSL is triggered by the use of methods named \lstinline{query} expecting an argument of type \lstinline{DBQuery} (line 1) or \lstinline{FeedQuery} (line 5) respectively.  These types define related but distinct syntax for queries, determined by the expected type of expression where the tilde appears (tildes need not appear only at the ends of lines). Queries are again delimited by indentation. This mechanism is similar to what can be expressed in languages with built-in query syntax like LINQ \cite{mslinq}, but in this case, it is entirely user-defined, rather than built into the language.\begin{figure}
%  \centering
%  \begin{lstlisting}
%serve(page, loc) where 
%  val page = ~ 
%    html:
%      head:
%        title: Hot Products
%        style: {myStylesheet}
%        body:
%          div id="search":
%            {SearchBox("products")}
%          div id="products":
%            {FeedBox(servlet.hotProds())}
%  val loc = ~
%    products.nameless.com
%  \end{lstlisting}
%  \caption{Wyvern DSLs: Presentation and URLs}
%  \label{f:dsl-presentation}
%\end{figure}
%
%Finally, Figure~\ref{f:dsl-presentation} shows a DSL for presenting the hot product application to a web browser, served at a particular URL. Here, two DSLs are used within a single function call. To allow this without introducing ambiguity, the user can use a \keyw{where} clause, similar to that found in Haskell \cite{jones2003haskell}. The presentation DSL is based on HTML and associated with a type, \lstinline{HTMLElement}. It uses an indentation-sensitive syntax and allows integration of Wyvern code of the appropriate type using curly braces. The second DSL simply canonicalizes URL literals into Wyvern values of type \lstinline{URL}.

% !TEX root = ecoop14.tex
\section{Syntax}
\label{s:approach}

\subsection{Concrete Syntax}
We will now describe the concrete syntax of Wyvern declaratively, using the same layout-sensitive formalism that we have introduced for TSL grammars, developed recently by Adams \cite{Adams:2013:PPI:2429069.2429129}. Such a formalism is useful because it allows us to implement  layout-sensitive syntax, like that we've been describing, without relying on context-sensitive lexers or parsers. Most existing layout-sensitive languages (e.g. Python and Haskell) use hand-rolled context-sensitive lexers or parsers (keeping track of, for example, the indentation level using special \li{INDENT} and \li{DEDENT} tokens), but these are more problematic because they cannot be used to generate editor modes, syntax highlighters and other tools automatically. In particular, we will show how the forward references we have described can be correctly encoded without requiring a context-sensitive parser or lexer using this formalism. It is also useful that the TSL for \li{Parser}, above, uses the same parser technology as the host language, so that it can be used to generate quasiquotes.

Wyvern's concrete syntax, with a few minor omissions for concision, is shown in Figure~\ref{f-grammar}. We first review Adams' formalism in some additional detail, then describe some key features of this syntax.

\subsection{Background: Adams' Formalism}
For each terminal and non-terminal in a rule, Adams proposed associating with them a relational operator, such as =, > and $\geq$ to specify the indentation at which those terms need to be with respect to the non-terminal on the left-hand side of the rule. The indentation level of a term can be identified as the column at which the left-most character of that term appears (not simply the first character, in the case of terms that span multiple lines). The meaning of the comparison operators is akin to their mathematical meaning: = means that the term on the right-hand side has to be at exactly the same indentation as the term on the left-hand side; >  means that the term on the right-hand side has to be indented strictly further to the right than the term on the left-hand side; $\geq$ is like >, except the term on the right could also be at the same indentation level as the term on the left-hand side. For example, the production rule of the form \lstinline{A $\rightarrow$ B$^=$ C$^\geq$ D$^>$} approximately reads as: ``Term \lstinline{B} must be at the same indentation level as term \lstinline{A}, term \lstinline{C} may be at the same or a greater indentation level as term \lstinline{A}, and term \lstinline{D} must be at an indentation level greater than term \lstinline{A}'s.'' In particular, if \li{D} contains a \lstinline{NEWLINE} character, the next line must be indented past the position of the left-most character of \lstinline{A} (typically constructed so that it must appear at the beginning of a line). There are no constraints relating \lstinline{D} to \lstinline{B} or \lstinline{C} other than the standard sequencing constraint: the first character of \lstinline{D} must be to further in the file than the others. Using Adam's formalism, the grammars of real-world languages like Python and Haskell can be written declaratively. This formalism can be integrated into LR and LALR parser generators.

% !TEX root = ecoop14.tex

\begin{figure}
\begin{lstlisting}[mathescape]
p $\rightarrow$ 'objtype'$^=$ ID$^>$ NEWLINE$^>$ objdecls$^>$ NEWLINE$^>$ metadatadecl$^>$ NEWLINE$^>$ p$^=$
p $\rightarrow$ 'casetype'$^=$ ID$^>$ NEWLINE$^>$ casedecls$^>$ NEWLINE$^>$ metadatadecl$^>$ NEWLINE$^>$ p$^=$
p $\rightarrow$ e$^=$

metadatadecl $\rightarrow$ 'metadata'$^=$ '='$^>$ e$^>$

e $\rightarrow$ $\overline{\texttt{e}}$$^=$
e $\rightarrow$ $\widetilde{\texttt{e}}$['~']$^=$ NEWLINE$^>$ chars$^>$
e $\rightarrow$ $\widetilde{\texttt{e}}$['new']$^=$ NEWLINE$^>$ d$^>$
e $\rightarrow$ $\widetilde{\texttt{e}}$['case(' $\overline{\texttt{e}}$ ')']$^=$ NEWLINE$^>$ c$^>$

$\overline{\texttt{e}}$ $\rightarrow$ ID$^=$
$\overline{\texttt{e}}$ $\rightarrow$ 'fn'$^=$ ID$^>$ ':'$^>$ type$^>$ '=>'$^>$ $\overline{\texttt{e}}$$^=$
$\overline{\texttt{e}}$ $\rightarrow$ $\overline{\texttt{e}}$$^=$ '('$^>$ $\overline{\texttt{al}}$$^=$ ')'$^>$
$\overline{\texttt{e}}$ $\rightarrow$ 'let'$^=$ ID$^>$ ':'$^>$ type$^>$ '='$^>$ e$^>$ NEWLINE$^>$ $\overline{\texttt{e}}$$^=$
$\overline{\texttt{e}}$ $\rightarrow$ $\overline{\texttt{e}}$$^=$ '.'$^>$ ID$^>$
$\overline{\texttt{e}}$ $\rightarrow$ type$^=$ '.'$^>$ ID$^>$ '('$^>$ $\overline{\texttt{e}}$$^>$ ')'$^>$
$\overline{\texttt{e}}$ $\rightarrow$ $\overline{\texttt{e}}$$^=$ ':'$^>$ type$^>$
$\overline{\texttt{e}}$ $\rightarrow$ 'valAST'$^=$ '('$^>$ $\overline{\texttt{e}}$$^>$ ')'$^>$
$\overline{\texttt{e}}$ $\rightarrow$ type$^=$ '.'$^>$ 'metadata'$^>$
$\overline{\texttt{e}}$ $\rightarrow$ inlinelit$^=$

$\widetilde{\texttt{e}}$[fwd] $\rightarrow$ fwd$^=$
$\widetilde{\texttt{e}}$[fwd] $\rightarrow$ 'fn'$^=$ ID$^>$ ':'$^>$ type$^>$ '=>'$^>$ $\widetilde{\texttt{e}}$[fwd]$^>$
$\widetilde{\texttt{e}}$[fwd] $\rightarrow$ $\widetilde{\texttt{e}}$[fwd]$^=$ '('$^>$ $\overline{\texttt{al}}$$^>$ ')'$^>$
$\widetilde{\texttt{e}}$[fwd] $\rightarrow$ 'let'$^=$ ID$^>$ ':'$^>$ type$^>$ '='$^>$ e$^>$ NEWLINE$^>$ $\widetilde{\texttt{e}}$[fwd]$^=$
$\widetilde{\texttt{e}}$[fwd] $\rightarrow$ $\overline{\texttt{e}}$$^=$ '('$^>$ $\widetilde{\texttt{al}}$[fwd]$^>$ ')'$^>$
$\widetilde{\texttt{e}}$[fwd] $\rightarrow$ $\widetilde{\texttt{e}}$[fwd]$^=$ '.'$^>$ ID$^>$
$\widetilde{\texttt{e}}$[fwd] $\rightarrow$ type$^=$ '.'$^>$ ID$^>$ '('$^>$ $\widetilde{\texttt{e}}$[fwd]$^>$ ')'$^>$
$\widetilde{\texttt{e}}$[fwd] $\rightarrow$ $\widetilde{\texttt{e}}$[fwd]$^=$ ':'$^>$ type$^>$
$\widetilde{\texttt{e}}$[fwd] $\rightarrow$ 'valAST'$^=$ '('$^>$ $\widetilde{\texttt{e}}$[fwd]$^>$ ')'$^>$

d $\rightarrow$ $\varepsilon$
d $\rightarrow$ 'val'$^=$ ID$^>$ ':'$^>$ type$^>$ '='$^>$ e$^>$ NEWLINE$^>$ d$^=$
d $\rightarrow$ 'def'$^=$ ID$^>$ '('$^>$ argsig$^>$ ')'$^>$ ':'$^>$ type$^>$ '='$^>$ e$^>$ NEWLINE$^>$ d$^=$

c $\rightarrow$ ID$^=$ '('$^>$ ID$^>$ ')'$^>$ '=>'$^>$ e$^>$
c $\rightarrow$ ID$^=$ '('$^>$ ID$^>$ ')'$^>$ '=>'$^>$ e$^>$ NEWLINE$^>$ c$^=$

$\overline{\texttt{al}}$ $\rightarrow$ $\varepsilon$ | $\overline{\texttt{al}}_{\texttt{nonempty}}$$^=$
$\overline{\texttt{al}}_{\texttt{nonempty}}$ $\rightarrow$ $\overline{\texttt{e}}$$^=$ | $\overline{\texttt{e}}$$^=$ ','$^>$ $\overline{\texttt{al}}_{\texttt{nonempty}}$$^>$

$\widetilde{\texttt{al}}$[fwd] $\rightarrow$ $\widetilde{\texttt{e}}$[fwd]$^=$
$\widetilde{\texttt{al}}$[fwd] $\rightarrow$ $\widetilde{\texttt{e}}$[fwd]$^=$ ','$^>$ $\overline{\texttt{al}}_{\texttt{nonempty}}$$^>$
$\widetilde{\texttt{al}}$[fwd] $\rightarrow$ $\overline{\texttt{e}}$$^=$ ','$^>$ $\widetilde{\texttt{al}}$[fwd]$^>$

inlinelit $\rightarrow$ chars1['`']$^=$ | chars1[''']$^=$ | chars1['"']$^=$ | ...
inlinelit $\rightarrow$ chars2['{', '}']$^=$ | chars2['<', '>']$^=$ | chars2['[', ']']$^=$ | ...
inlinelit $\rightarrow$ numlit$^=$
\end{lstlisting}
\caption{Concrete Syntax}
\label{f-grammar}
\end{figure}

\begin{figure}
\begin{lstlisting}[mathescape]
`TSL code here, ``inner backticks`` must be doubled`
'TSL code here, ''inner single quotes'' must be doubled'
'TSL code here, ""inner double quotes"" must be doubled'

{TSL code here, {inner braces} must be balanced}
[TSL code here, [inner brackets] must be balanced]
<TSL code here, <inner angle brackets> must be balanced>
\end{lstlisting}
\caption{TSL Delimiters}
\label{f-delims}
\end{figure}


\subsection{Programs}

\begin{figure}[t]
\begin{minipage}{.57\textwidth}
\begin{lstlisting}
objtype T
  val y : HTML
let page : HTML->HTML = fn x : HTML => ~
  :html
    :body
      {x}
page(case(5 : Nat))
  Z(_) => (new : T).y
    val y : HTML = ~
      :h1 Zero!
  S(x) => ~
    :h1 Successor!
\end{lstlisting}
\end{minipage}%
\begin{minipage}{.47\textwidth}
 \centering
\[
\begin{array}{l}
\keyw{objtype} \ T\ \{ \\
  ~~~\keyw{val}\ y : HTML, \\
  ~~~\keyw{metadata} = (\keyw{new}\ \{\}) : Unit\ \}; \\
  (\boldsymbol\lambda page : HTML \rightarrow HTML\ .\\
  ~page(\keyw{case}(\lfloor 5 \rfloor : Nat)\ \{ \\
  ~~~~Z(\_) \Rightarrow ((\keyw{new}\ \{ \\
    ~~~~~~~\keyw{val}\ y : HTML = \lfloor :h1\ Zero! \rfloor\}) : T).y\ | \\
   ~~~~S(x) \Rightarrow \lfloor :h1\ Successor! \rfloor\})) \\
  (\boldsymbol\lambda x : HTML\ .\ \lfloor :html \\
    ~~~~~~~~~~~:body \\
      ~~~~~~~~~~~~~~\{x\} \rfloor)
\end{array}
\]
\end{minipage}
\caption{An example Wyvern program demonstrating forward references}
\label{fig:fwd-ref}
\end{figure}
An example Wyvern program showing several unique syntactic features of TSL Wyvern is shown in Fig. \ref{f-frefs}. The top level of a program (the \lstinline{p} non-terminal) consists of a series of type declarations -- object types using \lstinline{objtype} or case types using \lstinline{casetype} -- followed by an expression, \lstinline{e}. Each type declaration contains associated declarations -- signatures for fields and methods in  \lstinline{objdecls} and case declarations in \lstinline{casedecls} (not shown on the figure). Each also can also include a metadata declaration. Metadata is simply an expression associated with the type, used to store TSL logic (and in future work, other logic). Sequences of top-level declarations use the form \lstinline{p$^=$} to signify that all the succeeding \lstinline{p} terms must begin at the same indentation.\todo{fig ref}
\subsection{Forward Referenced Blocks}
Wyvern makes extensive use of forward referenced blocks to make its syntax clean. In particular, layout-delimited TSLs, the general-purpose introductory form for object types and the elimination form for case types and product types all use forward referenced blocks. Fig. \ref{f-frefs} shows all of these in use (assuming suitable definitions of casetypes \li{Nat} and \li{HTML}, not included). In the grammar, note particularly the rules for \li{let} and that inline literals, even those containing nested expressions with forward references, can be treated as expressions not containing forward references -- \emph{in the initial phase of parsing, before typechecking commences, all literal forms are left unparsed}.

\subsection{Abstract Syntax}
The concrete syntax of a Wyvern program, \li{p}, is parsed to produce a program in the abstract syntax, $\rho$, shown on the left side of Fig. \ref{fig:core2-syntax}. Forward references are internalized. In particular, note that all literal forms are unified into the abstract literal form $\lfloor body \rfloor$, including the layout-delimited form and number literals. The abstract syntax contains a form, $\keyw{fromTS}[\Gamma](e,e)$, that has no analog in the concrete syntax. This will be used internally to ensure hygiene, as we will discuss in the next section.

\begin{figure}[t]
\centering
\[
\begin{array}[t]{lll} 
\rho & \bnfdef & \keyw{objtype}~ t~ \{ \omega, \keyw{metadata}=e \}; \rho \\
     & \bnfalt & \keyw{casetype}~ t~ \{ \chi, \keyw{metadata}=e \}; \rho\\
     & \bnfalt & e
     \\[1ex]
e    & \bnfdef & x \\
     & \bnfalt & \boldsymbol\lambda x{:}\tau . e \\ %
     & \bnfalt & e(e) \\
     & \bnfalt & (e, e) \\
     & \bnfalt & \keyw{case}(e) \{(x, y) \Rightarrow e\}\\
     & \bnfalt & t.C(e) \\
     & \bnfalt & \keyw{case}(e)~\{ c \} \\
     & \bnfalt & \keyw{new}~ \{ d \}\\
     & \bnfalt & e.x \\
     & \bnfalt & e : \tau\\
     & \bnfalt & \keyw{valAST}(e) \\
     & \bnfalt & t.\keyw{metadata}\\
     & \bnfalt & \lfloor literal \rfloor \\
     & \bnfalt & \keyw{fromTS}[\Gamma](e, e)
\\[1ex]	
c    & \bnfdef & C(x) \Rightarrow e\\
     & \bnfalt & c \bnfalt c
	 \\[1ex]
d   & \bnfdef & \varepsilon \\
     & \bnfalt & \keyw{val}~ f:\tau = e;~d \\
     & \bnfalt & \keyw{def}~ m:\tau = e;~d
\\[1ex] 
\end{array}
\begin{array}[t]{lll}
~~~
\end{array}
\begin{array}[t]{lll}


\hat\rho & \bnfdef & \keyw{objtype}~ t~ \{ \omega, \keyw{metadata}=\hat e \}; \hat\rho \\
     & \bnfalt & \keyw{casetype}~ t~ \{ \chi, \keyw{metadata}=\hat e \}; \hat\rho\\
     & \bnfalt & \hat e
     \\[1ex]
\hat{e}    & \bnfdef & x \\
     & \bnfalt & \boldsymbol\lambda x{:}\tau . \hat{e} \\ %
     & \bnfalt & \hat{e}(\hat{e}) \\
     & \bnfalt & \cdots \\
     & \bnfalt & t.\keyw{metadata} 
\\[1ex]
\hat c    & \bnfdef & ...
	 \\[1ex]
\hat d   & \bnfdef & ... 
\\[1ex] 
\chi & \bnfdef & C~\keyw{of}~\tau\\
     & \bnfalt & \chi \bnfalt \chi 
\\[1ex]
\omega &\bnfdef & \varepsilon \\  
         & \bnfalt & \keyw{val}~ f:\tau;~\omega\\
         & \bnfalt & \keyw{def}~ m:\tau;~\omega 
\\[1ex]
\tau & \bnfdef & t\\
     & \bnfalt & \tau \rightarrow \tau \\
     & \bnfalt & \tau \times \tau 
\\[1ex]
\Gamma & \bnfdef & \emptyset \bnfalt \Gamma, x:\tau
\\[1ex]
\Delta & \bnfdef & \emptyset \bnfalt \Delta, t:\{\chi, \hat e:\tau\} \bnfalt \Delta, t:\{\omega, \hat e:\tau\}
\\[1ex]

\end{array}
\]
\caption{Abstract Syntax}
\label{fig:core2-syntax}
\end{figure}

\section{Bidirectional Typechecking and Literal Rewriting}
We will now specify a type system for the abstract syntax in Fig. \ref{fig:core2-syntax}. Conventional type systems are specified using a typechecking judgement like $\Delta; \Gamma \vdash e : \tau$, where the variable context, $\Gamma$, tracks the types of variables, and the type context, $\Delta$, tracks types and their signatures. However, this conventional formulation does not separately consider how, when deriving this judgement, it will be considered algorithmically -- will a type be provided, so that we simply need to check $e$ against it, or do we need to synthesize a type for $e$? For our system, this distinction is crucial: a generic literal can only be used in the first situation. 

\emph{Bidirectional type systems}, as presented by Lovas and Pfenning \cite{Lovas08abidirectional}, make this distinction clear by specifying the type system instead using two simultaneously defined judgements: one for expressions that can \emph{synthesize} a type based on the surrounding context (e.g. variables and elimination forms), and another for expressions for which we know what type to \emph{check} or \emph{analyze} the term against (e.g. generic literals and some introductory forms). 
Our work builds upon this work, making the following core additions: the tye context $\Delta$ now tracks the metadata in addition to type signatures, and as we typecheck, we need to also perform literal rewriting by calling the parser associated with the type that a literal is being analyzed against, typechecking the AST it produces and ensuring that hygiene is maintained.

The judgement 
\fbox{$\Delta; \Gamma \vdash e\Rightarrow \tau \leadsto \hat{e}$} 
means that from the type context $\Delta$ and the variable context $\Gamma$ we synthesize the type $\tau$ for $e$. The  expression $e$ possibly containing $\lfloor literal \rfloor$ forms (and the \keyw{fromTS} form, which we will discuss shortly) is rewritten into the expression $\hat{e}$ without these forms.
The judgement 
\fbox{$\Delta; \Gamma \vdash e \Leftarrow \tau \leadsto \hat{e}$} 
means that we check $e$ against the type $\tau$ and the expression $e$ is rewritten into the expression $\hat{e}$. The forms of $\Gamma$ and $\Delta$ are given in Fig. \ref{fig:core2-syntax}. Note that $\Delta$ carries the type's signature as well as the rewritten form of the metadata. The rules for these judgements, as well as key rules for several auxiliary judgements that are needed in their premises, are given in Figs. 8-12. 

These rules assume that a collection of built-in types are included by default at the top of programs (e.g. \li{Unit}, \li{Parser}, \li{Exp} already mentioned, and a few others), captured by an intial type context $\Delta_0$. We show the concrete syntax for the two key ones in Fig. 7. The static semantics and the dynamic semantics (defined for $\hat{e}$ only) are that of a conventional functional language with functions, inductive datatypes, products and records  with the addition of a few new forms. The key new dynamic semantics rules are described in Fig. 13. We will now describe how some of the novel rules that support TSLs work below. We refer the reader to \cite{Lovas08abidirectional} and excellent textbooks on type theory, e.g. \cite{pfpl} and \cite{tapl}, for the remainder.

\begin{figure}[t]
\begin{lstlisting}
objtype Parser                                casetype Exp = 
  def parse(ts : TokenStream) : Exp             Var of ID
  metadata = new                              | Lam of ID * Type * Exp
    val parser : Parser = new                 | App of Exp * Exp
      val parse(ts : TokenStream) :           ... | FromTS of Exp * Exp
         Exp * TokenStream =                  | Literal of TokenStream
        (* ... parser generator based on      | Error of ErrorMessage
           Adams' formalism here ... *)       metadata = (* quasiquotes *)
\end{lstlisting}
\caption{Two of the built-in types included in $\Delta_0$.}
\label{fig:typeParser}
\end{figure}
%\subsection{Bidirectional Type System}
%Our type-checking is syntax-directed and it requires terms to be fully annotated with types where necessary. A commonly used approach for making the syntax more compact is bidirectional type systems, . They introduce two mutually recursive judgements: one for expressions that have enough information in the context to synthesize a type, and one for expressions for which we know what type to expect, thus only needing to check against that type. Unique types can be determined for synthesis expressions, while analytical expressions have to be verified to have the right types. We chose to use bidirectional type systems in our formalism because they leverage the simplicity of syntax-directed type-checking while not needing to carry much additional type information.
%
%In conventional bidirectional type systems, for constructors of a type one can propagate the type information $\tau$
%into the term $e$, which means it should be used in the analysis
%judgment $e \Leftarrow \tau$. When constructing a type, we do not have information about it and it is intuitive to use the analysis judgement, which is weaker than the synthesis judgement. On the other hand, destructors generate a result of a smaller type from a component of larger type and can be used for synthesis, propagating type information away from the term as in the synthesis judgement $e \Rightarrow \tau$. Our static semantics rules follow this conventional way of reasoning about constructors and destructors.

\subsection{Defining a TSL Manually}
In the example in Fig. \ref{f-htmltype}, we showed a TSL being defined using a parser generator based on Adams' formalism. A parser generator is itself merely a TSL for a parser, and a parser is the fundamental construct that becomes associated with a type to form a TSL. The signature for the built-in type \li{Parser} is shown in Fig. 7. It is an object type with a \li{parse} function taking in a \li{TokenStream} and producing an AST of a Wyvern expression, which is of type \li{Exp}. This built-in type is shown also in Fig. 7. Note that there is a form for each form in the abstract syntax, $e$, as well as an \li{Error} form for indicating error messages (in the theory, nothing is done with these messages). As previously mentioned, quasiquotes are merely a TSL that allows one to construct the abstract syntax, represented as this case type, using concrete syntax, with the addition of an unquote mechanism.

The \verb|parse| function for a type $t$ is called when checking a literal form against that type. This is seen in the key rule of our statics: \textit{T-literal}, in Fig. 9. The premises of these rules operate as follows:
\begin{enumerate}
\item A well-typed, rewritten parser object is extracted from the type's metadata. This is the step where the parser generator rewrites a grammar to a parse method, recursively using the TSL mechanism itself.
\item A tokenstream, of type \li{TokenStream}, is generated from the body of the literal. This type is an object that that allows the reading of tokens, as well as an additional method discussed in the next section for parsing the stream as a Wyvern expression.
\item The \li|parse| method is called with this extracted tokenstream to produce a syntax tree and a remaining tokenstream.
\item The syntax tree, $\hat{e}'$ is \emph{dereified} into its corresponding term, $e$ (the hat is gone because the generated syntax tree might itself use TSLs). The judgement for this is indexed by the variable context $\Gamma$ because that is needed to dereify \li{FromTS}, described below.
\item The dereified term is then recursively typechecked against the same type and rewritten, consistent with the semantics of TSLs as we have been describing them -- they must produce a term of the type they are being checked against. It is checked under the empty variable context to ensure hygiene (below).
\item The TSL must consume the entire token stream, so this is checked.
\end{enumerate}
\subsection{Hygiene}
A concern with any term rewriting system is \emph{hygiene} -- how should variables in the generated AST be bound? In particular, if the rewriting system generates an \emph{open term}, then it is making assumptions about the names of variables in scope at the site where the TSL is being used, which is incorrect - those variables should only be identifiable up to alpha renaming. Only the \emph{user} of a TSL knows which variables are in scope. Strict hygiene would simply reject all open terms, but this would prevent even nested Wyvern expressions which the user provided from referring to local variables.

The solution to being able to capture variables in portions of the tokenstream that are parsed as Wyvern only is to add a new term to the abstract syntax that has no corresponding form in the concrete syntax: $\keyw{fromTS}[\Gamma](ts, tok)$. This means: "this is a term that was parsed from the tokenstream $ts$ terminated by the token $tok$; it's surrounding context was $\Gamma$". It can be generated by calling \li{ts.as_wyv_exp(tok)}. When we attempt to typecheck it, which will be starting from an empty context (above), we add in the bindings available in Gamma (to any that were introduced by the TSL, such as the TSL for \li{Parser} does with named non-terminals). This can be seen in the rule \textit{T-fromTS}, which calls the parser for concrete syntax internally to generate the abstract syntax so that it can be checked (but does not expose this to the user).

For this mechanism to truly ensure hygiene, one must not be able to sidestep it by generating a tokenstream manually: expressions from a tokenstream must have actually come from the use site. This is ensured by preventing users from checking \li{new} against \li{TokenStream} in the statics. This creates a small subtlety in the safety theorem, discussed in the next section.

A second facet of hygiene is being able to refer to local variables available within the parser itself, such as local helper functions, for convenience. This can be done using the primitive $\keyw{valAST}(e)$. The semantics for this, shown in Fig. 13, first evaluate $e$ to a value, then \emph{reify} this value to an AST. This can be used to ``bake in'' a value known at compile time into the generated code safely. The rules for reification, used here, and dereification, used in the literal rule described above, are essentially dual, as seen in Figs. 11 and 12.

\subsection{Safety}
The semantics we have defined constitute a type safe language.

We begin with a lemma that shows that the statics for $e$ and $\hat{e}$ are consistent. This is nearly obvious, but for the slight difference in how \li{TokenStream} is treated. 
\begin{lemma}[Forward Consistency]
If $\Delta; \Gamma \vdash e \Leftarrow \tau \leadsto \hat{e}$ then $\Delta; \Gamma \vdash \hat{e} : \tau$. 
\end{lemma}
\begin{proof}
Here, $\hat{e}$ is more permissive (Fig. 10), so forward consistency is easily seen by observing that for each form shared by both $e$ and $\hat{e}$, the bidirectional system simply rewrites to the corresponding form of $\hat{e}$ recursively. Thus, these cases are direct applications of the IH. The only remaining cases are the literal form, where the lemma follows by applying the IH to the inductive premises.
% TODO: err, maybe? I think fromTS might be a bit more complex
\end{proof}

We then need to show type safety of $\hat{e}$. Because it doesn't contain any non-standard terms other than $\keyw{valAST}$ and $t.\keyw{metadata}$, both of which have straightforward semantics (Fig. 13), this follows by the standard progress and preservation techniques with the addition of a simple well-formedness check on $\Delta$, because metadata needs to be well-typed (not shown).

\begin{lemma}[Preservation]
If $\Delta; \emptyset \vdash \hat{e} : \tau$ and $\vdash \Delta~\texttt{ok}$ and $\hat{e} \xmapsto[\Delta]{} \hat{e}'$ then $\Delta; \emptyset \vdash \hat{e}' : \tau$.
\end{lemma}
\begin{lemma}[Progress]
If If $\Delta; \emptyset \vdash \hat{e} : \tau$ and $\vdash \Delta~\texttt{ok}$ then either $\hat{e}~\texttt{val}$ or $\hat{e} \xmapsto[\Delta]{} \hat{e}'$.
\end{lemma}
These lemmas and judgements can be lifted to the level of programs by applying them to the top-level expression the program contains (simple, not shown). As a result, we have type safety: well-typed programs cannot ``get stuck''.
\begin{theorem}[Type Safety]
If $\Delta_0; \emptyset \vdash \rho : \tau \leadsto \hat{\rho}$ then $\Delta_0; \emptyset \vdash \hat{\rho} : \tau$ and either $\hat{\rho}~\texttt{val}$ or $\hat{\rho} \xmapsto[\Delta_0]{} \hat{\rho}'$ such that $\Delta_0; \emptyset \vdash \hat{\rho}' : \tau$.
\end{theorem}

Because we are executing user-defined parsers during typechecking, we do not have a straightforward statement of decidability (i.e. termination) of typechecking. The parser might not terminate, or it might generate a term that contains itself. Non-decidability is strictly due to these parsers. Termination of the parsing process is well studied (e.g. \cite{DBLP:conf/sle/KrishnanW12}) and the techniques can be applied to user-defined parsers to increase confidence in termination.

% !TEX root = ecoop14.tex
\section{Statics}

\todo{This section on Statics is going to go and simply be integrated into approach in the right places, right? Right now it is just a list of figures and some text descrbining parts of them, not a coherent section as such, right?}

We present the abstract syntax of our system in Figure \ref{fig:core2-syntax}. A program $\rho$ is composed of a series of object type \keyw{objtype} and sum type \keyw{casetype} declarations, followed by expressions $e$. An object type is made of declarations of values \keyw{val}, methods \keyw{def} and \keyw{metadata}. A sum type is made of an enumeration of cases of the form $C$ \keyw{of} $\tau$, where $C$ is the name of the constructor and $\tau$ is the type of the expression constructed in this case, followed by metadata.

The metadata of a type $t$ (either an \keyw{objtype} or a \keyw{casetype}) can contain arbitrary data, for eg. documentation, but it will necessarily contain a $parser$ field of type $Parser$. The $parser$ field has a $parse$ method that takes as argument a stream of tokens and generates an abstract syntax tree expression. This $parse$ method is used for parsing new TSLs of type $t$ defined by the user. 

We differentiate between expressions that might contain a TSL expression and those that definitely do not contain a TSL expression by superscripting the latter with the symbol $\hat{}$. Thus we have two versions (one without $\hat{}$ and one with $\hat{}$) for programs $\rho$,  expressions $e$, cases $c$ of the sum types, and declarations $d$ of fields and methods. 

An expression can be a variable, a function, an application of a function to an expression, a pair of expressions, a case analysis on pairs, the constructor of a case type, the destructor of a case type, a new expression declaring fields and methods, the invocation $e.x$ of a field or a method. We do not have $e.m$ and $e.f$ in the abstract syntax because the parser cannot differentiate between the two, only the type checker can do that. An expression can also be an expression with a type ascribed to it, the abstract syntax tree of an expression, a metaobject of a type, a TSL literal or an expression obtained from a token stream. 








\begin{figure}
\centering
\[
\begin{array}[t]{lll} 
\rho & \bnfdef & \keyw{objtype}~ t~ \{ \omega, \keyw{metadata}=e \}; \rho \\
     & \bnfalt & \keyw{casetype}~ t~ \{ \chi, \keyw{metadata}=e \}; \rho\\
     & \bnfalt & e
     \\[1ex]
e    & \bnfdef & x \\
     & \bnfalt & \boldsymbol\lambda x{:}\tau . e \\ %
     & \bnfalt & e(e) \\
     & \bnfalt & (e, e) \\
     & \bnfalt & \keyw{case}(e) \{(x, y) \Rightarrow e\}\\
     & \bnfalt & t.C(e) \\
     & \bnfalt & \keyw{case}(e)~\{ c \} \\
     & \bnfalt & \keyw{new}~ \{ d \}\\
     & \bnfalt & e.x \\
     & \bnfalt & e : \tau\\
     & \bnfalt & \keyw{valAST}(e) \\
     & \bnfalt & t.\keyw{metaobject}\\
     & \bnfalt & \lfloor literal \rfloor \\
     & \bnfalt & \keyw{fromTS}[\Gamma](e, e)
\\[1ex]	
c    & \bnfdef & C(x) \Rightarrow e\\
     & \bnfalt & c \bnfalt c
	 \\[1ex]
d   & \bnfdef & \varepsilon \\
     & \bnfalt & \keyw{val}~ f:\tau = e;~d \\
     & \bnfalt & \keyw{def}~ m:\tau = e;~d
\\[1ex] 
\end{array}
\begin{array}[t]{lll}
~~~
\end{array}
\begin{array}[t]{lll}


\hat\rho & \bnfdef & \keyw{objtype}~ t~ \{ \omega, \keyw{metadata}=\hat e \}; \hat\rho \\
     & \bnfalt & \keyw{casetype}~ t~ \{ \chi, \keyw{metadata}=\hat e \}; \hat\rho\\
     & \bnfalt & \hat e
     \\[1ex]
\hat{e}    & \bnfdef & x \\
     & \bnfalt & \boldsymbol\lambda x{:}\tau . \hat{e} \\ %
     & \bnfalt & \hat{e}(\hat{e}) \\
     & \bnfalt & \cdots \\
     & \bnfalt & t.\keyw{metaobject} 
\\[1ex]
\hat c    & \bnfdef & ...
	 \\[1ex]
\hat d   & \bnfdef & ... 
\\[1ex] 
\chi & \bnfdef & C~\keyw{of}~\tau\\
     & \bnfalt & \chi \bnfalt \chi 
\\[1ex]
\omega &\bnfdef & \varepsilon \\  
         & \bnfalt & \keyw{val}~ f:\tau;~\omega\\
         & \bnfalt & \keyw{def}~ m:\tau;~\omega 
\\[1ex]
\tau & \bnfdef & t\\
     & \bnfalt & \tau \rightarrow \tau \\
     & \bnfalt & \tau \times \tau 
\\[1ex]
\Gamma & \bnfdef & \emptyset \bnfalt \Gamma, x:\tau
\\[1ex]
\Delta & \bnfdef & \emptyset \bnfalt \Delta, t:\{\chi, e:\tau\} \bnfalt \Delta, t:\{\omega, e:\tau\}
\\[1ex]

\end{array}
\]
\caption{Abstract Syntax}
\label{fig:core2-syntax}
\end{figure}


\begin{figure}
\centering
\[
\begin{array}{ll}
\keyw{casetype}\ & Exp=\\
& \ \ \ Var\ of\ ID\\
& \bnfalt \ Lam\ of\ ID\ *\ Ty\ *\ Exp\\
& \bnfalt \ App\ of\ Exp\ *\ Exp\\
& \cdots\\
& \bnfalt \ FromTokenStream\ of\ Exp\ *\ Exp\\
& \bnfalt \ Error\\
\\
\keyw{casetype}\ & Ty=\\
& \ \ \ Var\ of\ ID\\
& \bnfalt \ Arrow\ of\ Ty*Ty\\
\\ 
\end{array}
\]
\caption{Syntax Trees of Expressions and Types}
\end{figure}





\begin{figure}
\centering
\[
\begin{array}{c}

\infer[\textit{RT-objtype}]
          {\renewcommand{\arraystretch}{1}
	    \begin{array}{r}
	    \Delta; \Gamma \vdash  \keyw{objtype}~ t~=\{{\omega}, \keyw{metaobject}=e\}; \rho: \tau'\leadsto\\
            \keyw{objtype}~ t~=\{{\omega}, \keyw{metaobject}=\hat{e}\}; \hat{\rho}
            \end{array}
       }
	  {\Delta \vdash \omega & \Delta; \Gamma \vdash e \Rightarrow \tau \leadsto \hat{e} & \Delta, t:\{\omega, \hat e:\tau\}; \Gamma \vdash \rho :\tau'\leadsto \hat{\rho} }
	   \\[3ex] 


\infer[\textit{RT-casetype}]
          {\renewcommand{\arraystretch}{1}
	    \begin{array}{r}
	    \Delta; \Gamma \vdash  \keyw{casetype}~ t~=\{\chi, \keyw{metaobject}=e\}; \rho :\tau' \leadsto \\
            \keyw{casetype}~ t~=\{\chi, \keyw{metaobject}=\hat{e}\};\hat{\rho}
            \end{array}
       }
	  {\Delta \vdash \chi & \Delta; \Gamma \vdash e \Rightarrow \tau \leadsto \hat{e} & \Delta, t:\{\chi, \hat e:\tau\}; \Gamma \vdash \rho :\tau'\leadsto \hat{\rho} }
	   \\[3ex] 


\infer[\textit{RT-e}]
	{\Delta; \Gamma \vdash  e:\tau \leadsto \hat{e}} 
	{\Delta; \Gamma \vdash e \Rightarrow \tau \leadsto \hat{e}}\\[3ex]

\infer[\textit{C-decl}]
	{\Delta; \Gamma \vdash  C~\keyw{of}~\tau} 
	{\Delta \vdash \tau   }\\[3ex]

\infer[\textit{C-decls}]
	{\Delta; \Gamma \vdash  \chi_1 \bnfalt \chi_2} 
	{\Delta; \Gamma \vdash \chi_1 & \Delta; \Gamma \vdash \chi_2 & \text{dom}(\chi_1) \intersect \text{dom}(\chi_2) = \emptyset}\\[3ex]

\infer[\textit{O-val}]
	{\Delta; \Gamma \vdash \keyw{val}~ f:\tau \ \texttt{ok} }
	{\Delta \vdash \tau} \\[3ex]
	
\infer[\textit{O-def}]
	{\Delta; \Gamma \vdash \keyw{def}~ m:\tau \ \texttt{ok} }
	{\Delta \vdash \tau } \\[3ex]

\infer[\textit{O-defs}]
	{\Delta; \Gamma \vdash \omega_1\ \omega_2  }
	{\Delta \vdash \omega_1 & \Delta \vdash \omega_2 & \text{dom}(\omega_1) \intersect \text{dom}(\omega_2) = \emptyset } \\[3ex]

\infer[\textit{Syn2Check}]
	{\Delta; \Gamma \vdash  e \Leftarrow \tau \leadsto \hat{e}} 
	{\Delta;\Gamma \vdash e \Rightarrow \tau \leadsto \hat{e}   }\\[3ex]
	
\infer[\textit{T-varx}]
	{\Delta,\Gamma \vdash x\Rightarrow\tau } 
	{x:\tau \in \Gamma }\\[3ex]

\infer[\textit{T-abs}]
	{\Delta; \Gamma \vdash  \boldsymbol\lambda x{:}\tau . e \Leftarrow \tau \rightarrow \tau_1 \leadsto \boldsymbol\lambda x{:}\tau .\hat{e}} 
	{\Delta; \Gamma, x:\tau \vdash e\Leftarrow \tau_1 \leadsto \hat{e}  & \Delta\vdash \tau}\\[3ex]

\infer[\textit{T-appl}]
	{\Delta; \Gamma \vdash  e(e_1) \Rightarrow \tau_2  \leadsto \hat{e}(\hat{e}_1) } 
	{\Delta; \Gamma \vdash e \Rightarrow \tau_1 \rightarrow \tau_2  \leadsto \hat{e}  & \Gamma \vdash e_1 \Leftarrow \tau_1 \leadsto \hat{e}_1 }\\[3ex]

\infer[\textit{T-introcase}]
	{\Delta; \Gamma \vdash  t.C(e) \Rightarrow t  \leadsto t.C(\hat{e}) } 
	{t:\{\chi, e_0:\tau\} \in \Delta & C\ \keyw{of}\ \tau' \in \chi &\Delta; \Gamma \vdash e \Leftarrow \tau'  \leadsto \hat{e}}\\[3ex]

\infer[\textit{T-elimcase}]
	{\Delta; \Gamma \vdash  \keyw{case}~(e)~\{ c \} \Rightarrow \tau'  \leadsto \keyw{case}~(\hat{e})~\{ c \} } 
	{\Delta; \Gamma \vdash e \Rightarrow t  \leadsto \hat{e}  & t:\{ \chi,e_0:\tau\} \in \Delta & c:\chi \Rightarrow \tau'}\\[3ex]

\infer[\textit{T-casehelper1}]
	{\Delta; \Gamma \vdash  C(\chi)\Rightarrow e : C\ \keyw{of}\ \tau \Rightarrow \tau' \leadsto C(\chi)\Rightarrow \hat{e} : C\ \keyw{of}\ \tau} 
	{\Delta; \Gamma, x:\tau \vdash e \Rightarrow \tau' \leadsto \hat{e}}\\[3ex]

\infer[\textit{T-casehelper2}]
	{\Delta; \Gamma \vdash  c_1 \bnfalt c_2: \chi_1 \bnfalt \chi_2 \Rightarrow \tau' } 
	{\Delta; \Gamma \vdash c_1:\chi_1 \Rightarrow \tau' & \Delta; \Gamma \vdash c_2:\chi_2 \Rightarrow \tau'}\\[3ex]


\end{array}
\]
\caption{Static Semantics Rules}
\end{figure}

\begin{figure}
\centering
\[
\begin{array}{c}

\infer[\textit{T-new}]
	{\Delta; \Gamma \vdash \keyw{new}\ \{ d \} \Leftarrow  t \leadsto \keyw{new}\ \{\hat d\}}
	{ t:\{\omega, \hat e:\tau\} \in \Delta & \Delta;\Gamma \vdash d \Leftarrow \omega \leadsto \hat d & t\neq TokenStream} \\[3ex]

\infer[\textit{DT-val}]
	{\Delta; \Gamma \vdash \keyw{val}~ f:\tau = e \Leftarrow \keyw{val}~ f:\tau  \leadsto \keyw{val}~ f:\tau = \hat{e}}
	{\Delta \vdash \tau &\Delta; \Gamma \vdash e \Leftarrow \tau \leadsto \hat{e} } \\[3ex]
	
\infer[\textit{DT-def}]
	{\Delta; \Gamma \vdash \keyw{def}~ m:\tau = e \Leftarrow \keyw{def}~ m:\tau \leadsto \keyw{def}~ m:\tau = \hat{e} }
	{\Delta \vdash \tau  & \Delta; \Gamma \vdash e  \Leftarrow \tau \leadsto \hat{e} } \\[3ex]

	
\infer[\textit{DT-defs}]
	{\Delta; \Gamma \vdash d_1\ d_2 \Leftarrow \omega_1\ \omega_2 }
	{\Delta; \Gamma \vdash d_1 \Leftarrow \omega_1 &  \Delta; \Gamma \vdash d_2 \Leftarrow \omega_2 } \\[3ex]


\infer[\textit{T-field}]
	{\Delta; \Gamma \vdash  e.f \Rightarrow \tau' \leadsto \hat{e}.f} 
	{\Delta; \Gamma \vdash e \Rightarrow t \leadsto \hat{e} & t:\{\omega, e_0:\tau\}\in \Delta & \keyw{val}\ f:\tau' \in \omega  }\\[3ex]

 
\infer[\textit{T-def }]
	{\Delta; \Gamma \vdash  e.m \Rightarrow \tau' \leadsto \hat{e}.m} 
	{\Delta; \Gamma \vdash e \Rightarrow t \leadsto \hat{e} & t: \{\omega, e_0:\tau\} \in \Delta & \keyw{def}\ m:\tau' \in \omega }\\[3ex]

\infer[\textit{T-ascribe}]
	{\Delta; \Gamma  \vdash  e:\tau \Rightarrow \tau \leadsto \hat{e}:\tau}
	{\Delta \vdash \tau & \Delta; \Gamma \vdash e \Leftarrow \tau \leadsto \hat{e} } \\[3ex]

\infer[\textit{T-valAST}]
        {\Delta; \Gamma \vdash \keyw{valAST}(e) \Rightarrow Exp \leadsto \keyw{valAST}(\hat{e}) }
	{\Delta; \Gamma \vdash e \Rightarrow \tau \leadsto \hat{e}} \\[3ex]

\infer[\textit{T-metaobject}]
        {\Delta; \Gamma \vdash t.\keyw{metaobject} \Rightarrow \tau   }
	{t:\{\_, e_0:\tau\} \in \Delta} \\[3ex]


\infer[\textit{T-fromTS}]
	  {\Delta; \Gamma' \vdash \keyw{fromTS}[\Gamma](e_1,e_2) \Leftarrow \tau \leadsto \hat{e} }
	  {\renewcommand{\arraystretch}{1}
	    \begin{array}{r}
	    \Delta;\Gamma' \vdash e_1:TokenStream ~~~~~~ \Delta;\Gamma' \vdash e_2:Token\\
            \texttt{parseConcrete(}e_1,e_2\texttt{)}\ \texttt{is}\ e ~~~~~~\Delta; \Gamma', \Gamma \vdash e \Leftarrow \tau \leadsto \hat{e}
            \end{array}
       } \\[3ex]  

\infer[\textit{T-literal}]
	  {\Delta; \Gamma \vdash \lfloor literal \rfloor \Leftarrow t \leadsto \hat{e} }
	  {\renewcommand{\arraystretch}{1}
	    \begin{array}{r}
	    \Delta;\Gamma \vdash t.\keyw{metaobject}.parser\Leftarrow Parser \leadsto \hat{e}_p ~~~~~ \texttt{TokenStream(}\lfloor literal \rfloor \texttt{)}\ \texttt{is}\ \hat{e}_{ts}\\
            \hat{e}_p.parse(\hat{e}_{ts}, Token.EOS(())) \Downarrow_{\Delta} (\hat{e}', \hat e_{ts}') ~~~~~  e \triangleleft_\Gamma \hat{e}'~~~~~ \Delta;\Gamma\vdash e\Leftarrow t \leadsto \hat{e} ~~~~~ \hat e_{ts}'\ \texttt{empty}
            \end{array}
       } \\[3ex]   
\end{array}
\]
\caption{Static Semantics Rules 2}
\end{figure}

\begin{figure}
\centering
\[
\begin{array}{c}

\infer[\textit{Dyn-Meta}]
	{t.\keyw{metaobject} \xmapsto[\Delta]{} e} 
	{t:\{\_,e:\tau \} \in \Delta}\\[3ex]

\infer[\textit{Dyn-valAST1}]
	{\keyw{valAST}(\hat{e}) \xmapsto[\Delta]{} \keyw{valAST}(\hat{e}') } 
	{\hat{e} \xmapsto[\Delta]{} \hat{e}'}\\[3ex]

\infer[\textit{Dyn-valAST2}]
	{\keyw{valAST}(\hat{e}) \xmapsto[\Delta]{} \hat{e}' } 
	{\hat{e}\ \text{val} &\hat{e} \triangleright \hat{e}' }\\[3ex]




\end{array}
\]
\caption{Dynamic Semantics Rules}
\end{figure}




\begin{figure}
\centering
\begin{minipage}{.5\textwidth}
  \centering
   \[
\begin{array}{c}

\infer[\textit{DExp-Var}]
	{ x \triangleleft Exp.Var(\hat{e})   }
	{ ID(x)\ \text{is}\ \hat{e}} \\[3ex]

\infer[\textit{DExp-Lam}]
	{ \boldsymbol\lambda x{:}\tau . e'_1 \triangleleft Exp.Lam( \hat{e}, \tau, \hat{e}_1 )  }
	{ID(x)\ \text{is}\ \hat{e} & e'_1 \triangleleft \hat{e}_1  } \\[3ex]

\infer[\textit{DExp-App}]
	{ e'_1(e'_2)  \triangleleft Exp.App(\hat{e}_1,\hat{e}_2) }
	{ e'_1 \triangleleft \hat{e}_1  & e'_2 \triangleleft \hat{e}_2   } \\[3ex]

\infer[\textit{DExp-Literal}]
	{ \lfloor literal \rfloor \triangleleft Exp.Literal( \hat{e}_{ts} )  }
	{ \text{literal of}\ \hat{e}_{ts}\ \text{is}\ \lfloor literal \rfloor  } \\[3ex]

\infer[\textit{DTy-Var}]
	{ t \triangleleft Ty.Var(\hat{e})   }
	{ ID(t)\ \text{is}\ \hat{e}} \\[3ex]

\infer[\textit{DTy-Arrow}]
	{ \tau_1 \rightarrow \tau_2 \triangleleft Ty.Arrow(\hat{e}_1,\hat{e}_2 )  }
	{ \tau_1 \triangleleft \hat{e}_1 & \tau_2 \triangleleft \hat{e}_2 } \\[3ex]
   
\end{array}
\]
\caption{Dereification Rules}
\end{minipage}%
\vline
\begin{minipage}{.5\textwidth}
  \centering
  \[
\begin{array}{c}
\infer[\textit{RExp-Var}]
	{ x \triangleright Exp.Var(\hat{e})   }
	{ ID(x)\ \text{is}\ \hat{e}} \\[3ex]

\infer[\textit{RExp-Lam}]
	{ \boldsymbol\lambda x{:}\tau . \hat{e}'_1 \triangleright Exp.Lam( \hat{e}, \tau, \hat{e}_1 )  }
	{ID(x)\ \text{is}\ \hat{e} & \hat{e}'_1 \triangleright \hat{e}_1  } \\[3ex]

\infer[\textit{RExp-App}]
	{ \hat{e}'_1(\hat{e}'_2)  \triangleright Exp.App(\hat{e_1},\hat{e}_2) }
	{ \hat{e}'_1 \triangleright \hat{e}_1  & \hat{e}'_2 \triangleright \hat{e}_2   } \\[3ex]

\infer[\textit{RExp-Literal}]
	{ \lfloor literal \rfloor \triangleright Exp.Literal( \hat{e}_{ts} )  }
	{ \text{literal of}\ \hat{e}_{ts}\ \text{is}\ \lfloor literal \rfloor  } \\[3ex]

\infer[\textit{RTy-Var}]
	{ t \triangleright Ty.Var(\hat{e})   }
	{ ID(t)\ \text{is}\ \hat{e}} \\[3ex]

\infer[\textit{RTy-Arrow}]
	{ \tau_1 \rightarrow \tau_2 \triangleright Ty.Arrow(\hat{e}_1,\hat{e}_2 )  }
	{ \tau_1 \triangleright \hat{e}_1 & \tau_2 \triangleright \hat{e}_2 } \\[3ex]
   
\end{array}
\]
\caption{Reification Rules}
\end{minipage}
\end{figure}


\begin{figure}
\centering
\[
\infer[\textit{T-new-hat}]
	{\Delta; \Gamma \vdash \keyw{new}\ \{ \hat d \} :  t }
	{ t:\{\omega, \hat e:\tau\} \in \Delta & \Delta;\Gamma \vdash \hat d : \omega}
\]
\caption{Statics for $\hat e$}
\end{figure}


The \textit{Syn2Check} rule mediates between synthesis and type checking. 
The judgement 

\fbox{$\Delta; \Gamma \vdash e\Rightarrow \tau \leadsto \hat{e}$} 
\\
\noindent
from the type context $\Delta$ and the variable context $\Gamma$ we synthesize the type $\tau$ for $e$. The  expression $e$ possibly containing $\lfloor literal \rfloor$ forms is transformed into the expression $\hat{e}$ without literals.

In the static semantics rules, the context $\Delta$ contains types and their signatures. The context $\Gamma$ contains variables and their types. 

The judgement 

\fbox{$\Delta; \Gamma \vdash e \Leftarrow \tau \leadsto \hat{e}$} 

means that we check $e$ against the type $\tau$ . 

The \textit{Syn2Check} rule states that syntesis more powerful than type checking.

The rule \textit{RT-objtype} checks that the declaration of the object type $t$ is well-formed and the type of the expression $e$ is the same as the type of $t$'s metaobject.

The rule \textit{RT-casetype} checks that the declaration of the sum type $t$ is well-formed and the type of the expression $e$ is the same as the type of $t$'s metaobject.

The rule \textit{C-decl} checks that the type $\tau$ that is referenced by the name $C$ belongs to the type context $\Delta$.

The rule \textit{C-decls} allows a case type to have multiple cases, where each case will be checked by the rule \textit{C-decl}. We have to make sure that there are no two cases with the same names; we do this by checking that the domains of the $\chi$s are disjoint.

The rule \textit{O-val} checks that the type of a value belongs to the type context $\Delta$ and thus makes the declaration of a value well-formed (\texttt{ok}).

The rule \textit{O-def} checks that the type of a method ($\keyw{def}$) belongs to the type context $\Delta$ and thus makes the declaration of a method well-formed (\texttt{ok}).

The rule \textit{O-defs} allows multiple declarations of values $val$ and methods $def$ to appear one after the toher. Each declaration will either be checked by the rule \textit{O-val} or \textit{O-def}. We have to make sure that there are no two values or methods with the same name; we do this by checking that the domains of the $\omega$s are disjoint.

The rule \textit{T-varx} synthesizes the type of the variable $x$ to be $\tau$, after checking that the variable context $\Gamma$ contains the $x:\tau$ declaration. 

The rule \textit{T-abs} checks the type of the lambda abstraction and states what are the conditions for the checking to be performed.

The rule \textit{T-appl} synthesizes the type of the application of one expression to the other and states what are the premises needed for this synthesis to happen.

The rule \textit{T-introcase} introduces the way a case of the sum type should be used and the type of the resulting expression. To make it simpler, we precede the name of the case by the sum type that includes that case.

The rules \textit{T-elimcase} synthesizes the type of the resulting expression of a particular case of a sum type. Note that all the cases of the same sum type should synthesize to the same type. The rules  \textit{T-casehelper1} and \textit{T-casehelper2} help with the type checking of $c:\chi \Rightarrow \tau'$ in the rule \textit{T-elimcase}. Rule \textit{T-casehelper1} is used when $c$ is of the kind (matches) $C(\chi)\Rightarrow e$ , while rule \textit{T-casehelper2} is used when $c$ is of the kind $c_1 \bnfalt c_2$.

The rule \textit{T-new} checks the type of a \keyw{new} expression. 

The rule \textit{DT-val} checks the type of a value that is instatiated to the expression $e$.

The rule \textit{DT-def} checks the type of a method \keyw{def} that is instantiated to the expression $e$.

The rule \textit{DT-defs} allows for multiple instatiations of values or methods to take place one after the other. Each instantiation is checked with the rule \textit{DT-val} or \textit{DT-def}.

The rule \textit{T-field} synthesizes the type of the field of an expression. The premise mentions that the field is declared as a value .

The rule \textit{T-def}  synthesizes the type of the method (\keyw{def}) of an expression. 

The rule \textit{T-ascribe} ascribes (attributes) the type $\tau$ to $e$, after checking in the premise of the rule that the type of $e$ is $\tau$.

The rule \textit{T-metaobject} synthesizes the type of the \keyw{metaobject} of the type $t$ by checking that $t$ is in the type context $\Delta$ and it has the right type.

The rule \textit{T-literal} is a crucial rule of the our system. The conclusion checks that the $\lfloor literal \rfloor$ expression has the type $t$ and that it is translated to the expression $e$ that does not contain a $\lfloor literal \rfloor$ expression. The premise checks that the $parser$ method of the \keyw{metaobject} of the $t$ type is of type $Parser$. Instead of using $t.\keyw{metaobject}.parser$ we use $e_p$ in the rest of the rule. The premise continues by denoting the token stream of the $\lfloor literal \rfloor$ expression by $e_{ts}$. When the token stream is parsed, it evaluates to $Exp.C(e')$, which is an expression that does not contain a $\lfloor literal \rfloor$ expression. Note that only expressions with a hat $\hat{}$ contain a $\lfloor literal \rfloor$ expression. The expression $Exp.C(e')$ is then translated to the expression $\hat{e'}$, which in turn is recursively translated to the $e$ expression, which does not contain a $\lfloor literal \rfloor$. This rule might not terminate in the general case, but we prove that it terminates in the ways that we use it.















% !TEX root = ecoop14.tex

\section{Corpus Analysis}
\label{s:study}

We performed a corpus analysis to find out how often TSLs could be used in existing Java code. As a proxy for this goal we used \lstinline{String} arguments passed into Java constructors for two reasons:

\begin{enumerate}
\item The \lstinline{String} type is usually used for representing a large variety of notions, many of which may be expressed using TSLs.
\item Java constructors are a typical program construct and therefore are expected to reasonably well represent the overall composure of the program.
\end{enumerate}

\paragraph{Methodology.} We ran our analysis on a recent version (20130901r) of the Qualitas Corpus~\cite{QualitasCorpus:APSEC:2010}, consisting of 107 Java projects, and searched for constructors that used \lstinline{String}s that could be substituted with TSLs. To perform the search, we used command line tools, such as \lstinline{grep} and \lstinline{sed}, and a text editor features such as search and substitution. After we found the constructors, we chose those that took at least one \lstinline{String} as an argument. Via a visual scan of the names of the found constructors and their \lstinline{String} arguments, we inferred how the constructors and the arguments were intended to be used.


\paragraph{Results.} We found 124,873 constructors and that 19,288 (15\%) of them could use TSLs. Table~\ref{strings-in-constructors} gives more details on types of \lstinline{String} arguments we found that could be substituted with TSLs. The ``Identifier'' category comprises process IDs, user IDs, column or row IDs, etc. that usually must be unique; the ``Pattern'' category includes regular expressions, prefixes and suffixes, delimiters, format templates, etc.; the ``Other'' category contains \lstinline{String}s used for ZIP codes, passwords, queries, IP addresses, versions, HTML and XML code, etc.; and the ``Directory path'' and ``URL/URI'' categories are self-explanatory.

\begin{table}
   \centering
    \begin{tabular}[t]{l | c | c}
    \bf Type of String & \bf Number & \bf Percentage \\ \hline
    Identifier & 15,642 & 81\% \\
    Directory path& 823 & 4\% \\
    Pattern & 495 & 3\% \\
    URL/URI & 396 & 2\% \\
    Other (ZIP code, password, query, & 1,932 & 10\% \\
    ~~HTML/XML, IP address, version, etc.)  & & \\ \hline
    \bf Total: & \bf 19,288 & \bf 100\%
    \end{tabular}
    \vspace{0.15in}
    \caption{Types of \lstinline{String} Arguments in Java Constructors That Could Use TSLs}
    \label{strings-in-constructors}
\end{table}

\paragraph{Limitations.} There are three limitations to our corpus analysis. Firstly, the proxy that we chose for finding how often TSLs could be used in existing Java code is imprecise. Our corpus analysis focused exclusively on Java constructors and thus forwent many other types of programming constructs, such as method calls, variable assignments, etc., that could possibly use TSLs. Also, there are other datatypes, not just \lstinline{String}, that could be substituted with TSLs, e.g., \lstinline{Path} and \lstinline{URL}. Secondly, our search for constructors with the use of command line tools and text editor features may not have identified all the Java constructors present in the corpus. Finally, the inference of the intended functionality of the constructor and the passed in \lstinline{String} argument was based on the authors' programming experience and thus subjective.

\vspace{10px}

Despite the limitations of our corpus analysis, it showed that there is a potential for enhancing existing code with type-specific languages since numerous \lstinline{String}s could be substituted with TSLs and a significant portion of Java constructors could take advantage of this fact.


% !TEX root = ecoop14.tex

%\section{Implementation Considerations}
%\label{s:implementation}
%
%The Wyvern compiler
%was developed from the ground up to support the extensible parsing
%interface. The compiler is currently implemented in Java, with an eventual goal of self-hosting it within Wyvern itself.
%
%Wyvern uses a fixed whitespace-indented lexing approach similar to that used by languages like Python. The current indentation convention is based on the number of whitespace characters. In order for a block to be formed, each of its lines must share the same initial whitespace string. If the indentation level is decreased but does not return to the initial indentation level, a keyword block is created as shown in Figure \ref{f:dsl-multi}. This particular approach is not found in other whitespace-delimited languages to our knowledge.
%
%The Wyvern front-end effectively combines the parsing and type checking
%stages that are usually separated in more traditional compilers, such
%as \texttt{javac} or \texttt{gcc}. Before any block can be parsed,
%Wyvern requires a context that contains the types of the arguments of the surrounding function and any variables and types in scope. This context is passed into the delimited block for portions of the type-associated grammar that contain terms of sort \verb|Exp| or \verb|Type|, and thus may contain other function calls, variables, or types. 

%For example, the Wyvern default environment contains bindings for
%keywords like \keyw{class} or \keyw{meth} that are able to parse
%class and method declarations respectively. Parsers are bound to types and, as such, may be shadowed within application code. Only operators and their associated precedence are not exposed in the environment and, thus, have to be reused by the DSLs, though it is feasible to lift them up to the
%environment with an appropriate parser.
%
%% , an infix nature of operators
%% might make parsing definition more complex with arguably small gains
%% in expressiveness for potential DSLs.

%The core parsing in Wyvern is an extremely simple mechanism. The language uses a core parser that for every token encountered delegates to the appropriate parser in the environment. This extension parser processes the token and any relevant ones that follow. Once the extension parser is finished, the core parser continues through the remaining tokens, terminating when it either runs out of tokens or encounters a token with no associated extension parser.

%% In a similar manner to keywords, we also allow types and values to
%% attach appropriate bindings that can potentially define how to parse a
%% particular type and value as well as connecting them to an appropriate
%% type representation. For example, currently all primitive types (such
%% as integers, booleans, or strings) are defined using a binding as well
%% as basic values such as \texttt{true} or \texttt{false}.

%% Such flexible mechanism, especially when defined in Wyvern itself
%% makes any DSL extension as flexible and natural to define as writing
%% any normal Wyvern program itself.


%% \subsection{Examples}

%% We now present a small number of examples that demonstrate and explain
%% the operation of Wyvern.

%% \begin{figure}
%%   \centering
%%   \label{f:eg-let-eg}
%%   \caption{Wyvern Example: \texttt{let}}
%% \end{figure}
%% The \lstinline{let} parser knows that the first few lines following
%% the \lstinline{let} keyword will be variable definitions of the form
%% \lstinline{varname = value} and therefore can easily parse the lines
%% that following until a token \lstinline{in} on an indented line of its
%% own is encountered.  The body of the \lstinline{let} statement can now
%% be parsed using a continuation of the original Wyvern parser with an
%% addition of variables \lstinline{x} and \lstinline{z} to the current
%% environment. When the indented block finishes, the main Wyvern parser
%% can continue without \lstinline{let} variables in the environment. One
%% can observe that such parsing is very suitable to performing parsing
%% and type checking at the same time.

%% Our next example presents a simple \lstinline{if} statement in
%% Figure~\ref{f:eg-if}. Just like with the \lstinline{let} statement,
%% one can see how upon seeing the \lstinline{if} keyword, Wyvern parser
%% can invoke an appropriate \lstinline{if} parser that can easily be
%% replaced or modified. The default \lstinline{if} parsers parses the
%% conditional and is able to parse the body using the continuation of
%% the main parser with the right value of the condition taken into
%% account. One can observe that extending Wyvern to perform flow
%% sensitive checks such as non-nullness can be done easily even during
%% the parsing stage.

%% \begin{figure}
%%   \centering
%%   \begin{lstlisting}
%% if x = 5
%%     do stuff
%%   else
%%     do stuff
%%   \end{lstlisting}
%%   \label{f:eg-if}
%%   \caption{Wyvern Example: \texttt{if}}
%% \end{figure}

%% A more interesting variation of the \lstinline{if} statement shown in
%% Figure~\ref{f:eg-if-smalltalk} can be one that attaches a parser to
%% the boolean type rather than a specific keyword. In the spirit of
%% Smalltalk, one can provide a parser in Wyvern that is invoked when an
%% expression of a boolean type is encountered. Such parser can then
%% check if the next keyword is \lstinline{iftrue} that can be followed
%% by the true block parsed by the continuation of the normal
%% parser. Furthermore, an additional \lstinline{iffalse} keyword can be
%% allowed that contains the else part.

%% \begin{figure}
%%   \centering
%%   \begin{lstlisting}
%% condition iftrue
%%     [block]
%%   iffalse
%%     [block]
%%   \end{lstlisting}
%%   \label{f:eg-if-smalltalk}
%%   \caption{Wyvern Example: \texttt{if} in Smalltalk style}
%% \end{figure}

%% As a more complex example, consider an attempt to express a simple
%% Domain Specific Language that roughly corresponds to
%% HTML. Figure~\ref{f:eg-html} shows such extension that makes each
%% major html tag a keyword with its own parser attached to it. In fact,
%% each keyword can be treated as a first class value that can be stored
%% in a variable resulting in body being passed around or used as part of
%% a \lstinline{let} expression. The indentation makes clear when each
%% tag's parser needs to stop and in our opinion makes a body of a web
%% page much more readable, yet structured and even suitable for type
%% checking.

%% \begin{figure}
%%   \centering
%%   \begin{lstlisting}
%% let 
%%     theBody = body
%%         h1
%%             "This is a section header"
%%   in  
%% 	html
%% 	    head
%% 	        title: "The title of the page"
%% 	        script(src="http://.../")
%% 	        style
%% 	            body
%% 	                background-color: "white"
%% 	    theBody
%%   \end{lstlisting}
%%   \label{f:eg-html}
%%   \caption{Wyvern Example: HTML}
%% \end{figure}

%% \begin{figure}
%%   \centering
%%   \label{f:eg-arch}
%%   \begin{lstlisting}
%%   \end{lstlisting}
%%   \caption{Wyvern Example: Architecture}
%% \end{figure}

%% \begin{figure}
%%   \centering
%%   \label{f:eg-security}
%%   \begin{lstlisting}
%%   \end{lstlisting}
%%   \caption{Wyvern Example: Security Policy}
%% \end{figure}

%% \begin{figure*}
%%   \centering
%%   \label{f:eg-dsl-db}
%%   \begin{lstlisting}
%% assuming 
%%   theDB : OracleDB 
%% (that is, theDB is a DB that supports oracle-specific queries)

%% type-directed parsing:

%% theDB query
%%   (* stuff in the syntax of Oracle DBs *)

%% vs

%% keyword-directed parsing:

%% OracleDBquery theDB
%%   (* stuff in the syntax of Oracle DBs *)
%%   \end{lstlisting}
%%   \caption{Wyvern Example: DSL for DB Definition}
%% \end{figure*}



%For example, the Wyvern default environment contains bindings for
%keywords like \keyw{class} or \keyw{meth} that are able to parse
%class and method declarations respectively. Parsers are bound to types and, as such, may be shadowed within application code. Only operators and their associated precedence are not exposed in the environment and, thus, have to be reused by the DSLs, though it is feasible to lift them up to the
%environment with an appropriate parser.
%
%% , an infix nature of operators
%% might make parsing definition more complex with arguably small gains
%% in expressiveness for potential DSLs.

%The core parsing in Wyvern is an extremely simple mechanism. The language uses a core parser that for every token encountered delegates to the appropriate parser in the environment. This extension parser processes the token and any relevant ones that follow. Once the extension parser is finished, the core parser continues through the remaining tokens, terminating when it either runs out of tokens or encounters a token with no associated extension parser.

%% In a similar manner to keywords, we also allow types and values to
%% attach appropriate bindings that can potentially define how to parse a
%% particular type and value as well as connecting them to an appropriate
%% type representation. For example, currently all primitive types (such
%% as integers, booleans, or strings) are defined using a binding as well
%% as basic values such as \texttt{true} or \texttt{false}.

%% Such flexible mechanism, especially when defined in Wyvern itself
%% makes any DSL extension as flexible and natural to define as writing
%% any normal Wyvern program itself.


%% \subsection{Examples}

%% We now present a small number of examples that demonstrate and explain
%% the operation of Wyvern.

%% \begin{figure}
%%   \centering
%%   \label{f:eg-let-eg}
%%   \caption{Wyvern Example: \texttt{let}}
%% \end{figure}

\newpage

\section{Discussion and Future Work}\label{s:discussion}
\paragraph{Keyword-Directed Invocation}
In most DSL frameworks, a switch to a DSL is indicated by a keyword or function call naming the DSL to be used. Wyvern eliminates this overhead in many cases by determining the DSL based on the expected type of an expression. This lightweight mechanism is particularly useful for small DSLs, like the one associated with \lstinline{URL}. Keyword-directed invocation of a DSL is simply a special case of this approach. In particular, a keyword macro can be defined as a function with a single argument of a type specific to that keyword. The type contains the implementation of the domain-specific syntax associated with that keyword. In the most general sense, it may simply allow the entire \keyw{EXP} grammar, manipulating it in later phases of compilation. 

As an example, consider control flow operators like \verb|if|. This can be defined as a polymorphic method of the \verb|bool| type with signature $(\texttt{unit} \rightarrow \alpha, \texttt{unit} \rightarrow \alpha) \rightarrow \alpha$. That is, it takes the two branches as functions and chooses which to invoke based on the value of the boolean, using perhaps a more primitive control flow operator, like case analysis, or even a Church encoding of booleans as functions. In Wyvern, the branches could be packaged together into a type, \verb|IfBranches|, with an associated grammar that accepts the two branches as unwrapped expressions. Thus, \verb|if| could be defined entirely in a library and used as follows: 

\begin{lstlisting}
<guard>.if(~)
  then
    <any EXP>
  else
    <any EXP>
\end{lstlisting}

For methods like \verb|if| where constructing the argument explicitly will almost never be done, it may be useful to mark the method in a way that allows Wyvern to assume it is being called with a DSL argument immediately following its use. This would eliminate the need for the \verb|(~)| portion, supporting even more conventional notation. We have not considered this possibility in detail.

\paragraph{Explicit Delimiters}
Throughout this paper, DSLs have been delimited by whitespace. This allows arbitrary syntax within DSLs, since no delimiters need to be reserved to indicate the end of the DSL and thus there is no need for escaping internal uses of these delimiters. In cases where DSL expressions are expected to be reasonably short, such as the \lstinline{URL} example, or where delimiters are more natural than whitespace, such as for array or dictionary literals, it may be desirable to support other forms of delimited ``DSL literals''. 

One possible strategy for this is to reserve a number of common delimiter forms, such as quotation marks and  forms of braces, as equivalent DSL literal forms. The traditional meaning of these delimiters, such as quotation marks for strings and square brackets for lists, would then simply be convention in Wyvern. That is, the following expressions, as well as several similar ones, would be precisely equivalent (the programmer could choose the most convenient form, given the enclosed term):
\begin{verbatim}
  f("http://github.com/wyvernlang")  
  f([http://github.com/wyvernlang])
\end{verbatim}

Alternatively, types could specify the set of permitted delimiters so that conventions can be enforced by the compiler, improving identifiability. We have not yet explored either of these possibilities in detail, nor explored options that allow \emph{arbitrary} type-specified delimiters (a naive strategy for which would require that the first phase of parsing also be type-directed, which we wish to avoid).

\paragraph{Interaction with Subtyping}
The mechanism described here does not consider the case where multiple subtypes of a base type define a grammar. This can be resolved in several ways. We could require that only the \emph{declared} type's grammar is used (if a subtype's grammar is desired, an explicit type annotation on the tilde can be used). Alternatively, we could attempt to parse against all relevant subtypes, only requiring explicit disambiguation when ambiguities arise. Wyvern does not currently support subtyping, so we leave this as future work.
%%\paragraph{Custom Contexts}
%% One can observe a number of interesting issues that can be seen from
%% our Wyvern examples. For example, our \lstinline{let} statement
%% requires a corresponding \lstinline{in} part to be indented
%% differently from the other sub-blocks. This means that the entire
%% expression cannot be parsed by simply processing the appropriately
%% indented line sequence and then returing the control back to the
%% Wyvern parser, but rather there needs to be a way to detect that the
%% parser is now back to the level of the original \lstinline{let}
%% statement and the environment can be reset to what it was before the
%% statement.
%
%Presently, grammars cannot define their own contexts that are passed through nested grammars, but this would likely be a useful addition in many cases. As an example, consider a \lstinline{let} expression defined by a functional programming based DSL in 
%Wyvern:
%
%\begin{lstlisting}
%let 
%    x = 5
%    z = 4
%  in
%    x + z
%\end{lstlisting}
%
%\noindent 
%This expression requires maintaining a context of bound variables. This context must be maintained such that any expression within the \verb|in| block that is written in this DSL has access to the appropriate context, even if it appears as a subexpression within a different DSL. Multiple DSLs may define different contexts, and these must be threaded throughout nested DSLs in a modular manner.
%
%\paragraph{Custom Lexers} 
%Our existing lexing strategy may be too restrictive, requiring all DSLs to be hierarchical in nature. One potential expansion would be to enable DSLs to define their own lexers, still perhaps delimited by indentation or parentheses. Such an extension would sacrifice some readability.
%%However, we have been unable to find convincing use cases for this facet of extensibility. 
%
%Wyvern's operators are built in and their precedence follows that of
%C++ operators. We do not allow a replacement parser for infix
%operators as we considered it to unnecessarily complicate the current
%prefixed parsing approach. In the future, we plan to further support
%redefining operators.
%
%Finally, following Python and some other whitespace delimited
%languages, we may wish to allow parenthesized expressions to avoid the need for
%following the indentation level. This is still subject to debate and,
%as we try to express more and more DSL kinds in Wyvern, we may decide
%to enforce indentation levels even inside the parentheses.

% !TEX root = ecoop14.tex

\section{Discussion}\label{s:discussion}

\paragraph{Safe TSL Composition}

Our primary contribution is a strategy, where nesting of TSLs occurs by briefly entering the host language, ensures that ambiguities cannot occur. The host language ensures that TSLs are delimited unambiguously, and the TSL ensures that the host language is delimited unambiguously. The body of the TSL is interpreted by a fixed grammar -- the one associated with its expected type. This avoids the kinds of conflicts a simple merger of the grammars would cause.
%Because the language where a language switch 
Apart from the large number of TSLs that can be composed together in a short
piece of code while producing meaningful results, we aim to provide a safe
composability guarantee that other language extension solutions do not~\cite{Erdweg:2013:FEL:2517208.2517210,krahn2008monticore}.

\paragraph{Keyword-Directed Invocation}

In most domain-specific language frameworks, a switch to a different language is indicated by a keyword or function call naming the language to be used. Wyvern eliminates this overhead in many cases by determining the TSL based on the expected type of an expression. This lightweight mechanism is particularly useful for small languages.
%, like the one associated with \lstinline{URL}.
Keyword-directed invocation is simply a special case of our type-directed approach. In particular, a keyword macro can be defined as a function with a single argument of a type specific to that keyword. The type contains the implementation of the domain-specific syntax associated with that keyword. In the most general sense, it may simply allow the entire Wyvern
%\keyw{EXP}
grammar, manipulating it in later phases of compilation. 

As an example, consider control flow operators like \verb|if|. This can be defined as a polymorphic method of the \verb|bool| type with signature $(\texttt{unit} \rightarrow \alpha, \texttt{unit} \rightarrow \alpha) \rightarrow \alpha$. That is, it takes the two branches as functions and chooses which to invoke based on the value of the boolean, using perhaps a more primitive control flow operator, like case analysis, or even a Church encoding of booleans as functions. In Wyvern, the branches could be packaged together into a type, \verb|IfBranches|, with an associated grammar that accepts the two branches as unwrapped expressions. Thus, \verb|if| could be defined entirely in a library and used as follows: 

\begin{lstlisting}
<guard>.if(~)
  then
    <any Wyvern>
  else
    <any Wyvern>
\end{lstlisting}

For methods like \verb|if| where constructing the argument explicitly will almost never be done, it may be useful to mark the method in a way that allows Wyvern to assume it is being called with a TSL argument immediately following its use. This would eliminate the need for the \verb|(~)| portion, supporting even more conventional notation. We have not considered this possibility in detail.

%\paragraph{Explicit Delimiters}
%Throughout this paper, DSLs have been delimited by whitespace. This allows arbitrary syntax within DSLs, since no delimiters need to be reserved to indicate the end of the DSL and thus there is no need for escaping internal uses of these delimiters. In cases where DSL expressions are expected to be reasonably short, such as the \lstinline{URL} example, or where delimiters are more natural than whitespace, such as for array or dictionary literals, it may be desirable to support other forms of delimited ``DSL literals''. 
%
%One possible strategy for this is to reserve a number of common delimiter forms, such as quotation marks and  forms of braces, as equivalent DSL literal forms. The traditional meaning of these delimiters, such as quotation marks for strings and square brackets for lists, would then simply be convention in Wyvern. That is, the following expressions, as well as several similar ones, would be precisely equivalent (the programmer could choose the most convenient form, given the enclosed term):
%\begin{verbatim}
%  f("http://github.com/wyvernlang")  
%  f([http://github.com/wyvernlang])
%\end{verbatim}
%
%Alternatively, types could specify the set of permitted delimiters so that conventions can be enforced by the compiler, improving identifiability. We have not yet explored either of these possibilities in detail, nor explored options that allow \emph{arbitrary} type-specified delimiters (a naive strategy for which would require that the first phase of parsing also be type-directed, which we wish to avoid).

\paragraph{Interaction with Subtyping}

The mechanism described here does not consider the case where multiple subtypes of a base type define a grammar. This can be resolved in several ways. We could require that only the \emph{declared} type's grammar is used (if a subtype's grammar is desired, an explicit type annotation on the tilde can be used). Alternatively, we could attempt to parse against all relevant subtypes, only requiring explicit disambiguation when ambiguities arise. Wyvern does not currently support subtyping, so we leave this as future work.

%%\paragraph{Custom Contexts}
%% One can observe a number of interesting issues that can be seen from
%% our Wyvern examples. For example, our \lstinline{let} statement
%% requires a corresponding \lstinline{in} part to be indented
%% differently from the other sub-blocks. This means that the entire
%% expression cannot be parsed by simply processing the appropriately
%% indented line sequence and then returing the control back to the
%% Wyvern parser, but rather there needs to be a way to detect that the
%% parser is now back to the level of the original \lstinline{let}
%% statement and the environment can be reset to what it was before the
%% statement.

%Presently, grammars cannot define their own contexts that are passed through nested grammars, but this would likely be a useful %addition in many cases. As an example, consider a \lstinline{let} expression defined by a functional programming based DSL in 
%Wyvern:
%
%\begin{lstlisting}
%let 
%    x = 5
%    z = 4
%  in
%    x + z
%\end{lstlisting}
%
%\noindent 
%This expression requires maintaining a context of bound variables. This context must be maintained such that any expression within the \verb|in| block that is written in this DSL has access to the appropriate context, even if it appears as a subexpression within a different DSL. Multiple DSLs may define different contexts, and these must be threaded throughout nested DSLs in a modular manner.

\paragraph{Custom Lexers} 

Our existing lexing strategy may be too restrictive, requiring all DSLs to be hierarchical in nature. One potential expansion would be to enable DSLs to define their own lexers, still perhaps delimited by indentation or parentheses. Such an extension would sacrifice some readability.
%%However, we have been unable to find convincing use cases for this facet of extensibility. 

Wyvern's operators are built in and their precedence follows that of
C++ operators. We do not allow a replacement parser for infix
operators as we considered it to unnecessarily complicate the current
prefixed parsing approach. In the future, we plan to further support
redefining operators.

Finally, following Python and some other whitespace delimited
languages, we may wish to allow parenthesized expressions to avoid the need for
following the indentation level. This is still subject to debate and,
as we try to express more and more DSL kinds in Wyvern, we may decide
to enforce indentation levels even inside the parentheses.
% !TEX root = ecoop14.tex
\section{Related Work}
\label{s:related}

Language macros are the most explored way of extending programming languages, with Scheme and other Lisp-style languages' hygienic macros being the 'gold standard.' In those languages, macros are written in the language itself and benefit from the simple syntax -- parentheses universally serve as expression delimiters (although proposals for whitespace as a substitute for parentheses have been made \cite{srfi-49}). Our work is inspired by this flexibility, but aims to support richer syntax as well as static types. Wyvern's use of types to trigger parsing  avoids the overhead of needing to invoke macros explicitly by name and makes it easier to compose TSLs declaratively.

Another way to approach language extensibility is to go a level of abstraction above parsing as is done via metaprogramming facilities. For instance, OJ (previously, OpenJava)~\cite{Tatsubori00openjava:a} provides a macro system based on a meta-object protocol, and Backstage Java~\cite{Palmer:2011:BJM:2048066.2048137}, Template Haskell \cite{sheard2002template} and others employ compile-time meta-programming.  Each of these systems provide macro-style rewriting of source code, but they provide at most limited extension of language parsing.

Other systems aim at providing forms of syntax extension that change the host language, as opposed to our whitespace-delimited approach.  For example, Camlp4 \cite{camlp4} is a preprocessor for OCaml that offers the developer the ability to extend the concrete syntax of the language via the use of parsers and extensible grammars.  SugarJ \cite{Erdweg:2011:SLL:2048147.2048199} takes a library-centric approach which supports syntactic extension of the Java language by adding libraries. In Wyvern, the core language
is not extended directly, so conflicts cannot arise at link-time. 

Scoping TSLs to expressions of a single type comes at the expense of some flexibility, but we believe that many uses of domain-specific languages are of this form already. A previous approach has considered type-based disambiguation of parse forests for supporting quotation and anti-quotation of arbitrary object languages~\cite{bravenboer2005generalized}. Our work is similar in spirit, but does not rely on generation of parse forests and associates grammars with types, rather than types with grammar productions. We believe that this is a more simple and flexible methodology. 
 C\# expression trees \cite{Csharp} are similar in that, when the type of a term is \li{Expression<T->T'>}, it is parsed as a quotation. However, like the work just mentioned, this is \emph{specifically} to support quotations. Our work supports quotations in addition to a variety of other work.
 
Many approaches to syntax extension, such as XJ~\cite{DBLP:conf/scam/ClarkSW08} are keyword-directed in some form. We believe that a type-directed approach is more seamless and general, sacrificing a small amount of identifiability in some cases. 

In terms of work on safe language composition, Schwerdfeger and van Wyk~\cite{Schwerdfeger:2009:VCD:1542476.1542499} proposed a solution that make strong safety guarantees provided that the languages comply with certain grammar restrictions, concerning first and follow sets of the host language and the added new languages. It also relied on strongly named entry tokens, like keyword delimited approaches. Our approach does not impose any such restrictions while still making safety guarantees.

Domain-specific language frameworks and language workbenches, such as Spoofax \cite{KatsVisser2010}, Ens\={o}~\cite{enso} and others~\cite{krahn2008monticore,van1992pregmatic}, also provide a possible solution for the language extension task. They provide support for generating new programming languages and tooling in a modular manner.  The Marco language \cite{lee:2012:marco} similarly provides macro definition at a level of abstraction that is largely independent of the target language. In these approaches, each TSL is \emph{external} relative to the host language; in contrast, Wyvern focuses on extensibility \emph{internal} to the language, improving interoperability and composability.

In addition, there is an ongoing work on projectional editors (e.g., \cite{mps,Diekmann:2013}) that use special graphical user interface to allow the developer to implicitly mark where the extensions are placed in the code, essential specifying directly the underlying ASTs. This solution to the language extension problem poses several challenges such as defining and implementing the semantics for the composition of the languages and the channels for communication between them. In Wyvern, we do not encounter these problems as the semantic rules for a language composition are incorporated within the host language by design.


There is a relation between recent work on Active Code Completion and our approach in that
the Active Code Completion work associates code completion palettes with types~\cite{omar2012active} as well. Such palettes could be used for defining a TSL syntax for types. However, that syntax
is immediately translated to Java syntax at edit time, while this work
integrates with the core parsing facilities of the language.



%\begin{itemize}
%\item language boxes work discussed at Parsing workshop~\cite{Diekmann:2013}
%\item Tratt's Converge~\cite{Tratt:2005:CMD:1146841.1146846}
%\end{itemize}

% !TEX root = ecoop14.tex
\section{Conclusion} % and Future Work}
\label{s:conclusion}
%
%In this paper, we described how extensible parsing in Wyvern makes for
%a solid platform to support whitespace-delimited, type-directed embedded DSLs. In the
%future, we aim to implement a wide variety of DSLs in Wyvern tweaking
%our approach and implementation thereof to provide a comprehensive example of
%supporting multiple interacting DSLs in a safe and easy-to-use manner.
%
% \todo{tie features to goals}

% \todo{implementation and validation plans}

%\section*{Acknowledgements}
%We thank the anonymous reviewers for helpful comments, and acknowledge the support of the Department of Defense and the Air Force Research Laboratory. CO is supported by the NSF Graduate Research Fellowship.

\bibliographystyle{abbrv}
\bibliography{biblio,papers-cyrus}

\end{document}
