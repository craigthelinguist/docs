% !TEX root = ecoop14.tex
\section{Syntax}
\label{s:approach}

\subsection{Concrete Syntax}
We will now describe the concrete syntax of Wyvern declaratively, using the same layout-sensitive formalism that we have introduced for TSL grammars, developed recently by Adams \cite{Adams:2013:PPI:2429069.2429129}. Such a formalism is useful because it allows us to implement  layout-sensitive syntax, like that we've been describing, without relying on context-sensitive lexers or parsers. Most existing layout-sensitive languages (e.g. Python and Haskell) use hand-rolled context-sensitive lexers or parsers (keeping track of, for example, the indentation level using special \li{INDENT} and \li{DEDENT} tokens), but these are more problematic because they cannot be used to generate editor modes, syntax highlighters and other tools automatically. In particular, we will show how the forward references we have described can be correctly encoded without requiring a context-sensitive parser or lexer using this formalism. It is also useful that the TSL for \li{Parser}, above, uses the same parser technology as the host language, so that it can be used to generate quasiquotes.

Wyvern's concrete syntax, with a few minor omissions for concision, is shown in Figure~\ref{f-grammar}. We first review Adams' formalism in some additional detail, then describe some key features of this syntax.

\subsection{Background: Adams' Formalism}
For each terminal and non-terminal in a rule, Adams proposed associating with them a relational operator, such as =, > and $\geq$ to specify the indentation at which those terms need to be with respect to the non-terminal on the left-hand side of the rule. The indentation level of a term can be identified as the column at which the left-most character of that term appears (not simply the first character, in the case of terms that span multiple lines). The meaning of the comparison operators is akin to their mathematical meaning: = means that the term on the right-hand side has to be at exactly the same indentation as the term on the left-hand side; >  means that the term on the right-hand side has to be indented strictly further to the right than the term on the left-hand side; $\geq$ is like >, except the term on the right could also be at the same indentation level as the term on the left-hand side. For example, the production rule of the form \lstinline{A $\rightarrow$ B$^=$ C$^\geq$ D$^>$} approximately reads as: ``Term \lstinline{B} must be at the same indentation level as term \lstinline{A}, term \lstinline{C} may be at the same or a greater indentation level as term \lstinline{A}, and term \lstinline{D} must be at an indentation level greater than term \lstinline{A}'s.'' In particular, if \li{D} contains a \lstinline{NEWLINE} character, the next line must be indented past the position of the left-most character of \lstinline{A} (typically constructed so that it must appear at the beginning of a line). There are no constraints relating \lstinline{D} to \lstinline{B} or \lstinline{C} other than the standard sequencing constraint: the first character of \lstinline{D} must be to further in the file than the others. Using Adam's formalism, the grammars of real-world languages like Python and Haskell can be written declaratively. This formalism can be integrated into LR and LALR parser generators.

% !TEX root = ecoop14.tex

\begin{figure}
\begin{lstlisting}[mathescape]
p $\rightarrow$ 'objtype'$^=$ ID$^>$ NEWLINE$^>$ objdecls$^>$ NEWLINE$^>$ metadatadecl$^>$ NEWLINE$^>$ p$^=$
p $\rightarrow$ 'casetype'$^=$ ID$^>$ NEWLINE$^>$ casedecls$^>$ NEWLINE$^>$ metadatadecl$^>$ NEWLINE$^>$ p$^=$
p $\rightarrow$ e$^=$

metadatadecl $\rightarrow$ 'metadata'$^=$ '='$^>$ e$^>$

e $\rightarrow$ $\overline{\texttt{e}}$$^=$
e $\rightarrow$ $\widetilde{\texttt{e}}$['~']$^=$ NEWLINE$^>$ chars$^>$
e $\rightarrow$ $\widetilde{\texttt{e}}$['new']$^=$ NEWLINE$^>$ d$^>$
e $\rightarrow$ $\widetilde{\texttt{e}}$['case(' $\overline{\texttt{e}}$ ')']$^=$ NEWLINE$^>$ c$^>$

$\overline{\texttt{e}}$ $\rightarrow$ ID$^=$
$\overline{\texttt{e}}$ $\rightarrow$ 'fn'$^=$ ID$^>$ ':'$^>$ type$^>$ '=>'$^>$ $\overline{\texttt{e}}$$^=$
$\overline{\texttt{e}}$ $\rightarrow$ $\overline{\texttt{e}}$$^=$ '('$^>$ $\overline{\texttt{al}}$$^=$ ')'$^>$
$\overline{\texttt{e}}$ $\rightarrow$ 'let'$^=$ ID$^>$ ':'$^>$ type$^>$ '='$^>$ e$^>$ NEWLINE$^>$ $\overline{\texttt{e}}$$^=$
$\overline{\texttt{e}}$ $\rightarrow$ $\overline{\texttt{e}}$$^=$ '.'$^>$ ID$^>$
$\overline{\texttt{e}}$ $\rightarrow$ type$^=$ '.'$^>$ ID$^>$ '('$^>$ $\overline{\texttt{e}}$$^>$ ')'$^>$
$\overline{\texttt{e}}$ $\rightarrow$ $\overline{\texttt{e}}$$^=$ ':'$^>$ type$^>$
$\overline{\texttt{e}}$ $\rightarrow$ 'valAST'$^=$ '('$^>$ $\overline{\texttt{e}}$$^>$ ')'$^>$
$\overline{\texttt{e}}$ $\rightarrow$ type$^=$ '.'$^>$ 'metadata'$^>$
$\overline{\texttt{e}}$ $\rightarrow$ inlinelit$^=$

$\widetilde{\texttt{e}}$[fwd] $\rightarrow$ fwd$^=$
$\widetilde{\texttt{e}}$[fwd] $\rightarrow$ 'fn'$^=$ ID$^>$ ':'$^>$ type$^>$ '=>'$^>$ $\widetilde{\texttt{e}}$[fwd]$^>$
$\widetilde{\texttt{e}}$[fwd] $\rightarrow$ $\widetilde{\texttt{e}}$[fwd]$^=$ '('$^>$ $\overline{\texttt{al}}$$^>$ ')'$^>$
$\widetilde{\texttt{e}}$[fwd] $\rightarrow$ 'let'$^=$ ID$^>$ ':'$^>$ type$^>$ '='$^>$ e$^>$ NEWLINE$^>$ $\widetilde{\texttt{e}}$[fwd]$^=$
$\widetilde{\texttt{e}}$[fwd] $\rightarrow$ $\overline{\texttt{e}}$$^=$ '('$^>$ $\widetilde{\texttt{al}}$[fwd]$^>$ ')'$^>$
$\widetilde{\texttt{e}}$[fwd] $\rightarrow$ $\widetilde{\texttt{e}}$[fwd]$^=$ '.'$^>$ ID$^>$
$\widetilde{\texttt{e}}$[fwd] $\rightarrow$ type$^=$ '.'$^>$ ID$^>$ '('$^>$ $\widetilde{\texttt{e}}$[fwd]$^>$ ')'$^>$
$\widetilde{\texttt{e}}$[fwd] $\rightarrow$ $\widetilde{\texttt{e}}$[fwd]$^=$ ':'$^>$ type$^>$
$\widetilde{\texttt{e}}$[fwd] $\rightarrow$ 'valAST'$^=$ '('$^>$ $\widetilde{\texttt{e}}$[fwd]$^>$ ')'$^>$

d $\rightarrow$ $\varepsilon$
d $\rightarrow$ 'val'$^=$ ID$^>$ ':'$^>$ type$^>$ '='$^>$ e$^>$ NEWLINE$^>$ d$^=$
d $\rightarrow$ 'def'$^=$ ID$^>$ '('$^>$ argsig$^>$ ')'$^>$ ':'$^>$ type$^>$ '='$^>$ e$^>$ NEWLINE$^>$ d$^=$

c $\rightarrow$ ID$^=$ '('$^>$ ID$^>$ ')'$^>$ '=>'$^>$ e$^>$
c $\rightarrow$ ID$^=$ '('$^>$ ID$^>$ ')'$^>$ '=>'$^>$ e$^>$ NEWLINE$^>$ c$^=$

$\overline{\texttt{al}}$ $\rightarrow$ $\varepsilon$ | $\overline{\texttt{al}}_{\texttt{nonempty}}$$^=$
$\overline{\texttt{al}}_{\texttt{nonempty}}$ $\rightarrow$ $\overline{\texttt{e}}$$^=$ | $\overline{\texttt{e}}$$^=$ ','$^>$ $\overline{\texttt{al}}_{\texttt{nonempty}}$$^>$

$\widetilde{\texttt{al}}$[fwd] $\rightarrow$ $\widetilde{\texttt{e}}$[fwd]$^=$
$\widetilde{\texttt{al}}$[fwd] $\rightarrow$ $\widetilde{\texttt{e}}$[fwd]$^=$ ','$^>$ $\overline{\texttt{al}}_{\texttt{nonempty}}$$^>$
$\widetilde{\texttt{al}}$[fwd] $\rightarrow$ $\overline{\texttt{e}}$$^=$ ','$^>$ $\widetilde{\texttt{al}}$[fwd]$^>$

inlinelit $\rightarrow$ chars1['`']$^=$ | chars1[''']$^=$ | chars1['"']$^=$ | ...
inlinelit $\rightarrow$ chars2['{', '}']$^=$ | chars2['<', '>']$^=$ | chars2['[', ']']$^=$ | ...
inlinelit $\rightarrow$ numlit$^=$
\end{lstlisting}
\caption{Concrete Syntax}
\label{f-grammar}
\end{figure}

\begin{figure}
\begin{lstlisting}[mathescape]
`TSL code here, ``inner backticks`` must be doubled`
'TSL code here, ''inner single quotes'' must be doubled'
'TSL code here, ""inner double quotes"" must be doubled'

{TSL code here, {inner braces} must be balanced}
[TSL code here, [inner brackets] must be balanced]
<TSL code here, <inner angle brackets> must be balanced>
\end{lstlisting}
\caption{TSL Delimiters}
\label{f-delims}
\end{figure}


\subsection{Programs}

\begin{figure}[t]
\begin{minipage}{.57\textwidth}
\begin{lstlisting}
objtype T
  val y : HTML
let page : HTML->HTML = fn x : HTML => ~
  :html
    :body
      {x}
page(case(5 : Nat))
  Z(_) => (new : T).y
    val y : HTML = ~
      :h1 Zero!
  S(x) => ~
    :h1 Successor!
\end{lstlisting}
\end{minipage}%
\begin{minipage}{.47\textwidth}
 \centering
\[
\begin{array}{l}
\keyw{objtype} \ T\ \{ \\
  ~~~\keyw{val}\ y : HTML, \\
  ~~~\keyw{metadata} = (\keyw{new}\ \{\}) : Unit\ \}; \\
  (\boldsymbol\lambda page : HTML \rightarrow HTML\ .\\
  ~page(\keyw{case}(\lfloor 5 \rfloor : Nat)\ \{ \\
  ~~~~Z(\_) \Rightarrow ((\keyw{new}\ \{ \\
    ~~~~~~~\keyw{val}\ y : HTML = \lfloor :h1\ Zero! \rfloor\}) : T).y\ | \\
   ~~~~S(x) \Rightarrow \lfloor :h1\ Successor! \rfloor\})) \\
  (\boldsymbol\lambda x : HTML\ .\ \lfloor :html \\
    ~~~~~~~~~~~:body \\
      ~~~~~~~~~~~~~~\{x\} \rfloor)
\end{array}
\]
\end{minipage}
\caption{An example Wyvern program demonstrating forward references}
\label{fig:fwd-ref}
\end{figure}
An example Wyvern program showing several unique syntactic features of TSL Wyvern is shown in Fig. \ref{f-frefs}. The top level of a program (the \lstinline{p} non-terminal) consists of a series of type declarations -- object types using \lstinline{objtype} or case types using \lstinline{casetype} -- followed by an expression, \lstinline{e}. Each type declaration contains associated declarations -- signatures for fields and methods in  \lstinline{objdecls} and case declarations in \lstinline{casedecls} (not shown on the figure). Each also can also include a metadata declaration. Metadata is simply an expression associated with the type, used to store TSL logic (and in future work, other logic). Sequences of top-level declarations use the form \lstinline{p$^=$} to signify that all the succeeding \lstinline{p} terms must begin at the same indentation.\todo{fig ref}
\subsection{Forward Referenced Blocks}
Wyvern makes extensive use of forward referenced blocks to make its syntax clean. In particular, layout-delimited TSLs, the general-purpose introductory form for object types and the elimination form for case types and product types all use forward referenced blocks. Fig. \ref{f-frefs} shows all of these in use (assuming suitable definitions of casetypes \li{Nat} and \li{HTML}, not included). In the grammar, note particularly the rules for \li{let} and that inline literals, even those containing nested expressions with forward references, can be treated as expressions not containing forward references -- \emph{in the initial phase of parsing, before typechecking commences, all literal forms are left unparsed}.

\subsection{Abstract Syntax}
The concrete syntax of a Wyvern program, \li{p}, is parsed to produce a program in the abstract syntax, $\rho$, shown on the left side of Fig. \ref{fig:core2-syntax}. Forward references are internalized. In particular, note that all literal forms are unified into the abstract literal form $\lfloor body \rfloor$, including the layout-delimited form and number literals. The abstract syntax contains a form, $\keyw{fromTS}[\Gamma](e,e)$, that has no analog in the concrete syntax. This will be used internally to ensure hygiene, as we will discuss in the next section.

\begin{figure}[t]
\centering
\[
\begin{array}[t]{lll} 
\rho & \bnfdef & \keyw{objtype}~ t~ \{ \omega, \keyw{metadata}=e \}; \rho \\
     & \bnfalt & \keyw{casetype}~ t~ \{ \chi, \keyw{metadata}=e \}; \rho\\
     & \bnfalt & e
     \\[1ex]
e    & \bnfdef & x \\
     & \bnfalt & \boldsymbol\lambda x{:}\tau . e \\ %
     & \bnfalt & e(e) \\
     & \bnfalt & (e, e) \\
     & \bnfalt & \keyw{case}(e) \{(x, y) \Rightarrow e\}\\
     & \bnfalt & t.C(e) \\
     & \bnfalt & \keyw{case}(e)~\{ c \} \\
     & \bnfalt & \keyw{new}~ \{ d \}\\
     & \bnfalt & e.x \\
     & \bnfalt & e : \tau\\
     & \bnfalt & \keyw{valAST}(e) \\
     & \bnfalt & t.\keyw{metadata}\\
     & \bnfalt & \lfloor literal \rfloor \\
     & \bnfalt & \keyw{fromTS}[\Gamma](e, e)
\\[1ex]	
c    & \bnfdef & C(x) \Rightarrow e\\
     & \bnfalt & c \bnfalt c
	 \\[1ex]
d   & \bnfdef & \varepsilon \\
     & \bnfalt & \keyw{val}~ f:\tau = e;~d \\
     & \bnfalt & \keyw{def}~ m:\tau = e;~d
\\[1ex] 
\end{array}
\begin{array}[t]{lll}
~~~
\end{array}
\begin{array}[t]{lll}


\hat\rho & \bnfdef & \keyw{objtype}~ t~ \{ \omega, \keyw{metadata}=\hat e \}; \hat\rho \\
     & \bnfalt & \keyw{casetype}~ t~ \{ \chi, \keyw{metadata}=\hat e \}; \hat\rho\\
     & \bnfalt & \hat e
     \\[1ex]
\hat{e}    & \bnfdef & x \\
     & \bnfalt & \boldsymbol\lambda x{:}\tau . \hat{e} \\ %
     & \bnfalt & \hat{e}(\hat{e}) \\
     & \bnfalt & \cdots \\
     & \bnfalt & t.\keyw{metadata} 
\\[1ex]
\hat c    & \bnfdef & ...
	 \\[1ex]
\hat d   & \bnfdef & ... 
\\[1ex] 
\chi & \bnfdef & C~\keyw{of}~\tau\\
     & \bnfalt & \chi \bnfalt \chi 
\\[1ex]
\omega &\bnfdef & \varepsilon \\  
         & \bnfalt & \keyw{val}~ f:\tau;~\omega\\
         & \bnfalt & \keyw{def}~ m:\tau;~\omega 
\\[1ex]
\tau & \bnfdef & t\\
     & \bnfalt & \tau \rightarrow \tau \\
     & \bnfalt & \tau \times \tau 
\\[1ex]
\Gamma & \bnfdef & \emptyset \bnfalt \Gamma, x:\tau
\\[1ex]
\Delta & \bnfdef & \emptyset \bnfalt \Delta, t:\{\chi, \hat e:\tau\} \bnfalt \Delta, t:\{\omega, \hat e:\tau\}
\\[1ex]

\end{array}
\]
\caption{Abstract Syntax}
\label{fig:core2-syntax}
\end{figure}

\section{Bidirectional Typechecking and Literal Rewriting}
We will now specify a type system for the abstract syntax in Fig. \ref{fig:core2-syntax}. Conventional type systems are specified using a typechecking judgement like $\Delta; \Gamma \vdash e : \tau$, where the variable context, $\Gamma$, tracks the types of variables, and the type context, $\Delta$, tracks types and their signatures. However, this conventional formulation does not separately consider how, when deriving this judgement, it will be considered algorithmically -- will a type be provided, so that we simply need to check $e$ against it, or do we need to synthesize a type for $e$? For our system, this distinction is crucial: a generic literal can only be used in the first situation. 

\emph{Bidirectional type systems}, as presented by Lovas and Pfenning \cite{Lovas08abidirectional}, make this distinction clear by specifying the type system instead using two simultaneously defined judgements: one for expressions that can \emph{synthesize} a type based on the surrounding context (e.g. variables and elimination forms), and another for expressions for which we know what type to \emph{check} or \emph{analyze} the term against (e.g. generic literals and some introductory forms). 
Our work builds upon this work, making the following core additions: the tye context $\Delta$ now tracks the metadata in addition to type signatures, and as we typecheck, we need to also perform literal rewriting by calling the parser associated with the type that a literal is being analyzed against, typechecking the AST it produces and ensuring that hygiene is maintained.

The judgement 
\fbox{$\Delta; \Gamma \vdash e\Rightarrow \tau \leadsto \hat{e}$} 
means that from the type context $\Delta$ and the variable context $\Gamma$ we synthesize the type $\tau$ for $e$. The  expression $e$ possibly containing $\lfloor literal \rfloor$ forms (and the \keyw{fromTS} form, which we will discuss shortly) is rewritten into the expression $\hat{e}$ without these forms.
The judgement 
\fbox{$\Delta; \Gamma \vdash e \Leftarrow \tau \leadsto \hat{e}$} 
means that we check $e$ against the type $\tau$ and the expression $e$ is rewritten into the expression $\hat{e}$. The forms of $\Gamma$ and $\Delta$ are given in Fig. \ref{fig:core2-syntax}. Note that $\Delta$ carries the type's signature as well as the rewritten form of the metadata. The rules for these judgements, as well as key rules for several auxiliary judgements that are needed in their premises, are given in Figs. 8-12. 

These rules assume that a collection of built-in types are included by default at the top of programs (e.g. \li{Unit}, \li{Parser}, \li{Exp} already mentioned, and a few others), captured by an intial type context $\Delta_0$. We show the concrete syntax for the two key ones in Fig. 7. The static semantics and the dynamic semantics (defined for $\hat{e}$ only) are that of a conventional functional language with functions, inductive datatypes, products and records  with the addition of a few new forms. The key new dynamic semantics rules are described in Fig. 13. We will now describe how some of the novel rules that support TSLs work below. We refer the reader to \cite{Lovas08abidirectional} and excellent textbooks on type theory, e.g. \cite{pfpl} and \cite{tapl}, for the remainder.

\begin{figure}[t]
\begin{lstlisting}
objtype Parser                                casetype Exp = 
  def parse(ts : TokenStream) : Exp             Var of ID
  metadata = new                              | Lam of ID * Type * Exp
    val parser : Parser = new                 | App of Exp * Exp
      val parse(ts : TokenStream) :           ... | FromTS of Exp * Exp
         Exp * TokenStream =                  | Literal of TokenStream
        (* ... parser generator based on      | Error of ErrorMessage
           Adams' formalism here ... *)       metadata = (* quasiquotes *)
\end{lstlisting}
\caption{Two of the built-in types included in $\Delta_0$.}
\label{fig:typeParser}
\end{figure}
%\subsection{Bidirectional Type System}
%Our type-checking is syntax-directed and it requires terms to be fully annotated with types where necessary. A commonly used approach for making the syntax more compact is bidirectional type systems, . They introduce two mutually recursive judgements: one for expressions that have enough information in the context to synthesize a type, and one for expressions for which we know what type to expect, thus only needing to check against that type. Unique types can be determined for synthesis expressions, while analytical expressions have to be verified to have the right types. We chose to use bidirectional type systems in our formalism because they leverage the simplicity of syntax-directed type-checking while not needing to carry much additional type information.
%
%In conventional bidirectional type systems, for constructors of a type one can propagate the type information $\tau$
%into the term $e$, which means it should be used in the analysis
%judgment $e \Leftarrow \tau$. When constructing a type, we do not have information about it and it is intuitive to use the analysis judgement, which is weaker than the synthesis judgement. On the other hand, destructors generate a result of a smaller type from a component of larger type and can be used for synthesis, propagating type information away from the term as in the synthesis judgement $e \Rightarrow \tau$. Our static semantics rules follow this conventional way of reasoning about constructors and destructors.

\subsection{Defining a TSL Manually}
In the example in Fig. \ref{f-htmltype}, we showed a TSL being defined using a parser generator based on Adams' formalism. A parser generator is itself merely a TSL for a parser, and a parser is the fundamental construct that becomes associated with a type to form a TSL. The signature for the built-in type \li{Parser} is shown in Fig. 7. It is an object type with a \li{parse} function taking in a \li{TokenStream} and producing an AST of a Wyvern expression, which is of type \li{Exp}. This built-in type is shown also in Fig. 7. Note that there is a form for each form in the abstract syntax, $e$, as well as an \li{Error} form for indicating error messages (in the theory, nothing is done with these messages). As previously mentioned, quasiquotes are merely a TSL that allows one to construct the abstract syntax, represented as this case type, using concrete syntax, with the addition of an unquote mechanism.

The \verb|parse| function for a type $t$ is called when checking a literal form against that type. This is seen in the key rule of our statics: \textit{T-literal}, in Fig. 9. The premises of these rules operate as follows:
\begin{enumerate}
\item A well-typed, rewritten parser object is extracted from the type's metadata. This is the step where the parser generator rewrites a grammar to a parse method, recursively using the TSL mechanism itself.
\item A tokenstream, of type \li{TokenStream}, is generated from the body of the literal. This type is an object that that allows the reading of tokens, as well as an additional method discussed in the next section for parsing the stream as a Wyvern expression.
\item The \li|parse| method is called with this extracted tokenstream to produce a syntax tree and a remaining tokenstream.
\item The syntax tree, $\hat{e}'$ is \emph{dereified} into its corresponding term, $e$ (the hat is gone because the generated syntax tree might itself use TSLs). The judgement for this is indexed by the variable context $\Gamma$ because that is needed to dereify \li{FromTS}, described below.
\item The dereified term is then recursively typechecked against the same type and rewritten, consistent with the semantics of TSLs as we have been describing them -- they must produce a term of the type they are being checked against.
\item The TSL must consume the entire token stream, so this is checked.
\end{enumerate}
\subsection{Hygiene}
 (the previous restriction ensured hygiene, but would break type safety if maintained because even TokenStreams generated by the would not be well typed).
 
\subsection{Safety}
\todo{type safety - Cyrus}
\todo{discussion on decidability - Cyrus}
\todo{ambiguity - Cyrus}
