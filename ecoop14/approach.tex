% !TEX root = ecoop14.tex
\section{Approach}
Use a core language, TSL Wyvern, based on the statics we've developed...

\subsection{Concrete Syntax}
have all the delimiter forms here; talk about phase 1 parsing
\todo{adams formalism}
\label{s:approach}
\subsubsection{Literal Forms}
\subsubsection{Whitespace-Delimited}

\subsection{Abstract Syntax}
\subsubsection{Extraction to Abstract Delimited Form}

\subsection{Types and Metavalues}
\todo{object types, case types, metavalues}
\subsection{Bidirectional Type System}
\subsection{Procedural Parsing}
\subsubsection{Quotations as TSLs}
\subsection{Safety}
\todo{type safety}
\todo{discussion on decidability}
\todo{ambiguity}
\subsection{Grammars as TSLs}
\subsubsection{Ambiguity Checking (related to van wyk's work?)}
\section{Related Work}
\todo{related work}
- Eelco Visser et al have a paper on preventing injection attacks using string embedding http://dl.acm.org/citation.cfm?id=1289975

- mention language boxes work discussed at Parsing workshop
\section{Discussion}
\todo{discussion}
%The examples above demonstrate the core mechanism used in Wyvern: each expression or declaration can contain at most one tilde (\verb|~|). The line following the term containing the tilde must begin an indented block. This block will be parsed according to a grammar associated with the expected type of the expression where the tilde occurred. In Figure \ref{f:dsl-arch}, this type was determined by an explicit type annotation. In Figures \ref{f:dsl-query} and \ref{f:dsl-presentation}, it was determined implicitly by the argument types of the function being called. Although a single tilde per expression may initially seem limiting, we can see in Figure 3 that the use of a \keyw{where} clause allows for the use of multiple DSLs within a single expression.
%
%Figure~\ref{f:dsl-if} shows how users equip types with domain-specific grammars, here for the Architecture example in Figure \ref{f:dsl-arch}. The grammar associated with \lstinline{Architecture} is defined at the top level by a specially-named production, \keyw{grammar}, and it also defines two sub-productions for component and policy specifications. Each component specification includes a name, a type and an optional list of attributes. The name is specified using the \keyw{ID} production, which is a globally-available production that matches valid Wyvern (i.e. the GPL's) identifiers. Similarly, the \keyw{TYPE} production matches Wyvern types. These productions cannot be extended directly, but can be used within DSLs as needed and thus conflicts are detected when the library is compiled, rather than deferred to link-time. 
%
%One component attribute in the Architecture DSL is \lstinline{location}. On line \ref{f:dsl-if}.6, we see the use of a production defined in another type, \verb|URL|. This means that only URL literals are valid at that position, but \emph{not} any Wyvern expression of type \verb|URL|. To instead ask for any Wyvern expression of a particular type, we use the notation used in the \lstinline{policy} production on line \ref{f:dsl-if}.8. Here, a named policy is defined as an identifier followed by an equals sign and a Wyvern expression of type \lstinline{Policy}. This is denoted by the form \keyw{EXP}\verb| : T|, where \verb|T| is a type in scope of the grammar definition. Unlike in the URL example above, if there is a DSL associated with this type, it can only be used within a whitespace-delimited block introduced by a tilde. This key distinction can be seen in Figure \ref{f:dsl-arch}. Again, it can be modularly verified that this grammar does not contain conflicts, assuming that the core Wyvern grammar does not change. We expect it to become stable relatively early in its development, with most new features introduced via the embedded DSL mechanism.
%
%\begin{figure}
%  \centering\small
%  \begin{lstlisting}[language=Python,morekeywords={type,grammar,where,ID,TYPE,EXP}]
%type Architecture
%  grammar ::= (component|policy)+ 
%  component ::= "external"? "component" 
%                     ID ":" TYPE 
%                    ((componentAttr)*)?
%  componentAttr ::= "location " URL.grammar
%                       | "connects to" (ID ",")* ID
%  policy ::= "policy" ID "=" (EXP : Policy)
%  \end{lstlisting}
%  \caption{Type-Associated Grammar}
%  \label{f:dsl-if}
%\end{figure}
%
%%In this example, we first parse the if condition using a standard
%%expression parser \texttt{StdExpParser}.  The expression parser is
%%passed a sub-stream that contains only tokens preceding the end of
%%the line.  After parsing the condition expression, we then accept an
%%\texttt{INDENT} token, indicating that we are now expecting one or
%%more indented lines.  This token is consumed by using a DSL for
%%parsing, which is modeled after BNF syntax.  The definition of this
%%parsing DSL is not shown, but it can be defined using the same approach.
%%%described in this section.
%%
%%After that, we use a standard statement parser to parse the indented lines as
%%a series of one or more statements and construct and return
%%an \texttt{IfAST} object representing the if statement.  The
%%\texttt{IfAST} class is an ordinary class that implements the
%%\texttt{AST} interface, which requires the class to define its semantics via
%%an interpreter or via translation to lower-level constructs.  The
%%details of the AST interface are out of scope here.  Once the
%%semantics of \keyw{if} are defined, we can use it as shown in
%%the \texttt{abs} method, which computes the absolute value of its
%%argument.
%%
%%
%\paragraph{Parsing and Typechecking}
%Wyvern source code is parsed in two phases. The first phase uses a standard declarative whitespace-delimited parser (e.g. \cite{rule2013principled}) where all whitespace-delimited blocks are left as unparsed ``DSL literals''. The second phase occurs in tandem with typechecking. When a tilde is encountered, the compiler determines the expected type where the tilde appeared (based on function signatures, or explicit type annotations) and parses the subsequent whitespace-delimited block according to the grammar associated with that type. The baseline indentation is stripped from this block, so it appears to the parser as if the DSL begins on the leftmost column of the block. Any Wyvern expressions that occur internally to a DSL (such as the policy specification in Figure \ref{f:dsl-arch} or curly-brace-delimited expressions in Figure \ref{f:dsl-presentation}) are also parsed and typechecked at this time. After parsing a DSL block, it must be verified and translated to Wyvern code. The mechanism for doing this is similar, at a high level, to that for defining the grammar itself, but the details of these subsequent steps are beyond the scope of this paper.
%
%\paragraph{Procedural Parsing}
%The grammar definition in Figure 4 is declarative and relies on a parser generator included as part of Wyvern. It may be desirable in some cases, such as when a grammar cannot easily be expressed within our declarative framework, or when an existing parser can be called into, to allow a parsing algorithms to be encoded directly. An important use case is for interoperability layers between Wyvern and other full-scale programming languages, particularly ones for which parsing is known to be difficult (e.g. C, Haskell, Python). 
%
%It can be observed that a declarative grammar inside a class definition can be seen as inducing a class method (that is, an operation defined on the class itself, which can be invoked by the compiler or run-time system -- a concept borrowed from Smalltalk \cite{smalltalk}) called \lstinline|parse|, that transforms a string to some AST representation. This lower-level interface is exposed directly to programmers who wish to specify parsing in a procedural manner.
%
%
%%using a standard whitespace-delimited lexer followed by an initial typechecking and parsing phase that occurs in tandem.  The lexer for Wyvern generates a stream of tokens structured hierarchically, where indentation (and the equivalent bracket notation) increases the depth of the stream. Indentation and de-indentation can thus be simply  understood as a special pair of tokens, \texttt{INDENT} and \texttt{DEDENT}, generated by the lexer and delimiting portions of the stream that will be handled in a special way during the next phase.
%%
%%The standard Wyvern parser reads at a single level of the stream at a time. Initially, it parses the stream as a term of sort \verb|Exp|, which represents the core expressions and declaration forms of the language. This sort includes basic facilities for declaring and working with functions and variables, objects and perhaps other to-be-determined core data types. When an indented block is encountered, however, it is left in-place within the parsed syntax tree. 
%
%%When the parsing phase at a particular level is complete, it is typechecked according to a largely conventional protocol. 
%%In Wyvern, the type of a function is determined by its declaration, and cannot be affected by its use (this constraint may be relaxed in some instances in future iterations of the language). Thus, when a function is being applied, the expected types of its arguments are known. If, at a particular location where a function argument was expected, an indented block is found, then it is parsed not according to the \verb|Exp| grammar, but according to the \emph{type-associated grammar} contained in that type's definition. Figure 4 shows an example of such a grammar. That grammar may contain productions of the form $\texttt{EXP of }\tau$ where $\tau$ is another type in scope. A delimited block must be present at this location as well, and it will be parsed by the type-associated grammar of $\tau$.
%%
%%A particular function application can contain only one \emph{primary} indented block. If multiple arguments support DSL syntax, then all those other than the first in the argument ordering must be invoked by a partially-indented keyword. This can be seen in Figure \ref{f:dsl-multi}. Note that this partially-indented keyword ends the first delimited block and begins a new block for the keyword argument \verb|toServer| of the function \verb|db.sendQuery|. This can be seen as similar in form to keywords in functions in Smalltalk \cite{smalltalk}.
%
%%Our approach supports not just key{\-}word-based extension, as in
%%\keyw{if}, but also type-based syntax extensions.  Consider the
%%database query DSL used in Figure~\ref{f:dsl-query}.  When parsing
%%this code, the standard Wyvern parser reads the identifier
%%\texttt{productDB} and discovers, based on the architecture from
%%Figure~\ref{f:dsl-arch}, that \texttt{productDB} has type
%%\texttt{Database}.  To support type-based syntax extension, the Wyvern
%%parser examines this type, defined in Figure~\ref{f:dsl-define}, and
%%discovers that the \texttt{Database} class object implements the
%%\texttt{Parser} interface.  Note that we associate the parser with the
%%\texttt{Database} \textit{class}, not the \texttt{Database}
%%instance. This works well because the class is available at compile
%%time, during parsing,
%%but the instance is not available until execution.
%
%%\begin{figure}
%%\centering
%%\begin{lstlisting}
%%db.sendQuery
%%    select x where x > 10
%%  toServer
%%    www.server.org
%%\end{lstlisting}
%%\caption{Function applications involving multiple white-space delimited arguments use partially-indented keywords.}
%%\label{f:dsl-multi}
%%\end{figure}
%%\begin{figure}
%%  \centering
%%  \begin{lstlisting}
%%class Database class implements Parser
%%	class meth parse(db : AST,
%%	                 stream : TokStream):AST
%%		if (stream ::= "where")
%%			// parse where clause
%%		else ... // parse other constructs
%%	... // other definitions for Database
%%
%%// equivalent code for if using types
%%class IfKeyType class implements Parser
%%	class meth parse(ifKeyword : AST,
%%	                 stream:ParseStream):AST
%%		val cond = ... // same code as previous Figure
%%	class prop instance = new IfKeyType // singleton
%%val (@if@) = IfKeyType.instance
%%  \end{lstlisting}
%%  \caption{Type-based DSL definition}
%%  \label{f:dsl-define}
%%\end{figure}
%%
%
