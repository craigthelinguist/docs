% !TEX root = ecoop14.tex
\section{Syntax}
\label{s:approach}

\subsection{Concrete Syntax}
We will now describe the concrete syntax of Wyvern declaratively, using the same layout-sensitive formalism that we have introduced for TSL grammars, developed recently by Adams \cite{Adams:2013:PPI:2429069.2429129}. Such a formalism is useful because it allows us to implement  layout-sensitive syntax, like that we've been describing, without relying on context-sensitive lexers or parsers. Most existing layout-sensitive languages (e.g. Python and Haskell) use hand-rolled context-sensitive lexers or parsers (keeping track of, for example, the indentation level using special \li{INDENT} and \li{DEDENT} tokens), but these are more problematic because they cannot be used to generate editor modes, syntax highlighters and other tools automatically. In particular, we will show how the forward references we have described can be correctly encoded without requiring a context-sensitive parser or lexer using this formalism. It is also useful that the TSL for \li{Parser}, above, uses the same parser technology as the host language, so that it can be used to generate quasiquotes.

Wyvern's concrete syntax, with a few minor omissions for concision, is shown in Figure~\ref{f-grammar}. We first review Adams' formalism in some additional detail, then describe some key features of this syntax.

\subsection{Background: Adams' Formalism}
For each terminal and non-terminal in a rule, Adams proposed associating with them a relational operator, such as =, > and $\geq$ to specify the indentation at which those terms need to be with respect to the non-terminal on the left-hand side of the rule. The indentation level of a term can be identified as the column at which the left-most character of that term appears (not simply the first character, in the case of terms that span multiple lines). The meaning of the comparison operators is akin to their mathematical meaning: = means that the term on the right-hand side has to be at exactly the same indentation as the term on the left-hand side; >  means that the term on the right-hand side has to be indented strictly further to the right than the term on the left-hand side; $\geq$ is like >, except the term on the right could also be at the same indentation level as the term on the left-hand side. For example, the production rule of the form \lstinline{A $\rightarrow$ B$^=$ C$^\geq$ D$^>$} approximately reads as: ``Term \lstinline{B} must be at the same indentation level as term \lstinline{A}, term \lstinline{C} may be at the same or a greater indentation level as term \lstinline{A}, and term \lstinline{D} must be at an indentation level greater than term \lstinline{A}'s.'' In particular, if \li{D} contains a \lstinline{NEWLINE} character, the next line must be indented past the position of the left-most character of \lstinline{A} (typically constructed so that it must appear at the beginning of a line). There are no constraints relating \lstinline{D} to \lstinline{B} or \lstinline{C} other than the standard sequencing constraint: the first character of \lstinline{D} must be to further in the file than the others. Using Adam's formalism, the grammars of real-world languages like Python and Haskell can be written declaratively. This formalism can be integrated into LR and LALR parser generators.

% !TEX root = ecoop14.tex

\begin{figure}
\begin{lstlisting}[mathescape]
p $\rightarrow$ 'objtype'$^=$ ID$^>$ NEWLINE$^>$ objdecls$^>$ NEWLINE$^>$ metadatadecl$^>$ NEWLINE$^>$ p$^=$
p $\rightarrow$ 'casetype'$^=$ ID$^>$ NEWLINE$^>$ casedecls$^>$ NEWLINE$^>$ metadatadecl$^>$ NEWLINE$^>$ p$^=$
p $\rightarrow$ e$^=$

metadatadecl $\rightarrow$ 'metadata'$^=$ '='$^>$ e$^>$

e $\rightarrow$ $\overline{\texttt{e}}$$^=$
e $\rightarrow$ $\widetilde{\texttt{e}}$['~']$^=$ NEWLINE$^>$ chars$^>$
e $\rightarrow$ $\widetilde{\texttt{e}}$['new']$^=$ NEWLINE$^>$ d$^>$
e $\rightarrow$ $\widetilde{\texttt{e}}$['case(' $\overline{\texttt{e}}$ ')']$^=$ NEWLINE$^>$ c$^>$

$\overline{\texttt{e}}$ $\rightarrow$ ID$^=$
$\overline{\texttt{e}}$ $\rightarrow$ 'fn'$^=$ ID$^>$ ':'$^>$ type$^>$ '=>'$^>$ $\overline{\texttt{e}}$$^=$
$\overline{\texttt{e}}$ $\rightarrow$ $\overline{\texttt{e}}$$^=$ '('$^>$ $\overline{\texttt{al}}$$^=$ ')'$^>$
$\overline{\texttt{e}}$ $\rightarrow$ 'let'$^=$ ID$^>$ ':'$^>$ type$^>$ '='$^>$ e$^>$ NEWLINE$^>$ $\overline{\texttt{e}}$$^=$
$\overline{\texttt{e}}$ $\rightarrow$ $\overline{\texttt{e}}$$^=$ '.'$^>$ ID$^>$
$\overline{\texttt{e}}$ $\rightarrow$ type$^=$ '.'$^>$ ID$^>$ '('$^>$ $\overline{\texttt{e}}$$^>$ ')'$^>$
$\overline{\texttt{e}}$ $\rightarrow$ $\overline{\texttt{e}}$$^=$ ':'$^>$ type$^>$
$\overline{\texttt{e}}$ $\rightarrow$ 'valAST'$^=$ '('$^>$ $\overline{\texttt{e}}$$^>$ ')'$^>$
$\overline{\texttt{e}}$ $\rightarrow$ type$^=$ '.'$^>$ 'metadata'$^>$
$\overline{\texttt{e}}$ $\rightarrow$ inlinelit$^=$

$\widetilde{\texttt{e}}$[fwd] $\rightarrow$ fwd$^=$
$\widetilde{\texttt{e}}$[fwd] $\rightarrow$ 'fn'$^=$ ID$^>$ ':'$^>$ type$^>$ '=>'$^>$ $\widetilde{\texttt{e}}$[fwd]$^>$
$\widetilde{\texttt{e}}$[fwd] $\rightarrow$ $\widetilde{\texttt{e}}$[fwd]$^=$ '('$^>$ $\overline{\texttt{al}}$$^>$ ')'$^>$
$\widetilde{\texttt{e}}$[fwd] $\rightarrow$ 'let'$^=$ ID$^>$ ':'$^>$ type$^>$ '='$^>$ e$^>$ NEWLINE$^>$ $\widetilde{\texttt{e}}$[fwd]$^=$
$\widetilde{\texttt{e}}$[fwd] $\rightarrow$ $\overline{\texttt{e}}$$^=$ '('$^>$ $\widetilde{\texttt{al}}$[fwd]$^>$ ')'$^>$
$\widetilde{\texttt{e}}$[fwd] $\rightarrow$ $\widetilde{\texttt{e}}$[fwd]$^=$ '.'$^>$ ID$^>$
$\widetilde{\texttt{e}}$[fwd] $\rightarrow$ type$^=$ '.'$^>$ ID$^>$ '('$^>$ $\widetilde{\texttt{e}}$[fwd]$^>$ ')'$^>$
$\widetilde{\texttt{e}}$[fwd] $\rightarrow$ $\widetilde{\texttt{e}}$[fwd]$^=$ ':'$^>$ type$^>$
$\widetilde{\texttt{e}}$[fwd] $\rightarrow$ 'valAST'$^=$ '('$^>$ $\widetilde{\texttt{e}}$[fwd]$^>$ ')'$^>$

d $\rightarrow$ $\varepsilon$
d $\rightarrow$ 'val'$^=$ ID$^>$ ':'$^>$ type$^>$ '='$^>$ e$^>$ NEWLINE$^>$ d$^=$
d $\rightarrow$ 'def'$^=$ ID$^>$ '('$^>$ argsig$^>$ ')'$^>$ ':'$^>$ type$^>$ '='$^>$ e$^>$ NEWLINE$^>$ d$^=$

c $\rightarrow$ ID$^=$ '('$^>$ ID$^>$ ')'$^>$ '=>'$^>$ e$^>$
c $\rightarrow$ ID$^=$ '('$^>$ ID$^>$ ')'$^>$ '=>'$^>$ e$^>$ NEWLINE$^>$ c$^=$

$\overline{\texttt{al}}$ $\rightarrow$ $\varepsilon$ | $\overline{\texttt{al}}_{\texttt{nonempty}}$$^=$
$\overline{\texttt{al}}_{\texttt{nonempty}}$ $\rightarrow$ $\overline{\texttt{e}}$$^=$ | $\overline{\texttt{e}}$$^=$ ','$^>$ $\overline{\texttt{al}}_{\texttt{nonempty}}$$^>$

$\widetilde{\texttt{al}}$[fwd] $\rightarrow$ $\widetilde{\texttt{e}}$[fwd]$^=$
$\widetilde{\texttt{al}}$[fwd] $\rightarrow$ $\widetilde{\texttt{e}}$[fwd]$^=$ ','$^>$ $\overline{\texttt{al}}_{\texttt{nonempty}}$$^>$
$\widetilde{\texttt{al}}$[fwd] $\rightarrow$ $\overline{\texttt{e}}$$^=$ ','$^>$ $\widetilde{\texttt{al}}$[fwd]$^>$

inlinelit $\rightarrow$ chars1['`']$^=$ | chars1[''']$^=$ | chars1['"']$^=$ | ...
inlinelit $\rightarrow$ chars2['{', '}']$^=$ | chars2['<', '>']$^=$ | chars2['[', ']']$^=$ | ...
inlinelit $\rightarrow$ numlit$^=$
\end{lstlisting}
\caption{Concrete Syntax}
\label{f-grammar}
\end{figure}

\begin{figure}
\begin{lstlisting}[mathescape]
`TSL code here, ``inner backticks`` must be doubled`
'TSL code here, ''inner single quotes'' must be doubled'
'TSL code here, ""inner double quotes"" must be doubled'

{TSL code here, {inner braces} must be balanced}
[TSL code here, [inner brackets] must be balanced]
<TSL code here, <inner angle brackets> must be balanced>
\end{lstlisting}
\caption{TSL Delimiters}
\label{f-delims}
\end{figure}


\subsection{Programs}

\begin{figure}[t]
\begin{minipage}[t]{.54\textwidth}
\begin{lstlisting}
objtype T
  val y : HTML
let page : HTML->HTML = fn x:HTML => ~
  :html
    :body
      {x}
page(case(5 : Nat))
  Z(_) => (new : T).y
    val y : HTML = ~
      :h1 Zero!
  S(x) => ~
    :h1 Successor!
\end{lstlisting}
\end{minipage}%
\begin{minipage}[t]{.45\textwidth}
 \centering
\[
\begin{array}{l}
\keyw{objtype} \ T\ \{ \\
  ~~~\keyw{val}\ y : HTML, \\
  ~~~\keyw{metadata} = (\keyw{new}\ \{\}) : Unit\ \}; \\
  (\boldsymbol\lambda page : HTML \rightarrow HTML.\\
  ~page(\keyw{case}(\lfloor 5 \rfloor : Nat)\ \{ \\
  ~~~~Z(\_) \Rightarrow ((\keyw{new}\ \{ \\
    ~~~~~~~\keyw{val}\ y : HTML = \lfloor :h1\ Z! \rfloor\}) : T).y\ \\
   ~~~~| S(x) \Rightarrow \lfloor :h1\ S! \rfloor\})) \\
  ~~~~(\boldsymbol\lambda x : HTML.\ \lfloor :html \\
    ~~:body \\
      ~~~~\{x\} \rfloor)
\end{array}
\]
\end{minipage}
\caption{An example Wyvern program demonstrating forward references. The corresponding abstract syntax, where forward references are inlined, is on the right.}
\label{fig:fwd-ref}
\vspace{-10px}
\end{figure}
An example Wyvern program showing several unique syntactic features of TSL Wyvern is shown in Fig. \ref{f-example}. The top level of a program (the \lstinline{p} non-terminal) consists of a series of type declarations -- object types using \lstinline{objtype} or case types using \lstinline{casetype} -- followed by an expression, \lstinline{e}. Each type declaration contains associated declarations -- signatures for fields and methods in  \lstinline{objdecls} and case declarations in \lstinline{casedecls} (not shown on the figure). Each also can also include a metadata declaration. Metadata is simply an expression associated with the type, used to store TSL logic (and in future work, other logic). Sequences of top-level declarations use the form \lstinline{p$^=$} to signify that all the succeeding \lstinline{p} terms must begin at the same indentation.

\subsection{Forward Referenced Blocks}
Wyvern makes extensive use of forward referenced blocks to make its syntax clean. In particular, layout-delimited TSLs, the general-purpose introductory form for object types and the elimination form for case types and product types all use forward referenced blocks. Fig. \ref{fig:fwd-ref} shows all of these in use (assuming suitable definitions of casetypes \li{Nat} and \li{HTML}, not included). In the grammar, note particularly the rules for \li{let} and that inline literals, even those containing nested expressions with forward references, can be treated as expressions not containing forward references -- \emph{in the initial phase of parsing, before typechecking commences, all literal forms are left unparsed}.

\subsection{Abstract Syntax}
The concrete syntax of a Wyvern program, \li{p}, is parsed to produce a program in the abstract syntax, $\rho$, shown on the left side of Fig. \ref{fig:core2-syntax}. Forward references are internalized. In particular, note that all literal forms are unified into the abstract literal form $\lfloor body \rfloor$, including the layout-delimited form and number literals. The abstract syntax contains a form, $\keyw{fromTS}(e)$, that has no analog in the concrete syntax. This will be used internally to ensure hygiene, as we will discuss in the next section.
\newcommand{\ih}{\hat{\imath}}
\begin{figure}[t]
$
\begin{array}[t]{lcl} 
\rho & \bnfdef & {\keyw{objtype}[T, \omega, e]; \rho} \\
     & \bnfalt & {\keyw{casetype}[T, \chi, e]; \rho}\\
     & \bnfalt & \keyw{expr}[e]
\end{array}
$~~~
$\begin{array}[t]{lcl}
\omega & \bnfdef & \emptyset \bnfalt \keyw{membdecl}[\ell, \tau]; \omega\\
\chi & \bnfdef & \emptyset \bnfalt \keyw{casedecl}[C, \tau]; \chi\\
\tau & \bnfdef & \keyw{named}[T]  \bnfalt \keyw{arrow}[\tau, \tau]
\end{array}$\\
% & \bnfalt & \tau \times \tau
%\end{array}$~
%$\begin{array}[t]{lcl}
%\omega & \bnfdef & \keyw{membdecl}[f, \tau]; \omega\\
%% & \bnfalt & \keyw{defdecl}[f, \tau, \tau]\\
% & \bnfalt & \cdot
%\end{array}$~
%$\begin{array}[t]{lcl}
%\chi & \bnfdef & \keyw{casedecl}[C, \tau]; \chi\\
% & \bnfalt & \cdot
%\end{array}
%$\\
~\\
$
\begin{array}[t]{lcl} 
e    & \bnfdef & x \\
     & \bnfalt & \keyw{easc}[\tau](e)\\
     & \bnfalt & \keyw{elam}(x . e) \\ %
     & \bnfalt & \keyw{eap}(e; e) \\
%     & \bnfalt & \keyw{pair}(e; e) \\
%     & \bnfalt & \keyw{prj}(e; x,y.e)\\
     & \bnfalt & \keyw{enew}\, \{ m \}\\
     & \bnfalt & \keyw{eprj}[\ell](e) \\
%     & \bnfalt & \keyw{call}[m](e; e) \\
     & \bnfalt & \keyw{einj}[C](e) \\
     & \bnfalt & \keyw{ecase}(e)\,\{ r \} \\
     & \bnfalt & \keyw{etoast}(e) \\
     & \bnfalt & \keyw{emetadata}[T]\\
     & \bnfalt & \keyw{lit}[body]
\\[1ex]	
d   & \bnfdef & \emptyset\\
     & \bnfalt & \keyw{eval}[\ell](e); d\\
     & \bnfalt & \keyw{edef}[\ell](x.e); d
\\[1ex]
r    & \bnfdef & \emptyset\\
     & \bnfalt & \keyw{erule}[C](x.e); r
\end{array}
\begin{array}[t]{lcl}
~~~~~~
\end{array}
\begin{array}[t]{lcl}
\ih    & \bnfdef & x \\
     & \bnfalt & \keyw{hasc}[\tau](\ih)\\
     & \bnfalt & \keyw{hlam}(x . \ih) \\ %
     & \bnfalt & \keyw{hap}(\ih; \ih) \\
%     & \bnfalt & \keyw{pair}(\ih; \ih) \\
%     & \bnfalt & \keyw{prj}(\ih; x,y.\ih)\\
     & \bnfalt & \keyw{hnew}\, \{ \hat{d} \}\\
     & \bnfalt & \keyw{hprj}[\ell](\ih) \\
%     & \bnfalt & \keyw{call}[m](\ih; \ih) \\
     & \bnfalt & \keyw{hinj}[C](\ih) \\
     & \bnfalt & \keyw{hcase}(\ih)\,\{ \hat{r} \} \\
     & \bnfalt & \keyw{htoast}(\ih) \\
     & \bnfalt & \keyw{hmetadata}[T]\\
     & \bnfalt & \keyw{spliced}[e]
\\[1ex]	
\hat{d}   & \bnfdef & \emptyset \\
	 & \bnfalt & \keyw{hval}[\ell](\ih); \hat{d}\\
     & \bnfalt & \keyw{hdef}[\ell](x.\ih); \hat{d}
\\[1ex]
\hat{r}    & \bnfdef & \emptyset \\
      & \bnfalt & \keyw{hrule}[C](x.\ih); \hat{r}
%\\[1ex]
%\\
%\\
%\chi & \bnfdef & C~\keyw{of}~\tau\\
%     & \bnfalt & \chi \bnfalt \chi 
%\\[1ex]
%\omega &\bnfdef & \varepsilon \\  
%         & \bnfalt & \keyw{val}~ f:\tau;~\omega\\
%         & \bnfalt & \keyw{def}~ m:\tau;~\omega 
%\\[1ex]
\end{array}
\begin{array}[t]{lll}
~~~~~~
\end{array}
\begin{array}[t]{lcl} 
i    & \bnfdef & x \\
     & \bnfalt & \keyw{iasc}[\tau](i)\\
     & \bnfalt & \keyw{ilam}(x . i) \\ %
     & \bnfalt & \keyw{iap}(i; i) \\
%     & \bnfalt & \keyw{pair}(i; i) \\
%     & \bnfalt & \keyw{prj}(i; x,y.i)\\
     & \bnfalt & \keyw{inew}\, \{ \dot{d} \}\\
     & \bnfalt & \keyw{iprj}[\ell](i) \\
%     & \bnfalt & \keyw{call}[m](i; i) \\
     & \bnfalt & \keyw{iinj}[C](i) \\
     & \bnfalt & \keyw{icase}(i)\,\{ \dot{r} \} \\
     & \bnfalt & \keyw{itoast}(i) \\
     & \bnfalt & \keyw{imetadata}[T]\\
     & ~ & ~%& \bnfalt & \keyw{spliced}[e]
\\[1ex]	
\dot{d}   & \bnfdef & \emptyset\\
     & \bnfalt & \keyw{ival}[\ell](i); \dot{d} \\
     & \bnfalt & \keyw{idef}[\ell](x.i); \dot{d}
\\[1ex]
\dot{r}    & \bnfdef & \emptyset\\
     & \bnfalt & \keyw{irule}[C](x.i); \dot{r}
%\\[1ex] 
%\\
%\\
%\\
%\tau & \bnfdef & t\\
%     & \bnfalt & \tau \rightarrow \tau \\
%     & \bnfalt & \tau \times \tau 
%\\[1ex]
%\Gamma & \bnfdef & \emptyset \bnfalt \Gamma, x:\tau~~~~~~~\delta \bnfdef - \bnfalt \hat{\imath} : \tau
%\\[1ex]
%\Sigma & \bnfdef & \emptyset \bnfalt \Sigma, t:\{\chi, \delta\} \bnfalt \Sigma, t:\{\omega, \delta\}
\end{array}
$
%\vspace{-12px}
\caption{Abstract Syntax of TSL Wyvern. $T$ ranges over type name, $\ell$ ranges over object member labels, $C$ ranges over case labels, $x$ ranges over variables and $body$ ranges over literal bodies. Tuples can be introduced as derived forms: $(i_1, i_2) := \keyw{inew}\,\{\keyw{ival}[\ell_1](i_1); \keyw{ival}[\ell_2](i_2)\}$.
}
\vspace{-15px}
\label{fig:core2-syntax}
\end{figure}



\section{Bidirectional Typechecking and Literal Rewriting}
We will now specify a type system for the abstract syntax in Fig. \ref{fig:core2-syntax}. Conventional type systems are specified using a typechecking judgement like $\Delta; \Gamma \vdash e : \tau$, where the variable context, $\Gamma$, tracks the types of variables, and the type context, $\Delta$, tracks types and their signatures. However, this conventional formulation does not separately consider how, when deriving this judgement, it will be considered algorithmically -- will a type be provided, so that we simply need to check $e$ against it, or do we need to synthesize a type for $e$? For our system, this distinction is crucial: a generic literal can only be used in the first situation. 

\emph{Bidirectional type systems}, as presented by Lovas and Pfenning \cite{Lovas08abidirectional}, make this distinction clear by specifying the type system instead using two simultaneously defined judgements: one for expressions that can \emph{synthesize} a type based on the surrounding context (e.g. variables and elimination forms), and another for expressions for which we know what type to \emph{check} or \emph{analyze} the term against (e.g. generic literals and some introductory forms). 
Our work builds upon this work, making the following core additions: the tye context $\Delta$ now tracks the metadata in addition to type signatures, and as we typecheck, we need to also perform literal rewriting by calling the parser associated with the type that a literal is being analyzed against, typechecking the AST it produces and ensuring that hygiene is maintained.

The judgement 
\fbox{$\Delta; \Gamma'; \Gamma \vdash e\Rightarrow \tau \leadsto \hat{e}$} 
means that from the type context $\Delta$, the \emph{surrounding variable context}, $\Gamma'$, and the \emph{local variable context}, $\Gamma$, we synthesize the type $\tau$ for $e$ and rewrite it to  $\hat{e}$ (which does not contain literals or the special form $\keyw{fromTS}(e)$, which makes the surrounding context available; see below).
The judgement 
\fbox{$\Delta; \Gamma'; \Gamma \vdash e \Leftarrow \tau \leadsto \hat{e}$} similarly 
means that we check $e$ against the type $\tau$ and the expression $e$ is rewritten into the expression $\hat{e}$. The forms of $\Gamma$ and $\Delta$ are given in Fig. \ref{fig:core2-syntax}. Note that $\Delta$ carries the type's signature as well as the rewritten form of the metadata. The rules for these judgements, as well as key rules for several auxiliary judgements that are needed in their premises, are given in Figs. 8-11. 

These rules assume that a collection of built-in types are included by default at the top of programs (e.g. \li{Unit}, \li{Parser}, \li{Exp} already mentioned, and a few others), captured by an intial type context $\Delta_0$. We show the concrete syntax for the two key ones in Fig. 7. The static semantics and the dynamic semantics (defined for $\hat{e}$ only) are that of a conventional functional language with functions, inductive datatypes, products and records  with the addition of a few new forms. The key new dynamic semantics rules are described in Figs. 12 and 13. We will now describe how some of the novel rules that support TSLs work below. We refer the reader to \cite{Lovas08abidirectional} and texts on type systems, e.g. \cite{pfpl,tapl}, for the remainder.

\begin{figure}[t]
\begin{subfigure}[t]{.55\textwidth}
\begin{lstlisting}
objtype Parser                          
  def parse(ts : TokenStream) : (Exp * 
    TokenStream)
  metadata = new                        
    val parser : Parser = new           
      val parse(ts : TokenStream) : (
          Exp * TokenStream) =            
        (* parser generator based
           on Adams' formalism *)
\end{lstlisting}
\end{subfigure}
\begin{subfigure}[t]{.55\textwidth}
\begin{lstlisting}[linewidth=.5\textwidth]
casetype Exp 
    Var of ID
  | Lam of ID * Type * Exp
  | App of Exp * Exp
  ... 
  | FromTS of Exp * Exp
  | Literal of TokenStream
  | Error of ErrorMessage
  metadata = (* quasiquotes *)
\end{lstlisting}
\end{subfigure}
%\begin{lstlisting}
%objtype Parser                                casetype Exp 
%  def parse(ts : TokenStream) : Exp               Var of ID
%  metadata = new                                | Lam of ID * Type * Exp
%    val parser : Parser = new                   | App of Exp * Exp
%      val parse(ts : TokenStream) :             ... | FromTS of Exp * Exp
%         Exp * TokenStream =                    | Literal of TokenStream
%           ... parser generator based on        | Error of ErrorMessage
%           Adams' formalism here ...            metadata = (* quasiquotes *)
%\end{lstlisting}
\caption{Two of the built-in types included in $\Delta_0$ (concrete syntax).}
\vspace{-10px}
%\label{fig:typeParser}
\end{figure}
%\subsection{Bidirectional Type System}
%Our type-checking is syntax-directed and it requires terms to be fully annotated with types where necessary. A commonly used approach for making the syntax more compact is bidirectional type systems, . They introduce two mutually recursive judgements: one for expressions that have enough information in the context to synthesize a type, and one for expressions for which we know what type to expect, thus only needing to check against that type. Unique types can be determined for synthesis expressions, while analytical expressions have to be verified to have the right types. We chose to use bidirectional type systems in our formalism because they leverage the simplicity of syntax-directed type-checking while not needing to carry much additional type information.
%
%In conventional bidirectional type systems, for constructors of a type one can propagate the type information $\tau$
%into the term $e$, which means it should be used in the analysis
%judgment $e \Leftarrow \tau$. When constructing a type, we do not have information about it and it is intuitive to use the analysis judgement, which is weaker than the synthesis judgement. On the other hand, destructors generate a result of a smaller type from a component of larger type and can be used for synthesis, propagating type information away from the term as in the synthesis judgement $e \Rightarrow \tau$. Our static semantics rules follow this conventional way of reasoning about constructors and destructors.

\subsection{Defining a TSL Manually}
% !TEX root = ecoop14.tex
\section{Statics}

\todo{This section on Statics is going to go and simply be integrated into approach in the right places, right? Right now it is just a list of figures and some text descrbining parts of them, not a coherent section as such, right?}

We present the abstract syntax of our system in Figure \ref{fig:core2-syntax}. A program $\rho$ is composed of a series of object type \keyw{objtype} and sum type \keyw{casetype} declarations, followed by expressions $e$. An object type is made of declarations of values \keyw{val}, methods \keyw{def} and \keyw{metadata}. A sum type is made of an enumeration of cases of the form $C$ \keyw{of} $\tau$, where $C$ is the name of the constructor and $\tau$ is the type of the expression constructed in this case, followed by metadata.

The metadata of a type $t$ (either an \keyw{objtype} or a \keyw{casetype}) can contain arbitrary data, for eg. documentation, but it will necessarily contain a $parser$ field of type $Parser$. The $parser$ field has a $parse$ method that takes as argument a stream of tokens and generates an abstract syntax tree expression. This $parse$ method is used for parsing new TSLs of type $t$ defined by the user. 

We differentiate between expressions that might contain a TSL expression and those that definitely do not contain a TSL expression by superscripting the latter with the symbol $\hat{}$. Thus we have two versions (one without $\hat{}$ and one with $\hat{}$) for programs $\rho$,  expressions $e$, cases $c$ of the sum types, and declarations $d$ of fields and methods. 

An expression can be a variable, a function, an application of a function to an expression, a pair of expressions, a case analysis on pairs, the constructor of a case type, the destructor of a case type, a new expression declaring fields and methods, the invocation $e.x$ of a field or a method. We do not have $e.m$ and $e.f$ in the abstract syntax because the parser cannot differentiate between the two, only the type checker can do that. An expression can also be an expression with a type ascribed to it, the abstract syntax tree of an expression, a metaobject of a type, a TSL literal or an expression obtained from a token stream. 








\begin{figure}
\centering
\[
\begin{array}[t]{lll} 
\rho & \bnfdef & \keyw{objtype}~ t~ \{ \omega, \keyw{metadata}=e \}; \rho \\
     & \bnfalt & \keyw{casetype}~ t~ \{ \chi, \keyw{metadata}=e \}; \rho\\
     & \bnfalt & e
     \\[1ex]
e    & \bnfdef & x \\
     & \bnfalt & \boldsymbol\lambda x{:}\tau . e \\ %
     & \bnfalt & e(e) \\
     & \bnfalt & (e, e) \\
     & \bnfalt & \keyw{case}(e) \{(x, y) \Rightarrow e\}\\
     & \bnfalt & t.C(e) \\
     & \bnfalt & \keyw{case}(e)~\{ c \} \\
     & \bnfalt & \keyw{new}~ \{ d \}\\
     & \bnfalt & e.x \\
     & \bnfalt & e : \tau\\
     & \bnfalt & \keyw{valAST}(e) \\
     & \bnfalt & t.\keyw{metaobject}\\
     & \bnfalt & \lfloor literal \rfloor \\
     & \bnfalt & \keyw{fromTS}[\Gamma](e, e)
\\[1ex]	
c    & \bnfdef & C(x) \Rightarrow e\\
     & \bnfalt & c \bnfalt c
	 \\[1ex]
d   & \bnfdef & \varepsilon \\
     & \bnfalt & \keyw{val}~ f:\tau = e;~d \\
     & \bnfalt & \keyw{def}~ m:\tau = e;~d
\\[1ex] 
\end{array}
\begin{array}[t]{lll}
~~~
\end{array}
\begin{array}[t]{lll}


\hat\rho & \bnfdef & \keyw{objtype}~ t~ \{ \omega, \keyw{metadata}=\hat e \}; \hat\rho \\
     & \bnfalt & \keyw{casetype}~ t~ \{ \chi, \keyw{metadata}=\hat e \}; \hat\rho\\
     & \bnfalt & \hat e
     \\[1ex]
\hat{e}    & \bnfdef & x \\
     & \bnfalt & \boldsymbol\lambda x{:}\tau . \hat{e} \\ %
     & \bnfalt & \hat{e}(\hat{e}) \\
     & \bnfalt & \cdots \\
     & \bnfalt & t.\keyw{metaobject} 
\\[1ex]
\hat c    & \bnfdef & ...
	 \\[1ex]
\hat d   & \bnfdef & ... 
\\[1ex] 
\chi & \bnfdef & C~\keyw{of}~\tau\\
     & \bnfalt & \chi \bnfalt \chi 
\\[1ex]
\omega &\bnfdef & \varepsilon \\  
         & \bnfalt & \keyw{val}~ f:\tau;~\omega\\
         & \bnfalt & \keyw{def}~ m:\tau;~\omega 
\\[1ex]
\tau & \bnfdef & t\\
     & \bnfalt & \tau \rightarrow \tau \\
     & \bnfalt & \tau \times \tau 
\\[1ex]
\Gamma & \bnfdef & \emptyset \bnfalt \Gamma, x:\tau
\\[1ex]
\Delta & \bnfdef & \emptyset \bnfalt \Delta, t:\{\chi, e:\tau\} \bnfalt \Delta, t:\{\omega, e:\tau\}
\\[1ex]

\end{array}
\]
\caption{Abstract Syntax}
\label{fig:core2-syntax}
\end{figure}


\begin{figure}
\centering
\[
\begin{array}{ll}
\keyw{casetype}\ & Exp=\\
& \ \ \ Var\ of\ ID\\
& \bnfalt \ Lam\ of\ ID\ *\ Ty\ *\ Exp\\
& \bnfalt \ App\ of\ Exp\ *\ Exp\\
& \cdots\\
& \bnfalt \ FromTokenStream\ of\ Exp\ *\ Exp\\
& \bnfalt \ Error\\
\\
\keyw{casetype}\ & Ty=\\
& \ \ \ Var\ of\ ID\\
& \bnfalt \ Arrow\ of\ Ty*Ty\\
\\ 
\end{array}
\]
\caption{Syntax Trees of Expressions and Types}
\end{figure}





\begin{figure}
\centering
\[
\begin{array}{c}

\infer[\textit{RT-objtype}]
          {\renewcommand{\arraystretch}{1}
	    \begin{array}{r}
	    \Delta; \Gamma \vdash  \keyw{objtype}~ t~=\{{\omega}, \keyw{metaobject}=e\}; \rho: \tau'\leadsto\\
            \keyw{objtype}~ t~=\{{\omega}, \keyw{metaobject}=\hat{e}\}; \hat{\rho}
            \end{array}
       }
	  {\Delta \vdash \omega & \Delta; \Gamma \vdash e \Rightarrow \tau \leadsto \hat{e} & \Delta, t:\{\omega, \hat e:\tau\}; \Gamma \vdash \rho :\tau'\leadsto \hat{\rho} }
	   \\[3ex] 


\infer[\textit{RT-casetype}]
          {\renewcommand{\arraystretch}{1}
	    \begin{array}{r}
	    \Delta; \Gamma \vdash  \keyw{casetype}~ t~=\{\chi, \keyw{metaobject}=e\}; \rho :\tau' \leadsto \\
            \keyw{casetype}~ t~=\{\chi, \keyw{metaobject}=\hat{e}\};\hat{\rho}
            \end{array}
       }
	  {\Delta \vdash \chi & \Delta; \Gamma \vdash e \Rightarrow \tau \leadsto \hat{e} & \Delta, t:\{\chi, \hat e:\tau\}; \Gamma \vdash \rho :\tau'\leadsto \hat{\rho} }
	   \\[3ex] 


\infer[\textit{RT-e}]
	{\Delta; \Gamma \vdash  e:\tau \leadsto \hat{e}} 
	{\Delta; \Gamma \vdash e \Rightarrow \tau \leadsto \hat{e}}\\[3ex]

\infer[\textit{C-decl}]
	{\Delta; \Gamma \vdash  C~\keyw{of}~\tau} 
	{\Delta \vdash \tau   }\\[3ex]

\infer[\textit{C-decls}]
	{\Delta; \Gamma \vdash  \chi_1 \bnfalt \chi_2} 
	{\Delta; \Gamma \vdash \chi_1 & \Delta; \Gamma \vdash \chi_2 & \text{dom}(\chi_1) \intersect \text{dom}(\chi_2) = \emptyset}\\[3ex]

\infer[\textit{O-val}]
	{\Delta; \Gamma \vdash \keyw{val}~ f:\tau \ \texttt{ok} }
	{\Delta \vdash \tau} \\[3ex]
	
\infer[\textit{O-def}]
	{\Delta; \Gamma \vdash \keyw{def}~ m:\tau \ \texttt{ok} }
	{\Delta \vdash \tau } \\[3ex]

\infer[\textit{O-defs}]
	{\Delta; \Gamma \vdash \omega_1\ \omega_2  }
	{\Delta \vdash \omega_1 & \Delta \vdash \omega_2 & \text{dom}(\omega_1) \intersect \text{dom}(\omega_2) = \emptyset } \\[3ex]

\infer[\textit{Syn2Check}]
	{\Delta; \Gamma \vdash  e \Leftarrow \tau \leadsto \hat{e}} 
	{\Delta;\Gamma \vdash e \Rightarrow \tau \leadsto \hat{e}   }\\[3ex]
	
\infer[\textit{T-varx}]
	{\Delta,\Gamma \vdash x\Rightarrow\tau } 
	{x:\tau \in \Gamma }\\[3ex]

\infer[\textit{T-abs}]
	{\Delta; \Gamma \vdash  \boldsymbol\lambda x{:}\tau . e \Leftarrow \tau \rightarrow \tau_1 \leadsto \boldsymbol\lambda x{:}\tau .\hat{e}} 
	{\Delta; \Gamma, x:\tau \vdash e\Leftarrow \tau_1 \leadsto \hat{e}  & \Delta\vdash \tau}\\[3ex]

\infer[\textit{T-appl}]
	{\Delta; \Gamma \vdash  e(e_1) \Rightarrow \tau_2  \leadsto \hat{e}(\hat{e}_1) } 
	{\Delta; \Gamma \vdash e \Rightarrow \tau_1 \rightarrow \tau_2  \leadsto \hat{e}  & \Gamma \vdash e_1 \Leftarrow \tau_1 \leadsto \hat{e}_1 }\\[3ex]

\infer[\textit{T-introcase}]
	{\Delta; \Gamma \vdash  t.C(e) \Rightarrow t  \leadsto t.C(\hat{e}) } 
	{t:\{\chi, e_0:\tau\} \in \Delta & C\ \keyw{of}\ \tau' \in \chi &\Delta; \Gamma \vdash e \Leftarrow \tau'  \leadsto \hat{e}}\\[3ex]

\infer[\textit{T-elimcase}]
	{\Delta; \Gamma \vdash  \keyw{case}~(e)~\{ c \} \Rightarrow \tau'  \leadsto \keyw{case}~(\hat{e})~\{ c \} } 
	{\Delta; \Gamma \vdash e \Rightarrow t  \leadsto \hat{e}  & t:\{ \chi,e_0:\tau\} \in \Delta & c:\chi \Rightarrow \tau'}\\[3ex]

\infer[\textit{T-casehelper1}]
	{\Delta; \Gamma \vdash  C(\chi)\Rightarrow e : C\ \keyw{of}\ \tau \Rightarrow \tau' \leadsto C(\chi)\Rightarrow \hat{e} : C\ \keyw{of}\ \tau} 
	{\Delta; \Gamma, x:\tau \vdash e \Rightarrow \tau' \leadsto \hat{e}}\\[3ex]

\infer[\textit{T-casehelper2}]
	{\Delta; \Gamma \vdash  c_1 \bnfalt c_2: \chi_1 \bnfalt \chi_2 \Rightarrow \tau' } 
	{\Delta; \Gamma \vdash c_1:\chi_1 \Rightarrow \tau' & \Delta; \Gamma \vdash c_2:\chi_2 \Rightarrow \tau'}\\[3ex]


\end{array}
\]
\caption{Static Semantics Rules}
\end{figure}

\begin{figure}
\centering
\[
\begin{array}{c}

\infer[\textit{T-new}]
	{\Delta; \Gamma \vdash \keyw{new}\ \{ d \} \Leftarrow  t \leadsto \keyw{new}\ \{\hat d\}}
	{ t:\{\omega, \hat e:\tau\} \in \Delta & \Delta;\Gamma \vdash d \Leftarrow \omega \leadsto \hat d & t\neq TokenStream} \\[3ex]

\infer[\textit{DT-val}]
	{\Delta; \Gamma \vdash \keyw{val}~ f:\tau = e \Leftarrow \keyw{val}~ f:\tau  \leadsto \keyw{val}~ f:\tau = \hat{e}}
	{\Delta \vdash \tau &\Delta; \Gamma \vdash e \Leftarrow \tau \leadsto \hat{e} } \\[3ex]
	
\infer[\textit{DT-def}]
	{\Delta; \Gamma \vdash \keyw{def}~ m:\tau = e \Leftarrow \keyw{def}~ m:\tau \leadsto \keyw{def}~ m:\tau = \hat{e} }
	{\Delta \vdash \tau  & \Delta; \Gamma \vdash e  \Leftarrow \tau \leadsto \hat{e} } \\[3ex]

	
\infer[\textit{DT-defs}]
	{\Delta; \Gamma \vdash d_1\ d_2 \Leftarrow \omega_1\ \omega_2 }
	{\Delta; \Gamma \vdash d_1 \Leftarrow \omega_1 &  \Delta; \Gamma \vdash d_2 \Leftarrow \omega_2 } \\[3ex]


\infer[\textit{T-field}]
	{\Delta; \Gamma \vdash  e.f \Rightarrow \tau' \leadsto \hat{e}.f} 
	{\Delta; \Gamma \vdash e \Rightarrow t \leadsto \hat{e} & t:\{\omega, e_0:\tau\}\in \Delta & \keyw{val}\ f:\tau' \in \omega  }\\[3ex]

 
\infer[\textit{T-def }]
	{\Delta; \Gamma \vdash  e.m \Rightarrow \tau' \leadsto \hat{e}.m} 
	{\Delta; \Gamma \vdash e \Rightarrow t \leadsto \hat{e} & t: \{\omega, e_0:\tau\} \in \Delta & \keyw{def}\ m:\tau' \in \omega }\\[3ex]

\infer[\textit{T-ascribe}]
	{\Delta; \Gamma  \vdash  e:\tau \Rightarrow \tau \leadsto \hat{e}:\tau}
	{\Delta \vdash \tau & \Delta; \Gamma \vdash e \Leftarrow \tau \leadsto \hat{e} } \\[3ex]

\infer[\textit{T-valAST}]
        {\Delta; \Gamma \vdash \keyw{valAST}(e) \Rightarrow Exp \leadsto \keyw{valAST}(\hat{e}) }
	{\Delta; \Gamma \vdash e \Rightarrow \tau \leadsto \hat{e}} \\[3ex]

\infer[\textit{T-metaobject}]
        {\Delta; \Gamma \vdash t.\keyw{metaobject} \Rightarrow \tau   }
	{t:\{\_, e_0:\tau\} \in \Delta} \\[3ex]


\infer[\textit{T-fromTS}]
	  {\Delta; \Gamma' \vdash \keyw{fromTS}[\Gamma](e_1,e_2) \Leftarrow \tau \leadsto \hat{e} }
	  {\renewcommand{\arraystretch}{1}
	    \begin{array}{r}
	    \Delta;\Gamma' \vdash e_1:TokenStream ~~~~~~ \Delta;\Gamma' \vdash e_2:Token\\
            \texttt{parseConcrete(}e_1,e_2\texttt{)}\ \texttt{is}\ e ~~~~~~\Delta; \Gamma', \Gamma \vdash e \Leftarrow \tau \leadsto \hat{e}
            \end{array}
       } \\[3ex]  

\infer[\textit{T-literal}]
	  {\Delta; \Gamma \vdash \lfloor literal \rfloor \Leftarrow t \leadsto \hat{e} }
	  {\renewcommand{\arraystretch}{1}
	    \begin{array}{r}
	    \Delta;\Gamma \vdash t.\keyw{metaobject}.parser\Leftarrow Parser \leadsto \hat{e}_p ~~~~~ \texttt{TokenStream(}\lfloor literal \rfloor \texttt{)}\ \texttt{is}\ \hat{e}_{ts}\\
            \hat{e}_p.parse(\hat{e}_{ts}, Token.EOS(())) \Downarrow_{\Delta} (\hat{e}', \hat e_{ts}') ~~~~~  e \triangleleft_\Gamma \hat{e}'~~~~~ \Delta;\Gamma\vdash e\Leftarrow t \leadsto \hat{e} ~~~~~ \hat e_{ts}'\ \texttt{empty}
            \end{array}
       } \\[3ex]   
\end{array}
\]
\caption{Static Semantics Rules 2}
\end{figure}

\begin{figure}
\centering
\[
\begin{array}{c}

\infer[\textit{Dyn-Meta}]
	{t.\keyw{metaobject} \xmapsto[\Delta]{} e} 
	{t:\{\_,e:\tau \} \in \Delta}\\[3ex]

\infer[\textit{Dyn-valAST1}]
	{\keyw{valAST}(\hat{e}) \xmapsto[\Delta]{} \keyw{valAST}(\hat{e}') } 
	{\hat{e} \xmapsto[\Delta]{} \hat{e}'}\\[3ex]

\infer[\textit{Dyn-valAST2}]
	{\keyw{valAST}(\hat{e}) \xmapsto[\Delta]{} \hat{e}' } 
	{\hat{e}\ \text{val} &\hat{e} \triangleright \hat{e}' }\\[3ex]




\end{array}
\]
\caption{Dynamic Semantics Rules}
\end{figure}




\begin{figure}
\centering
\begin{minipage}{.5\textwidth}
  \centering
   \[
\begin{array}{c}

\infer[\textit{DExp-Var}]
	{ x \triangleleft Exp.Var(\hat{e})   }
	{ ID(x)\ \text{is}\ \hat{e}} \\[3ex]

\infer[\textit{DExp-Lam}]
	{ \boldsymbol\lambda x{:}\tau . e'_1 \triangleleft Exp.Lam( \hat{e}, \tau, \hat{e}_1 )  }
	{ID(x)\ \text{is}\ \hat{e} & e'_1 \triangleleft \hat{e}_1  } \\[3ex]

\infer[\textit{DExp-App}]
	{ e'_1(e'_2)  \triangleleft Exp.App(\hat{e}_1,\hat{e}_2) }
	{ e'_1 \triangleleft \hat{e}_1  & e'_2 \triangleleft \hat{e}_2   } \\[3ex]

\infer[\textit{DExp-Literal}]
	{ \lfloor literal \rfloor \triangleleft Exp.Literal( \hat{e}_{ts} )  }
	{ \text{literal of}\ \hat{e}_{ts}\ \text{is}\ \lfloor literal \rfloor  } \\[3ex]

\infer[\textit{DTy-Var}]
	{ t \triangleleft Ty.Var(\hat{e})   }
	{ ID(t)\ \text{is}\ \hat{e}} \\[3ex]

\infer[\textit{DTy-Arrow}]
	{ \tau_1 \rightarrow \tau_2 \triangleleft Ty.Arrow(\hat{e}_1,\hat{e}_2 )  }
	{ \tau_1 \triangleleft \hat{e}_1 & \tau_2 \triangleleft \hat{e}_2 } \\[3ex]
   
\end{array}
\]
\caption{Dereification Rules}
\end{minipage}%
\vline
\begin{minipage}{.5\textwidth}
  \centering
  \[
\begin{array}{c}
\infer[\textit{RExp-Var}]
	{ x \triangleright Exp.Var(\hat{e})   }
	{ ID(x)\ \text{is}\ \hat{e}} \\[3ex]

\infer[\textit{RExp-Lam}]
	{ \boldsymbol\lambda x{:}\tau . \hat{e}'_1 \triangleright Exp.Lam( \hat{e}, \tau, \hat{e}_1 )  }
	{ID(x)\ \text{is}\ \hat{e} & \hat{e}'_1 \triangleright \hat{e}_1  } \\[3ex]

\infer[\textit{RExp-App}]
	{ \hat{e}'_1(\hat{e}'_2)  \triangleright Exp.App(\hat{e_1},\hat{e}_2) }
	{ \hat{e}'_1 \triangleright \hat{e}_1  & \hat{e}'_2 \triangleright \hat{e}_2   } \\[3ex]

\infer[\textit{RExp-Literal}]
	{ \lfloor literal \rfloor \triangleright Exp.Literal( \hat{e}_{ts} )  }
	{ \text{literal of}\ \hat{e}_{ts}\ \text{is}\ \lfloor literal \rfloor  } \\[3ex]

\infer[\textit{RTy-Var}]
	{ t \triangleright Ty.Var(\hat{e})   }
	{ ID(t)\ \text{is}\ \hat{e}} \\[3ex]

\infer[\textit{RTy-Arrow}]
	{ \tau_1 \rightarrow \tau_2 \triangleright Ty.Arrow(\hat{e}_1,\hat{e}_2 )  }
	{ \tau_1 \triangleright \hat{e}_1 & \tau_2 \triangleright \hat{e}_2 } \\[3ex]
   
\end{array}
\]
\caption{Reification Rules}
\end{minipage}
\end{figure}


\begin{figure}
\centering
\[
\infer[\textit{T-new-hat}]
	{\Delta; \Gamma \vdash \keyw{new}\ \{ \hat d \} :  t }
	{ t:\{\omega, \hat e:\tau\} \in \Delta & \Delta;\Gamma \vdash \hat d : \omega}
\]
\caption{Statics for $\hat e$}
\end{figure}


The \textit{Syn2Check} rule mediates between synthesis and type checking. 
The judgement 

\fbox{$\Delta; \Gamma \vdash e\Rightarrow \tau \leadsto \hat{e}$} 
\\
\noindent
from the type context $\Delta$ and the variable context $\Gamma$ we synthesize the type $\tau$ for $e$. The  expression $e$ possibly containing $\lfloor literal \rfloor$ forms is transformed into the expression $\hat{e}$ without literals.

In the static semantics rules, the context $\Delta$ contains types and their signatures. The context $\Gamma$ contains variables and their types. 

The judgement 

\fbox{$\Delta; \Gamma \vdash e \Leftarrow \tau \leadsto \hat{e}$} 

means that we check $e$ against the type $\tau$ . 

The \textit{Syn2Check} rule states that syntesis more powerful than type checking.

The rule \textit{RT-objtype} checks that the declaration of the object type $t$ is well-formed and the type of the expression $e$ is the same as the type of $t$'s metaobject.

The rule \textit{RT-casetype} checks that the declaration of the sum type $t$ is well-formed and the type of the expression $e$ is the same as the type of $t$'s metaobject.

The rule \textit{C-decl} checks that the type $\tau$ that is referenced by the name $C$ belongs to the type context $\Delta$.

The rule \textit{C-decls} allows a case type to have multiple cases, where each case will be checked by the rule \textit{C-decl}. We have to make sure that there are no two cases with the same names; we do this by checking that the domains of the $\chi$s are disjoint.

The rule \textit{O-val} checks that the type of a value belongs to the type context $\Delta$ and thus makes the declaration of a value well-formed (\texttt{ok}).

The rule \textit{O-def} checks that the type of a method ($\keyw{def}$) belongs to the type context $\Delta$ and thus makes the declaration of a method well-formed (\texttt{ok}).

The rule \textit{O-defs} allows multiple declarations of values $val$ and methods $def$ to appear one after the toher. Each declaration will either be checked by the rule \textit{O-val} or \textit{O-def}. We have to make sure that there are no two values or methods with the same name; we do this by checking that the domains of the $\omega$s are disjoint.

The rule \textit{T-varx} synthesizes the type of the variable $x$ to be $\tau$, after checking that the variable context $\Gamma$ contains the $x:\tau$ declaration. 

The rule \textit{T-abs} checks the type of the lambda abstraction and states what are the conditions for the checking to be performed.

The rule \textit{T-appl} synthesizes the type of the application of one expression to the other and states what are the premises needed for this synthesis to happen.

The rule \textit{T-introcase} introduces the way a case of the sum type should be used and the type of the resulting expression. To make it simpler, we precede the name of the case by the sum type that includes that case.

The rules \textit{T-elimcase} synthesizes the type of the resulting expression of a particular case of a sum type. Note that all the cases of the same sum type should synthesize to the same type. The rules  \textit{T-casehelper1} and \textit{T-casehelper2} help with the type checking of $c:\chi \Rightarrow \tau'$ in the rule \textit{T-elimcase}. Rule \textit{T-casehelper1} is used when $c$ is of the kind (matches) $C(\chi)\Rightarrow e$ , while rule \textit{T-casehelper2} is used when $c$ is of the kind $c_1 \bnfalt c_2$.

The rule \textit{T-new} checks the type of a \keyw{new} expression. 

The rule \textit{DT-val} checks the type of a value that is instatiated to the expression $e$.

The rule \textit{DT-def} checks the type of a method \keyw{def} that is instantiated to the expression $e$.

The rule \textit{DT-defs} allows for multiple instatiations of values or methods to take place one after the other. Each instantiation is checked with the rule \textit{DT-val} or \textit{DT-def}.

The rule \textit{T-field} synthesizes the type of the field of an expression. The premise mentions that the field is declared as a value .

The rule \textit{T-def}  synthesizes the type of the method (\keyw{def}) of an expression. 

The rule \textit{T-ascribe} ascribes (attributes) the type $\tau$ to $e$, after checking in the premise of the rule that the type of $e$ is $\tau$.

The rule \textit{T-metaobject} synthesizes the type of the \keyw{metaobject} of the type $t$ by checking that $t$ is in the type context $\Delta$ and it has the right type.

The rule \textit{T-literal} is a crucial rule of the our system. The conclusion checks that the $\lfloor literal \rfloor$ expression has the type $t$ and that it is translated to the expression $e$ that does not contain a $\lfloor literal \rfloor$ expression. The premise checks that the $parser$ method of the \keyw{metaobject} of the $t$ type is of type $Parser$. Instead of using $t.\keyw{metaobject}.parser$ we use $e_p$ in the rest of the rule. The premise continues by denoting the token stream of the $\lfloor literal \rfloor$ expression by $e_{ts}$. When the token stream is parsed, it evaluates to $Exp.C(e')$, which is an expression that does not contain a $\lfloor literal \rfloor$ expression. Note that only expressions with a hat $\hat{}$ contain a $\lfloor literal \rfloor$ expression. The expression $Exp.C(e')$ is then translated to the expression $\hat{e'}$, which in turn is recursively translated to the $e$ expression, which does not contain a $\lfloor literal \rfloor$. This rule might not terminate in the general case, but we prove that it terminates in the ways that we use it.















In the example in Fig. \ref{f-htmltype}, we showed a TSL being defined using a parser generator based on Adams' formalism. A parser generator is itself merely a TSL for a parser, and a parser is the fundamental construct that becomes associated with a type to form a TSL. The signature for the built-in type \li{Parser} is shown in Fig. 7. It is an object type with a \li{parse} function taking in a \li{TokenStream} and producing an AST of a Wyvern expression, which is of type \li{Exp}. This built-in type is shown also in Fig. 7. Note that there is a form for each form in the abstract syntax, $e$, as well as an \li{Error} form for indicating error messages (in the theory, nothing is done with these messages). As previously mentioned, quasiquotes are merely a TSL that allows one to construct the abstract syntax, represented as this case type, using concrete syntax, with the addition of an unquote mechanism.

The \verb|parse| function for a type $t$ is called when checking a literal form against that type. This is seen in the key rule of our statics: \textit{T-lit}, in Fig. 9. The premises of these rules operate as follows:
\begin{enumerate}
\setlength{\itemsep}{1pt}
\item This rule uses some built-in types. We first ensure they are available.
\item A well-typed, rewritten parser object is extracted from the type's metadata. This is the step where the parser generator rewrites a grammar to a parse method, recursively using the TSL mechanism itself.
\item A tokenstream, of type \li{TokenStream}, is generated from the body of the literal. This type is an object that that allows the reading of tokens, as well as an additional method discussed in the next section for parsing the stream as a Wyvern expression.
\item The \li|parse| method is called with this extracted tokenstream to produce a syntax tree and a remaining tokenstream.
\item The syntax tree, $\hat{e}'$ is \emph{dereified} into its corresponding term, $e$ (the hat is gone because the generated syntax tree might itself use TSLs). This is the only way terms of the form $\keyw{fromTS}(e)$ can be generated (see below).
\item The dereified term is then recursively typechecked against the same type and rewritten, consistent with the semantics of TSLs as we have been describing them -- they must produce a term of the type they are being checked against. It is checked under the empty local context to ensure hygiene (below).
\item The TSL must consume the entire token stream, so this is checked.
\end{enumerate}
\subsection{Hygiene}
A concern with any term rewriting system is \emph{hygiene} -- how should variables in the generated AST be bound? In particular, if the rewriting system generates an \emph{open term}, then it is making assumptions about the names of variables in scope at the site where the TSL is being used, which is incorrect - those variables should only be identifiable up to alpha renaming. Only the \emph{user} of a TSL knows which variables are in scope. Strict hygiene would simply reject all open terms, but this would prevent even nested Wyvern expressions which the user provided from referring to local variables.

The solution to being able to capture variables in portions of the tokenstream that are parsed as Wyvern only is to add a new term to the abstract syntax that has no corresponding form in the concrete syntax: $\keyw{fromTS}(e)$. This means: "this is a term that was parsed from the user's tokenstream". It can be generated by calling \li{ts.as_wyv_exp(tok)}, which returns a the remaining tokenstream as well as the value  \li{Exp.FromTS(ts, tok)}. When we dereify this term, we turn this into the form $\keyw{fromTS}(e)$, where $e$ is the result of parsing the tokenstream as a Wyvern expression until an unexpected token \li{tok} appears.

When we attempt to typecheck this form, which will be starting from an empty local variable context by moving all the available variables into the surrounding variable context (in the \textit{T-Lit} rule, above), we add in the bindings available in the surrounding variable context (to any that were introduced by the TSL, such as the TSL for \li{Parser} does with named non-terminals). In other words, variables in the surrounding variable context can only be used within a term of the form $\keyw{fromTS}(e)$. These variables are precisely those that only the user can know exist, but not the extension.

For this mechanism to truly ensure hygiene, one must not be able to sidestep it by generating a tokenstream manually: expressions from a tokenstream must have actually come from the use site. This is ensured by preventing users from checking \li{new} against \li{TokenStream} in the statics.

A second facet of hygiene is being able to refer to local variables available within the parser itself, such as local helper functions, for convenience. This can be done using the primitive $\keyw{valAST}(e)$. The semantics for this, shown in Fig. 13, first evaluate $e$ to a value, then \emph{reify} this value to an AST. This can be used to ``bake in'' a value known at compile time into the generated code safely. The rules for reification, used here, and dereification, used in the literal rule described above, are essentially dual, as seen in Figs. 11 and 12.

\subsection{Safety}
\begin{figure}[t]
$\fbox{$\Gamma \vdash_\Sigma i \Leftarrow \tau$}$~
$\fbox{$\Gamma \vdash_\Sigma i \Rightarrow \tau$}$~~$\cdots$
\vspace{-25px}\[
\begin{array}{c}
~~~~~~~~~~~~~~~~~~~~~~~~~~~~~~~~~~~~\infer[\textit{IT-obj-intro}]
	{\Gamma \vdash_\Sigma \keyw{inew}\,\{\dot{d}\} \Leftarrow \keyw{named}[T]}
	{T[\omega,\mu] \in \Sigma & 
	\Gamma \vdash_\Sigma^T \dot{d} \Leftarrow \omega}
\end{array}
\]
\vspace{-12px}
\caption{Statics for internal terms, $i$. Each rule in Fig. \ref{fig:statics1} corresponds to an analogous rule here by removing the elaboration portion. Only the rule for object introduction differs, in that it does not restrict the introduction of parse streams. }
\label{it-statics}
\vspace{-5px}
\end{figure}
\begin{figure}[t]
$\fbox{$i \xmapsto[\Sigma]{} i$}$~~$\cdots$
%$\fbox{$i~\texttt{val}$}$
\vspace{-10px}\[
\begin{array}{c}
\infer[\textit{D1}]
	{\keyw{imetadata}[T] \xmapsto[\Sigma]{} i} 
	{T[\delta,i:\tau] \in \Sigma}
~~~~
\infer[\textit{D2}]
	{\keyw{itoast}(i) \xmapsto[\Sigma]{} \keyw{itoast}(i') } 
	{i \xmapsto[\Sigma]{} i'}
~~~~
\infer[\textit{D3}]
	{\keyw{itoast}(i)  \xmapsto[\Sigma]{} i' } 
	{i\ \texttt{val} & i \downarrow i' }\\[2ex]
\end{array}
\]
\vspace{-15px}
\caption{Dynamics for internal terms, $i$. Only internal terms have a dynamic semantics. Most constructs in TSL Wyvern are standard and omitted, as our focus in this paper is on the statics (see \cite{pfpl} Chs. XXXX). The rules for the operators for extracting metadata and extracting an AST from a value are shown.}
\label{fig:dynsemantics}
\end{figure}

The semantics we have defined constitute a type safe language.

We begin with a lemma that shows that the statics for $e$ and $\hat{e}$ are consistent. This makes us sure that the splitting of variable contexts to maintain hygiene was done correctly (because they can be brought back together at the end). 
\begin{lemma}[Forward Consistency]
\begin{enumerate}
\item \hspace{-5px}
If $\vdash \Delta$ and $\Delta \vdash \Gamma'$ and $\Delta \vdash \Gamma$ and $\Delta; \Gamma'; \Gamma \vdash e \Leftarrow \tau \leadsto \hat{e}$ then $\Delta; \Gamma', \Gamma \vdash \hat{e} : \tau$. 
\item \hspace{-5px} If $\vdash \Delta$ and $\Delta \vdash \Gamma'$ and $\Delta \vdash \Gamma$ and $\Delta; \Gamma'; \Gamma \vdash e \Rightarrow \tau \leadsto \hat{e}$ then $\Delta; \Gamma', \Gamma \vdash \hat{e} : \tau$. 
\end{enumerate}
\end{lemma}
\begin{proof}
Forward consistency is easily seen by observing that for each form shared by both $e$ and $\hat{e}$, the bidirectional system simply rewrites to the corresponding form of $\hat{e}$ recursively. Thus, these cases are direct applications of the IH. For the literal form, we can apply the IH to arrive at the fact that $\Delta_0, \Delta; \Gamma', \Gamma, \emptyset \vdash \hat{e} : t$ which by congruence (removing the empty context at the end) is what we wish to show. Similarly, for the form $\keyw{fromTS}(e)$ we have by the IH that $\Delta; \emptyset, \Gamma', \Gamma \vdash \hat{e} : \tau$ which again implies what we wish to show by simple congruence of contexts.
% TODO: err, maybe? I think fromTS might be a bit more complex
\end{proof}

We then need to show type safety of $\hat{e}$. Because it doesn't contain any non-standard terms other than $\keyw{valAST}$ and $t.\keyw{metadata}$, both of which have straightforward semantics (Fig. 13), this follows by the standard progress and preservation techniques. The only tricky case is \textit{Dyn-valAST2}, which requires the following straightforward lemma about the reification rules in Fig. 12, as well as standard structural properties for the contexts (weakening; not shown).
\begin{lemma}[Reification]
If $\hat{e} \triangleright \hat{e}'$ then $\Delta_0; \emptyset \vdash \hat{e}' : Exp$. 
\end{lemma}

\begin{lemma}[Preservation]
If $\vdash \Delta$ and $\Delta; \emptyset \vdash \hat{e} : \tau$ and  $\hat{e} \xmapsto[\Delta]{} \hat{e}'$ then $\Delta; \emptyset \vdash \hat{e}' : \tau$.
\end{lemma}
\begin{lemma}[Progress]
If  $\vdash \Delta$ and $\Delta; \emptyset \vdash \hat{e} : \tau$ then either $\hat{e}~\texttt{val}$ or $\hat{e} \xmapsto[\Delta]{} \hat{e}'$.
\end{lemma}
These lemmas and associated judgements can be lifted to the level of programs by applying them to the top-level expression the program contains (simple, not shown). As a result, we have type safety: well-typed programs cannot ``get stuck''.
\begin{theorem}[Type Safety]
If $\Delta_0 \vdash \rho : \tau \leadsto \hat{\rho}$ then $\Delta_0 \vdash \hat{\rho} : \tau$ and either $\hat{\rho}~\texttt{val}$ or $\hat{\rho} \xmapsto[\Delta_0]{} \hat{\rho}'$ such that $\Delta_0 \vdash \hat{\rho}' : \tau$.
\end{theorem}
\subsection{Decidability}
Because we are executing user-defined parsers during typechecking, we do not have a straightforward statement of decidability (i.e. termination) of typechecking. The parser might not terminate, or it might generate a term that contains itself. Non-decidability is strictly due to user-defined parsing code. Typechecking of programs that do not contain literals is guaranteed to terminate, as is typechecking of $\hat{e}$ (which we do not actually need to do in practice by Lemma 1). Termination of parsers and parser generators has previously been studied (e.g. \cite{DBLP:conf/sle/KrishnanW12}) and the techniques can be applied to user-defined parsing code to increase confidence in termination. Few compilers, even those with high demands for correctness (e.g. CompCert \cite{compcert}), have made it a priority to fully verify and prove termination of the parser. This is because it is perceived that most bugs in compilers arise due to incorrect optimization passes, not initial parsing and elaboration logic.
