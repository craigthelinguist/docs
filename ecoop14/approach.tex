% !TEX root = ecoop14.tex
\section{Syntax}
\label{s:approach}

\subsection{Concrete Syntax}
% !TEX root = ecoop14.tex

\begin{figure}
\begin{lstlisting}[mathescape]
p $\rightarrow$ 'objtype'$^=$ ID$^>$ NEWLINE$^>$ objdecls$^>$ NEWLINE$^>$ metadatadecl$^>$ NEWLINE$^>$ p$^=$
p $\rightarrow$ 'casetype'$^=$ ID$^>$ NEWLINE$^>$ casedecls$^>$ NEWLINE$^>$ metadatadecl$^>$ NEWLINE$^>$ p$^=$
p $\rightarrow$ e$^=$

metadatadecl $\rightarrow$ 'metadata'$^=$ '='$^>$ e$^>$

e $\rightarrow$ $\overline{\texttt{e}}$$^=$
e $\rightarrow$ $\widetilde{\texttt{e}}$['~']$^=$ NEWLINE$^>$ chars$^>$
e $\rightarrow$ $\widetilde{\texttt{e}}$['new']$^=$ NEWLINE$^>$ d$^>$
e $\rightarrow$ $\widetilde{\texttt{e}}$['case(' $\overline{\texttt{e}}$ ')']$^=$ NEWLINE$^>$ c$^>$

$\overline{\texttt{e}}$ $\rightarrow$ ID$^=$
$\overline{\texttt{e}}$ $\rightarrow$ 'fn'$^=$ ID$^>$ ':'$^>$ type$^>$ '=>'$^>$ $\overline{\texttt{e}}$$^=$
$\overline{\texttt{e}}$ $\rightarrow$ $\overline{\texttt{e}}$$^=$ '('$^>$ $\overline{\texttt{al}}$$^=$ ')'$^>$
$\overline{\texttt{e}}$ $\rightarrow$ 'let'$^=$ ID$^>$ ':'$^>$ type$^>$ '='$^>$ e$^>$ NEWLINE$^>$ $\overline{\texttt{e}}$$^=$
$\overline{\texttt{e}}$ $\rightarrow$ $\overline{\texttt{e}}$$^=$ '.'$^>$ ID$^>$
$\overline{\texttt{e}}$ $\rightarrow$ type$^=$ '.'$^>$ ID$^>$ '('$^>$ $\overline{\texttt{e}}$$^>$ ')'$^>$
$\overline{\texttt{e}}$ $\rightarrow$ $\overline{\texttt{e}}$$^=$ ':'$^>$ type$^>$
$\overline{\texttt{e}}$ $\rightarrow$ 'valAST'$^=$ '('$^>$ $\overline{\texttt{e}}$$^>$ ')'$^>$
$\overline{\texttt{e}}$ $\rightarrow$ type$^=$ '.'$^>$ 'metadata'$^>$
$\overline{\texttt{e}}$ $\rightarrow$ inlinelit$^=$

$\widetilde{\texttt{e}}$[fwd] $\rightarrow$ fwd$^=$
$\widetilde{\texttt{e}}$[fwd] $\rightarrow$ 'fn'$^=$ ID$^>$ ':'$^>$ type$^>$ '=>'$^>$ $\widetilde{\texttt{e}}$[fwd]$^>$
$\widetilde{\texttt{e}}$[fwd] $\rightarrow$ $\widetilde{\texttt{e}}$[fwd]$^=$ '('$^>$ $\overline{\texttt{al}}$$^>$ ')'$^>$
$\widetilde{\texttt{e}}$[fwd] $\rightarrow$ 'let'$^=$ ID$^>$ ':'$^>$ type$^>$ '='$^>$ e$^>$ NEWLINE$^>$ $\widetilde{\texttt{e}}$[fwd]$^=$
$\widetilde{\texttt{e}}$[fwd] $\rightarrow$ $\overline{\texttt{e}}$$^=$ '('$^>$ $\widetilde{\texttt{al}}$[fwd]$^>$ ')'$^>$
$\widetilde{\texttt{e}}$[fwd] $\rightarrow$ $\widetilde{\texttt{e}}$[fwd]$^=$ '.'$^>$ ID$^>$
$\widetilde{\texttt{e}}$[fwd] $\rightarrow$ type$^=$ '.'$^>$ ID$^>$ '('$^>$ $\widetilde{\texttt{e}}$[fwd]$^>$ ')'$^>$
$\widetilde{\texttt{e}}$[fwd] $\rightarrow$ $\widetilde{\texttt{e}}$[fwd]$^=$ ':'$^>$ type$^>$
$\widetilde{\texttt{e}}$[fwd] $\rightarrow$ 'valAST'$^=$ '('$^>$ $\widetilde{\texttt{e}}$[fwd]$^>$ ')'$^>$

d $\rightarrow$ $\varepsilon$
d $\rightarrow$ 'val'$^=$ ID$^>$ ':'$^>$ type$^>$ '='$^>$ e$^>$ NEWLINE$^>$ d$^=$
d $\rightarrow$ 'def'$^=$ ID$^>$ '('$^>$ argsig$^>$ ')'$^>$ ':'$^>$ type$^>$ '='$^>$ e$^>$ NEWLINE$^>$ d$^=$

c $\rightarrow$ ID$^=$ '('$^>$ ID$^>$ ')'$^>$ '=>'$^>$ e$^>$
c $\rightarrow$ ID$^=$ '('$^>$ ID$^>$ ')'$^>$ '=>'$^>$ e$^>$ NEWLINE$^>$ c$^=$

$\overline{\texttt{al}}$ $\rightarrow$ $\varepsilon$ | $\overline{\texttt{al}}_{\texttt{nonempty}}$$^=$
$\overline{\texttt{al}}_{\texttt{nonempty}}$ $\rightarrow$ $\overline{\texttt{e}}$$^=$ | $\overline{\texttt{e}}$$^=$ ','$^>$ $\overline{\texttt{al}}_{\texttt{nonempty}}$$^>$

$\widetilde{\texttt{al}}$[fwd] $\rightarrow$ $\widetilde{\texttt{e}}$[fwd]$^=$
$\widetilde{\texttt{al}}$[fwd] $\rightarrow$ $\widetilde{\texttt{e}}$[fwd]$^=$ ','$^>$ $\overline{\texttt{al}}_{\texttt{nonempty}}$$^>$
$\widetilde{\texttt{al}}$[fwd] $\rightarrow$ $\overline{\texttt{e}}$$^=$ ','$^>$ $\widetilde{\texttt{al}}$[fwd]$^>$

inlinelit $\rightarrow$ chars1['`']$^=$ | chars1[''']$^=$ | chars1['"']$^=$ | ...
inlinelit $\rightarrow$ chars2['{', '}']$^=$ | chars2['<', '>']$^=$ | chars2['[', ']']$^=$ | ...
inlinelit $\rightarrow$ numlit$^=$
\end{lstlisting}
\caption{Concrete Syntax}
\label{f-grammar}
\end{figure}

\begin{figure}
\begin{lstlisting}[mathescape]
`TSL code here, ``inner backticks`` must be doubled`
'TSL code here, ''inner single quotes'' must be doubled'
'TSL code here, ""inner double quotes"" must be doubled'

{TSL code here, {inner braces} must be balanced}
[TSL code here, [inner brackets] must be balanced]
<TSL code here, <inner angle brackets> must be balanced>
\end{lstlisting}
\caption{TSL Delimiters}
\label{f-delims}
\end{figure}

We will now describe the concrete syntax of Wyvern declaratively, using the same layout-sensitive formalism that we have introduced for TSL grammars, developed recently by Adams \cite{Adams:2013:PPI:2429069.2429129}. Such a formalism is useful because it allows us to implement  layout-sensitive syntax, like that we've been describing, without relying on context-sensitive lexers or parsers. Most existing layout-sensitive languages (e.g. Python and Haskell) use hand-rolled context-sensitive lexers or parsers (keeping track of, for example, the indentation level using special \li{INDENT} and \li{DEDENT} tokens), but these are more problematic because they cannot be used to generate editor modes, syntax highlighters and other tools automatically. In particular, we will show how the forward references we have described can be correctly encoded without requiring a context-sensitive parser or lexer using this formalism. It is also useful that the TSL for \li{Parser}, above, uses the same parser technology as the host language, so that it can be used to generate quasiquotes.

Wyvern's concrete syntax, with a few minor omissions for concision, is shown in Figure~\ref{f-grammar}. We first review Adams' formalism in some additional detail, then describe some key features of this syntax.

\subsection{Background: Adams' Formalism}
For each terminal and non-terminal in a rule, Adams proposed associating with them a relational operator, such as =, > and $\geq$ to specify the indentation at which those terms need to be with respect to the non-terminal on the left-hand side of the rule. The indentation level of a term can be identified as the column at which the left-most character of that term appears (not simply the first character, in the case of terms that span multiple lines). The meaning of the comparison operators is akin to their mathematical meaning: = means that the term on the right-hand side has to be at exactly the same indentation as the term on the left-hand side; >  means that the term on the right-hand side has to be indented strictly further to the right than the term on the left-hand side; $\geq$ is like >, except the term on the right could also be at the same indentation level as the term on the left-hand side. For example, the production rule of the form \lstinline{A $\rightarrow$ B$^=$ C$^\geq$ D$^>$} approximately reads as: ``Term \lstinline{B} must be at the same indentation level as term \lstinline{A}, term \lstinline{C} may be at the same or a greater indentation level as term \lstinline{A}, and term \lstinline{D} must be at an indentation level greater than term \lstinline{A}'s.'' In particular, if \li{D} contains a \lstinline{NEWLINE} character, the next line must be indented past the position of the left-most character of \lstinline{A} (typically constructed so that it must appear at the beginning of a line). There are no constraints relating \lstinline{D} to \lstinline{B} or \lstinline{C} other than the standard sequencing constraint: the first character of \lstinline{D} must be further along in the file than the others. Using Adam's formalism, the grammars of real-world languages like Python and Haskell can be written declaratively. This formalism can be integrated into LR and LALR parser generators.

\subsection{Programs}

\begin{figure}[t]
\begin{minipage}[t]{.54\textwidth}
\begin{lstlisting}
objtype T
  val y : HTML
let page : HTML->HTML = fn x:HTML => ~
  :html
    :body
      {x}
page(case(5 : Nat))
  Z(_) => (new : T).y
    val y : HTML = ~
      :h1 Zero!
  S(x) => ~
    :h1 Successor!
\end{lstlisting}
\end{minipage}%
\begin{minipage}[t]{.45\textwidth}
 \centering
\[
\begin{array}{l}
\keyw{objtype} \ T\ \{ \\
  ~~~\keyw{val}\ y : HTML, \\
  ~~~\keyw{metadata} = (\keyw{new}\ \{\}) : Unit\ \}; \\
  (\boldsymbol\lambda page : HTML \rightarrow HTML.\\
  ~page(\keyw{case}(\lfloor 5 \rfloor : Nat)\ \{ \\
  ~~~~Z(\_) \Rightarrow ((\keyw{new}\ \{ \\
    ~~~~~~~\keyw{val}\ y : HTML = \lfloor :h1\ Z! \rfloor\}) : T).y\ \\
   ~~~~| S(x) \Rightarrow \lfloor :h1\ S! \rfloor\})) \\
  ~~~~(\boldsymbol\lambda x : HTML.\ \lfloor :html \\
    ~~:body \\
      ~~~~\{x\} \rfloor)
\end{array}
\]
\end{minipage}
\caption{An example Wyvern program demonstrating forward references. The corresponding abstract syntax, where forward references are inlined, is on the right.}
\label{fig:fwd-ref}
\end{figure}
An example Wyvern program showing several unique syntactic features of TSL Wyvern is shown in Fig.~\ref{f-prelude}. The top level of a program (the \lstinline{p} non-terminal) consists of a series of type declarations -- object types using \lstinline{objtype} or case types using \lstinline{casetype} -- followed by an expression, \lstinline{e}. Each type declaration contains associated declarations -- signatures for fields and methods in  \lstinline{objdecls} and case declarations in \lstinline{casedecls}. Each also can also include a metadata declaration. Metadata is simply an expression associated with the type, used to store TSL logic (and in future work, other logic). Sequences of top-level declarations use the form \lstinline{p$^=$} to signify that all the succeeding \lstinline{p} terms must begin at the same indentation.

\subsection{Forward Referenced Blocks}
Wyvern makes extensive use of forward referenced blocks to make its syntax clean. In particular, layout-delimited TSLs, \keyw{new} expressions for creating objects, and the \keyw{case} statement for eliminating case types all use forward referenced blocks. Fig. \ref{fig:fwd-ref} shows all of these in use (assuming suitable definitions of casetypes \li{Nat} and \li{HTML}, not included). In the grammar, note particularly the rules for \li{let} and that inline literals, even those containing nested expressions with forward references, can be treated as expressions not containing forward references -- \emph{in the initial phase of parsing, before typechecking commences, all literal forms are left unparsed}.

\subsection{Abstract Syntax}
The concrete syntax of a Wyvern program, \li{p}, is parsed to produce a program in the abstract syntax, $\rho$, shown on the left side of Fig. \ref{fig:core2-syntax}. Forward references are internalized. In particular, note that all literal forms are unified into the abstract literal form $\lfloor body \rfloor$, including the layout-delimited form and number literals. The abstract syntax contains a form, $\keyw{fromTS}(e)$, that has no analog in the concrete syntax. This will be used internally to ensure hygiene, as we will discuss in the next section.
\newcommand{\ih}{\hat{e}}
\begin{figure}[t]
$
\begin{array}[t]{lcl} 
\rho & \bnfdef & \theta; e\\
\theta & \bnfdef & \emptyset\\
 & \bnfalt & \keyw{objtype}[T, \omega, e]; \theta \\
     & \bnfalt & \keyw{casetype}[T, \chi, e]; \theta
\end{array}
$~~~~~~
$\begin{array}[t]{lcl}
\tau & \bnfdef & \keyw{named}[T]  \bnfalt \keyw{arrow}[\tau, \tau]\\
~\\
\omega & \bnfdef & \emptyset \bnfalt \ell[\tau]; \omega\\
\chi & \bnfdef & \emptyset \bnfalt C[\tau]; \chi \\
\end{array}$\\
% & \bnfalt & \tau \times \tau
%\end{array}$~
%$\begin{array}[t]{lcl}
%\omega & \bnfdef & \keyw{membdecl}[f, \tau]; \omega\\
%% & \bnfalt & \keyw{defdecl}[f, \tau, \tau]\\
% & \bnfalt & \cdot
%\end{array}$~
%$\begin{array}[t]{lcl}
%\chi & \bnfdef & \keyw{casedecl}[C, \tau]; \chi\\
% & \bnfalt & \cdot
%\end{array}
%$\\
~\\
$
\begin{array}[t]{lcl} 
e    & \bnfdef & x \\
     & \bnfalt & \keyw{easc}[\tau](e)\\
     & \bnfalt & \keyw{elet}(e; x.e)\\
     & \bnfalt & \keyw{elam}(x . e) \\ %
     & \bnfalt & \keyw{eap}(e; e) \\
%     & \bnfalt & \keyw{pair}(e; e) \\
%     & \bnfalt & \keyw{prj}(e; x,y.e)\\
     & \bnfalt & \keyw{enew}\, \{ m \}\\
     & \bnfalt & \keyw{eprj}[\ell](e) \\
%     & \bnfalt & \keyw{call}[m](e; e) \\
     & \bnfalt & \keyw{einj}[C](e) \\
     & \bnfalt & \keyw{ecase}(e)\,\{ r \} \\
     & \bnfalt & \keyw{etoast}(e) \\
     & \bnfalt & \keyw{emetadata}[T]\\
     & \bnfalt & \keyw{lit}[\mathit{body}]
\\[1ex]	
m   & \bnfdef & \emptyset\\
     & \bnfalt & \keyw{eval}[\ell](e); m\\
     & \bnfalt & \keyw{edef}[\ell](x.e); m
\\[1ex]
r    & \bnfdef & \emptyset\\
     & \bnfalt & \keyw{erule}[C](x.e); r
\end{array}
\begin{array}[t]{lcl}
~~~~~~
\end{array}
\begin{array}[t]{lcl}
\ih    & \bnfdef & x \\
     & \bnfalt & \keyw{hasc}[\tau](\ih)\\
     & \bnfalt & \keyw{hlet}(\ih; x.\ih)\\
     & \bnfalt & \keyw{hlam}(x . \ih) \\ %
     & \bnfalt & \keyw{hap}(\ih; \ih) \\
%     & \bnfalt & \keyw{pair}(\ih; \ih) \\
%     & \bnfalt & \keyw{prj}(\ih; x,y.\ih)\\
     & \bnfalt & \keyw{hnew}\, \{ \hat{m} \}\\
     & \bnfalt & \keyw{hprj}[\ell](\ih) \\
%     & \bnfalt & \keyw{call}[m](\ih; \ih) \\
     & \bnfalt & \keyw{hinj}[C](\ih) \\
     & \bnfalt & \keyw{hcase}(\ih)\,\{ \hat{r} \} \\
     & \bnfalt & \keyw{htoast}(\ih) \\
     & \bnfalt & \keyw{hmetadata}[T]\\
     & \bnfalt & \keyw{spliced}[e]
\\[1ex]	
\hat{m}   & \bnfdef & \emptyset \\
	 & \bnfalt & \keyw{hval}[\ell](\ih); \hat{m}\\
     & \bnfalt & \keyw{hdef}[\ell](x.\ih); \hat{m}
\\[1ex]
\hat{r}    & \bnfdef & \emptyset \\
      & \bnfalt & \keyw{hrule}[C](x.\ih); \hat{r}
%\\[1ex]
%\\
%\\
%\chi & \bnfdef & C~\keyw{of}~\tau\\
%     & \bnfalt & \chi \bnfalt \chi 
%\\[1ex]
%\omega &\bnfdef & \varepsilon \\  
%         & \bnfalt & \keyw{val}~ f:\tau;~\omega\\
%         & \bnfalt & \keyw{def}~ m:\tau;~\omega 
%\\[1ex]
\end{array}
\begin{array}[t]{lll}
~~~~~~
\end{array}
\begin{array}[t]{lcl} 
i    & \bnfdef & x \\
     & \bnfalt & \keyw{iasc}[\tau](i)\\
     & \bnfalt & \keyw{ilet}(i; x.i)\\
     & \bnfalt & \keyw{ilam}(x . i) \\ %
     & \bnfalt & \keyw{iap}(i; i) \\
%     & \bnfalt & \keyw{pair}(i; i) \\
%     & \bnfalt & \keyw{prj}(i; x,y.i)\\
     & \bnfalt & \keyw{inew}\, \{ \dot{m} \}\\
     & \bnfalt & \keyw{iprj}[\ell](i) \\
%     & \bnfalt & \keyw{call}[m](i; i) \\
     & \bnfalt & \keyw{iinj}[C](i) \\
     & \bnfalt & \keyw{icase}(i)\,\{ \dot{r} \} \\
     & \bnfalt & \keyw{itoast}(i) \\
     & ~\\
     & ~%& \bnfalt & \keyw{spliced}[e]
\\[1ex]	
\dot{m}   & \bnfdef & \emptyset\\
     & \bnfalt & \keyw{ival}[\ell](i); \dot{m} \\
     & \bnfalt & \keyw{idef}[\ell](x.i); \dot{m}
\\[1ex]
\dot{r}    & \bnfdef & \emptyset\\
     & \bnfalt & \keyw{irule}[C](x.i); \dot{r}
%\\[1ex] 
%\\
%\\
%\\
%\tau & \bnfdef & t\\
%     & \bnfalt & \tau \rightarrow \tau \\
%     & \bnfalt & \tau \times \tau 
%\\[1ex]
%\Gamma & \bnfdef & \emptyset \bnfalt \Gamma, x:\tau~~~~~~~\delta \bnfdef - \bnfalt \hat{\imath} : \tau
%\\[1ex]
%\Theta & \bnfdef & \emptyset \bnfalt \Theta, t:\{\chi, \delta\} \bnfalt \Theta, t:\{\omega, \delta\}
\end{array}
$\\[2ex]
%\vspace{-12px}
\caption{Abstract Syntax of TSL Wyvern programs ($\rho$), type declarations ($\theta$), types ($\tau$), external terms ($e$), translational terms ($\ih$) and internal terms ($i$) and auxiliary forms. Metavariable $T$ ranges over type names, $\ell$ over object member (field and method) labels, $C$ over case labels, $x$ over variables and $\mathit{body}$ over literal bodies. Tuple types are a mode of use of object types, so they are not included in the abstract syntax. For concision, we continue to write pairs as $(i_1, i_2)$ in the rules below.
}
\label{fig:core2-syntax}
\end{figure}



\section{Bidirectional Typechecking and Elaboration}
\label{s:statics}
We will now specify a type system for the abstract syntax in Fig. \ref{fig:core2-syntax}. Conventional type systems are specified using a typing judgement written like $\Gamma \vdash_{\Theta} e : \tau$, where the typing context, $\Gamma$, maps bound variables to types, and the named type context, $\Theta$, maps type names to their declarations. Typing judgements do not consider how, when writing a typechecker, it should be considered algorithmically: will a type be provided from the surrounding syntactic context (e.g. when the term appears as a function argument, or an explicit ascription has been provided), so that we simply need to \emph{analyze} $e$ against it, or do we need to \emph{synthesize} a type for $e$ (e.g. when the term appears at the  top-level)? Here, this distinction is crucial: a literal can only appear in an analytic context. 

\emph{Bidirectional type systems} \cite{Pierce:2000:LTI:345099.345100} make this distinction explicit by specifying the type system instead using two simultaneously defined typechecking judgements corresponding to these two situations. %In the latter situation, type annotations are unnecessary. 
For TSL Wyvern, we need to also simultaneously perform an elaboration of the external language, which contains literals, to an ``internal language'', $i$, the syntax for which is shown on the right side of Fig. \ref{fig:core2-syntax}. The internal language does not have literals, nor a form for accessing the metadata of a named type explicitly (the elaboration process inserts the statically known metadata value, tracked by the named type context, directly). The judgement $\Gamma \vdash_\Theta e \leadsto i \Rightarrow \tau$ means that under typing context $\Gamma$ and named type context $\Theta$, external term $e$ elaborates to internal term $i$ and synthesizes type $\tau$. The judgement $\Gamma \vdash_\Theta e \leadsto i \Leftarrow \tau$ is analagous but for situations where we are analyzing $e$ against type $\tau$. This manner of specifying a type-directed mapping from external terms to a smaller collection internal terms, which are the only terms that are given a dynamic semantics, is stylistically related to the Harper-Stone elaboration semantics for Standard ML \cite{Harper00atype-theoretic} so  our semantics for TSL Wyvern is a form of \emph{bidirectionally typed elaboration semantics}.% Note that both external and internal terms are classified by the same types (we will state this more precisely when we return to the metatheory in Sec. \ref{s:metatheory}).

\begin{figure}[t]
$\fbox{$\rho \sim \Theta \leadsto i : \tau$}$
~~~~$\Theta ::= \emptyset \bnfalt \Theta, T[\delta, \mu]$~~~~$\delta ::=~? \bnfalt \keyw{ot}[\omega] \bnfalt \keyw{ct}[\chi]$ ~~~~ $\mu ::=~? \bnfalt i : \tau$
\vspace{-5px}\[\begin{array}{c}
\infer[\textit{Compile}]
{
\theta; e \sim \Theta \leadsto i : \tau
}
{
\vdash_{\Theta_0} \theta \sim \Theta &
\emptyset \vdash_{\Theta_0\Theta} e \leadsto i \Rightarrow \tau
}
\end{array}
\vspace{-15px}\]
$\fbox{$\vdash_\Theta \theta \sim \Theta$}$~
\[
\begin{array}{c}
\infer[\textit{OT}]
          {%\renewcommand{\arraystretch}{1}
	    %\begin{array}{c}
	    \vdash_\Theta  \keyw{objtype}[T, \omega, e_{m}]; \theta \sim T[\keyw{ot}[\omega], i_m : \tau_m]; \Theta'
       }
	  {T \notin \text{dom}(\Theta) &
	  \vdash_{\Theta, T[?,?]} \omega &
	   \emptyset \vdash_{\Theta, T[\keyw{ot}[\omega],?]} e_{m} \leadsto i_{m} \Rightarrow \tau_{m} &  \vdash_{\Theta, T[\keyw{ot}[\omega], i_{m} : \tau_{m}]} \theta \leadsto \Theta' }
	   \\[2ex] 
\infer[\textit{CT}]
          {%\renewcommand{\arraystretch}{1}
	    %\begin{array}{c}
	    \vdash_\Theta  \keyw{casetype}[T, \chi, e_{m}]; \theta \sim T[\keyw{ct}[\chi], i_m : \tau_m]; \Theta'
       }
	  {T \notin \text{dom}(\Theta) &
	  \vdash_{\Theta, T[?,?]} \chi &
	   \emptyset \vdash_{\Theta, T[\keyw{ct}[\chi],?]} e_{m} \leadsto i_{m} \Rightarrow \tau_{m} &  \vdash_{\Theta, T[\keyw{ct}[\chi], i_{m} : \tau_{m}]} \theta \leadsto \Theta' }
\end{array}
\]
$\fbox{$\vdash_\Theta \omega$}$~~
%\vspace{pt}
$
\begin{array}{c}
\infer[\textit{M-decl}]
	{\vdash_\Theta \ell[\tau]; \omega}
	{ \ell \notin \text{dom}(\omega) & \vdash_\Theta \tau & \vdash_\Theta \omega}
%~~~~~~	
%\infer[\textit{O-d}]
%	{\vdash_\Theta \keyw{defdecl}[f, \tau]}
%	{\vdash_\Theta \tau_1 & \vdash_\Theta \tau_2}
%~~~~~~
%\infer[\textit{M-decls}]
%	{\vdash_\Theta \keyw{memberdecls}[\omega_1, \omega_2]}
%	{\vdash_\Theta \omega_1 &
%	\vdash_\Theta \omega_2 &
%	\text{dom}(\omega_1) \intersect \text{dom}(\omega_2) = \emptyset}
\end{array}
$~~~~~
$\fbox{$\vdash_\Theta \chi$}$~~
$
\begin{array}{c}
\infer[\textit{C-decl}]
	{\vdash_\Theta C[\tau]; \chi} 
	{C \notin \text{dom}(\chi) & \vdash_\Theta \tau & \vdash_\Theta \chi}
\end{array}
$\\[1ex]
$\fbox{$\vdash_\Theta \tau$}$~~
$\begin{array}{c}\infer[\textit{Ty-named}]
{\vdash_\Theta \keyw{named}[T]}
{T[\delta,\mu] \in \Theta}~~~~~~
\infer[\textit{Ty-arrow}]
{\vdash_\Theta \keyw{arrow}[\tau_1, \tau_2]}
{\vdash_\Theta \tau_1 & \vdash_\Theta \tau_2}
\end{array}$\\[1ex]
\caption{Typechecking and elaboration of programs, $\rho$. Note that type declarations can only be recursive, not mutually recursive, with these rules. The prelude $\Theta_0$ (see Fig. \ref{f-prelude}) defines mutually recursive types, so we cannot write a $\theta_0$ corresponding to $\Theta_0$ given the rules above. For concision, the rules to support mutual recursion as well as omitted rules for empty declarations are available in a technical report \cite{TR}.}
\label{f-statics-programs}
\end{figure}
\subsection{Programs and Type Declarations}
Before considering these judgements in detail, let us briefly discuss the steps leading up to typechecking and elaboration of the top-level term, specified by the compilation judgement, $\rho \sim \Theta \leadsto i : \tau$, defined in Fig. \ref{f-statics-programs}. We first load the prelude, $\Theta_0$ (see Fig. \ref{f-prelude}),  then validate the provided user-defined type declarations, $\theta$, to produce a corresponding named typed context, $\Theta$. During this process, we synthesize a type for the associated metadata terms (under the empty typing context) and store their elaborations in the type context $\Theta$ (we do not evaluate the elaboration to a value immediately here, though in a language with effects, the choice of when to evaluate the term is important). Note that type names must be unique (we plan to use a URI-based mechanism in practice). Finally, the top-level external term must synthesize a type $\tau$ and produces an elaboration $i$ under an empty typing context and a named type context combining the prelude with the named type context induced by the user-defined types, written $\Theta_0 \Theta$.

\begin{figure}
$\fbox{$\Gamma \vdash_\Theta e \leadsto i \Rightarrow \tau$}$~
$\fbox{$\Gamma \vdash_\Theta e \leadsto i \Leftarrow \tau$}$~~~~
$\Gamma \bnfdef \emptyset \bnfalt \Gamma, x : \tau$
\[
\begin{array}{c}
\infer[\textit{T-syn-to-ana}]
	{\Gamma \vdash_\Theta  e \leadsto i\Leftarrow \tau } 
	{\Gamma \vdash_\Theta e  \leadsto i\Rightarrow \tau  }
~~~~~~
\infer[\textit{T-asc}]
	{\Gamma  \vdash_\Theta \keyw{easc}[\tau](e) \leadsto \keyw{iasc}[\tau](i) \Rightarrow \tau}
	{\vdash_\Theta \tau & \Gamma \vdash_\Theta e \leadsto i \Leftarrow \tau} \\[2ex]

\infer[\textit{T-var}]
	{\Gamma \vdash_\Theta x \leadsto x \Rightarrow\tau } 
	{x:\tau \in \Gamma }
~~~~~~
\infer[\textit{T-let}]
    {\Gamma \vdash_\Theta \keyw{elet}(e_1; x.e_2) \leadsto \keyw{ilet}(i_1; x.i_2) \Rightarrow \tau}
    {\Gamma \vdash_\Theta e_1 \leadsto i_1 \Rightarrow \tau_1 &
    \Gamma, x : \tau_1 \vdash_\Theta e_2 \leadsto i_2 \Rightarrow \tau}\\[2ex]
\infer[\textit{T-abs}]
	{\Gamma \vdash_\Theta  \keyw{elam}(x . e) \leadsto \keyw{ilam}(x .i) \Leftarrow \keyw{arrow}[\tau_1, \tau_2]} 
	{\Gamma, x:\tau_1 \vdash_\Theta e\leadsto i\Leftarrow \tau_2 }\\[2ex]

\infer[\textit{T-ap}]
	{\Gamma \vdash_\Theta  \keyw{eap}(e_1; e_2) \leadsto \keyw{iap}(i_1; i_2) \Rightarrow \tau_2  } 
	{\Gamma \vdash_\Theta e_1 \leadsto i_1 \Rightarrow \tau_1 \rightarrow \tau_2    & \Gamma \vdash_\Theta e_2  \leadsto i_2 \Leftarrow \tau_1}\\[2ex]

%\infer[\textit{T-prod-intro}]
%	{\Gamma \vdash_\Theta (e_1, e_2)  \leadsto (i_1, i_2) \Leftarrow \tau_1 \times \tau_2}
%	{\Gamma \vdash_\Theta e_1 \leadsto i_1 \Leftarrow \tau_1 & 
%	 \Gamma \vdash_\Theta e_2  \leadsto i_2 \Leftarrow \tau_2}
%	 \\[2ex]
%\infer[\textit{T-prod-elim}]
%	{\Gamma \vdash_\Theta \keyw{case}(e)~\{ (x, y) \Rightarrow e' \} \leadsto \keyw{case}(i)~\{ (x, y) \Rightarrow i' \} \Rightarrow \tau}
%	{\Gamma \vdash_\Theta e \leadsto i \Rightarrow \tau_1 \times \tau_2& 
%	 \Gamma, x : \tau_1, y : \tau_2 \vdash_\Theta e' \leadsto i' \Rightarrow \tau}
%	\\[2ex]
\infer[\textit{T-new}]
	{\Gamma \vdash_\Theta \keyw{enew}\,\{ m \}  \leadsto \keyw{inew}\,\{\dot m\} \Leftarrow  \keyw{named}[T]}
	{ T\neq ParseStream & T[\keyw{ot}[\omega], \mu] \in \Theta & \Gamma \vdash_\Theta^T m \leadsto \dot m \Leftarrow \omega} \\[2ex]
\infer[\textit{T-prj}]
	{\Gamma \vdash_\Theta  \keyw{eprj}[\ell](e) \leadsto \keyw{iprj}[\ell](i) \Rightarrow \tau} 
	{\Gamma \vdash_\Theta e \leadsto i \Rightarrow \keyw{named}[T] & T[\keyw{ot}[\omega], \mu] \in \Theta & \ell[\tau] \in \omega}\\[2ex]
%
%\infer[\textit{T-meth}]
%	{\Gamma \vdash_\Theta  e_1.m(e_2) \leadsto i_1.m(i_2) \Rightarrow \tau_2} 
%	{\Gamma \vdash_\Theta e_1 \leadsto i_1 \Rightarrow t & t\,\{\omega, \_\} \in \Theta & \keyw{def}\ m(\tau_1) \rightarrow \tau_2 \in \omega & \Gamma \vdash_\Theta e_2 \leadsto i_2 \Leftarrow \tau_1}\\[2ex]
%	
	\infer[\textit{T-inj}]
	{\Gamma \vdash_\Theta  \keyw{einj}[C](e) \leadsto \keyw{iinj}[C](i) \Leftarrow \keyw{named}[T]} 
	{T[\keyw{ct}[\chi], \mu] \in \Theta & C[\tau] \in \chi &\Gamma \vdash_\Theta e \leadsto i \Leftarrow \tau }\\[2ex]

\infer[\textit{T-case}]
	{\Gamma \vdash_\Theta  \keyw{ecase}(e)\,\{r \}   \leadsto \keyw{icase}(i)\,\{ \dot{r} \} \Rightarrow \tau} 
	{\Gamma \vdash_\Theta e   \leadsto i \Rightarrow \keyw{named}[T] & T[\keyw{ct}[\chi], \mu] \in \Theta & \Gamma \vdash_\Theta r \leadsto \dot{r} \Leftarrow \chi \Rightarrow \tau}\\[2ex]
	
\infer[\textit{T-toast}]
        {\Gamma \vdash_\Theta \keyw{etoast}(e) \leadsto \keyw{itoast}(i) \Rightarrow \keyw{named}[Exp]}
	{\Theta_0 \subset \Theta & \Gamma \vdash_\Theta e \leadsto i \Rightarrow \tau } \\[2ex]

\infer[\textit{T-metadata}]
        {\Gamma \vdash_\Theta \keyw{emetadata}[T]  \leadsto i  \Rightarrow \tau}
	{T[\delta, i : \tau] \in \Theta}
\end{array}
\]
$\fbox{$\Gamma \vdash_\Theta^T m \leadsto \dot m \Leftarrow \omega$}$
\vspace{-15px}\[
\begin{array}{c}
%\infer[\textit{T-emp}]
%	{\Gamma \vdash_\Theta^t \epsilon  \leadsto \epsilon \Leftarrow \epsilon}
%	{ }
%~~~~~~
\infer[\textit{T-unit}]{\Gamma \vdash_\Theta^T \emptyset \leadsto \emptyset \Leftarrow \emptyset}{ }
\\[2ex]
\infer[\textit{T-val}]
	{\Gamma \vdash_\Theta^T \keyw{eval}[\ell](e); m \leadsto \keyw{ival}[\ell](i); \dot{m} \Leftarrow \ell[\tau]; \omega}
	{\Gamma \vdash_\Theta e \leadsto i \Leftarrow \tau & \Gamma \vdash_\Theta^T m \leadsto \dot{m} \Leftarrow \omega} \\[2ex]
	
\infer[\textit{T-def}]
	{\Gamma \vdash_\Theta^T \keyw{edef}[\ell](x.e); m  \leadsto \keyw{idef}[\ell](x.i); \dot m \Leftarrow \ell[\tau]; \omega}
	{\Gamma, x : \keyw{named}[T] \vdash_\Theta e \leadsto i \Leftarrow \tau & \Gamma \vdash_\Theta^T m \leadsto \dot m \Leftarrow \omega}
\end{array}
\]
$\fbox{$\Gamma \vdash_\Theta r \leadsto \dot{r} \Leftarrow \chi \Rightarrow \tau$}$
\vspace{-15px}\[
\begin{array}{c}
\infer[\textit{T-void}]
	{\Gamma \vdash_\Theta \emptyset \leadsto \emptyset \Leftarrow \emptyset \Rightarrow \tau}{ }\\[2ex]
\infer[\textit{T-rule}]
	{\Gamma \vdash_\Theta  \keyw{erule}[C](x.e); r \leadsto \keyw{irule}[C](x.i); \dot{r} \Leftarrow C[\tau_1]; \chi \Rightarrow \tau_2} 
	{\Gamma, x:\tau_1 \vdash_\Theta e \leadsto i \Rightarrow \tau_2
	 & \Gamma\vdash_\Theta r \leadsto \dot{r} \Leftarrow \chi \Rightarrow \tau_2}\\[2ex]
%
%\infer[\textit{T-cases}]
%	{\Gamma \vdash_\Theta \keyw{erules}[c_1, c_2] \leadsto \keyw{irules}[\dot{c}_1, \dot{c}_2] \Leftarrow \keyw{casedecls}[\chi_1, \chi_2] \Rightarrow \tau  } 
%	{\Gamma \vdash_\Theta c_1 \leadsto \dot{c_1} \Leftarrow \chi_1 \Rightarrow \tau & \Gamma \vdash_\Theta c_2 \leadsto \dot{c_2} \Leftarrow \chi_2 \Rightarrow \tau}
\end{array}
\]
\vspace{-10px}
\caption{Statics for external terms, $e$. The rule for literals is shown in Fig. \ref{fig:statics-lit}.}
\label{fig:statics1}
\end{figure}
\subsection{External Terms}
\begin{figure}[t]
\centering
\[
\footnotesize
\begin{array}{c}
\infer[\textit{T-lit}]
	  {\Gamma \vdash_\Theta \keyw{lit}[\mathit{body}] \leadsto i \Leftarrow \keyw{named}[T]}
	  {\begin{array}{c}
	   \Theta_0 \subset \Theta ~~~~
	   T[\delta, i_m : HasTSL] \in \Theta ~~~~
	   \mathtt{parsestream}(\mathit{body})=i_{ps} \\
	   \keyw{iap}(\keyw{iprj}[parse](\keyw{iprj}[parser](i_m)); i_{ps}) \Downarrow \keyw{iinj}[OK]((i_{ast}, i'_{ps}))\\
	   i_{ast} \uparrow \ih~~~~
	   \Gamma; \emptyset \vdash_\Theta \ih \leadsto i \Leftarrow \keyw{named}[T]
	   \end{array}}
%	  {\renewcommand{\arraystretch}{1}
%	    \begin{array}{r}
%	    \vdash \Delta_0, \Delta ~~~ \Delta_0, \Delta ;\emptyset; \emptyset \vdash t.\keyw{metadata}.parser\Leftarrow Parser \leadsto i_p ~~~ \texttt{TS(}\lfloor body \rfloor \texttt{)}\ \texttt{is}\ i_{ts}\\
%            i_p.parse(i_{ts}) \Downarrow_{\Delta_0, \Delta} (i', \hat e_{ts}') ~~~  e \triangleleft  i'~~~ \Delta_0, \Delta;\Gamma', \Gamma; \emptyset\vdash e\Leftarrow t \leadsto i ~~~ \hat e_{ts}'\ \texttt{empty}
%            \end{array}
%       }
%\infer[\textit{T-Tvar}]{
%	\vdash_\Theta t
%}{
%	t : \{\_, \delta \} \in \Delta
%}
%~~~~~~~
%\infer[\textit{T-Arr}]{
%	\vdash_\Theta \tau_1 \rightarrow \tau_2
%}{
%	\vdash_\Theta \tau_1 & \vdash_\Theta \tau_2
%}
%~~~~~~~
%\infer[?]{
%	\vdash \emptyset
%}{ }
%~~~~
%\infer[?]{
%	\vdash \Delta, t : \{ \omega, \delta \}
%}{
%	\vdash \Delta & \vdash_\Theta \omega~\texttt{ok} & \Delta, t : \{\omega, -\} \vdash \delta
%}
\end{array}
\]
\caption{Statics for external terms, $e$, continued. This is the key rule (see text).}
\label{fig:statics-lit}
%\vspace{-10px}
\end{figure}

The bidirectional typechecking and elaboration rules for external terms are shown beginning in Fig. \ref{fig:statics1}. Nearly all the rules are standard for simply typed lambda calculus with labeled sums and labeled products, and the elaborations are direct. We refer the reader to standard texts on type systems (e.g. \cite{pfpl}) to understand the basic constructs, and to course material\footnote{\small \url{http://www.cs.cmu.edu/~fp/courses/15312-f04/handouts/15-bidirectional.pdf}} on bidirectional typechecking for background. In our presentation, all introductory forms are analytic and all elimination forms are synthetic. 

The introductory form for object types, $\keyw{enew}\,\{m\}$, prevents the manual introduction of parse streams (only the semantics can introduce parse streams, to permit us to enforce hygiene, as we will discuss below). The auxiliary judgement $\Gamma \vdash_\Theta^T m \leadsto \dot{m} \Leftarrow \omega$ analyzes the member definitions $m$ against the member declarations $\omega$ while rewriting them to the internal member definitions, $\dot{m}$. Method definitions involve a self-reference, so the judgement keeps track of the type  name, $T$. We implicitly assume that member definitions and declarations are congruent up to reordering.

The introduction form for case types is written $\keyw{einj}[C](e)$, where $C$ is the case name and $e$ is the associated data. The type of the data associated with each case is stored in the case type's declaration, $\chi$. Because the introductory form is analytic, multiple case types can use the same case names (unlike in, for example, ML). The elimination form, $\keyw{ecase}(e)\,\{r\}$, performs simple exhaustive case analysis (we leave support for nested pattern matching as future work) using the auxiliary judgement $\Gamma \vdash_\Theta r \leadsto \dot{r} \Leftarrow \chi \Rightarrow \tau$, which checks that each case in $\chi$ appears in a rules in the rule sequence $r$, rewriting it to the internal rule sequence $\dot{r}$. Every rule must synthesize the same type, $\tau$.

The rule \textit{T-metadata} shows how the appropriate metadata is extracted from the named type context and inserted directly in the elaboration. We will return to the rule \textit{T-toast} when discussing hygiene.
%\begin{figure}[t]
%\begin{subfigure}[t]{.55\textwidth}
%\begin{lstlisting}
%objtype Parser                          
%  def parse(ts : TokenStream) : (Exp * 
%    TokenStream)
%  metadata = new                        
%    val parser : Parser = new           
%      val parse(ts : TokenStream) : (
%          Exp * TokenStream) =            
%        (* parser generator based
%           on Adams' formalism *)
%\end{lstlisting}
%\end{subfigure}
%\begin{subfigure}[t]{.55\textwidth}
%\begin{lstlisting}[linewidth=.5\textwidth]
%casetype Exp 
%    Var of ID
%  | Lam of ID * Type * Exp
%  | App of Exp * Exp
%  ... 
%  | FromTS of Exp * Exp
%  | Literal of TokenStream
%  | Error of ErrorMessage
%  metadata = (* quasiquotes *)
%\end{lstlisting}
%\end{subfigure}
%%\begin{lstlisting}
%%objtype Parser                                casetype Exp 
%%  def parse(ts : TokenStream) : Exp               Var of ID
%%  metadata = new                                | Lam of ID * Type * Exp
%%    val parser : Parser = new                   | App of Exp * Exp
%%      val parse(ts : TokenStream) :             ... | FromTS of Exp * Exp
%%         Exp * TokenStream =                    | Literal of TokenStream
%%           ... parser generator based on        | Error of ErrorMessage
%%           Adams' formalism here ...            metadata = (* quasiquotes *)
%%\end{lstlisting}
%\caption{Two of the built-in types included in $\Delta_0$ (concrete syntax).}
%\vspace{-10px}
%%\label{fig:typeParser}
%\end{figure}
%\subsection{Bidirectional Type System}
%Our type-checking is syntax-directed and it requires terms to be fully annotated with types where necessary. A commonly used approach for making the syntax more compact is bidirectional type systems, . They introduce two mutually recursive judgements: one for expressions that have enough information in the context to synthesize a type, and one for expressions for which we know what type to expect, thus only needing to check against that type. Unique types can be determined for synthesis expressions, while analytical expressions have to be verified to have the right types. We chose to use bidirectional type systems in our formalism because they leverage the simplicity of syntax-directed type-checking while not needing to carry much additional type information.
%
%In conventional bidirectional type systems, for constructors of a type one can propagate the type information $\tau$
%into the term $e$, which means it should be used in the analysis
%judgment $e \Leftarrow \tau$. When constructing a type, we do not have information about it and it is intuitive to use the analysis judgement, which is weaker than the synthesis judgement. On the other hand, destructors generate a result of a smaller type from a component of larger type and can be used for synthesis, propagating type information away from the term as in the synthesis judgement $e \Rightarrow \tau$. Our static semantics rules follow this conventional way of reasoning about constructors and destructors.

\subsection{Literals}
% !TEX root = ecoop14.tex
\section{Statics}

\todo{This section on Statics is going to go and simply be integrated into approach in the right places, right? Right now it is just a list of figures and some text descrbining parts of them, not a coherent section as such, right?}

We present the abstract syntax of our system in Figure \ref{fig:core2-syntax}. A program $\rho$ is composed of a series of object type \keyw{objtype} and sum type \keyw{casetype} declarations, followed by expressions $e$. An object type is made of declarations of values \keyw{val}, methods \keyw{def} and \keyw{metadata}. A sum type is made of an enumeration of cases of the form $C$ \keyw{of} $\tau$, where $C$ is the name of the constructor and $\tau$ is the type of the expression constructed in this case, followed by metadata.

The metadata of a type $t$ (either an \keyw{objtype} or a \keyw{casetype}) can contain arbitrary data, for eg. documentation, but it will necessarily contain a $parser$ field of type $Parser$. The $parser$ field has a $parse$ method that takes as argument a stream of tokens and generates an abstract syntax tree expression. This $parse$ method is used for parsing new TSLs of type $t$ defined by the user. 

We differentiate between expressions that might contain a TSL expression and those that definitely do not contain a TSL expression by superscripting the latter with the symbol $\hat{}$. Thus we have two versions (one without $\hat{}$ and one with $\hat{}$) for programs $\rho$,  expressions $e$, cases $c$ of the sum types, and declarations $d$ of fields and methods. 

An expression can be a variable, a function, an application of a function to an expression, a pair of expressions, a case analysis on pairs, the constructor of a case type, the destructor of a case type, a new expression declaring fields and methods, the invocation $e.x$ of a field or a method. We do not have $e.m$ and $e.f$ in the abstract syntax because the parser cannot differentiate between the two, only the type checker can do that. An expression can also be an expression with a type ascribed to it, the abstract syntax tree of an expression, a metaobject of a type, a TSL literal or an expression obtained from a token stream. 








\begin{figure}
\centering
\[
\begin{array}[t]{lll} 
\rho & \bnfdef & \keyw{objtype}~ t~ \{ \omega, \keyw{metadata}=e \}; \rho \\
     & \bnfalt & \keyw{casetype}~ t~ \{ \chi, \keyw{metadata}=e \}; \rho\\
     & \bnfalt & e
     \\[1ex]
e    & \bnfdef & x \\
     & \bnfalt & \boldsymbol\lambda x{:}\tau . e \\ %
     & \bnfalt & e(e) \\
     & \bnfalt & (e, e) \\
     & \bnfalt & \keyw{case}(e) \{(x, y) \Rightarrow e\}\\
     & \bnfalt & t.C(e) \\
     & \bnfalt & \keyw{case}(e)~\{ c \} \\
     & \bnfalt & \keyw{new}~ \{ d \}\\
     & \bnfalt & e.x \\
     & \bnfalt & e : \tau\\
     & \bnfalt & \keyw{valAST}(e) \\
     & \bnfalt & t.\keyw{metaobject}\\
     & \bnfalt & \lfloor literal \rfloor \\
     & \bnfalt & \keyw{fromTS}[\Gamma](e, e)
\\[1ex]	
c    & \bnfdef & C(x) \Rightarrow e\\
     & \bnfalt & c \bnfalt c
	 \\[1ex]
d   & \bnfdef & \varepsilon \\
     & \bnfalt & \keyw{val}~ f:\tau = e;~d \\
     & \bnfalt & \keyw{def}~ m:\tau = e;~d
\\[1ex] 
\end{array}
\begin{array}[t]{lll}
~~~
\end{array}
\begin{array}[t]{lll}


\hat\rho & \bnfdef & \keyw{objtype}~ t~ \{ \omega, \keyw{metadata}=\hat e \}; \hat\rho \\
     & \bnfalt & \keyw{casetype}~ t~ \{ \chi, \keyw{metadata}=\hat e \}; \hat\rho\\
     & \bnfalt & \hat e
     \\[1ex]
\hat{e}    & \bnfdef & x \\
     & \bnfalt & \boldsymbol\lambda x{:}\tau . \hat{e} \\ %
     & \bnfalt & \hat{e}(\hat{e}) \\
     & \bnfalt & \cdots \\
     & \bnfalt & t.\keyw{metaobject} 
\\[1ex]
\hat c    & \bnfdef & ...
	 \\[1ex]
\hat d   & \bnfdef & ... 
\\[1ex] 
\chi & \bnfdef & C~\keyw{of}~\tau\\
     & \bnfalt & \chi \bnfalt \chi 
\\[1ex]
\omega &\bnfdef & \varepsilon \\  
         & \bnfalt & \keyw{val}~ f:\tau;~\omega\\
         & \bnfalt & \keyw{def}~ m:\tau;~\omega 
\\[1ex]
\tau & \bnfdef & t\\
     & \bnfalt & \tau \rightarrow \tau \\
     & \bnfalt & \tau \times \tau 
\\[1ex]
\Gamma & \bnfdef & \emptyset \bnfalt \Gamma, x:\tau
\\[1ex]
\Delta & \bnfdef & \emptyset \bnfalt \Delta, t:\{\chi, e:\tau\} \bnfalt \Delta, t:\{\omega, e:\tau\}
\\[1ex]

\end{array}
\]
\caption{Abstract Syntax}
\label{fig:core2-syntax}
\end{figure}


\begin{figure}
\centering
\[
\begin{array}{ll}
\keyw{casetype}\ & Exp=\\
& \ \ \ Var\ of\ ID\\
& \bnfalt \ Lam\ of\ ID\ *\ Ty\ *\ Exp\\
& \bnfalt \ App\ of\ Exp\ *\ Exp\\
& \cdots\\
& \bnfalt \ FromTokenStream\ of\ Exp\ *\ Exp\\
& \bnfalt \ Error\\
\\
\keyw{casetype}\ & Ty=\\
& \ \ \ Var\ of\ ID\\
& \bnfalt \ Arrow\ of\ Ty*Ty\\
\\ 
\end{array}
\]
\caption{Syntax Trees of Expressions and Types}
\end{figure}





\begin{figure}
\centering
\[
\begin{array}{c}

\infer[\textit{RT-objtype}]
          {\renewcommand{\arraystretch}{1}
	    \begin{array}{r}
	    \Delta; \Gamma \vdash  \keyw{objtype}~ t~=\{{\omega}, \keyw{metaobject}=e\}; \rho: \tau'\leadsto\\
            \keyw{objtype}~ t~=\{{\omega}, \keyw{metaobject}=\hat{e}\}; \hat{\rho}
            \end{array}
       }
	  {\Delta \vdash \omega & \Delta; \Gamma \vdash e \Rightarrow \tau \leadsto \hat{e} & \Delta, t:\{\omega, \hat e:\tau\}; \Gamma \vdash \rho :\tau'\leadsto \hat{\rho} }
	   \\[3ex] 


\infer[\textit{RT-casetype}]
          {\renewcommand{\arraystretch}{1}
	    \begin{array}{r}
	    \Delta; \Gamma \vdash  \keyw{casetype}~ t~=\{\chi, \keyw{metaobject}=e\}; \rho :\tau' \leadsto \\
            \keyw{casetype}~ t~=\{\chi, \keyw{metaobject}=\hat{e}\};\hat{\rho}
            \end{array}
       }
	  {\Delta \vdash \chi & \Delta; \Gamma \vdash e \Rightarrow \tau \leadsto \hat{e} & \Delta, t:\{\chi, \hat e:\tau\}; \Gamma \vdash \rho :\tau'\leadsto \hat{\rho} }
	   \\[3ex] 


\infer[\textit{RT-e}]
	{\Delta; \Gamma \vdash  e:\tau \leadsto \hat{e}} 
	{\Delta; \Gamma \vdash e \Rightarrow \tau \leadsto \hat{e}}\\[3ex]

\infer[\textit{C-decl}]
	{\Delta; \Gamma \vdash  C~\keyw{of}~\tau} 
	{\Delta \vdash \tau   }\\[3ex]

\infer[\textit{C-decls}]
	{\Delta; \Gamma \vdash  \chi_1 \bnfalt \chi_2} 
	{\Delta; \Gamma \vdash \chi_1 & \Delta; \Gamma \vdash \chi_2 & \text{dom}(\chi_1) \intersect \text{dom}(\chi_2) = \emptyset}\\[3ex]

\infer[\textit{O-val}]
	{\Delta; \Gamma \vdash \keyw{val}~ f:\tau \ \texttt{ok} }
	{\Delta \vdash \tau} \\[3ex]
	
\infer[\textit{O-def}]
	{\Delta; \Gamma \vdash \keyw{def}~ m:\tau \ \texttt{ok} }
	{\Delta \vdash \tau } \\[3ex]

\infer[\textit{O-defs}]
	{\Delta; \Gamma \vdash \omega_1\ \omega_2  }
	{\Delta \vdash \omega_1 & \Delta \vdash \omega_2 & \text{dom}(\omega_1) \intersect \text{dom}(\omega_2) = \emptyset } \\[3ex]

\infer[\textit{Syn2Check}]
	{\Delta; \Gamma \vdash  e \Leftarrow \tau \leadsto \hat{e}} 
	{\Delta;\Gamma \vdash e \Rightarrow \tau \leadsto \hat{e}   }\\[3ex]
	
\infer[\textit{T-varx}]
	{\Delta,\Gamma \vdash x\Rightarrow\tau } 
	{x:\tau \in \Gamma }\\[3ex]

\infer[\textit{T-abs}]
	{\Delta; \Gamma \vdash  \boldsymbol\lambda x{:}\tau . e \Leftarrow \tau \rightarrow \tau_1 \leadsto \boldsymbol\lambda x{:}\tau .\hat{e}} 
	{\Delta; \Gamma, x:\tau \vdash e\Leftarrow \tau_1 \leadsto \hat{e}  & \Delta\vdash \tau}\\[3ex]

\infer[\textit{T-appl}]
	{\Delta; \Gamma \vdash  e(e_1) \Rightarrow \tau_2  \leadsto \hat{e}(\hat{e}_1) } 
	{\Delta; \Gamma \vdash e \Rightarrow \tau_1 \rightarrow \tau_2  \leadsto \hat{e}  & \Gamma \vdash e_1 \Leftarrow \tau_1 \leadsto \hat{e}_1 }\\[3ex]

\infer[\textit{T-introcase}]
	{\Delta; \Gamma \vdash  t.C(e) \Rightarrow t  \leadsto t.C(\hat{e}) } 
	{t:\{\chi, e_0:\tau\} \in \Delta & C\ \keyw{of}\ \tau' \in \chi &\Delta; \Gamma \vdash e \Leftarrow \tau'  \leadsto \hat{e}}\\[3ex]

\infer[\textit{T-elimcase}]
	{\Delta; \Gamma \vdash  \keyw{case}~(e)~\{ c \} \Rightarrow \tau'  \leadsto \keyw{case}~(\hat{e})~\{ c \} } 
	{\Delta; \Gamma \vdash e \Rightarrow t  \leadsto \hat{e}  & t:\{ \chi,e_0:\tau\} \in \Delta & c:\chi \Rightarrow \tau'}\\[3ex]

\infer[\textit{T-casehelper1}]
	{\Delta; \Gamma \vdash  C(\chi)\Rightarrow e : C\ \keyw{of}\ \tau \Rightarrow \tau' \leadsto C(\chi)\Rightarrow \hat{e} : C\ \keyw{of}\ \tau} 
	{\Delta; \Gamma, x:\tau \vdash e \Rightarrow \tau' \leadsto \hat{e}}\\[3ex]

\infer[\textit{T-casehelper2}]
	{\Delta; \Gamma \vdash  c_1 \bnfalt c_2: \chi_1 \bnfalt \chi_2 \Rightarrow \tau' } 
	{\Delta; \Gamma \vdash c_1:\chi_1 \Rightarrow \tau' & \Delta; \Gamma \vdash c_2:\chi_2 \Rightarrow \tau'}\\[3ex]


\end{array}
\]
\caption{Static Semantics Rules}
\end{figure}

\begin{figure}
\centering
\[
\begin{array}{c}

\infer[\textit{T-new}]
	{\Delta; \Gamma \vdash \keyw{new}\ \{ d \} \Leftarrow  t \leadsto \keyw{new}\ \{\hat d\}}
	{ t:\{\omega, \hat e:\tau\} \in \Delta & \Delta;\Gamma \vdash d \Leftarrow \omega \leadsto \hat d & t\neq TokenStream} \\[3ex]

\infer[\textit{DT-val}]
	{\Delta; \Gamma \vdash \keyw{val}~ f:\tau = e \Leftarrow \keyw{val}~ f:\tau  \leadsto \keyw{val}~ f:\tau = \hat{e}}
	{\Delta \vdash \tau &\Delta; \Gamma \vdash e \Leftarrow \tau \leadsto \hat{e} } \\[3ex]
	
\infer[\textit{DT-def}]
	{\Delta; \Gamma \vdash \keyw{def}~ m:\tau = e \Leftarrow \keyw{def}~ m:\tau \leadsto \keyw{def}~ m:\tau = \hat{e} }
	{\Delta \vdash \tau  & \Delta; \Gamma \vdash e  \Leftarrow \tau \leadsto \hat{e} } \\[3ex]

	
\infer[\textit{DT-defs}]
	{\Delta; \Gamma \vdash d_1\ d_2 \Leftarrow \omega_1\ \omega_2 }
	{\Delta; \Gamma \vdash d_1 \Leftarrow \omega_1 &  \Delta; \Gamma \vdash d_2 \Leftarrow \omega_2 } \\[3ex]


\infer[\textit{T-field}]
	{\Delta; \Gamma \vdash  e.f \Rightarrow \tau' \leadsto \hat{e}.f} 
	{\Delta; \Gamma \vdash e \Rightarrow t \leadsto \hat{e} & t:\{\omega, e_0:\tau\}\in \Delta & \keyw{val}\ f:\tau' \in \omega  }\\[3ex]

 
\infer[\textit{T-def }]
	{\Delta; \Gamma \vdash  e.m \Rightarrow \tau' \leadsto \hat{e}.m} 
	{\Delta; \Gamma \vdash e \Rightarrow t \leadsto \hat{e} & t: \{\omega, e_0:\tau\} \in \Delta & \keyw{def}\ m:\tau' \in \omega }\\[3ex]

\infer[\textit{T-ascribe}]
	{\Delta; \Gamma  \vdash  e:\tau \Rightarrow \tau \leadsto \hat{e}:\tau}
	{\Delta \vdash \tau & \Delta; \Gamma \vdash e \Leftarrow \tau \leadsto \hat{e} } \\[3ex]

\infer[\textit{T-valAST}]
        {\Delta; \Gamma \vdash \keyw{valAST}(e) \Rightarrow Exp \leadsto \keyw{valAST}(\hat{e}) }
	{\Delta; \Gamma \vdash e \Rightarrow \tau \leadsto \hat{e}} \\[3ex]

\infer[\textit{T-metaobject}]
        {\Delta; \Gamma \vdash t.\keyw{metaobject} \Rightarrow \tau   }
	{t:\{\_, e_0:\tau\} \in \Delta} \\[3ex]


\infer[\textit{T-fromTS}]
	  {\Delta; \Gamma' \vdash \keyw{fromTS}[\Gamma](e_1,e_2) \Leftarrow \tau \leadsto \hat{e} }
	  {\renewcommand{\arraystretch}{1}
	    \begin{array}{r}
	    \Delta;\Gamma' \vdash e_1:TokenStream ~~~~~~ \Delta;\Gamma' \vdash e_2:Token\\
            \texttt{parseConcrete(}e_1,e_2\texttt{)}\ \texttt{is}\ e ~~~~~~\Delta; \Gamma', \Gamma \vdash e \Leftarrow \tau \leadsto \hat{e}
            \end{array}
       } \\[3ex]  

\infer[\textit{T-literal}]
	  {\Delta; \Gamma \vdash \lfloor literal \rfloor \Leftarrow t \leadsto \hat{e} }
	  {\renewcommand{\arraystretch}{1}
	    \begin{array}{r}
	    \Delta;\Gamma \vdash t.\keyw{metaobject}.parser\Leftarrow Parser \leadsto \hat{e}_p ~~~~~ \texttt{TokenStream(}\lfloor literal \rfloor \texttt{)}\ \texttt{is}\ \hat{e}_{ts}\\
            \hat{e}_p.parse(\hat{e}_{ts}, Token.EOS(())) \Downarrow_{\Delta} (\hat{e}', \hat e_{ts}') ~~~~~  e \triangleleft_\Gamma \hat{e}'~~~~~ \Delta;\Gamma\vdash e\Leftarrow t \leadsto \hat{e} ~~~~~ \hat e_{ts}'\ \texttt{empty}
            \end{array}
       } \\[3ex]   
\end{array}
\]
\caption{Static Semantics Rules 2}
\end{figure}

\begin{figure}
\centering
\[
\begin{array}{c}

\infer[\textit{Dyn-Meta}]
	{t.\keyw{metaobject} \xmapsto[\Delta]{} e} 
	{t:\{\_,e:\tau \} \in \Delta}\\[3ex]

\infer[\textit{Dyn-valAST1}]
	{\keyw{valAST}(\hat{e}) \xmapsto[\Delta]{} \keyw{valAST}(\hat{e}') } 
	{\hat{e} \xmapsto[\Delta]{} \hat{e}'}\\[3ex]

\infer[\textit{Dyn-valAST2}]
	{\keyw{valAST}(\hat{e}) \xmapsto[\Delta]{} \hat{e}' } 
	{\hat{e}\ \text{val} &\hat{e} \triangleright \hat{e}' }\\[3ex]




\end{array}
\]
\caption{Dynamic Semantics Rules}
\end{figure}




\begin{figure}
\centering
\begin{minipage}{.5\textwidth}
  \centering
   \[
\begin{array}{c}

\infer[\textit{DExp-Var}]
	{ x \triangleleft Exp.Var(\hat{e})   }
	{ ID(x)\ \text{is}\ \hat{e}} \\[3ex]

\infer[\textit{DExp-Lam}]
	{ \boldsymbol\lambda x{:}\tau . e'_1 \triangleleft Exp.Lam( \hat{e}, \tau, \hat{e}_1 )  }
	{ID(x)\ \text{is}\ \hat{e} & e'_1 \triangleleft \hat{e}_1  } \\[3ex]

\infer[\textit{DExp-App}]
	{ e'_1(e'_2)  \triangleleft Exp.App(\hat{e}_1,\hat{e}_2) }
	{ e'_1 \triangleleft \hat{e}_1  & e'_2 \triangleleft \hat{e}_2   } \\[3ex]

\infer[\textit{DExp-Literal}]
	{ \lfloor literal \rfloor \triangleleft Exp.Literal( \hat{e}_{ts} )  }
	{ \text{literal of}\ \hat{e}_{ts}\ \text{is}\ \lfloor literal \rfloor  } \\[3ex]

\infer[\textit{DTy-Var}]
	{ t \triangleleft Ty.Var(\hat{e})   }
	{ ID(t)\ \text{is}\ \hat{e}} \\[3ex]

\infer[\textit{DTy-Arrow}]
	{ \tau_1 \rightarrow \tau_2 \triangleleft Ty.Arrow(\hat{e}_1,\hat{e}_2 )  }
	{ \tau_1 \triangleleft \hat{e}_1 & \tau_2 \triangleleft \hat{e}_2 } \\[3ex]
   
\end{array}
\]
\caption{Dereification Rules}
\end{minipage}%
\vline
\begin{minipage}{.5\textwidth}
  \centering
  \[
\begin{array}{c}
\infer[\textit{RExp-Var}]
	{ x \triangleright Exp.Var(\hat{e})   }
	{ ID(x)\ \text{is}\ \hat{e}} \\[3ex]

\infer[\textit{RExp-Lam}]
	{ \boldsymbol\lambda x{:}\tau . \hat{e}'_1 \triangleright Exp.Lam( \hat{e}, \tau, \hat{e}_1 )  }
	{ID(x)\ \text{is}\ \hat{e} & \hat{e}'_1 \triangleright \hat{e}_1  } \\[3ex]

\infer[\textit{RExp-App}]
	{ \hat{e}'_1(\hat{e}'_2)  \triangleright Exp.App(\hat{e_1},\hat{e}_2) }
	{ \hat{e}'_1 \triangleright \hat{e}_1  & \hat{e}'_2 \triangleright \hat{e}_2   } \\[3ex]

\infer[\textit{RExp-Literal}]
	{ \lfloor literal \rfloor \triangleright Exp.Literal( \hat{e}_{ts} )  }
	{ \text{literal of}\ \hat{e}_{ts}\ \text{is}\ \lfloor literal \rfloor  } \\[3ex]

\infer[\textit{RTy-Var}]
	{ t \triangleright Ty.Var(\hat{e})   }
	{ ID(t)\ \text{is}\ \hat{e}} \\[3ex]

\infer[\textit{RTy-Arrow}]
	{ \tau_1 \rightarrow \tau_2 \triangleright Ty.Arrow(\hat{e}_1,\hat{e}_2 )  }
	{ \tau_1 \triangleright \hat{e}_1 & \tau_2 \triangleright \hat{e}_2 } \\[3ex]
   
\end{array}
\]
\caption{Reification Rules}
\end{minipage}
\end{figure}


\begin{figure}
\centering
\[
\infer[\textit{T-new-hat}]
	{\Delta; \Gamma \vdash \keyw{new}\ \{ \hat d \} :  t }
	{ t:\{\omega, \hat e:\tau\} \in \Delta & \Delta;\Gamma \vdash \hat d : \omega}
\]
\caption{Statics for $\hat e$}
\end{figure}


The \textit{Syn2Check} rule mediates between synthesis and type checking. 
The judgement 

\fbox{$\Delta; \Gamma \vdash e\Rightarrow \tau \leadsto \hat{e}$} 
\\
\noindent
from the type context $\Delta$ and the variable context $\Gamma$ we synthesize the type $\tau$ for $e$. The  expression $e$ possibly containing $\lfloor literal \rfloor$ forms is transformed into the expression $\hat{e}$ without literals.

In the static semantics rules, the context $\Delta$ contains types and their signatures. The context $\Gamma$ contains variables and their types. 

The judgement 

\fbox{$\Delta; \Gamma \vdash e \Leftarrow \tau \leadsto \hat{e}$} 

means that we check $e$ against the type $\tau$ . 

The \textit{Syn2Check} rule states that syntesis more powerful than type checking.

The rule \textit{RT-objtype} checks that the declaration of the object type $t$ is well-formed and the type of the expression $e$ is the same as the type of $t$'s metaobject.

The rule \textit{RT-casetype} checks that the declaration of the sum type $t$ is well-formed and the type of the expression $e$ is the same as the type of $t$'s metaobject.

The rule \textit{C-decl} checks that the type $\tau$ that is referenced by the name $C$ belongs to the type context $\Delta$.

The rule \textit{C-decls} allows a case type to have multiple cases, where each case will be checked by the rule \textit{C-decl}. We have to make sure that there are no two cases with the same names; we do this by checking that the domains of the $\chi$s are disjoint.

The rule \textit{O-val} checks that the type of a value belongs to the type context $\Delta$ and thus makes the declaration of a value well-formed (\texttt{ok}).

The rule \textit{O-def} checks that the type of a method ($\keyw{def}$) belongs to the type context $\Delta$ and thus makes the declaration of a method well-formed (\texttt{ok}).

The rule \textit{O-defs} allows multiple declarations of values $val$ and methods $def$ to appear one after the toher. Each declaration will either be checked by the rule \textit{O-val} or \textit{O-def}. We have to make sure that there are no two values or methods with the same name; we do this by checking that the domains of the $\omega$s are disjoint.

The rule \textit{T-varx} synthesizes the type of the variable $x$ to be $\tau$, after checking that the variable context $\Gamma$ contains the $x:\tau$ declaration. 

The rule \textit{T-abs} checks the type of the lambda abstraction and states what are the conditions for the checking to be performed.

The rule \textit{T-appl} synthesizes the type of the application of one expression to the other and states what are the premises needed for this synthesis to happen.

The rule \textit{T-introcase} introduces the way a case of the sum type should be used and the type of the resulting expression. To make it simpler, we precede the name of the case by the sum type that includes that case.

The rules \textit{T-elimcase} synthesizes the type of the resulting expression of a particular case of a sum type. Note that all the cases of the same sum type should synthesize to the same type. The rules  \textit{T-casehelper1} and \textit{T-casehelper2} help with the type checking of $c:\chi \Rightarrow \tau'$ in the rule \textit{T-elimcase}. Rule \textit{T-casehelper1} is used when $c$ is of the kind (matches) $C(\chi)\Rightarrow e$ , while rule \textit{T-casehelper2} is used when $c$ is of the kind $c_1 \bnfalt c_2$.

The rule \textit{T-new} checks the type of a \keyw{new} expression. 

The rule \textit{DT-val} checks the type of a value that is instatiated to the expression $e$.

The rule \textit{DT-def} checks the type of a method \keyw{def} that is instantiated to the expression $e$.

The rule \textit{DT-defs} allows for multiple instatiations of values or methods to take place one after the other. Each instantiation is checked with the rule \textit{DT-val} or \textit{DT-def}.

The rule \textit{T-field} synthesizes the type of the field of an expression. The premise mentions that the field is declared as a value .

The rule \textit{T-def}  synthesizes the type of the method (\keyw{def}) of an expression. 

The rule \textit{T-ascribe} ascribes (attributes) the type $\tau$ to $e$, after checking in the premise of the rule that the type of $e$ is $\tau$.

The rule \textit{T-metaobject} synthesizes the type of the \keyw{metaobject} of the type $t$ by checking that $t$ is in the type context $\Delta$ and it has the right type.

The rule \textit{T-literal} is a crucial rule of the our system. The conclusion checks that the $\lfloor literal \rfloor$ expression has the type $t$ and that it is translated to the expression $e$ that does not contain a $\lfloor literal \rfloor$ expression. The premise checks that the $parser$ method of the \keyw{metaobject} of the $t$ type is of type $Parser$. Instead of using $t.\keyw{metaobject}.parser$ we use $e_p$ in the rest of the rule. The premise continues by denoting the token stream of the $\lfloor literal \rfloor$ expression by $e_{ts}$. When the token stream is parsed, it evaluates to $Exp.C(e')$, which is an expression that does not contain a $\lfloor literal \rfloor$ expression. Note that only expressions with a hat $\hat{}$ contain a $\lfloor literal \rfloor$ expression. The expression $Exp.C(e')$ is then translated to the expression $\hat{e'}$, which in turn is recursively translated to the $e$ expression, which does not contain a $\lfloor literal \rfloor$. This rule might not terminate in the general case, but we prove that it terminates in the ways that we use it.















In the example in Fig. \ref{f-htmltype}, we showed a TSL being defined using a parser generator based an Adams grammars. As we noted, a parser generator can itself be seen as a TSL for a parser, and a parser is the fundamental construct that becomes associated with a type to form a TSL. The declaration for the prelude type \li{Parser}, shown in Fig. \ref{f-prelude}, shows that it is an object type with a \li{parse} function taking in a \li{ParseStream} and producing a \li{Result}, which is a case type that indicates either that parsing succeeded, in which case an elaboration of type \li{Exp} is paired with the remaining parse stream (to allow one parser to call another), or that parsing failed, in which case an error message and location is provided. This function is called by the typechecker when analyzing the literal form, as shown in the key rule of our system, \textit{T-lit}, shown in Fig. \ref{fig:statics-lit}. Note that we do not explicitly handle failure in the specification, but in practice we would use the data provided for the failure case to report to the user. 

\noindent
The rule \textit{T-lit} operates as follows:
\begin{enumerate}
\setlength{\itemsep}{1pt}
\item This rule requires that the prelude is available. For technical reasons, we include a check that the prelude was actually included in the named type context.
\item The metadata of the type the literal is being checked against, which must be of type $HasTSL$, is extracted from the named type context. Note that in a language with subtyping or richer forms of type equality, which would be necessary for situations where the metadata might serve other roles, the check that $i_m$ defines a TSL would require an additional premise. 
\item A parse stream, an internal term of type \li{ParseStream}, $i_{ps}$, is generated from the body of the literal. This type is an object that that allows the reading of tokens, as well as additional methods, discussed further below.
\item The \li|parse| method is called with this parse stream. If it produces an reified elaboration, $i_{ast}$ (of type \li{Exp}) and a remaining parse stream, $i_{ps}$, then parsing was successful. Note that we use shorthand for pairs in the rule for concision. 
\item The reified elaboration is \emph{dereified} into a corresponding \emph{translational term}, $\ih$, as specified in Fig. \ref{fig:dereification}. The syntax for translational terms mirrors that of external terms, but does not include literal forms. It adds the form $\keyw{spliced}[e]$, representing an external term  spliced into a literal body. 

The key rule is \textit{U-Spl} -- the only way to generate a translational term of this form is by asking for (a portion of) a parse stream to be parsed as a Wyvern expression or identifier. The reified form, unlike the translational form it corresponds to, does not contain the expression itself, but rather just a portion of the parse stream that should be recognized. Because parse streams (and thus portions thereof) can originate only metatheoretically, we know that $e$ must be an external term written concretely by the TSL client in the body of the literal being analyzed. This is key to guaranteeing hygiene in the final step.

The prelude methods \li{parse_exp} and \li{parse_id} return a value having this reified form corresponding to the first external term found in the parse stream (but, as just described, not necessarily the term itself) paired with the remainder of the parse stream. These methods themselves are not treated specially by the compiler but, for convenience, are associated with  \li{ParseStream}.
\item The final step is to typecheck and elaborate this translational term. This involves the bidirectional typing judgements shown in Fig. \ref{fig:staticsHat}. This judgement has a form similar to that for external terms, but with the addition of an ``outer typing context'', written $\Gamma_{\text{out}}$ in the rules. This holds the context that the literal appeared in, so that the ``main'' typing context can be emptied to ensure that elaborations are closed except for portions derived from the parse stream. 
Each rule in Fig. \ref{fig:statics1} should be thought of as having a corresponding rule in Fig. \ref{fig:staticsHat}. Two examples are shown for concision. The outer context is threaded two opaquely in all cases except the rule for spliced external terms. We discuss these rules further below.
\end{enumerate}

\begin{figure}[t]
\centering
\begin{minipage}[t]{.48\textwidth}
$\fbox{$i \uparrow \ih$}$
\vspace{-20px}
   \[
\begin{array}{c}
\infer[\textit{U-Var}]
	{ \keyw{iinj}[Var](i_{id}) \uparrow x}
	{ i_{id} \uparrow x} \\[2ex]

\infer[\textit{U-Asc}]
	{\keyw{iinj}[Asc]((i_1, i_2)) \uparrow \keyw{hasc}[\tau](\ih)}
	{i_1 \uparrow \tau & i_2 \uparrow \ih}\\[2ex]
	
\infer[\textit{U-Lam}]
	{ \keyw{iinj}[Lam]((i_{id}, i)) \uparrow \keyw{hlam}(x.\ih) }
	{ i_{id} \uparrow x & i \uparrow \ih } \\[2ex]

\infer[\textit{U-Ap}]
	{ \keyw{iinj}[Ap]((i_1, i_2)) \uparrow  \keyw{hap}(\ih_1, \ih_2)}
	{ i_1 \uparrow \ih_1 & i_2 \uparrow \ih_2  } \\[1ex]
\cdots\\[1ex]
\infer[\textit{U-Spl}]
      {\keyw{iinj}[Spliced](i_{ps}) \uparrow \keyw{spliced}[e]}
	  {\texttt{body}(i_{ps})\texttt{=}body & \texttt{eparse}(body)\texttt{=}e}
\end{array}
\]
$\fbox{$i \uparrow \tau$}$
\vspace{-15px}
\[
\begin{array}{c}
\infer[\textit{U-N}]
	{ \keyw{iinj}[Named](i_{name}) \uparrow \keyw{named}[T]}
	{ i_{name} \uparrow T} \\[2ex]

\infer[\textit{U-A}]
	{ \keyw{iinj}[Arrow]((i_1, i_2)) \uparrow \keyw{arrow}[\tau_1, \tau_2]}
	{ i_1 \uparrow \tau_1 & i_2 \uparrow \tau_2}
\end{array}
\]
\caption{Dereification rules, used by rule \textit{T-lit} (above) to determine the translational term encoded by the internal term of type $\keyw{named}[Exp]$.}
\label{fig:dereification}
\end{minipage}%
~\vline\,
\begin{minipage}[t]{.44\textwidth}
$\fbox{$i \downarrow i$}$
\vspace{-20px}
  \[
\begin{array}{c}
\infer[\textit{R-Var}]
	{x \downarrow \keyw{iinj}[Var](i_{id})}
	{x \downarrow i_{id}} \\[2ex]

\infer[\textit{R-Asc}]
	{\keyw{iasc}[\tau](i) \downarrow \keyw{iinj}[Asc]((i_1, i_2))}
	{\tau \downarrow i_1 & i \downarrow i_2}\\[2ex]
	
\infer[\textit{R-Lam}]
	{ \keyw{ilam}(x.i) \downarrow \keyw{iinj}[Lam]((i_{id}, i')) }
	{x \downarrow i_{id} & i \downarrow i'} \\[2ex]

\infer[\textit{R-Ap}]
	{ \keyw{iap}(i_1; i_2) \downarrow \keyw{iinj}[Ap]((i_1', i_2))}
	{ i_1 \downarrow i_1' & i_2 \downarrow i_2' }\\[1ex]
\cdots
\vspace{-10px}
\end{array}
\]
$\fbox{$\tau \downarrow i$}$
\vspace{-12px}
\[
\begin{array}{c}
\infer[\textit{R-N}]
	{ \keyw{named}[T] \downarrow \keyw{iinj}[Named](i_{name}) }
	{ T \downarrow i_{name}} \\[2ex]

\infer[\textit{R-A}]
	{ \keyw{arrow}[\tau_1,\tau_2] \downarrow \keyw{iinj}[Arrow]((i_1, i_2))  }
	{ \tau_1 \downarrow i_1 & \tau_2 \downarrow i_2 }
\end{array}
\]
\caption{Reification rules, used by the $\keyw{itoast}$ (``to AST'') operator (Fig. \ref{fig:dynsemantics}) to permit generating an internal term of type $\keyw{named}[Exp]$ corresponding to the value of the argument (a form of serialization).}
\label{fig:reification}
\end{minipage}
%\vspace{-15px}
\end{figure}
\newcommand{\Gout}{\Gamma_{\text{out}}}
\newcommand{\Gin}{\Gamma}
\begin{figure}[t]
$\fbox{$\Gamma; \Gamma \vdash_\Theta \ih \leadsto i \Rightarrow \tau$}$~
$\fbox{$\Gamma; \Gamma \vdash_\Theta \ih \leadsto i \Leftarrow \tau$}$
\[
\begin{array}{c}
\infer[\textit{H-var}]
	{\Gout; \Gin \vdash_\Theta x \leadsto x \Rightarrow\tau } 
	{x:\tau \in \Gin }
~~~~~
\infer[\textit{H-abs}]
	{\Gout; \Gin \vdash_\Theta  \keyw{hlam}(x . \ih) \leadsto \keyw{ilam}(x.i) \Leftarrow \keyw{arrow}[\tau_1,  \tau_2] } 
	{\Gout; \Gin, x:\tau_1 \vdash_\Theta \ih\leadsto i\Leftarrow \tau_2 }\\[1ex]
\cdots\\[1ex]
\infer[\textit{H-spl-A}]
	{\Gout; \Gin \vdash_\Theta \keyw{spliced}[e] \leadsto i \Leftarrow \tau}
	{\Gout \vdash_\Theta e \leadsto i \Leftarrow \tau}~~~~

\infer[\textit{H-spl-S}]
	{\Gout; \Gin \vdash_\Theta \keyw{spliced}[e] \leadsto i \Rightarrow \tau}
	{\Gout \vdash_\Theta e \leadsto i \Rightarrow \tau}
\end{array}
\]
\caption{Statics for translational terms, $\ih$. Each rule in Fig. \ref{fig:statics1} corresponds to an analagous rule here by threading the outer context through opaquely (e.g. the rules for variables and functions, shown here). The outer context is only used by the rules for $\keyw{spliced}[e]$, representing external terms that were spliced into TSL bodies. Only these terms can access outer variables, achieving hygiene (see text). Note that elaboration is implicitly capture-avoiding here (we assume unique names for internal variables can be generated whenever necessary, see Sec. \ref{s:implementation}).}
\label{fig:staticsHat}
\end{figure}
\begin{figure}[t]
$\fbox{$i \xmapsto{} i$}$~~~~~~$\cdots$~~~~$\begin{array}{c}
\infer[\textit{D-Toast-1}]
	{\keyw{itoast}(i) \xmapsto{} \keyw{itoast}(i') } 
	{i \xmapsto{} i'}
~~~~~~
\infer[\textit{D-Toast-2}]
	{\keyw{itoast}(i)  \xmapsto{} i' } 
	{i\ \texttt{val} & i \downarrow i' }\\[2ex]
\end{array}
$
\caption{Dynamics for internal terms, $i$. Only internal terms have a dynamic semantics. Most constructs in TSL Wyvern are standard and omitted, as our focus in this paper is on the statics. The only novel internal form, $\keyw{itoast}(i)$, extracts an AST (of type $\keyw{named}[Exp]$) from the value of $i$, shown.}
\label{fig:dynsemantics}
\end{figure}
\subsection{Hygiene}


A concern with any term rewriting system is \emph{hygiene} -- how should variables in the generated AST be bound? In particular, if the rewriting system generates an \emph{open term}, then it is making assumptions about the names of variables in scope at the site where the TSL is being used, which is incorrect. Those variables should only be identifiable up to alpha renaming. Only the \emph{user} of a TSL knows which variables are in scope. The strictest rule would simply reject all open terms, but this would prevent even spliced terms written by the TSL client, who presumably is aware of variable bindings at the use site, from referring to local variables. Moreover, the variables in these terms should be bound to what the client expects. The elaboration should not be able to surreptitiously or accidentally shadow variables in spliced terms that {may} be otherwise bound at the use site (e.g. variables named \li{tmp}).

The solution to both of these issues, which we have outlined above, is quite simple: we construct the system so that we know which sub-terms originate from the TSL client, marking them as $\keyw{spliced}[e]$. These terms can refer only to variables in the client's context, $\Gout$, as seen in the premises of the two rules for this form (one for analysis, one for synthesis). The remainder of the term is generated by the TSL provider, so it can refer only to variables introduced earlier in the elaboration, tracked by the context $\Gamma$. The two are kept separate. If the TSL wishes to introduce variables in spliced terms, it must do so by via a function application (as in the TSL for \li{Parser} discussed earlier), ensuring that the client has full control over variable binding.

\subsection{Lifting Values to ASTs}
In some rewriting systems, free variables become bound to their values at the generation site, rather than the use site. In the formulation just discussed, this does not directly occur -- all free variables lead to errors when returned by a TSL definition. To permit lifting values bound at the generation site to ASTs for use at the use site, we include the primitive operator \li{toast(e)}. This simply takes the value of \li{e} and reifies it, producing a term of type \li{Exp}, as specified in Figs. \ref{fig:dynsemantics} and Fig. \ref{fig:reification}. This can be used to ``bake in'' a value known at compile time into the generated code safely. The rules for reification, used here, and dereification, used in the literal rule described above, are notionally dual.

The TSL associated with \li{Exp}, implementing quasiquotes, can perform a free variable analysis and insert these terms automatically (by itself treating the free variables as spliced terms), so they are only explicitly needed when generating an AST manually.

\subsection{Safety}
\begin{figure}[t]
$\fbox{$\Gamma \vdash_\Theta i \Rightarrow \tau$}$~
$\fbox{$\Gamma \vdash_\Theta i \Leftarrow \tau$}$~~~~~~$\cdots$
$
\begin{array}{c}
~~~~~~\infer[\textit{IT-new}]
	{\Gamma \vdash_\Theta \keyw{inew}\,\{\dot{m}\} \Leftarrow \keyw{named}[T]}
	{T[\omega,\mu] \in \Theta & 
	\Gamma \vdash_\Theta^T \dot{m} \Leftarrow \omega}
\end{array}
$\\[1ex]
\caption{Statics for internal terms, $i$. Each rule in Fig. \ref{fig:statics1} corresponds to an analogous rule here by removing the elaboration portion. Only the rule for object introduction differs, in that it does not restrict the introduction of parse streams. }
\label{it-statics}
\vspace{-5px}
\end{figure}


The semantics we have defined constitute a type safe language.

We begin with a lemma that shows that the statics for $e$ and $\hat{e}$ are consistent. This makes us sure that the splitting of variable contexts to maintain hygiene was done correctly (because they can be brought back together at the end). 
\begin{lemma}[Forward Consistency]
\begin{enumerate}
\item \hspace{-5px}
If $\vdash \Delta$ and $\Delta \vdash \Gamma'$ and $\Delta \vdash \Gamma$ and $\Delta; \Gamma'; \Gamma \vdash e \Leftarrow \tau \leadsto \hat{e}$ then $\Delta; \Gamma', \Gamma \vdash \hat{e} : \tau$. 
\item \hspace{-5px} If $\vdash \Delta$ and $\Delta \vdash \Gamma'$ and $\Delta \vdash \Gamma$ and $\Delta; \Gamma'; \Gamma \vdash e \Rightarrow \tau \leadsto \hat{e}$ then $\Delta; \Gamma', \Gamma \vdash \hat{e} : \tau$. 
\end{enumerate}
\end{lemma}
\begin{proof}
Forward consistency is easily seen by observing that for each form shared by both $e$ and $\hat{e}$, the bidirectional system simply rewrites to the corresponding form of $\hat{e}$ recursively. Thus, these cases are direct applications of the IH. For the literal form, we can apply the IH to arrive at the fact that $\Delta_0, \Delta; \Gamma', \Gamma, \emptyset \vdash \hat{e} : t$ which by congruence (removing the empty context at the end) is what we wish to show. Similarly, for the form $\keyw{fromTS}(e)$ we have by the IH that $\Delta; \emptyset, \Gamma', \Gamma \vdash \hat{e} : \tau$ which again implies what we wish to show by simple congruence of contexts.
% TODO: err, maybe? I think fromTS might be a bit more complex
\end{proof}

We then need to show type safety of $\hat{e}$. Because it doesn't contain any non-standard terms other than $\keyw{valAST}$ and $t.\keyw{metadata}$, both of which have straightforward semantics (Fig. 13), this follows by the standard progress and preservation techniques. The only tricky case is \textit{Dyn-valAST2}, which requires the following straightforward lemma about the reification rules in Fig. 12, as well as standard structural properties for the contexts (weakening; not shown).
\begin{lemma}[Reification]
If $\hat{e} \triangleright \hat{e}'$ then $\Delta_0; \emptyset \vdash \hat{e}' : Exp$. 
\end{lemma}

\begin{lemma}[Preservation]
If $\vdash \Delta$ and $\Delta; \emptyset \vdash \hat{e} : \tau$ and  $\hat{e} \xmapsto[\Delta]{} \hat{e}'$ then $\Delta; \emptyset \vdash \hat{e}' : \tau$.
\end{lemma}
\begin{lemma}[Progress]
If  $\vdash \Delta$ and $\Delta; \emptyset \vdash \hat{e} : \tau$ then either $\hat{e}~\texttt{val}$ or $\hat{e} \xmapsto[\Delta]{} \hat{e}'$.
\end{lemma}
These lemmas and associated judgements can be lifted to the level of programs by applying them to the top-level expression the program contains (simple, not shown). As a result, we have type safety: well-typed programs cannot ``get stuck''.
\begin{theorem}[Type Safety]
If $\Delta_0 \vdash \rho : \tau \leadsto \hat{\rho}$ then $\Delta_0 \vdash \hat{\rho} : \tau$ and either $\hat{\rho}~\texttt{val}$ or $\hat{\rho} \xmapsto[\Delta_0]{} \hat{\rho}'$ such that $\Delta_0 \vdash \hat{\rho}' : \tau$.
\end{theorem}
\subsection{Decidability}
Because we are executing user-defined parsers during typechecking, we do not have a straightforward statement of decidability (i.e. termination) of typechecking. The parser might not terminate, or it might generate a term that contains itself. Non-decidability is strictly due to user-defined parsing code. Typechecking of programs that do not contain literals is guaranteed to terminate, as is typechecking of $\hat{e}$ (which we do not actually need to do in practice by Lemma 1). Termination of parsers and parser generators has previously been studied (e.g. \cite{DBLP:conf/sle/KrishnanW12}) and the techniques can be applied to user-defined parsing code to increase confidence in termination. Few compilers, even those with high demands for correctness (e.g. CompCert \cite{Leroy-Compcert-CACM}), have made it a priority to fully verify and prove termination of the parser. This is because it is perceived that most bugs in compilers arise due to incorrect optimization passes, not initial parsing and elaboration logic.
