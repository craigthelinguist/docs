% !TEX root = ecoop14.tex
\section{Introduction}
\label{s:intro}
\todo{intro - Cyrus}
%https://en.wikipedia.org/wiki/Cognitive_dimensions_of_notations
%Domain-specific languages (DSLs) \cite{fowler2010domain} % have been widely-studied because they 
%allow developers to work with   
%specialized abstractions in a natural manner, and allow for specialized 
%verification and compilation strategies that can improve verifiability and performance. However, for DSLs to reach their full
%potential, it must be simple to define a new DSL, invoke it when needed, and to use multiple
%DSLs within a host general-purpose language (GPL), such that pieces of DSL code
% can interoperate to form a complete application. These intuitions are captured by the following core design criteria that govern our work:
%
%\begin{itemize}
%
%\item \emph {Composability}: It should be possible to use multiple DSLs and a GPL
%%, in addition to a GPL,
%within a single program unit.  %Here 
%Within the file-based paradigm used by most contemporary languages, this 
%means including multiple DSLs within a single file.
% Moreover, it should be possible to embed code written in one DSL
%  within another DSL when appropriate, without requiring them to have specific knowledge of each other. This should be possible without interference between DSLs used in any combination: DSLs should be \emph{safely composable}.
%
%\item \emph{Interoperability}: It should be
%  possible to pass around and operate on values 
%%such as functions or data structures
%  that were defined in foreign DSLs in a reasonably natural manner (that is, without requiring large amounts of ``glue code''). Additional requirements, such as the ability to do so with the safety guarantees provided in the foreign DSL, may also be relevant in many settings. 
%  
%%Moreover,
%  %Minimally, DSLs should be able to define a function or data structure that satisfies an interface specified in
%  %a common interface description language (such as the type system of
%  %a host GPL), code written in another DSL should be able to use those
%  %values according to that interface, without requiring that the client DSL
%  %have knowledge of the details of the provider DSL or \emph{vice versa}.
%\end{itemize}
%
%
%%\item type system: one DSL depends on types defined by another DSL,
%%  can use objects of that type in special ways [this goes beyond the
%%    scope - save for a later paper]
%
%In addition to these fundamental criteria, we believe that to be most useful, a system supporting DSLs should satisfy the following related design criteria:
%
%\begin{itemize}
%\item \emph{Flexibility}: Support a variety of notations and new language mechanisms, with minimal bias;
%\item \emph{Modularity}: Support defining DSLs as combinations of reusable components distributed directly within  libraries;
%\item \emph{Identifiability}: Make it easy for programmers to identify which code is written in which DSL and what it means;
%%\item \emph{Consistency}: Encourage DSLs written in similar styles whenever possible in order to enhance readability and learnability of each DSL;
%\item \emph{Simplicity}: Keep the complexity and cost of both defining and invoking a DSL as low as possible.
%%\item Share conventions between DSLs and a host language, making each DSL easier for programmers to learn, helping programmers to identify which code is written in which DSL, and avoiding unintended conflicts between DSLs. \todo{Avoid conflicts (visual and real), enhance learnability}
%%\item Reuse low-level mechanisms and design decisions from a host language, thereby reducing the cost of defining DSLs.  \todo{Easy to define DSLs}
%\end{itemize}
%
%We are developing a comprehensive language design, \emph{Wyvern}, that we believe can satisfy these design criteria well, and that specifically considers language-internal extensibility from the start. In Wyvern, DSL developers define the run-time semantics of DSL constructs via translation
%into a common host language, as in many other DSL frameworks. The novelty of the proposed extensibility mechanism lies in the ways in which we delimit and determine the \emph{scope} of DSL code:
%\begin{itemize}
%%\item The host language and its DSLs share a tokenization and lexing 
%%  strategy, standardizing conventions for identifiers, operators,
%%  constants, and comments.  This avoids the cost of defining lexing
%%  within each DSL, avoids many kinds of low-level clashes between
%%  languages, and makes the composed language more readable. Note that this does not limit the ability of a DSL to define new keywords and other constructs.
%% 
%\item Wyvern is a \emph{whitespace-delimited} language. Source code that is governed by a DSL, rather than the GPL, occurs in whitespace-delimited blocks and must be indented further than the GPL line introducing it. A decrease in indentation relative to the baseline of the DSL block signals its end. This scheme delimits the scope of each DSL in a clear manner, both to 
%  the programmer and the top-level parser, supporting the principle of identifiability.  It also allows Wyvern to avoid restrictions on a DSL's use of delimiters internally. Because the GPL grammar is not extended in a global manner, it also guarantees that syntactic 
%  conflicts cannot arise at link-time.
%
%%There is also a flexible mechanism for explicit delimiters, for small DSLs, that we will discuss briefly later in the paper.
%
% % \emph{whitespace-delimited} at the top level, according to a particular strategy that we will describe.  Various forms of parentheses 
%  %can also serve as delimiters.  Thus, indentation levels or
%  %parenthesized expressions clearly delimit blocks that are governed by a particular 
%  %language.  This makes the boundaries of each DSL clear to
%  %the programmer and the compiler, enhancing usability and guaranteeing that 
%  %ubtle conflicts cannot arise.
%  
%%\end{itemize}
%
%\item Within this basic syntactic framework, we then propose a novel
%type-directed dispatch mechanism: the \emph{expected type} of an expression, rather than an explicit keyword, 
%%, rather than a keyword, 
%determines which DSL grammar should parse the delimited block that generates that expression. That is, \emph{grammars are associated with types}. We will show that the  more common keyword-directed strategy arises as a special case of this strategy. 
%\end{itemize}
%
%This mechanism allows us to satisfy many of the criteria above, including safe composability, while still being quite expressive, as we will show with examples in the next section. We will continue by describing our approach in more detail (\S\ref{s:approach}), discuss ongoing research directions (\S\ref{s:discussion}), and conclude with related work (\S\ref{s:related}).
%
%

% keep discussion of type-based parsing brief - active typing for parsing (only)
% avoids conflict with Cyrus' paper

%The rest of the paper is organized as follows.  The next section
%illustrates our approach by example, discussing the components of the
%solution in more detail. Section~\ref{s:approach} outlines our
%approach, shows a wider variety of examples and discusses variations
%of our approach.  Our in-progress implementation of the proposal is
%described in Section~\ref{s:implementation}.  Section~\ref{s:related}
%compares to related work, and Section~\ref{s:conclusion} concludes.

% with a discussion of future work.
