% !TEX root = ecoop14.tex
\section{Corpus Analysis}
\label{s:study}

To further motivate our approach, we performed an empirical study that examined potential usage of the proposed language composition mechanism. For this purpose, we looked at the recent version (20130901r) of 107 Java projects in the Qualitas Corpus~\cite{QualitasCorpus:APSEC:2010} and performed two analyses to see how TSLs could be used directly and indirectly. Our analyses' methodology and the results are described below.

\subsubsection{Methodology}

To perform our analyses, we used command line tools, such as \lstinline{grep} and \lstinline{sed}, and editor features such as search and substitution. In a semi-manual procedure, we scanned though the Java code and picked out class constructors. After that, we chose constructors that take at least one \lstinline{String} as an argument, and looking at the names of the constructors and their arguments, we inferred the intended use of the classes associated with them. The summarized results of our analysis are shown in Table~\ref{t-summary}.

\begin{table}
   \centering
    \begin{tabular}{l | c | c}
    \bf Constructors & \bf Number & \bf \% of Total \\ \hline
    Total analyzed & 124,873 & 100 \\
    Have a String argument & 30,161 & 24 \\
    Could be equipped with a TSL & 603 & 0.5 \\
    Could use a TSL & 19,288 & 15 \\
    \end{tabular}
    \vspace{0.15in}
    \caption{Summary of the Analyses Results}
    \label{t-summary}
\end{table}

\subsubsection{Direct Substitution with TSLs}

As the first part of our analysis of the Java projects, we looked into how developers could directly benefit  from using TSLs. To do that, we looked though the Java constructors and found those that could be substituted by a type. What we were looking for is the constructors such as:

\begin{lstlisting}
Path(String path) {...}
\end{lstlisting}

That is the constructors that take in a single \lstinline{String} argument, which must be of a specific format and which serves as a basis for the underlying class. In the example above, if it was to be written in Wyvern, the \lstinline{Path} class could be represented as a type called \lstinline{Path} with a \lstinline{parse} method that would verify the adherence to the necessary format.

Looking through the Java constructors, we found that there is x \% (y out of z examined constructors) which comply with this pattern. Those constructors were used for classes that represent URLs and URIs, identification numbers, versions, directory paths, and various types of names (e.g., user name, database name, column name, etc.).

\subsubsection{Potential Use of TSLs}

In the second part of our analysis, we continued examining the identified Java constructors, but this time we looked at the constructors that could benefit from the substitution of classes with TSLs, which we talked about in the previous subsection. In particular, we again looked at the constructors that take at least one \lstinline{String} argument that has to conform to a specific format but the \lstinline{String} argument is not the basis for the underlying class. For instance, constructors such as:

\begin{lstlisting}
FileUpdatedEvent(Object source, String path) {...}
\end{lstlisting}

Here, the second argument \lstinline{path}, which is of type \lstinline{String}, could be represented using a Wyvern type \lstinline{Path} that would guarantee that the passed in argument is of the required format.

\begin{table}
   \centering
    \begin{tabular}{l | c | c}
    \bf Type of String & \bf Number & \bf \% of Total \\ \hline
    Name & 14,307 & 74.2 \\
    ID	& 1,335 & 6.9 \\
    Directory path& 823 & 4.3 \\
    Pattern & 495 & 2.6 \\
    URL/URI & 396 & 2.0 \\
    Other (zip code, password, query, & 1,932 & 10.0 \\
    ~~HTML/XML, IP address, version, etc.)  & & \\ \hline
    \bf Total: & \bf 19,288 & \bf 100.0
    \end{tabular}
    \vspace{0.15in}
    \caption{Types of \lstinline{String} Arguments in Java Constructors}
    \label{t-strs-in-constrs}
\end{table}

Our analysis found that there is x \% (y out of z constructors) of this kind of constructors (see Table~\ref{t-summary}). More details on the kinds of \lstinline{String} arguments that are passed into constructors can be found in Table~\ref{t-strs-in-constrs}. The ``Name'' category refers to the name of a file, a user, a class, etc. that do not have to be unique; the ``ID'' category comprises process IDs, user IDs, column or row IDs, etc. that must have the uniqueness property; the ``Pattern'' category includes regular expressions, prefixes and suffixes, delimiters, format templates, etc.; the ``Other'' category contains \lstinline{String}s used for ZIP codes, passwords, queries, IP addresses, versions, HTML and XML code, etc.; and the ``Directory path'' and ``URL/URI'' categories are self-explanatory.

Hence, our empirical study has shown that there is at least x \% of Java constructors that have a potential of directly or indirectly taking advantage of TSLs. It is important to keep in mind that our analyses were fairly narrow: they focused exclusively on the constructors and thus forwent many other types of programming constructs, such as methods, variable assignments, etc., that could possibly also benefit from our approach.

