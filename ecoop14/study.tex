% !TEX root = ecoop14.tex
\section{Corpus Analysis}
\label{s:study}

An important question when introducing a new approach is how it would change the existing solutions and at what places it could be used. To answer these questions we performed a code analysis and tried to identify potential uses of the TSLs in the existing Java code. Our analysis is limited in scope and focused only on the class constructors. We are interested in class constructors because they are the programmatic constructs that potentially could be equipped with Wyvern types. For example, a Java class constructor such as 

\begin{lstlisting}
Path(String path) {...}
\end{lstlisting}

\noindent which takes in a single \lstinline{String} and makes sure that it is of a specific format, could be equipped with a Wyvern type \lstinline{Path} that would check for the format of the string. We examined this type of Java constructors in the first part of our analysis.

Further, having the pool of Java constructors we wanted to see in how many of them a TSL could be used. A place where we believe a TSL could be used is a \lstinline{String} argument that has a specific format, for instance, constructors such as:

\begin{lstlisting}
FileUpdatedEvent(Object source, String path) {...}
\end{lstlisting}

\noindent Here, the second argument \lstinline{path}, which is of type \lstinline{String}, could be represented using a Wyvern type \lstinline{Path} that would guarantee that the passed in argument is of the required format.

\paragraph{Methodology}

To perform our analysis, we used a recent version (20130901r) of 107 Java projects in the Qualitas Corpus~\cite{QualitasCorpus:APSEC:2010} and searched for the two types of constructors described above. We used command line tools, such as \lstinline{grep} and \lstinline{sed}, and editor features such as search and substitution. In a semi-manual procedure, we scanned though the Java code and picked out class constructors. After that, we chose constructors that take at least one \lstinline{String} as an argument, and looking at the names of the constructors and their arguments, we inferred the intended use of the classes associated with them.

\paragraph{Results} 

\begin{table}
   \centering
    \begin{tabular}[t]{l | c | c}
    \bf Constructors & \bf Number & \bf \% of Total \\ \hline
    Total analyzed & 124,873 & 100 \\
    Have a String argument & 30,161 & 24 \\
    Could be equipped with a TSL & 603 & 0.5 \\
    Could use a TSL & 19,288 & 15 \\
    \end{tabular}
    \vspace{0.15in}
    \caption{Summary of the Analyses Results}
    \label{t-summary}
\end{table}

\begin{table}
   \centering
    \begin{tabular}[t]{l | c | c}
    \bf Type of String & \bf Number & \bf \% of Total \\ \hline
    Name & 14,307 & 74.2 \\
    ID	& 1,335 & 6.9 \\
    Directory path& 823 & 4.3 \\
    Pattern & 495 & 2.6 \\
    URL/URI & 396 & 2.0 \\
    Other (zip code, password, query, & 1,932 & 10.0 \\
    ~~HTML/XML, IP address, version, etc.)  & & \\ \hline
    \bf Total: & \bf 19,288 & \bf 100.0
    \end{tabular}
    \vspace{0.15in}
    \caption{Types of \lstinline{String} Arguments in Java Constructors}
    \label{t-strs-in-constrs}
\end{table}

Our findings are summarized in Figure~\ref{t-summary} and~\ref{t-strs-in-constrs}. For the first part of our analysis, looking through the Java constructors, we found that there is 0.5\% (603 out of 124,873 examined constructors) which could be equipped with a TSL. Those constructors were used for classes that represent URLs and URIs, identification numbers, versions, directory paths, and various types of names (e.g., user name, database name, column name, etc.).

For the second part of our analysis, we found that there is 15\% (19,288 out of 124,873 constructors) of constructors that could use a TSL. More details on the kinds of \lstinline{String} arguments that are passed into constructors can be found in Table~\ref{t-strs-in-constrs}. The ``Name'' category refers to the name of a file, a user, a class, etc. that do not have to be unique; the ``ID'' category comprises process IDs, user IDs, column or row IDs, etc. that must have the uniqueness property; the ``Pattern'' category includes regular expressions, prefixes and suffixes, delimiters, format templates, etc.; the ``Other'' category contains \lstinline{String}s used for ZIP codes, passwords, queries, IP addresses, versions, HTML and XML code, etc.; and the ``Directory path'' and ``URL/URI'' categories are self-explanatory.

Hence, our empirical study has shown that there is significant portion of Java constructors that have a potential of taking advantage of TSLs. It is important to keep in mind that our analysis was narrow: it focused exclusively on the constructors and thus forwent many other types of programming constructs, such as methods, variable assignments, etc., that could possibly also benefit from our approach.

