% !TEX root = ecoop14.tex

\section{Corpus Analysis}
\label{s:study}

We performed a corpus analysis on existing Java code to assess how frequently there are opportunities to use TSLs. As a proxy for this goal, we examined \lstinline{String} arguments passed into Java constructors, for two reasons:

\begin{enumerate}
\item The \lstinline{String} type may be used to represent a large variety of notations, many of which may be expressed using TSLs.
\item We hypothesized that opportunities to use TSLs would often come when initializing an object with state described by the TSL.
\end{enumerate}

\paragraph{Methodology.} We ran our analysis on a recent version (20130901r) of the Qualitas Corpus~\cite{QualitasCorpus:APSEC:2010}, consisting of 107 Java projects, and searched for constructors that used \lstinline{String}s that could be substituted with TSLs. To perform the search, we used command line tools, such as \lstinline{grep} and \lstinline{sed}, and a text editor features such as search and substitution. After we found the constructors, we chose those that took at least one \lstinline{String} as an argument. Via a visual scan of the names of the constructors and their \lstinline{String} arguments, we inferred how the constructors and the arguments were intended to be used.


\paragraph{Results.} We found 124,873 constructors and that 19,288 (15\%) of them could use TSLs. Table~\ref{strings-in-constructors} gives more details on types of \lstinline{String} arguments we found that could be substituted with TSLs. The ``Identifier'' category comprises process IDs, user IDs, column or row IDs, etc. that usually must be unique; the ``Pattern'' category includes regular expressions, prefixes and suffixes, delimiters, format templates, etc.; the ``Other'' category contains \lstinline{String}s used for ZIP codes, passwords, queries, IP addresses, versions, HTML and XML code, etc.; and the ``Directory path'' and ``URL/URI'' categories are self-explanatory.

\begin{table}
   \centering
    \begin{tabular}[t]{l | c | c}
    \bf Type of String & \bf Number & \bf Percentage \\ \hline
    Identifier & 15,642 & 81\% \\
    Directory path& 823 & 4\% \\
    Pattern & 495 & 3\% \\
    URL/URI & 396 & 2\% \\
    Other (ZIP code, password, query, & 1,932 & 10\% \\
    ~~HTML/XML, IP address, version, etc.)  & & \\ \hline
    \bf Total: & \bf 19,288 & \bf 100\%
    \end{tabular}
    \vspace{0.15in}
    \caption{Types of \lstinline{String} arguments in Java constructors that could use TSLs}
    \label{strings-in-constructors}
\end{table}

\paragraph{Limitations.} There are three limitations to our corpus analysis. First, the proxy that we chose for finding how often TSLs could be used in existing Java code is imprecise. Our corpus analysis focused exclusively on Java constructors and thus did not consider other programming constructs, such as method calls, variable assignments, etc., that could possibly use TSLs. Also, there are other datatypes, not just \lstinline{String}, that could be substituted with TSLs, e.g. \lstinline{Path} and \lstinline{URL}. Second, our search for constructors with the use of command line tools and text editor features may not have identified all the Java constructors present in the corpus. Finally, the inference of the intended functionality of the constructor and the passed in \lstinline{String} argument was based on the authors' programming experience and was thus subjective.

\vspace{10px}

Despite the limitations of our corpus analysis, it shows that there is the potential to enhance existing code with type-specific languages since numerous \lstinline{String}s could be substituted with TSLs and a significant portion of Java constructors could take advantage of this fact.

