% !TEX root = ecoop14.tex

\section{Implementation}
\label{s:implementation}
Because Wyvern itself is an evolving language and we believe that the techniques herein are broadly applicable, we have implemented the abstract syntax, typechecking and elaboration rules precisely as specified in this paper, including the hygiene mechanism, in Scala as a stable resource. We have also included a simple compiler from our representation of internal terms, which includes explicit type information at each node, to Scala source code. We represent both external terms and translational terms using the same case classes, using traits to distinguish them when necessary. This code can be used to better understand the implementation overhead of our mechanisms. The key ``trick'' is to make sure that the typing context also maps each source variable to a unique internal variable, so that elaboration of spliced terms is capture-avoiding. This code can be found at \url{http://github.com/wyvernlang/tslwyvern}.

Wyvern itself also supports a variant of this mechanism. The Wyvern language is an evolving effort involving a number of techniques other than TSLs, so the implementation does not precisely coincide with the specification presented herein. In particular, Wyvern's object types and case types have substantially different semantics. Moreover, Adams grammars do not presently have a robust implementation, so their presentation here is merely expository. The top-level parser for Wyvern is instead produced by the Copper parser generator~\cite{van2007context}
which uses stateful LALR parsing to handle whitespace. Forward references, such as the TSL tilde, the new keyword, and case expressions, are handled by inserting a special ``signal'' token into the parse stream at the end of an expression containing a forward reference. When the parser subsequently reads this signal token, it enters the appropriate state depending on the type of forward reference encountered. TSL blocks are handled as if they were strings, preserving all non-leading whitespace, and new and case expression bodies are parsed using their respective grammars. Wyvern performs literal parsing during typechecking essentially as described, using a standard bidirectional type system. It does not enforce the constraints on parse streams and the hygiene mechanisms as of this writing. Some of the API is implemented using a Java interoperability layer rather than directly in Wyvern. This implementation does support some simpler examples fully, however (unlike the implementation above, which does not have a concrete syntax at all). The code can be found at \url{http://github.com/wyvernlang/wyvern}. 
