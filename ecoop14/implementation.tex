% !TEX root = ecoop14.tex

\section{Implementation}
\label{s:implementation}
The Wyvern implementation is written in Java, based around a custom recursive-descent parser, with self-hosting left to future work. Our use of a custom parser was predicated by three unique requirements: layout sensitivity, arbitrary substreams that are completely unparsed until a later stage, and dynamic construction of parsers at run time. To our knowledge, no existing parser system supports all three.

Layout-sensitive parsing is implemented with a custom stateful lexer as in Python, producing \li{INDENT} and \li{DEDENT} tokens. The token stream produced by the lexer is then passed into the Wyvern parser. When a language transition occurs, the Wyvern core parser extracts a substream from the current token stream, using either \li{INDENT} and \li{DEDENT} or any of the TSL delimiters to indicate where the substream should begin or end. This substream is then passed to the extension parser as an argument. By subdividing the token stream, the parsers can avoid complicated issues with delegation of responsibility caused by a single shared stream. 

In order to invoke the correct extension parser, the Wyvern compiler requires a typing context to be present when parsing. To implement this, we combine the typechecking and parsing stages of the compiler, so that typechecking happens incrementally as the source is parsed. Once the first stage of parsing is complete and all Wyvern expressions are known, the Wyvern constructs are typechecked. Then, types for TSL blocks are inferred from the local type context, and the associated parsers are invoked in the next stage on the substreams inside the TSL blocks. This process then continued recursively, as TSL blocks can contain Wyvern code that contains TSLs and etc., until all expressions have been parsed and typechecked.

Extension parsers are added though the interpreter's Java interoperability support, which allows Wyvern types to be structural subtypes of Java interfaces. Using this system, we convert type metaobjects into Java objects extending the Java Parser interface. Then, they are used just as if they were defined in Java code.
