% !TEX root = ecoop14.tex
\section{Statics}

\begin{figure}
\centering
\[
\begin{array}{lll}
\rho & \bnfdef & \keyw{objtype}~ t~ = \{ \omega, \keyw{metaobject}=e \}; \rho \\
     & \bnfalt & \keyw{casetype}~ t~ = \{ \chi, \keyw{metaobject}=e \}; \rho\\
     & \bnfalt & e\\
     \\[1ex]
\chi & \bnfdef & C~\keyw{of}~\tau\\
     & \bnfalt & \chi \bnfalt \chi \\
\\[1ex]
e    & \bnfdef & x \\
     & \bnfalt & \boldsymbol\lambda x{:}\tau . e \\ %
     & \bnfalt & e(e) \\
     & \bnfalt & t.C(e) \\
     & \bnfalt & \keyw{case}~e~\keyw{of}~\{ c \} \\
     & \bnfalt & \keyw{new}~ \{ d \}\\
     & \bnfalt & e.f \\
%     & \bnfalt & e.f = e \\
     & \bnfalt & e.m \\
     & \bnfalt & e : \tau\\
     & \bnfalt & t.\keyw{metaobject}\\
\\[1ex]	 
\end{array}
\begin{array}{lll}
~~~~
\end{array}
\begin{array}{lll}
\hat{e}    & \bnfdef & x \\
     & \bnfalt & \boldsymbol\lambda x{:}\tau . \hat{e} \\ %
     & \bnfalt & \hat{e}(\hat{e}) \\
     & \bnfalt & \cdots \\
     & \bnfalt & t.\keyw{metaobject}\\
     & \bnfalt & \lfloor dsl \rfloor \\
\\[1ex]
c    & \bnfdef & C(x) \Rightarrow e\\
     & \bnfalt & c \bnfalt c\\
	 \\[1ex]
d    & \bnfdef & \keyw{val}~ f:\tau = e \\
     & \bnfalt & \keyw{def}~ m:\tau = e \\
     & \bnfalt & d~d
\\[1ex]
\tau & \bnfdef & t\\
     & \bnfalt & \tau \rightarrow \tau \\
\\[1ex]
\omega   & \bnfdef & \keyw{val}~ f:\tau\\
         & \bnfalt & \keyw{def}~ m:\tau \\
         & \bnfalt & \omega~\omega 
\\[1ex]
\end{array}
\]
\caption{Featherweight Wyvern Syntax}
\label{fig:core2-syntax}
\end{figure}


\begin{figure}
\centering
\[
\begin{array}{c}

\infer[\textit{Syn2Check}]
	{\Delta; \Gamma \vdash  e \downarrow \tau} 
	{\Delta;\Gamma \vdash e \uparrow \tau   }\\[3ex]

\infer[\textit{RT-objtype}]
	{\Delta; \Gamma \vdash  \keyw{objtype}~ t~=\{{\omega}, \keyw{metaobject}=e\}; \rho} 
	{\Delta \vdash \omega & \Delta; \Gamma \vdash e \uparrow\tau & \Delta, t:\{\omega,\tau\}; \Gamma \vdash \rho }\\[3ex]

\infer[\textit{RT-casetype}]
	{\Delta; \Gamma \vdash  \keyw{casetype}~ t~=\{\chi, \keyw{metaobject}=e\}; \rho} 
	{\Delta \vdash \chi & \Delta; \Gamma \vdash e \uparrow\tau & \Delta, t:\{\chi,\tau\}; \Gamma \vdash \rho  }\\[3ex]

\infer[\textit{C-decl}]
	{\Delta; \Gamma \vdash  C~\keyw{of}~\tau} 
	{\Delta \vdash \tau   }\\[3ex]

\infer[\textit{C-decls}]
	{\Delta; \Gamma \vdash  \chi_1 \bnfalt \chi_2} 
	{\Delta; \Gamma \vdash \chi_1 & \Delta; \Gamma \vdash \chi_2 & \text{dom}(\chi_1) \intersect \text{dom}(\chi_2) = \emptyset}\\[3ex]

\infer[\textit{O-val}]
	{\Delta; \Gamma \vdash \keyw{val}~ f:\tau \ \texttt{ok} }
	{\Delta \vdash \tau} \\[3ex]
	
\infer[\textit{O-def}]
	{\Delta; \Gamma \vdash \keyw{def}~ m:\tau \ \texttt{ok} }
	{\Delta \vdash \tau } \\[3ex]

\infer[\textit{O-defs}]
	{\Delta; \Gamma \vdash \omega_1\ \omega_2  }
	{\Delta \vdash \omega_1 & \Delta \vdash \omega_2 & \text{dom}(\omega_1) \intersect \text{dom}(\omega_2) = \emptyset } \\[3ex]
	
\infer[\textit{T-varx}]
	{\Delta,\Gamma \vdash x\uparrow\tau } 
	{x:\tau \in \Gamma }\\[3ex]

\infer[\textit{T-abs}]
	{\Delta; \Gamma \vdash  \boldsymbol\lambda x{:}\tau . e \downarrow \tau \rightarrow \tau_1} 
	{\Delta; \Gamma, x:\tau \vdash e\downarrow\tau_1  & \Delta\vdash \tau}\\[3ex]

\infer[\textit{T-appl}]
	{\Delta; \Gamma \vdash  e(e_1) \uparrow\tau_2} 
	{\Delta; \Gamma \vdash e \uparrow \tau_1 \rightarrow \tau_2 & \Gamma \vdash e_1 \downarrow \tau_1}\\[3ex]

\infer[\textit{T-introcase}]
	{\Delta; \Gamma \vdash  t.C(e) \uparrow t} 
	{t:\{\chi, \tau\} \in \Delta & C\ \keyw{of}\ \tau' \in \chi &\Delta; \Gamma \vdash e \downarrow \tau'}\\[3ex]

\infer[\textit{T-elimcase}]
	{\Delta; \Gamma \vdash  \keyw{case}~e~\keyw{of}~\{ c \} \uparrow\tau'} 
	{\Delta; \Gamma \vdash e \uparrow t & t:\{ \chi,\tau\} \in \Delta & c:\chi \uparrow \tau'}\\[3ex]

\infer[\textit{T-casehelper1}]
	{\Delta; \Gamma \vdash  C(\chi)\Rightarrow e : C\ \keyw{of}\ \tau \uparrow \tau'} 
	{\Delta; \Gamma, x:\tau \vdash e \uparrow \tau'}\\[3ex]

\infer[\textit{T-casehelper2}]
	{\Delta; \Gamma \vdash  c_1 \bnfalt c_2: \chi_1 \bnfalt \chi_2 \uparrow \tau' } 
	{\Delta; \Gamma \vdash c_1:\chi_1 \uparrow \tau' & \Delta; \Gamma \vdash c_2:\chi_2 \uparrow \tau'}\\[3ex]


\end{array}
\]
\caption{Static Semantics Rules Core 2}
\end{figure}

\begin{figure}
\centering
\[
\begin{array}{c}

\infer[\textit{T-new}]
	{\Delta; \Gamma \vdash \keyw{new}\ \{ d \} \downarrow  t }
	{ t:\{\omega,\tau\} \in \Delta & \Delta;\Gamma \vdash d \downarrow \omega} \\[3ex]


\infer[\textit{DT-val}]
	{\Delta; \Gamma \vdash \keyw{val}~ f:\tau = e \downarrow \keyw{val}~ f:\tau }
	{\Delta \vdash \tau &\Delta; \Gamma \vdash e \downarrow \tau  } \\[3ex]
	
\infer[\textit{DT-def}]
	{\Delta; \Gamma \vdash \keyw{def}~ m:\tau = e \downarrow \keyw{def}~ m:\tau }
	{\Delta \vdash \tau  & \Delta; \Gamma \vdash e  \downarrow \tau } \\[3ex]

	
\infer[\textit{DT-defs}]
	{\Delta; \Gamma \vdash d_1\ d_2 \downarrow \omega_1\ \omega_2 }
	{\Delta; \Gamma \vdash d_1 \downarrow \omega_1 &  \Delta; \Gamma \vdash d_2 \downarrow \omega_2 } \\[3ex]


\infer[\textit{T-field}]
	{\Delta; \Gamma \vdash  e.f \uparrow \tau'} 
	{\Delta; \Gamma \vdash e \uparrow t & t:\{\omega, \tau\}\in \Delta & \keyw{val}\ f:\tau' \in \omega  }\\[3ex]

 
\infer[\textit{T-def }]
	{\Delta; \Gamma \vdash  e.m \uparrow\tau'} 
	{\Delta; \Gamma \vdash e \uparrow t & t: \{\omega, \tau\} \in \Delta & \keyw{def}\ m:\tau' \in \omega }\\[3ex]

\infer[\textit{T-ascribe}]
	{\Delta; \Gamma  \vdash  e:\tau \uparrow \tau }
	{\Delta \vdash \tau & \Delta; \Gamma \vdash e \downarrow \tau } \\[3ex]


\infer[\textit{T-metaobject}]
        {\Delta; \Gamma \vdash t.\keyw{metaobject} \uparrow \tau   }
	{t:\{\_, \tau\} \in \Delta} \\[3ex]

\infer[\textit{T-dsl}]
        {\Delta; \Gamma \vdash \lfloor dsl \rfloor \downarrow t \leadsto e }
	{\Delta;\Gamma \vdash t.\keyw{metaobject}.parser\downarrow Parser \leadsto e_p & \text{TokenStream of} \lfloor dsl \rfloor\ \text{is}\ e_{ts}\\
 e_p.parse(e_{ts}) \Downarrow Exp.C(e') & Exp.C(e') \hookrightarrow \hat{e'} & \Delta;\Gamma\vdash \hat{e'}\downarrow t \leadsto e } \\[3ex]





\end{array}
\]
\caption{Static Semantics Rules Core 2}
\end{figure}

The \textit{Syn2Check} rule mediates between synthesis and type checking. 
The judgement 

\fbox{$\Delta; \Gamma \vdash e\uparrow\tau $} 

states that from the type context $\Delta$ and the variable context $\Gamma$ we synthesize (or we deduce) the type of $e$ to be $\tau$.

In the static semantics rules, the context $\Delta$ contains types and their signatures. The context $\Gamma$ contains variables and their types. 

The judgement 

\fbox{$\Delta; \Gamma \vdash e\downarrow\tau $} 

means that we check $e$ against the type $\tau$. 

The \textit{Syn2Check} rule states that syntesis more powerful than type checking.

The rule \textit{RT-objtype} checks that the declaration of the object type $t$ is well-formed and the type of the expression $e$ is the same as the type of $t$'s metaobject.

The rule \textit{RT-casetype} checks that the declaration of the sum type $t$ is well-formed and the type of the expression $e$ is the same as the type of $t$'s metaobject.

The rule \textit{C-decl} checks that the type $\tau$ that is referenced by the name $C$ belongs to the type context $\Delta$.

The rule \textit{C-decls} allows a case type to have multiple cases, where each case will be checked by the rule \textit{C-decl}. We have to make sure that there are no two cases with the same names; we do this by checking that the domains of the $\chi$s are disjoint.

The rule \textit{O-val} checks that the type of a value belongs to the type context $\Delta$ and thus makes the declaration of a value well-formed (\texttt{ok}).

The rule \textit{O-def} checks that the type of a method ($\keyw{def}$) belongs to the type context $\Delta$ and thus makes the declaration of a method well-formed (\texttt{ok}).

The rule \textit{O-defs} allows multiple declarations of values $val$ and methods $def$ to appear one after the toher. Each declaration will either be checked by the rule \textit{O-val} or \textit{O-def}. We have to make sure that there are no two values or methods with the same name; we do this by checking that the domains of the $\omega$s are disjoint.

The rule \textit{T-varx} synthesizes the type of the variable $x$ to be $\tau$, after checking that the variable context $\Gamma$ contains the $x:\tau$ declaration. 

The rule \textit{T-abs} checks the type of the lambda abstraction and states what are the conditions for the checking to be performed.

The rule \textit{T-appl} synthesizes the type of the application of one expression to the other and states what are the premises needed for this synthesis to happen.

The rule \textit{T-introcase} introduces the way a case of the sum type should be used and the type of the resulting expression. To make it simpler, we precede the name of the case by the sum type that includes that case.

The rules \textit{T-elimcase} synthesizes the type of the resulting expression of a particular case of a sum type. Note that all the cases of the same sum type should synthesize to the same type. The rules  \textit{T-casehelper1} and \textit{T-casehelper2} help with the type checking of $c:\chi \uparrow \tau'$ in the rule \textit{T-elimcase}. Rule \textit{T-casehelper1} is used when $c$ is of the kind (matches) $C(\chi)\Rightarrow e$ , while rule \textit{T-casehelper2} is used when $c$ is of the kind $c_1 \bnfalt c_2$.

The rule \textit{T-new} checks the type of a \keyw{new} expression. 

The rule \textit{DT-val} checks the type of a value that is instatiated to the expression $e$.

The rule \textit{DT-def} checks the type of a method \keyw{def} that is instantiated to the expression $e$.

The rule \textit{DT-defs} allows for multiple instatiations of values or methods to take place one after the other. Each instantiation is checked with the rule \textit{DT-val} or \textit{DT-def}.

The rule \textit{T-field} synthesizes the type of the field of an expression. The premise mentions that the field is declared as a value .

The rule \textit{T-def}  synthesizes the type of the method (\keyw{def}) of an expression. 

The rule \textit{T-ascribe} ascribes (attributes) the type $\tau$ to $e$, after checking in the premise of the rule that the type of $e$ is $\tau$.

The rule \textit{T-metaobject} synthesizes the type of the \keyw{metaobject} of the type $t$ by checking that $t$ is in the type context $\Delta$ and it has the right type.

The rule \textit{T-dsl} is a crucial rule of the our system. The conclusion checks that the $\lfloor dsl \rfloor$ expression has the type $t$ and that it is translated to the expression $e$ that does not contain a $\lfloor dsl \rfloor$ expression. The premise checks that the $parser$ method of the \keyw{metaobject} of the $t$ type is of type $Parser$. Instead of using $t.\keyw{metaobject}.parser$ we use $e_p$ in the rest of the rule. The premise continues by denoting the token stream of the $\lfloor dsl \rfloor$ expression by $e_{ts}$. When the token stream is parsed, it evaluates to $Exp.C(e')$, which is an expression that does not contain a $\lfloor dsl \rfloor$ expression. Note that only expressions with a hat $\hat{}$ contain a $\lfloor dsl \rfloor$ expression. The expression $Exp.C(e')$ is then translated to the expression \hat{e'}, which in turn is recursively translated to the $e$ expression, which does not contain a $\lfloor dsl \rfloor$. This rule might not terminate in the general case, but we prove that it terminates in the ways that we use it.















