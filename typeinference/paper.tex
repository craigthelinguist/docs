\documentclass{sig-alternate}

\usepackage{subfiles}
\usepackage{cite}
\usepackage{graphicx}
\usepackage{amsmath, amssymb, mathpartir}
\usepackage{array}
\numberofauthors{3}
\newcommand{\code}[1]{\texttt{\footnotesize #1}}
\newcommand{\todo}[1]{{\bf \{TODO: {#1}\}}}

\title{}

\author{
}

\begin{document}
\maketitle
\begin{abstract}

\end{abstract}

\section{Introduction}
Bidirectional typechecking is a commonly used method for implementing a wide range of type systems in new languages. The mechanism is simple and easy to implement, a attractive combination for language implementers. However, the theoretical basis of bidirectional type checking has fallen behind implementation, with a number of important components remaining informal, most notably higher-rank polymorphism. Dunfield and Krishnaswami provide a method for handling higher-rank polymorphism, but are unable to handle cases where the underlying type system has an inherent subtyping mechanism.

Our primary contribution is an adaptation of Dunfield's system to allow it to handle higher-rank polymorphism and subtyping, while retaining the bidirectional structure and the underlying simplicity of the system. Additionally, our approach is both sound and complete over a predicative system $F_{\st}$, and therefore retains the general applicability of Dunfield et al's approach.

\section{Prior Work}

% The seminal work in this area is that by Pierce and Turner~\cite{Pierce:2000:LTI:345099.345100}. Their approach is built around subtyping for higher-order functions, and introduced the idea of using type argument synthesis. In their system, they develop a set of constraints for each type parameter, then manipulate these sets to find a consistent argument type, akin to a simplified unification system. This approach is quite limited in comparison, however, as their approach cannot handle transitive subtyping relations among existential type variables.

% Odersky's 2001 paper~\cite{odersky2001colored} simplifies the issue, but raises problems of its own. His approach also uses a bidirectional type system, extending the bidirectional concept to the types and type variables themselves. The system colors types to indicate if they are synthetic or analytic, then using this information to inform type argument instantiation. However, this system does not handle subtyping well (requiring a unification system for the single special-case), making it unsuitable for our purposes.

% We base our system of Dunfield and Krishnaswami's 2013 paper~\cite{Dunfield:2013:CEB:2544174.2500582} as a basis for our work. This is a development of the 2009 effort, fixing the completness issues as well as simplifying the system substantially through the introduction of new judgments. The system uses existentials to replace type parameters in parameterized types, then resolves them using relations discovered during subtyping.

% This approach is quite simple and elegant in construction, but has a major drawback for our application, namely that the system fails for type systems with subtyping. As seen in figure~\ref{fig:intro-examples}, this ``greedy'' approach to instantiating variables fails in cases where the first constraint is tighter than (or disjoint from) the second, making it hard to apply in many modern languages. 

% Systems for providing sound and complete typechecking with subtyping do exist, all from the unification community. While these approaches provide the soundness and completeness guarantees that we do, they are also substantially more complex. Additionally, unification is much more difficult for a user to understand than bidirectional subtyping, as the internal operation of the unification mechanism is not as intuitive as bidirectional type inference systems. 

\section{Declarative Type System}

\begin{figure}
\subfile{decl}
\small
\caption{Subtyping in the declarative system}
\end{figure}
\begin{figure}
\subfile{declCons}
\small
\label{fig:declCons}
\caption{Structure of the declarative system}
\end{figure}
\begin{figure}
\subfile{terms}

\small
\label{fig:terms}
\caption{Source Expressions}
\end{figure}
\begin{figure}
\small
\subfile{typing}
\caption{Typing rules for our system}
\end{figure}

A subtyping relationship can be described as a partial ordering over a set of types. To represent types with some relation of this form, we use the variables $X,Y,Z$. We then define $X < Y$ to mean that $X$ is a subtype of $Y$ in the external relationship. This structure allows us to simply describe nominal type systems without multiple inheritance. To support multiple inheritance, we can use a simple power set construction.

% Note - there's a bunch of explanatory text that I am leaving out here, mainly about justifying why the system is built this way, citing Dunfield et al.
We add two new rules to the basic type system, $\leq$Sub and Synth, both for handling types with some arbitrary subtype relation. $\leq$Sub refers the internal judgement $X \leq Y$ to the external judgement $X < Y$, and rule Synth provides a means for introducing a new nominal type.

\section{Algorithmic Type System}
Rule $\leq\forall L$ causes the declarative type system to be difficult to implement. We propose a new algorithmic system, in the vein of that of Dunfield, to implement the declarative type system with an algorithmic approach.

Our algorithmic subtyping approach is identical to that of Dunfield et al, with two minor modifications. The first is the modification of $\st\forall$R to introduce a new existential variable instead of a simple type variable, a change made to enable proper bounding, as we cannot infer bounds for simple type variables. Second, we add the rule $\st$Sub, which again refers the algorithmic subtyping judgment $X \st Y$ to the external mechanism.

\begin{figure}

\small
\subfile{subtyping}
\caption{Algorithmic subtyping relationship}
\end{figure}

The bulk of our modifications are to instantiation. Dunfield et al's system eagerly instantiates existential variables to types, causing the system to become incomplete in the presence of subtyping. onsider the instantiations $. \vdash \text{Cat} \instr \alphahat \Theta$ and $\Theta \vdash \text{Dog} \instr \alphahat \Delta$, where $\text{Cat} \not\st \text{Dog}$ and vice versa, but both $\text{Cat} < \text{Animal}$ and $\text{Dog} < \text{Animal}$. In Dunfield's system, it would infer $\alphahat = \text{Cat}$ then be unable to derive the second judgement, as $\text{Dog}$ is not a subtype of Cat.

To solve this problem, we extend Dunfield's instantiation mechanism with a bounding mechanism similar to that of Pierce and Turner. Like Pierce and Turner, instead of directly instantiating an existential to a type $\tau$, we find a range of valid instantiations for the existential. However, we cannot directly use Pierce and Turner's mechanism, as it cannot maintain transitive relationships between existential variables. To solve this, we have to keep track of the relationships between existential variables, in addition to the relations between existential variables and atomic types. We implement this by adding a list of existentials to each bound (e.g. $\pair{\{\alphahat_1,\alphahat_2\}}{\tau} \st \alphahat$), enabling us to maintain the relationship between existential variables.

Additionally, we need to be able to compare a \emph{bound} to a \emph{type}. If an existential is bounded below by some $\sigma_l$, and we want to bound it above with some type $\tau$, then every member in $\sigma_l$ must be a subtype of $\tau$. To represent this in our rules, we use the $\sigbndl$ and $\sigbndr$ judgments. $\Gamma \vdash \sigma \sigbndl \tau \dashv \Delta$ ensures that every member of $\sigma$ is a subtype of $\tau$, producing the context $\Delta$, and $\Gamma \vdash \tau \sigbndr \sigma \dashv \Delta$ is the converse.
\begin{figure}[h]
\small
\subfile{instantiation}
\end{figure}

To this end, we redefine instantiation entirely. InstLVar is the case where an existential $\alphahat$, with bounds $\sigma_l$ and $\sigma_u$, should be bounded \emph{above} by some type $\tau$. Ensuring this case is obvious - we need to make sure that $\sigma_l \st \tau$ and that the upper bound's monotype $\tau_u$ is a supertype of $\tau$. We use the operator $\sigbndl$ to ensure that $\sigma_l$ on the left is a subtype of $\tau$ on the right, as it operates by simply checking the subtyping relationship between every member in the bound and the type $\tau$.

We then use the $\lub$ operator. The $\lub$ operator takes a bound $\sigma$ and a type $\tau$ and outputs a new bound $\sigma'$, where the monotype in $\sigma'$ is guaranteed to be a supertype of both the monotype in $\sigma$ and the type $\tau$. By using both of these operators, we ensure that $\alphahat$ \emph{can} be bounded above by $\tau$ (checking that the lower bound is below $\tau$ with $\sigbndl$) and that it is a subtype of $\tau$ (using $\lub$ to update the upper context).

The rules InstLVar and InstLReach work similarly. InstLVar instantiates $\alphahat$ to be a subtype of $\betahat$, by making sure $\betahat$ is a supertype of $\alphahat$'s lower bound (using the $\sigbndr$ operator, repeatedly instantiating $\betahat$) and simply adds it to the upper bound of $\alphahat$.

To ensure that the ordered property holds (e.g. $\alphahat$ references $\betahat$ iff $\alphahat$ follows $\betahat$ in the context), we use InstLReach, which works when $\alphahat$ (the variable being bounded) precedes $\betahat$ in the context. To maintain the invariant, we ensure that $\betahat$ is a supertype of $\alphahat$.

InstLArrow constrains $\alphahat$ such that it is bounded both above and below by the type $\alphahat_1 \rightarrow \alphahat_2$, equivalent to forcing equality. It then ensures that the correct relations between the new $\alphahat$ variables hold, which then implicitly enforces the constraint on $\alphahat$ itself.

InstLAIIL is identical in operation to Dunfield's InstRAIIL but for InstL, as defining $\lub$ and $\glb$ over $\alpha$ and $\alphahat$ is not possible.
\begin{figure}[h]
\small
\subfile{sigbnd}
\end{figure}

Bounding is defined in the obvious manner. We simply iterate through the existential variables, instantiating where required, and then check the subtyping relation on the final type $\tau$.

\begin{figure}[h]
\small
\subfile{bound_adapt}
\end{figure}

\section{Verification}

To be able to express soundness and completeness for the system, we relate our algorithmic context to the complete contexts of Dunfield et al by checking that the bounds expressed by our system map to the instantiation provided by the complete context. We express this relationship as $\Gamma \mapsto \Omega$, read as ``$\Omega$ satisfies $\Gamma$''.


\subfile{cext}

Additionally, note that our system only adds constraints, so constraints will continue to be valid in the future, and any later valid assignment will also satisfy earlier constraints. We introduce the operator $\hookrightarrow$ to represent this ``tightening'' relationship, allowing us to state a number of important lemmas.

\emph{Constraint addition} $\Gamma \vdash \alphahat \instl \tau \dashv \Delta$ implies $\Gamma \hookrightarrow \Delta$

\emph{Transitivity of tightening} $\Gamma \hookrightarrow \Theta$ and $\Theta \hookrightarrow \Delta$ implies $\Gamma \hookrightarrow \Delta$

\emph{Repeatability of instantiation} $\Gamma \vdash \alphahat \instl \tau \dashv \Theta$ and $\Theta \hookrightarrow \Delta$ implies $\Delta \vdash \alphahat \instl \tau \dashv \Delta$.

\emph{Stability of instantiation over tightening} $\Gamma \hookrightarrow \Theta$ and $\Theta \mapsto \Omega$ implies $\Gamma \mapsto \Omega$



To validate our system itself, we prove that it is sound and complete with reference to our algorithmic approach. Our theorems follow those of Dunfield, with the minor loss of one precondition to completeness of instantiation.

\emph{Soundness of instantiation} If $\Gamma \vdash \alphahat\instl A \dashv\Delta$ and $\Delta \mapsto \Omega$, then $[\Omega]\Gamma \vdash [\Omega]\alphahat \leq [\Omega]A$

\emph{Completeness of instantiation} 
Given that $\Gamma \longmapsto \Omega$, $\alphahat\not\in \text{FV}(A)$, and $\Gamma = \Gamma_1,\bound{\sigma_l}{\alphahat}{\sigma_u},\Gamma_2$\\
if $[\Omega]\Gamma\vdash [\Omega]\alphahat \leq [\Omega]A$, then there are some $\Delta,\Omega'$ such that $\Gamma \vdash \alphahat \instl A \dashv \Delta$, $\Delta \longmapsto \Omega'$, and $\Omega \longrightarrow \Omega'$.\\
if $[\Omega]\Gamma\vdash [\Omega]A \leq [\Omega]\alphahat$, then there are some $\Delta,\Omega'$ such that $\Gamma \vdash A \instr \alphahat \dashv \Delta$, $\Delta \longmapsto \Omega'$, and $\Omega \longrightarrow \Omega'$.

\end{document}