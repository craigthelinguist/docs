\documentclass{sig-alternate}

\usepackage{subfiles}
\usepackage{cite}
\usepackage{graphicx}
\usepackage{amsmath, amssymb, mathpartir}
\usepackage{array}
\numberofauthors{3}
\newcommand{\code}[1]{\texttt{\footnotesize #1}}
\newcommand{\todo}[1]{{\bf \{TODO: {#1}\}}}

\title{}

\author{
}

\begin{document}
\maketitle
\begin{abstract}

\end{abstract}

\section{Prior Work}

% The seminal work in this area is that by Pierce and Turner~\cite{Pierce:2000:LTI:345099.345100}. Their approach is built around subtyping for higher-order functions, and introduced the idea of using type argument synthesis. In their system, they develop a set of constraints for each type parameter, then manipulate these sets to find a consistent argument type, akin to a simplified unification system. This approach is quite limited in comparison, however, as their approach cannot handle transitive subtyping relations among existential type variables.

% Odersky's 2001 paper~\cite{odersky2001colored} simplifies the issue, but raises problems of its own. His approach also uses a bidirectional type system, extending the bidirectional concept to the types and type variables themselves. The system colors types to indicate if they are synthetic or analytic, then using this information to inform type argument instantiation. However, this system does not handle subtyping well (requiring a unification system for the single special-case), making it unsuitable for our purposes.

% We base our system of Dunfield and Krishnaswami's 2013 paper~\cite{Dunfield:2013:CEB:2544174.2500582} as a basis for our work. This is a development of the 2009 effort, fixing the completness issues as well as simplifying the system substantially through the introduction of new judgments. The system uses existentials to replace type parameters in parameterized types, then resolves them using relations discovered during subtyping.

% This approach is quite simple and elegant in construction, but has a major drawback for our application, namely that the system fails for type systems with subtyping. As seen in figure~\ref{fig:intro-examples}, this ``greedy'' approach to instantiating variables fails in cases where the first constraint is tighter than (or disjoint from) the second, making it hard to apply in many modern languages. 

% Systems for providing sound and complete typechecking with subtyping do exist, all from the unification community. While these approaches provide the soundness and completeness guarantees that we do, they are also substantially more complex. Additionally, unification is much more difficult for a user to understand than bidirectional subtyping, as the internal operation of the unification mechanism is not as intuitive as bidirectional type inference systems. 

\section{Declarative Type System}

\begin{figure}
\subfile{decl}
\small
\caption{Subtyping in the declarative system}
\end{figure}
\begin{figure}
\subfile{declCons}
\small
\label{fig:declCons}
\caption{Structure of the declarative system}
\end{figure}
\begin{figure}
\subfile{terms}
\small
\label{fig:terms}
\caption{Source Expressions}
\end{figure}
\begin{figure}
\small
\subfile{typing}
\caption{Typing rules for our system}
\end{figure}

\section{Algorithmic Type System}

To justify the construction of our algorithmic type system, consider the problematic rule $\st\forall$App. If we are able to find a value for $\tau$ that satisfies the given constraints, we're done. Finding this type is difficult, however, as it requires solving the type constraints implied by the subtyping relationship. Dunfield solves this problem elegantly by replacing $\tau$ with the existential variable $\alphahat$, then determining values for $\alphahat$ based on what subtyping judgements arise for it. In this way, if $\alphahat$ has a valid instantiation, then the same value will work for $\tau$.

One problem with this approach, however, is that it eagerly determines instantiations for $\alphahat$ and is thus incomplete in the presence of subtyping. Suppose we say that $\alphahat$ is a subtype of $\tau_1$ and $\tau_2$, and also that $\tau_1 \st \tau_2$. If we set $\alphahat$ to $\tau_2$ first, then try to check if $\alphahat \st \tau_1$, it does not work, despite even though we could have set $\alphahat$ to $\tau_1$ and both constraints would have held.

To get around this, we replace the notion of instantiation, taking an existential and a type and setting the existential to the type, with the idea of bounding, where an existential must be a subtype of another type. This enables us to represent the above constraint, as because we know that $\tau_1 \st \tau_2$, we can safely replace the upper bound $\tau_2$ on $\alphahat$ with $\tau_1$. 

This system, using upper and lower bounds consisting of types has a flaw, however. Consider two existential variables $\alphahat$ and $\betahat$, where $\alphahat \st \betahat$. $\alphahat$ can be a subtype of some $\tau$ with no additional constraint on $\betahat$, and $\betahat$ can be a supertype of some other $\tau'$ with no implications for $\alphahat$. As such, we cannot represent relationships between two existentials and retain soundness while only maintaining a single type for bounds.

Instead, our approach uses a pair of a type $\tau$ and a list of existential variables $\rho$ for bounds. Suppose that $\alphahat$ is bounded below by both some $\tau$ and an existential $\betahat$, represented as $\pair{\{\betahat\}}{\tau}$, referred to as $\sigma$. This representation enables us to maintain relationships between existentials, while still ensuring that bounding relationships hold.

We need two new concepts to extend Dunfield's formalism to our approach. The first is the notion of sigma bounding, where a bound of the form $\pair{\rho}{\tau}$ is bounded below or above by some other $\tau'$, and the second is greatest lower/least upper bounds.

Sigma bounding enables us to check types against bounds. Consider the case where $\alphahat$ is constrained below by $\sigma$ and we add the constraint $\tau$. It suffices to show that every member (that is, the type and the list of existential type vriables) in $\sigma$ is a subtype of $\tau$, a constraint we represent with sigma bounding, and notated as $\sigma \sigbndl \tau$.

Our notion of greatest lower ($\glb$) and least upper bounds ($\lub$) is similar to that of Pierce and Turner, with some modifications to adapt it to our system. Our operators work over a bound and a type, and produce a new bound. For example, suppose that $\sigma \glb \tau = \sigma'$. Then, $\tau \sigbndr \sigma'$ and every member of $\sigma$ is bounded above by $\sigma'$.
\begin{figure}

\small
\subfile{subtyping}
\caption{Algorithmic subtyping relationship}
\end{figure}

\begin{figure}[h]
\small
\subfile{instantiation}
\end{figure}

Bounding works by additionally constraining an existential type variable in the context. Suppose that $\alphahat$ is bounded above by both some $\betahat$ and a $\tau$. If we then add a $\tau' \st \tau$, then $\alphahat$ would be bounded above by $\betahat$ and the new $\tau'$. Likewise, if we added a $\hat{\gamma}$, then $\alphahat$ should be bounded above by both $\betahat$ and $\hat{\gamma}$. We represent this action with the notation $\Gamma \vdash \alphahat \instl A \dashv \Delta$ (for upper constraints) and $\Gamma \vdash A \instr \alphahat \dashv \Delta$ (for lower constraints).

This can be easily seen in the rule InstLAbs, where $\tau$ is a simple type. We are trying to constrain $\alphahat$, already bounded by $\sigma_l$ from below and $\sigma_t$ from above, and want to add the new constraint $\tau$. To make sure that $\alphahat$ can possibly take on $\tau$ as a value, we need to make sure that the lower bound will admit it. We do so using the first predicate, $\Gamma \vdash \sigma_l \sigbndl \tau \dashv \Theta$, which, against the input context $\Gamma$ checks that every constraint in $\sigma_l$ is a subtype of $\tau$. Now that we know that it is possible for the constraint to hold, we then ensure that the upper bound constraint holds via the least upper bound operator, producing the new upper bound $\sigma'$.

InstLReach is critical to maintaining the ordered context property. To ensure that no cycles arise in the typing relationships, we apply an ordering to the context. However, in the case where we are bounding one existential $\alphahat$ above with another $\betahat$, where $\alphahat$ is \emph{above} $\betahat$ in the context. Since we cannot add $\betahat$ to the bound on $\alphahat$, we have to add $\alphahat$ to the lower bound of $\betahat$, using InstLReach

Typing the arrow type is difficult, as it requires that we infer constraints on both the argument and return type. We do so by constraining $\alphahat$ such that $\alphahat = \alphahat_1 \rightarrow \alphahat_2$, where $\alphahat_1$ and $\alphahat_2$ are fresh existential variables, then bound $\alphahat_1$ and $\alphahat_2$ with the constraint types.

Our InstLAIIL and InstRAIIL rules are a copy of those proposed by Dunfield for InstLAIIL. As our system does not have pure type variables, only existential variables, we use the form of InstLAIIL in InstLAIIR.

The InstR rules are entirely symmetric, considering new lower bounds instead of new upper bounds.
 

Bounding is defined in the obvious manner. We simply iterate through the existential variables, instantiating where required, and then check the subtyping relation on the final type $\tau$.


\section{Verification}

\subfile{cext}



To validate our system, we prove that it is sound and complete with reference to our algorithmic approach. Our theorems follow those of Dunfield, with the minor loss of one precondition to completeness of instantiation.

\emph{Soundness of instantiation} If $\Gamma \vdash \alphahat\instl A \dashv\Delta$ and $\Delta \mapsto \Omega$, then $[\Omega]\Gamma \vdash [\Omega]\alphahat \leq [\Omega]A$

\emph{Completeness of instantiation} 
Given that $\Gamma \longmapsto \Omega$, $\alphahat\not\in \text{FV}(A)$, and $\Gamma = \Gamma_1,\bound{\sigma_l}{\alphahat}{\sigma_u},\Gamma_2$\\
if $[\Omega]\Gamma\vdash [\Omega]\alphahat \leq [\Omega]A$, then there are some $\Delta,\Omega'$ such that $\Gamma \vdash \alphahat \instl A \dashv \Delta$, $\Delta \longmapsto \Omega'$, and $\Omega \longrightarrow \Omega'$.\\
if $[\Omega]\Gamma\vdash [\Omega]A \leq [\Omega]\alphahat$, then there are some $\Delta,\Omega'$ such that $\Gamma \vdash A \instr \alphahat \dashv \Delta$, $\Delta \longmapsto \Omega'$, and $\Omega \longrightarrow \Omega'$.

\end{document}