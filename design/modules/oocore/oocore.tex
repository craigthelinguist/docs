\documentclass{llncs}

\usepackage{listings}
\usepackage{proof}
\usepackage{amssymb}
\usepackage[margin=.9in]{geometry}
\usepackage{amsmath}

\lstset{tabsize=3, basicstyle=\ttfamily\small, commentstyle=\itshape\rmfamily, numbers=left, numberstyle=\tiny, language=java,moredelim=[il][\sffamily]{?},mathescape=true,showspaces=false,showstringspaces=false,columns=fullflexible,xleftmargin=15pt,escapeinside={(@}{@)}, morekeywords=[1]{objtype,module,import,let,in,fn,var,type,rec,fold,unfold,letrec,alloc,ref,application,policy,external,component,connects,to,meth,val,where,return,group,by,within,count,connect,with,attr,html,head,title,style,body,div,keyword,unit,def}}
\lstloadlanguages{Java,VBScript,XML,HTML}

\newcommand{\keywadj}[1]{\mathtt{#1}}
\newcommand{\keyw}[1]{\keywadj{#1}~}
\newcommand{\reftt}{\mathtt{ref}~}
\newcommand{\Reftt}{\mathtt{Ref}~}
\newcommand{\inttt}{\mathtt{int}~}
\newcommand{\Inttt}{\mathtt{Int}~}
\newcommand{\stepsto}{\leadsto}

\begin{document}


\section{Modules}

\subsection{Syntax}

\[
\begin{array}{lll}
\begin{array}{lllr}
e & ::= & x & expressions \\
& | & \keywadj{new}_{[p]}(x \Rightarrow d) \\
& | & e.m(e)\\
& | & e.f \\
& | & e.f = e \\
& | & l & run-time forms\\
& | & l.m(l) \rhd e \\
&&\\
d & ::= & \bullet & declarations \\
  & |   & \keyw{def} m(x:\tau):\tau = e; d \\
  & |   & \keyw{var} f:\tau = e; d \\
&&\\
\tau & ::= & \{ x \Rightarrow \sigma \}_{[s]} & types \\
&&\\
\sigma & ::= & \bullet & decl.~ types \\
       & |   & \keyw{def} m:\tau \rightarrow \tau; \sigma \\
       & |   & \keyw{var} f:\tau; \sigma \\
&&\\
\Gamma & :: = & \varnothing & contexts\\
& | & \Gamma,~x : \tau\\
\end{array}
& ~~~~~~
&
\begin{array}{lllr}
\mu & :: = & \bullet & store\\
& | & \mu,~l \mapsto \{ x \Rightarrow d_v \}_{[s]}\\
&&\\
d_v & ::= & \bullet & decl.~ values \\
  & |   & \keyw{def} m(x:\tau):\tau = e; d_v \\
  & |   & \keyw{var} f:\tau = l; d_v \\
&&\\
\Sigma & :: = & \bullet & store type\\
& | & \Sigma,~l : \tau\\
&&\\
E & ::= & \diamond & eval.~ contexts\\
  & |   & \keyw{new}_{[p]}(x \Rightarrow D) \\
  & |   & E.m(e)\\
  & |   & l.m(E)\\
  & |   & E.f \\
  & |   & E.f = e \\
  & |   & l.f = E \\
  & |   & l.m(l) \rhd E \\
&&\\
D & ::= & \keyw{def} m(x:\tau):\tau = e; D & decl.~ contexts\\
  & |   & \keyw{var} f:\tau = E; d \\
  & |   & \keyw{var} f:\tau = e; D \\
&&\\
\end{array}
\end{array}
\]

\newpage

\subsection{Authority}

\[
\begin{array}{c}
\infer[\textsc{(auth-config)}]
  {auth(l,\mu \rhd e) = auth(l,\mu) \cup auth(l,e,\mu)}
  {}\\[5ex]

\infer[\textsc{(auth-store)}]
  {auth(l,\mu) = pointsto(d_v,\mu)}
  {\mu(l) = \{ x \Rightarrow d_v \}_{[s]}}\\[5ex]
  
\infer[\textsc{(auth-stack)}]
  {auth(l,E[l.m(l') \rhd e'],\mu) = pointsto(e',\mu) \cup auth(l,e',\mu)}
  {l.m(l'') \rhd E' \not\in E}\\[5ex]
  
\infer[\textsc{(auth-stack-nocall)}]
  {auth(l,e,\mu) = \varnothing}
  {l.m(l') \rhd e' \not\in e}\\[5ex]
  
\infer[\textsc{(pointsto-def)}]
  {pointsto(\keyw{def} m(x:\tau):\tau = e; d,\mu) = pointsto(e,\mu) \cup pointsto(d,\mu)}
  {}\\[5ex]
  
\infer[\textsc{(pointsto-var)}]
  {pointsto(\keyw{var} f:\tau = e; d,\mu) = pointsto(e,\mu) \cup pointsto(d,\mu)}
  {}\\[5ex]
  
\infer[\textsc{(pointsto-principal)}]
  {pointsto(l,\mu) = \{ l \}}
  {\mu(l) = \{ x \Rightarrow d_v \}_s}\\[5ex]
  
\infer[\textsc{(pointsto-data)}]
  {pointsto(l,\mu) = \varnothing }
  {\mu(l) = \{ x \Rightarrow d_v \}}\\[5ex]

\infer[\textsc{(pointsto-call-principal)}]
  {pointsto(l.m(l') \rhd e,\mu) = \{l\} }
  {\mu(l) = \{ x \Rightarrow d_v \}_s}\\[5ex]
  
\infer[\textsc{(pointsto-call-data)}]
  {pointsto(l.m(l') \rhd e,\mu) = pointsto(e,\mu)}
  {\mu(l) = \{ x \Rightarrow d_v \}}\\[5ex]
  
\infer[\textsc{(pointsto-new)}]
  {pointsto(\keywadj{new}_{[p]}(x \Rightarrow d),\mu) = pointsto(d,\mu) }
  {}\\[5ex]
  
\infer[\textsc{(pointsto-otherexp)}]
  {pointsto(e,\mu) = pointsto(subexprs(e),\mu) }
  {\textit{when $e$ is not one of the expression forms defined above}}\\[5ex]

\end{array}
\]

note: the last rule is a shorthand; the rest of the \textit{pointsto} rules for expressions are just congruence - look inside the e's


\newpage

\subsection{Theory}

\textbf{Theorem [Authority Safety].}  If $\Sigma \vdash \mu$, $\bullet, \Sigma \vdash e : \tau$, and $\mu \rhd e \stepsto \mu' \rhd e'$ then $\forall l \in domain(\mu)$ if $auth(l,\mu' \rhd e') \supset auth(l,\mu \rhd e)$ then one of the following holds:

\begin{itemize}
  \item \textbf{Creation.} $e = E[\keywadj{new}_p(x \Rightarrow d_v)]$, $e' = E[l']$, and $auth(l,\mu' \rhd e') = auth(l,\mu \rhd e) \cup \{ l' \}$.  \textit{check that the creator was $l$}
  \item \textbf{Call.} $e = E[l.m(l')]$, $e' = E[l.m(l') \rhd e]$, and $auth(l,\mu' \rhd e') = auth(l,\mu \rhd e) \cup \{ pointsto(l',\mu) \}$  \textit{check that the caller previously had the authority that $l$ gained}
  \item \textbf{Return.}
\end{itemize}

Here, the Authority Safety theorem follows the definition in Maffeis et al. 2010.

To fix

\begin{itemize}
  \item have auth-stack and auth-heap variants of auth
  \item use s consistently instead of p
  \item don't overload triangle
\end{itemize}


\end{document}