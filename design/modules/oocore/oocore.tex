\documentclass{llncs}

\usepackage{listings}
\usepackage{proof}
\usepackage{amssymb}
\usepackage[margin=.9in]{geometry}
\usepackage{amsmath}
\usepackage[english]{babel}
\usepackage[utf8]{inputenc}
\usepackage{enumitem}
 
\usepackage{fancyhdr}
\renewcommand{\headrulewidth}{0pt}
\pagestyle{fancy}
\fancyhf{}
\rhead{\thepage}

\lstset{tabsize=3, basicstyle=\ttfamily\small, commentstyle=\itshape\rmfamily, numbers=left, numberstyle=\tiny, language=java,moredelim=[il][\sffamily]{?},mathescape=true,showspaces=false,showstringspaces=false,columns=fullflexible,xleftmargin=15pt,escapeinside={(@}{@)}, morekeywords=[1]{objtype,module,import,let,in,fn,var,type,rec,fold,unfold,letrec,alloc,ref,application,policy,external,component,connects,to,meth,val,where,return,group,by,within,count,connect,with,attr,html,head,title,style,body,div,keyword,unit,def}}
\lstloadlanguages{Java,VBScript,XML,HTML}

\newcommand{\keywadj}[1]{\mathtt{#1}}
\newcommand{\keyw}[1]{\keywadj{#1}~}
\newcommand{\reftt}{\mathtt{ref}~}
\newcommand{\Reftt}{\mathtt{Ref}~}
\newcommand{\inttt}{\mathtt{int}~}
\newcommand{\Inttt}{\mathtt{Int}~}
\newcommand{\stepsto}{\leadsto}

\newlist{pcases}{enumerate}{1}
\setlist[pcases]{
  label=\textit{Case}\protect\thiscase\textit{:}~,
  ref=\arabic*,
  align=left,
  labelsep=0pt,
  leftmargin=0pt,
  labelwidth=0pt,
  parsep=0pt
}
\newcommand{\pcase}[1][]{
  \if\relax\detokenize{#1}\relax
    \def\thiscase{}
  \else
    \def\thiscase{~#1}
  \fi
  \item
}

\begin{document}

\section{Syntax}

\[
\begin{array}{lll}
\begin{array}{lllr}
e & ::= & x & expressions \\
& | & \keywadj{new}_{s}(x \Rightarrow d) \\
& | & e.m(e)\\
& | & e.f \\
& | & e.f = e \\
& | & \keyw{bind} x = e~\keyw{in} e \\
& | & l & (run\mbox{-}time~forms)\\
%& | & \keywadj{new}_{s}(x \Rightarrow d_v)\\
& | & l.m(l) \rhd e \\
&&\\
s & ::= & \keyw{stateful} | ~\keyw{pure} \\
&&\\
d & ::= & \epsilon & declarations \\
  & |   & \keyw{def} m(x:\tau):\tau = e; d \\
  & |   & \keyw{var} f:\tau = x; d \\
  & |   & \keyw{var} f:\tau = l; d & (run\mbox{-}time~form)\\
&&\\
%d_v & ::= & \epsilon & decl.~values \\
%  & |   & \keyw{def} m(x:\tau):\tau = e; d_v \\
%  & |   & \keyw{var} f:\tau = l; d_v \\
%&&\\
\tau & ::= & \{ \sigma \}_{s} & types \\
&&\\
\sigma & ::= & \epsilon & decl.~ types \\
       & |   & \keyw{def} m:\tau \rightarrow \tau; \sigma \\
       & |   & \keyw{var} f:\tau; \sigma \\
&&\\
\end{array}
& ~~~~~~
&
\begin{array}{lllr}
\Gamma & :: = & \varnothing & contexts\\
& | & \Gamma,~x : \tau\\
&&\\
\mu & :: = & \varnothing & store\\
%& | & \mu,~l \mapsto \{ x \Rightarrow d_v \}_{s}\\
& | & \mu,~l \mapsto \{ x \Rightarrow d \}_{s}\\
&&\\
\Sigma & :: = & \varnothing & store~type\\
& | & \Sigma,~l : \tau\\
&&\\
E & ::= & [~] & evaluation~ contexts\\
%  & |   & \keywadj{new}_{s}(x \Rightarrow D) \\
%  & |   & \keywadj{new}_{s}(x \Rightarrow d) \\
  & |   & E.m(e)\\
  & |   & l.m(E)\\
  & |   & E.f \\
  & |   & E.f = e \\
  & |   & \keyw{bind} x = E~\keyw{in} e \\
  & |   & \keyw{bind} x = l~\keyw{in} E \\
  & |   & l.f = E \\
  & |   & l.m(l) \rhd E \\
&&\\
%D & ::= & [~] & decl.~eval.~ contexts\\
%  & |   & \keyw{def} m(x:\tau):\tau = e; D \\
%  & |   & \keyw{var} f:\tau = E; d \\
%  & |   & \keyw{var} f:\tau = l; D \\
%&&\\
\end{array}
\end{array}
\]

\noindent Notes:

\begin{itemize}
\item The $s$ tag indicates whether an object is stateful (i.e. captures mutable state) or pure (i.e. captures no mutable state).
\item Example of how to initialize and use object declarations (and potentially encode modules):
\[
\keywadj{new}_{s}(x \Rightarrow \keyw{def} m(y : \tau) : \tau' = \keywadj{new}_{s}(x \Rightarrow \keyw{var} f_1 : \tau = y;~\keyw{var} f_2 : \tau = y;~\keyw{var} f_3 : \tau = y)).m(\keywadj{new}_{s} (x \Rightarrow \epsilon))
\]
which in memory looks like 
\[
l_2 \mapsto \{l_1,l_1,l_1\},~l_1 \mapsto \varnothing
\]
\item In the \textsc{DT-DefPure} rule, the argument $x$ may be stateful, but because all other variables in $\Gamma$ are pure, $x$ cannot be used (e.g. be assigned to a variable) inside $e$.
\item In the preservation of types under substitution lemma, $y$ and $z$ have to be values because we don't have $\keyw{var} f:\tau = e; d$ and instead have a more restrictive $\keyw{var} f:\tau = l;d$.
\end{itemize}


\newpage

\section{Semantics}

$\fbox{$\Gamma~|~\Sigma \vdash e : \tau$}$
\[
\begin{array}{c}
\infer[\textsc{(T-Var)}]
  {\Gamma~|~\Sigma \vdash x : \tau}
  {x : \tau \in \Gamma}%\\[5ex]
~~~~~~~~~~
\infer[\textsc{(T-New)}]
	{\Gamma~|~\Sigma \vdash \keywadj{new}_{s}(x \Rightarrow d) : \{ \sigma \}_{s}}
	{\Gamma,~x : \{ \sigma \}_{s}~|~\Sigma \vdash_s d : \sigma} \\[5ex]

\infer[\textsc{(T-Meth)}]
	{\Gamma~|~\Sigma \vdash e_1.m(e_2) : \tau_1} 
	{\Gamma~|~\Sigma \vdash e_1 : \{\sigma\}_s  & \keyw{def}~ m : \tau_2 \rightarrow \tau_1 \in \sigma & \Gamma~|~\Sigma \vdash e_2 : \tau_2}\\[5ex]

\infer[\textsc{(T-Field)}]
	{\Gamma~|~\Sigma \vdash  e.f : \tau} 
	{\Gamma~|~\Sigma \vdash e : \{\sigma\}_s & \keyw{var}~ f : \tau \in \sigma}\\[5ex]
	
\infer[\textsc{(T-Assign)}]
	{\Gamma~|~\Sigma \vdash  e_1.f=e_2 : \tau} 
	{\Gamma~|~\Sigma \vdash e_1 : \{\sigma\}_s & \keyw{var}~ f : \tau \in \sigma & \Gamma~|~\Sigma \vdash e_2 : \tau}\\[5ex]

\infer[\textsc{(T-Bind)}]
  {\Gamma~|~\Sigma \vdash \keyw{bind} x = e_1~\keyw{in} e_2 : \tau_2}
  {\Gamma~|~\Sigma \vdash e_1 : \tau_1 & x : \tau_1~|~\Sigma \vdash e_2 : \tau_2}%\\[5ex]
~~~~~~~~~~
\infer[\textsc{(T-Loc)}]
  {\Gamma~|~\Sigma \vdash l : \tau}
  {l : \tau \in \Sigma}\\[5ex]

\infer[\textsc{(T-StackFrames)}]
	{\Gamma~|~\Sigma \vdash l_1.m(l_2) \rhd e : \tau_1}
	{\Gamma~|~\Sigma \vdash l_1 : \{\sigma\}_s & \keyw{def}~ m : \tau_2 \rightarrow \tau_1 \in \sigma & \Gamma~|~\Sigma \vdash l_2 : \tau_2 & \Gamma~|~\Sigma \vdash e : \tau_1} \\[5ex]

\infer[\textsc{(T-Sub)}]
  {\Gamma~|~\Sigma \vdash e : \tau_2}
  {\Gamma~|~\Sigma \vdash e : \tau_1 & \tau_1 <: \tau_2}\\[5ex]

\end{array}
\]

$\fbox{$\Gamma~|~\Sigma \vdash_s d : \sigma$}$
\[
\begin{array}{c}

\infer[\textsc{(DT-DefPure)}]
  {\Gamma~|~\Sigma \vdash_{\keyw{pure}} \keyw{def} m(x : \tau_1) : \tau_2 = e; d~:~\keyw{def} m : \tau_1 \rightarrow \tau_2; \sigma}
  {\def\arraystretch{1.6}
  \begin{array}{c}
\Gamma_{stateful} = \{x : \{ \sigma \}_{\keyw{stateful}} ~|~ x : \{ \sigma \}_{\keyw{stateful}} \in \Gamma\} \\
\Gamma_{pure} = \Gamma \setminus \Gamma_{stateful}~~~~~~~~~~\Gamma_{pure},~x : \tau_1~|~\Sigma \vdash e : \tau_2~~~~~~~~~~\Gamma_{pure}~|~\Sigma \vdash_s d : \sigma
  \end{array}}\\[5ex]

\infer[\textsc{(DT-DefStateful)}]
  {\Gamma~|~\Sigma \vdash_{\keyw{stateful}} \keyw{def} m(x : \tau_1) : \tau_2 = e; d~:~\keyw{def} m : \tau_1 \rightarrow \tau_2; \sigma}
  {\Gamma,~x : \tau_1~|~\Sigma \vdash e : \tau_2 & \Gamma~|~\Sigma \vdash_s d : \sigma}\\[5ex]

\infer[\textsc{(DT-Varx)}]
  {\Gamma~|~\Sigma \vdash_{\keyw{stateful}} \keyw{var} f : \tau = x; d~:~\keyw{var} f : \tau; \sigma}
  {\Gamma~|~\Sigma \vdash x : \tau & \Gamma~|~\Sigma \vdash_s d : \sigma}\\[5ex]

\infer[\textsc{(DT-Varl)}]
  {\Gamma~|~\Sigma \vdash_{\keyw{stateful}} \keyw{var} f : \tau = l; d~:~\keyw{var} f : \tau; \sigma}
  {\Gamma~|~\Sigma \vdash l : \tau & \Gamma~|~\Sigma \vdash_s d : \sigma}\\[5ex]

\end{array}
\]

$\fbox{$\tau <: \tau'$}$
\[
\begin{array}{c}
\infer[\textsc{(S-Refl)}]
  {\tau <: \tau}
  {}%\\[5ex]
~~~~~~~~~~
\infer[\textsc{(S-Trans)}]
  {\tau_1 <: \tau_3}
  {\tau_1 <: \tau_2 & \tau_2 <: \tau_3}\\[5ex]

\infer[\textsc{(S-State)}]
  {\{ \sigma \}_{\keyw{pure}} <: \{ \sigma' \}_{\keyw{stateful}}}
  {\sigma <: \sigma'}%\\[5ex]  
~~~~~~~~~~
\infer[\textsc{(S-Obj)}]
  {\{ \sigma \}_{s} <: \{ \sigma' \}_{s}}
  {\sigma <: \sigma'}\\[5ex]  

\end{array}
\]

$\fbox{$\sigma <: \sigma'$}$
\[
\begin{array}{c}
\infer[\textsc{(S-Refl)}]
  {\sigma <: \sigma}
  {}%\\[5ex]
~~~~~~~~~~
\infer[\textsc{(S-Trans)}]
  {\sigma_1 <: \sigma_3}
  {\sigma_1 <: \sigma_2 & \sigma_2 <: \sigma_3}%\\[5ex]
~~~~~~~~~~
\infer[\textsc{(S-Eps)}]
  {\sigma <: \epsilon}
  {}\\[5ex]

\infer[\textsc{(S-Def)}]
  {\keyw{def} m:\tau_1 \rightarrow \tau_2; \sigma <: \keyw{def} m:\tau_1' \rightarrow \tau_2'; \sigma'}
  {\tau_1' <: \tau_1 & \tau_2 <: \tau_2' & \sigma <: \sigma'}%\\[5ex]  
~~~~~~~~~~
\infer[\textsc{(S-Var)}]
  {\keyw{var} f:\tau; \sigma <: \keyw{var} f:\tau; \sigma'}
  {\sigma <: \sigma'}\\[5ex]  

\end{array}
\]

$\fbox{$\mu : \Sigma$}$
\[
\begin{array}{c}

\infer[\textsc{(T-StoreEmpty)}]
  {\varnothing : \varnothing}
  {}%\\[5ex]
~~~~~~~~~~
\infer[\textsc{(T-Store)}]
  {\mu,~l \mapsto \{ x \Rightarrow d \}_s~:~\Sigma,~l : \{ \sigma \}_s}
  {\mu : \Sigma & \Gamma,~x : \{ \sigma \}_s~|~\Sigma \vdash_s d : \sigma}\\[5ex]

\end{array}
\]

$\fbox{$E[e]~|~\mu \longrightarrow E[e']~|~\mu'$}$
\[
\begin{array}{c}
\infer[\textsc{(E-Congruence)}]
  {E[e]~|~\mu \longrightarrow E[e']~|~\mu'}
  {e~|~\mu \longrightarrow e'~|~\mu'}\\[5ex]  
\end{array}
\]


$\fbox{$e~|~\mu \longrightarrow e'~|~\mu'$}$
\[
\begin{array}{c}
\infer[\textsc{(E-New)}]
  {\keywadj{new}_{s}(x \Rightarrow d)~|~\mu \longrightarrow l~|~\mu, l \mapsto \{ x \Rightarrow d \}_{s}}
  {l \not\in dom(\mu)}\\[5ex]

\infer[\textsc{(E-Meth)}]
  {l_1.m(l_2)~|~\mu \longrightarrow l_1.m(l_2) \rhd [l_2/y][l_1/x]e~|~\mu}
  {l_1 \mapsto \{ x \Rightarrow d \}_{s} \in \mu & \keyw{def} m(y : \tau_1) : \tau_2 = e \in d}\\[5ex]

\infer[\textsc{(E-Field)}]
  {l.f~|~\mu \longrightarrow l_1~|~\mu}
  {l \mapsto \{ x \Rightarrow d \}_{s} \in \mu & \keyw{var} f:\tau = l_1 \in d}\\[5ex]

\infer[\textsc{(E-Assign)}]
  {l_1.f = l_2~|~\mu \longrightarrow l_2~|~\mu'}
  {\def\arraystretch{1.6}
  \begin{array}{c}
l_1 \mapsto \{ x \Rightarrow d \}_{s} \in \mu~~~~~~~~~~~\keyw{var} f:\tau = l \in d \\
d' = [\keyw{var} f:\tau = l_2/\keyw{var} f:\tau = l]d~~~~~\mu' = [l_1 \mapsto \{ x \Rightarrow d' \}_{s}/l_1 \mapsto \{ x \Rightarrow d \}_{s}]\mu
  \end{array}}\\[5ex]

\infer[\textsc{(E-Bind)}]
  {\keyw{bind} x = l_1~\keyw{in} l_2~|~\mu \longrightarrow l_2~|~\mu'}
  {}%\\[5ex]
~~~~~~~~~~
\infer[\textsc{(E-StackFrame)}]
  {l.m(l_1) \rhd l_2~|~\mu \longrightarrow l_2~|~\mu'}
  {}\\[5ex]
  
\end{array}
\]

\newpage

\section{Theorems}

\subsection{Preservation}

\begin{lemma}[Permutation]
If $\Gamma~|~\varnothing \vdash e : \tau$ and $\Delta$ is a permutation of $\Gamma$, then $\Delta~|~\varnothing \vdash e : \tau$, and the latter derivation has the same depth as the former.
\end{lemma}

\begin{proof}
Straightforward induction on typing derivations. \qed
\end{proof}

\vspace{8pt}

\begin{lemma}[Weakening]
If $\Gamma~|~\varnothing \vdash e : \tau$ and $x \not\in dom(\Gamma)$, then $\Gamma,~x : \tau'~|~\varnothing \vdash e : \tau$, and the latter derivation has the same depth as the former.
\end{lemma}

\begin{proof}
Straightforward induction on typing derivations. \qed
\end{proof}

\vspace{8pt}

\begin{lemma}[Preservation of types under substitution]
If $\Gamma,~z : \tau'~|~\Sigma \vdash e : \tau$ and $\Gamma~|~\Sigma \vdash e' : \tau'$, then $\Gamma~|~\Sigma \vdash [e'/z]e : \tau$.
\end{lemma}

\begin{proof} The proof is by induction on a derivation of $\Gamma,~z : \tau'~|~\Sigma \vdash e : \tau$ and $\Gamma,~z : \tau'~|~\Sigma \vdash_s d : \sigma$. For a given derivation, we proceed by cases on the final typing rule used in the proof:

\begin{pcases}
\pcase[\textsc{T-Var}]
$e = x$ and, from the premise, we get $x : \tau \in (\Gamma,~z : \tau')$. There are two subcases to consider, depending on whether $x$ is $z$ or another variable. If $x = z$, then $[e'/z]x = e'$. The required result is then $\Gamma~|~\varnothing \vdash e' : \tau'$, which is among the assumptions of the lemma. Otherwise, $[e'/z]x = x$, and the desired result is immediate.
\\
\pcase[\textsc{T-New}]
$e = \keywadj{new}_{s}(x \Rightarrow d)$ and, from the premise, we get
\[
\Gamma,~x : \{ \sigma \}_{s},~z : \tau'~|~\Sigma \vdash d : \sigma
\]
Here, $d$ is a sequence of declarations each of which are either a method declaration or a field, and so we consider each of the two possibilities in order next.
\\
\begin{itemize}
\item[] \textit{Case $d$ is a method definition:} $d = \keyw{def} m(x : \tau_1) : \tau_2 = e$. We do substitution on the method body expression, which is subject to alpha-conversion. There are two cases: either the method is pure or it is stateful:
\\
\begin{itemize}
\item[] \textit{Subcase} \textsc{DT-DefPure}\textit{:} From the premise, we get
\[
\Gamma_{stateful} = \{x : \{ \sigma \}_{\keyw{stateful}} ~|~ x : \{ \sigma \}_{\keyw{stateful}} \in \Gamma\}~~~~~\Gamma_{pure} = \Gamma \setminus \Gamma_{stateful}
\]\[
\Gamma_{pure},~x : \tau_1,~z : \tau'~|~\Sigma \vdash e : \tau_2~~~~~~~~~~\Gamma_{pure},~z : \tau'~|~\Sigma \vdash_s d : \sigma
\]
By the induction hypothesis, $\Gamma_{pure},~x : \tau_1~|~\Sigma \vdash [e'/z]e : \tau_2$, and substituting $e'$ for $z$ in $d$ preserves the original types. (The latter can be proved by induction on $d$ starting with the end of the sequence when $d = \epsilon$.) Then, by \textsc{DT-DefPure}, $\Gamma~|~\Sigma \vdash_{\keyw{pure}} \keyw{def} m(x : \tau_1) : \tau_2 = [e'/z]e; [e'/z]d~:~\keyw{def} m : \tau_1 \rightarrow \tau_2; \sigma$, i.e. $\Gamma~|~\Sigma \vdash_{\keyw{pure}} [e'/z](\keyw{def} m(x : \tau_1) : \tau_2 = e; d)~:~\keyw{def} m : \tau_1 \rightarrow \tau_2; \sigma$.
\\
\item[] \textit{Subcase} \textsc{DT-DefStateful}\textit{:} From the premise, we get
\[
\Gamma, x : \tau_1,~z : \tau'~|~\Sigma \vdash e : \tau_2~~~~~\Gamma,~z : \tau'~|~\Sigma \vdash_s d : \sigma
\]
By the induction hypothesis, $\Gamma, x : \tau_1~|~\Sigma \vdash [e'/z]e : \tau_2$, and substituting $e'$ for $z$ in $d$ preserves the original types. (The latter can be proved by induction on $d$ starting with the end of the sequence when $d = \epsilon$.) Then, by \textsc{DT-DefStateful}, $\Gamma~|~\Sigma \vdash_{\keyw{stateful}} \keyw{def} m(x : \tau_1) : \tau_2 = [e'/z]e; [e'/z]d~:~\keyw{def} m : \tau_1 \rightarrow \tau_2; \sigma$, i.e. $\Gamma~|~\Sigma \vdash_{\keyw{stateful}} [e'/z](\keyw{def} m(x : \tau_1) : \tau_2 = e; d)~:~\keyw{def} m : \tau_1 \rightarrow \tau_2; \sigma$.
\\
\end{itemize}


\item[] \textit{Case $d$ is a field:} There are two cases depending on whether it is run-time or not:
\\
\begin{itemize}
\item[] \textit{Subcase} \textsc{DT-Varx}\textit{:} $d = \keyw{var} f : \tau = x$, and, from the premise, we get
\[
\Gamma,~z : \tau'~|~\Sigma \vdash x : \tau~~~~~\Gamma,~z : \tau'~|~\Sigma \vdash_s d : \sigma
\]
There are two subcases to consider, depending on whether $x$ is $z$ or another variable. If $x = z$, then
$\Gamma,~z : \tau'~|~\Sigma \vdash [e'/z]x : \tau$ yields $\Gamma,~z : \tau'~|~\Sigma \vdash e' : \tau$ and $\tau = \tau'$. Substituting $e'$ for $z$ in $d$ preserves the original types. (The latter can be proved by induction on $d$ starting with the end of the sequence when $d = \epsilon$.) Thus, $\Gamma~|~\Sigma \vdash_{\keyw{stateful}} \keyw{var} f : \tau = e'; d~:~\keyw{var} f : \tau; \sigma$ as required. If $x \not = z$, then $\Gamma,~z : \tau'~|~\Sigma \vdash [e'/z]x : \tau$ yields $\Gamma,~z : \tau'~|~\Sigma \vdash x : \tau$, and the desired result is immediate.
\\
\item[] \textit{Subcase} \textsc{DT-Varl}\textit{:} $d = \keyw{var} f : \tau = l$, i.e. the field is resolved to a location $l$. This is not affected by the substitution, and the desired result is immediate.
\\
\end{itemize}

\end{itemize}

Having considered all the possibilities for $d$, we see that $\Gamma,~x : \{ \sigma \}_{s}~|~\Sigma \vdash_s [e'/z]d : \sigma$. Then, by \textsc{T-New}, $\Gamma~|~\Sigma \vdash \keywadj{new}_{s}(x \Rightarrow [e'/z]d) : \{ \sigma \}_{s}$, i.e. $\Gamma~|~\Sigma \vdash [e'/z](\keywadj{new}_{s}(x \Rightarrow d)) : \{ \sigma \}_{s}$.
\\
\pcase[\textsc{T-Meth}]
$e = e_1.m(e_2)$ and, from the premise, we get
\[
\Gamma,~z : \tau'~|~\Sigma \vdash e_1 : \{\sigma\}_s~~~~~\keyw{def}~ m : \tau_2 \rightarrow \tau_1 \in \sigma~~~~~\Gamma,~z : \tau'~|~\Sigma \vdash e_2 : \tau_2
\]
By the induction hypothesis, $\Gamma~|~\Sigma \vdash [e'/z]e_1 : \{\sigma\}_s$ and $\Gamma~|~\Sigma \vdash [e'/z]e_2 : \tau_2$. Then, by \textsc{T-Meth}, $\Gamma~|~\Sigma \vdash [e'/z]e_1.m([e'/z]e_2) : \tau_1$, i.e. $\Gamma~|~\Sigma \vdash [e'/z](e_1.m(e_2)) : \tau_1$.
\\
\pcase[\textsc{T-Field}]
$e = e.f$ and, from the premise, we get
\[
\Gamma,~z : \tau'~|~\Sigma \vdash e : \{\sigma\}_s~~~~~\keyw{var}~ f : \tau \in \sigma
\]
By the induction hypothesis, $\Gamma~|~\Sigma \vdash [e'/z]e : \{\sigma\}_s$. Then, by \textsc{T-Field}, $\Gamma~|~\Sigma \vdash ([e'/z]e).f : \tau$, i.e. $\Gamma~|~\Sigma \vdash [e'/z](e.f) : \tau$.
\\
\pcase[\textsc{T-Assign}]
$e = e_1.f=e_2$ and, from the premise, we get
\[
\Gamma,~z : \tau'~|~\Sigma \vdash e_1 : \{\sigma\}_s~~~~~\keyw{var}~ f : \tau \in \sigma~~~~~\Gamma,~z : \tau'~|~\Sigma \vdash e_2 : \tau
\]
By the induction hypothesis, $\Gamma~|~\Sigma \vdash [e'/z]e_1 : \{\sigma\}_s$ and $\Gamma~|~\Sigma \vdash [e'/z]e_2 : \tau$. Then, by \textsc{T-Assign}, $\Gamma~|~\Sigma \vdash  [e'/z]e_1.f = [e'/z]e_2 : \tau$, i.e. $\Gamma~|~\Sigma \vdash [e'/z](e_1.f=e_2) : \tau$.
\\
\pcase[\textsc{T-Loc}]
$e = l$, $[e'/z]l = l$, and the desired result is immediate.
\\
\pcase[\textsc{T-StackFrame}]
$e = l_1.m(l_2) \rhd e$ and, from the premise, we get
\[
\Gamma,~z : \tau'~|~\Sigma \vdash l_1 : \{\sigma\}_s~~~~~\keyw{def}~ m : \tau_2 \rightarrow \tau_1 \in \sigma~~~~~\Gamma,~z : \tau'~|~\Sigma \vdash l_2 : \tau_2~~~~~\Gamma,~z : \tau'~|~\Sigma \vdash e : \tau_1
\]
Locations are not affected by the substitution, and by the induction hypothesis, $\Gamma~|~\Sigma \vdash [e'/z]e : \tau_1$. Then, by \textsc{T-StackFrame}, $\Gamma~|~\Sigma \vdash l_1.m(l_2) \rhd [e'/z]e : \tau_1$, i.e. $\Gamma~|~\Sigma \vdash [e'/z](l_1.m(l_2) \rhd e) : \tau_1$.
\\
\pcase[\textsc{T-Sub}]
$e = e$ and, from the premise, we get
\[
\Gamma,~z : \tau'~|~\Sigma \vdash e : \tau_1~~~~~\tau_1 <: \tau_2
\]
By the induction hypothesis, $\Gamma~|~\Sigma \vdash [e'/z]e : \tau_1$ and $\tau_1 <: \tau_2$. Then, by \textsc{T-Sub}, $\Gamma~|~\Sigma \vdash [e'/z]e : \tau_2$.
\end{pcases}

\noindent Thus, substituting terms in a well-typed expression preserves the typing. \qed
\end{proof}

\newpage

\begin{theorem}[Preservation]
If
\begin{enumerate}
\item $\Gamma~|~\Sigma \vdash e : \tau$,
\item $\mu : \Sigma$, and
\item $e~|~\mu \longrightarrow e'~|~\mu'$,
\end{enumerate}
then
\begin{enumerate}
\item $\exists \Sigma' \supseteq \Sigma$,
\item $\mu' : \Sigma'$, and
\item $\Gamma~|~\Sigma' \vdash e' : \tau$.
\end{enumerate}

\end{theorem}

\begin{proof} The proof is by induction on a derivation of $\Gamma~|~\Sigma \vdash e : \tau$. At each step of the induction, we assume that the desired property holds for all subderivations and proceed by case analysis on the final rule in the derivation. The cases when $e$ is a variable (\textsc{T-Var}) or a location (\textsc{T-Loc}) cannot arise, since we assumed $e \longrightarrow e'$, and there are no evaluation rules corresponding to variables or locations. For the other cases, we argue as follows:

\begin{pcases}
\pcase[\textsc{T-New}]
$e = \keywadj{new}_{s}(x \Rightarrow d)$, and by inversion on \textsc{T-New}, we get $\Gamma,~x : \{ \sigma \}_{s}~|~\Sigma \vdash d : \sigma$. The store changes from $\mu$ to $\mu' = \mu,~l \mapsto \{ x \Rightarrow d \}_s$, i.e. the new store is the old store augmented with a new mapping for the location $l$, which was not in the old store. From the premise of the theorem, we know that $\mu : \Sigma$, and then by \textsc{T-Store}, we have $\mu,~l \mapsto \{ x \Rightarrow d \}_s~:~\Sigma,~l : \{ \sigma \}_s$, which implies that $\Sigma' = \Sigma,~l : \{ \sigma \}_s$. Finally, by \textsc{T-Loc}, $\Gamma~|~\Sigma \vdash l : \{ \sigma \}_s$. Hence, the right-hand side is well typed.
\\
\pcase[\textsc{T-Meth}]
$e = e_1.m(e_2)$, and by the definition of the evaluation relation, there are two subcases:
\\
\begin{itemize}
\item[]  \textit{Subcase} \textsc{E-Congruence}\textit{:} In this case, either $e_1 \longrightarrow e_1'$ or $e_1$ is a value and $e_2 \longrightarrow e_2'$. Then, the result follows from the induction hypothesis and \textsc{T-Meth}.
\\
\item[]  \textit{Subcase} \textsc{E-Meth}\textit{:} In this case, both $e_1$ and $e_2$ are values, namely locations $l_1$ and $l_2$ respectively. Then, by inversion on \textsc{T-Meth}, we get that $\Gamma~|~\Sigma \vdash l_1 : \{\sigma\}_s$, $\keyw{def}~ m : \tau_2 \rightarrow \tau_1 \in \sigma$, and $\Gamma~|~\Sigma \vdash l_2 : \tau_2$. The store $\mu$ does not change, and since \textsc{T-Store} has been applied throughout, the store is well typed, and thus, \mbox{$\Gamma~|~\Sigma \vdash_s \keyw{def}~ m(l_2 : \tau_2) : \tau_1 = e; d~:~\keyw{def}~ m : \tau_2 \rightarrow \tau_1; \sigma$}. Then, by inversion on both \textsc{DT-DefPure} and \textsc{DT-DefStateful}, we know that $\Gamma~|~\Sigma \vdash e : \tau_1$, and by \textsc{T-StackFrame}, we have $\Gamma~|~\Sigma \vdash l_1.m(l_2) \rhd e : \tau_1$. Finally, by the preservation under subsumption lemma, substituting locations for variables in $e$ preserve its type, and therefore, the right-hand side is well typed.
\\
\end{itemize}

\pcase[\textsc{T-Field}]
$e = e.f$, and by the definition of the evaluation relation, there are two subcases:
\\
\begin{itemize}
\item[]  \textit{Subcase} \textsc{E-Congruence}\textit{:} In this case, $e \longrightarrow e'$, and the result follows from the induction hypothesis and \textsc{T-Field}.
\\
\item[]  \textit{Subcase} \textsc{E-Field}\textit{:} In this case, $e$ is a value, i.e. a location $l_1$. Then, by inversion on  \textsc{T-Field}, we have $\Gamma~|~\Sigma \vdash e : \{\sigma\}_s$ and $\keyw{var}~ f : \tau \in \sigma$. The store $\mu$ does not change, and since \textsc{T-Store} has been applied throughout, the store is well typed, and thus, \mbox{$\Gamma~|~\Sigma \vdash_s \keyw{var}~ f : \tau = l_1 : \keyw{var}~ f : \tau$}. Then, by inversion on \textsc{DT-Varl}, we know that $\Gamma~|~\Sigma \vdash l_1 : \tau$, and the right-hand side is well typed.
\\
\end{itemize}

\pcase[\textsc{T-Assign}]
$e = e_1.f=e_2$, and by the definition of the evaluation relation, there are two subcases:
\\
\begin{itemize}
\item[]  \textit{Subcase} \textsc{E-Congruence}\textit{:} In this case, either $e_1 \longrightarrow e_1'$ or $e_1$ is a value and $e_2 \longrightarrow e_2'$. Then, the result follows from the induction hypothesis and \textsc{T-Assign}.
\\
\item[]  \textit{Subcase} \textsc{E-Assign}\textit{:} In this case, both $e_1$ and $e_2$ are values, namely locations $l_1$ and $l_2$ respectively. Then, by inversion on \textsc{T-Assign}, we get that $\Gamma~|~\Sigma \vdash l_1 : \{\sigma\}_s$, $\keyw{var}~ f : \tau \in \sigma$, and $\Gamma~|~\Sigma \vdash l_2 : \tau$. The store changes as follows: $\mu' = [l_1 \mapsto \{ x \Rightarrow d' \}_{s}/l_1 \mapsto \{ x \Rightarrow d \}_{s}]\mu$, where $d' = [\keyw{var} f:\tau = l_2/\keyw{var} f:\tau = l]d$. However, since \textsc{T-Store} has been applied throughout and the substituted location has the type expected by \textsc{T-Store}, the new store is well typed (as well as the old store), and thus, \mbox{$\Gamma~|~\Sigma \vdash_s \keyw{var}~ f : \tau = l_2 : \keyw{var}~ f : \tau$}. Then, by inversion on \textsc{DT-Varl}, we know that $\Gamma~|~\Sigma \vdash l_2 : \tau$, and the right-hand side is well typed.

the substituted location has the type expected by \textsc{T-Store} and \textsc{DT-Varl}; the fact that it is the same as the old one is mainly relevant by virtue of this fact.

the location you substitute has the type expected by T-Store and DT-Varl; the fact that it is the same as the old one is mainly relevant by virtue of this fact.
\\
\end{itemize}

\pcase[\textsc{T-StackFrame}]
$e = l.m(l_1) \rhd e_2$, and by the definition of the evaluation relation, there are two subcases:
\\
\begin{itemize}
\item[]  \textit{Subcase} \textsc{E-Congruence}\textit{:} In this case, $e \longrightarrow e'$, and the result follows from the induction hypothesis and \textsc{T-StackFrame}.
\\
\item[]  \textit{Subcase} \textsc{E-StackFrame}\textit{:} In this case, $e_2$ is a value, i.e. a location $l_2$, and the result follows directly from \textsc{T-StackFrame}.
\\
\end{itemize}

\pcase[\textsc{T-Sub}]
The result follows directly from the induction hypothesis.
\\
\end{pcases}

\noindent Thus, the program written in this language is always well typed. \qed

\end{proof}


\subsection{Progress}

\begin{theorem}[Progress]
If $\varnothing~|~\Sigma \vdash e : \tau$ (i.e. $e$ is a closed, well-typed expression), then either
\begin{enumerate}
\item $e$ is a value (i.e. a location) or
\item $\forall \mu$ such that $\mu : \Sigma$,
   $\exists e', \mu'$ such that $e~|~\mu \longrightarrow e'~|~\mu'$.
\end{enumerate}
\end{theorem}
\begin{proof} The proof is by induction on the derivation of $\Gamma~|~\Sigma \vdash e : \tau$, with a case analysis on the last typing rule used. The case when $e$ is a variable (\textsc{T-Var}) cannot occur, and the case when $e$ is a location (\textsc{T-Loc}) is immediate, since in that case $e$ is a value. For the other cases, we argue as follows:

\begin{pcases}
\pcase[\textsc{T-New}]
$e = \keywadj{new}_{s}(x \Rightarrow d)$, and by \textsc{E-New}, $e$ can make a step of evaluation if there is a location available that is not in the current store $\mu$. There are infinitely many such locations, and therefore, $e$ indeed can take a step and become a value (i.e. a location $l$). The new store $\mu'$ then is $\mu, l \mapsto \{ x \Rightarrow d \}_{s}$, and all the declarations $d$ inside the new objects are mapped in the new store.
\\
\pcase[\textsc{T-Meth}]
$e = e_1.m(e_2)$, and by the induction hypothesis applied to $\Gamma~|~\Sigma \vdash e_1 : \{\sigma\}_s$, either $e_1$ is a value or else it can make a step of evaluation, and, similarly, by the induction hypothesis applied to $\Gamma~|~\Sigma \vdash e_2 : \tau_2$, either $e_2$ is a value or else it can make a step of evaluation. Then, there are two subcases:
\\
\begin{itemize}
\item[]  \textit{Subcase $e_1 \longrightarrow e_1'$ or $e_2 \longrightarrow e_2'$:} If $e_1$ can take a step or if $e_1$ is a value and $e_2$ can take a step, then rule \textsc{E-Congruence} applies to $e$, and $e$ can take a step.
\\
\item[]  \textit{Subcase $e_1$ and $e_2$ are values:} If both $e_1$ and $e_2$ are values, i.e. they are locations $l_1$ and $l_2$ respectively, then by inversion on \textsc{T-Meth}, we have $\Gamma~|~\Sigma \vdash l_1 : \{\sigma\}_s$ and $\keyw{def}~ m : \tau_2 \rightarrow \tau_1 \in \sigma$. By inversion on \textsc{T-Loc}, we know that the store contains an appropriate mapping for the location $l_1$, and since \textsc{T-Store} has been applied throughout, the store is well typed and $l_1 \mapsto \{ x \Rightarrow d \}_{s} \in \mu$ with $\keyw{def} m(y : \tau_1) : \tau_2 = e \in d$. Therefore, the rule \textsc{E-Meth} applies to $e$, $e$ can take a step, and $\mu' = \mu$.
\\
\end{itemize}

\pcase[\textsc{T-Field}]
$e = e_1.f$, and by the induction hypothesis, either $e_1$ is a value or else it can make a step of evaluation. Then, there are two subcases:
\\
\begin{itemize}
\item[]  \textit{Subcase $e_1 \longrightarrow e_1'$:} If $e_1$ can take a step, then rule \textsc{E-Congruence} applies to $e$, and $e$ can take a step.
\\
\item[]  \textit{Subcase $e_1$ is a value:} If $e_1$ is a value, i.e. a location $l$, then by inversion on \textsc{T-Field}, we have $\Gamma~|~\Sigma \vdash l : \{\sigma\}_s$ and $\keyw{var}~ f : \tau \in \sigma
$. By inversion on \textsc{T-Loc}, we know that the store contains an appropriate mapping for the location $l$, and since \textsc{T-Store} has been applied throughout, the store is well typed and $l \mapsto \{ x \Rightarrow d \}_{s} \in \mu$ with $\keyw{var} f : \tau = l_1 \in d$. Therefore, the rule \textsc{E-Field} applies to $e$, $e$ can take a step, and $\mu' = \mu$.
\\
\end{itemize}

\pcase[\textsc{T-Assign}]
$e = e_1.f=e_2$, and by the induction hypothesis, either $e_1$ is a value or else it can make a step of evaluation, and likewise $e_2$. Then, there are two subcases:
\\
\begin{itemize}
\item[]  \textit{Subcase $e_1 \longrightarrow e_1'$ or $e_2 \longrightarrow e_2'$:} If $e_1$ can take a step or if $e_1$ is a value and $e_2$ can take a step, then rule \textsc{E-Congruence} applies to $e$, and $e$ can take a step.
\\
\item[]  \textit{Subcase $e_1$ and $e_2$ are values:} If both $e_1$ and $e_2$ are values, i.e. they are locations $l_1$ and $l_2$ respectively, then by inversion on \textsc{T-Assign}, we have $\Gamma~|~\Sigma \vdash l_1 : \{\sigma\}_s$, $\keyw{var}~ f:\tau \in \sigma$, and $\Gamma~|~\Sigma \vdash l_2 : \tau$. By inversion on \textsc{T-Loc}, we know that the store contains an appropriate mapping for the locations $l_1$ and $l_2$, and 
since \textsc{T-Store} has been applied throughout, the store is well typed and $l_1 \mapsto \{ x \Rightarrow d \}_{s} \in \mu$ with $\keyw{var} f:\tau = l \in d$. A new well-typed store can be created as follows: $\mu' = [l_1 \mapsto \{ x \Rightarrow d' \}_{s}/l_1 \mapsto \{ x \Rightarrow d \}_{s}]\mu$, where $d' = [\keyw{var} f:\tau = l_2/\keyw{var} f:\tau = l]d$. Then, the rule \textsc{E-Assign} applies to $e$, and $e$ can take a step.
\\
\end{itemize}

\pcase[\textsc{T-StackFrame}]
$e = l.m(l_1) \rhd e_2$, and by the induction hypothesis, either $e_2$ is a value or else it can make a step of evaluation. Then, there are two subcases:
\\
\begin{itemize}
\item[]  \textit{Subcase $e_2 \longrightarrow e_2'$:} If $e_2$ can take a step, then rule \textsc{E-Congruence} applies to $e$, and $e$ can take a step.
\\
\item[]  \textit{Subcase $e_2$ is a value:} If $e_2$ is a value, i.e. a location $l_2$, the rule \textsc{E-StackFrame} applies, and $e$ can take a step.
\\
\end{itemize}

\pcase[\textsc{T-Sub}]
The result follows directly from the induction hypothesis.

\end{pcases}

\noindent Thus, the program written in this language never gets stuck. \qed

\end{proof}

\newpage

\section{Module Translation}

\subsection{Code Example 1}

\begin{tabular}{p{0.53\textwidth}p{0.47\textwidth}}
\begin{minipage}[t]{\textwidth}
\begin{lstlisting}
signature resource sigA =
  var f : Int
  def m(y : Int) : Int

signature resource sigB =
  var k : Int

signature resource sigC =
  def n(Unit) : Int

signature resource sigD =
  def m(y : Int) : Int

resource module A : sigA
  var f : Int = 3
  def m(y : Int) : Int = y

module D : sigD
  def m(y : Int) : Int = 42

resource module B : sigB
  import D as myD
  require sigA as myA
  var k : Int = myA.m(3)

resource module C : sigC
  require sigA as myA
  def n(Unit) : Int = myA.m(5)

resource module Main
  instantiate A() as copyOfAForB 
  instantiate A() as copyOfAForC 
  instantiate B(copyOfAForB)
  instantiate C(copyOfAForC)

\end{lstlisting}
\end{minipage}
&
\hspace{-10ex}
\begin{minipage}[t]{\textwidth}
\begin{lstlisting}
signature resource sigA =
  var f : Int
  def m : Int -> Int

signature resource sigB =
  var k : Int

signature resource sigC =
  def n : Unit -> Int

signature resource sigD =
  def m : Int -> Int
  
val A : Unit -> sigA = fn _ : Unit =>
    bind in
    new (A =>
        var f : Int = 3;
        def m(y : Int) : Int = y;
    );
    
val D : sigD =
    bind in
    new (D =>
        def m(y : Int) : Int = 42;
    );

val B : sigA -> sigB = fn myA : sigA =>
  bind
     val myD = D
  in
  new(B =>
    var k : Int = myA.m(3);
  );

val C : sigA -> sigC = fn myA : sigA =>
  bind in
  new(C =>
    def n (Unit) : Int = myA.m(5);
  );

val Main = fn _ : Unit =>
    bind
        val copyOfAForB = A()
        val copyOfAForC = A()
        val B = B(copyOfAForB)
        val C = C(copyOfAForC)
    in
        // nothing
        
\end{lstlisting}
\end{minipage}
\end{tabular}

\newpage

\subsection{Code Example 2}

\begin{tabular}{p{0.6\textwidth}p{0.6\textwidth}}
\begin{minipage}[t]{\textwidth}
\begin{lstlisting}
signature sigA = 
  def a(Unit) : Int

signature sigB =
  def b(y : Int) : Int

resource signature sigC =
  var c : Int

resource signature sigD =
  var d : Int
  
resource signature sigX =
  def sumX(y : Int) : Int



module A : sigA
  def a(Unit) : Int = 42

module B : sigB
  def b(y : Int) : Int = y

resource module C : sigC
  var c : Int = 2

resource module D : sigD
  var d : Int = 0

\end{lstlisting}
\end{minipage}
&
\hspace{-10ex}
\begin{minipage}[t]{\textwidth}
\begin{lstlisting}
resource module X : sigX

  resource signature sigE =
    var e : Int
  
  resource signature sigF =
    def f(y : Int)
  
  resource signature sigG =
    def g(Unit)

  resource module E : sigE
    var e : Int = 5

  resource module F : sigF
    require sigE as myE
    def f(y : Int) = y + myE.e

  resource module G : sigG
    require sig E as myE
    def g(Unit) = myE.e

  import sigA as myA
  import sigB as myB
  require sigC as myC
  require sigD as myD
  instantiate E() as myE // necessary to access?
  instantiate E() as eForF
  instantiate E() as eForG
  instantiate F(eForF) as myF
  instantiate G(eForG) as myG

  def sumX(y : Int) : Int = 
      myA.a() + myB.b(y) + myC.c + myD.d
      + myE.E + myF.f(y) + myG.g() // + E.e?



resource module Main
  instantiate C() as cForX
  instantiate D() as dForX
  instantiate X(cForX, dForX) as x
// instantiate X(instantiate C(), instantiate D()) ?



\end{lstlisting}
\end{minipage}
\end{tabular}


\newpage

\subsection{Abstract Grammar}
\[
\begin{array}{cllr}
m & ::= & h~\overline{i}~\overline{d} & module \\
&&\\
h & ::= & [\keywadj{resource}]~\keyw{module} x : URI & module~header \\
&&\\
i & ::= & \keyw{import} URI~[\keyw{as} x] \\
  & |   & \keyw{instantiate} URI(\overline{x})~[\keyw{as} x] \\
  & |   & \keyw{requires} URI~[\keyw{as} x]\\
&&\\
sm & ::= & \keyw{signature} [\keywadj{resource}]~x = \tau &~~~signatures~module \\
&&\\
d & ::= & ... & declarations \\
&&\\
\end{array}
\]

\noindent\textbf{Explanation and comments:}
\begin{itemize}
\item Declarations are defined by regular Wyvern.
\item $sm$ is a module that defines only module signatures.
\item Signatures are translated to top-level type abbreviations.
\item Imported, instantiated, and required modules specified by URI.
\item $\keywadj{import}$ is for non-resource modules whose implementation you specify directly.
\item $\keywadj{instantiate}$ is for specifying the parameter of some parameterized module and for creating a resource module. (It is assumed that modules can call native code only if they are an FFI module.)
\item In $\keywadj{instantiate}$, $\overline{x}$ must come either from $\keywadj{import}$s or $\keywadj{requires}$s.
\item $\keywadj{requires}$ is used to specify module parameters; the URI is the module signature.
%\item $\keywadj{requires}$ is for resource modules created by someone else and for modules the module is  parameterized by.
\end{itemize}


\newpage

\subsection{Translation Rules}

$\textit{trans}(\keyw{def} m(x : \tau_1) : \tau_2; \sigma) \equiv \keyw{def} m : \tau_1 \rightarrow \tau_2; trans(\sigma)$\\
\\
$\textit{trans}(\keyw{signature} M = \sigma) \equiv \keyw{signature} M = trans(\sigma)$ \\
$\textit{trans}(\keyw{signature} \keyw{resource} M = \sigma) \equiv \keyw{signature} \keyw{resource} M = trans(\sigma)$ \\

\noindent$trans(\keyw{module} M : \tau; d) \equiv \keyw{val} M : \tau = \keyw{bind} \keyw{in} \keywadj{new}(M \Rightarrow d)$
\\\\
\noindent$\textit{trans}(\overline{i}) \equiv$\\
\indent if $\keyw{import} M_1~\keyw{as} M; \overline{i'}$ then $\keyw{val} M = M_1; trans(\overline{i'})$\\
\indent if $\keyw{instantiate} M_1(\overline{M_2})~\keyw{as} M; \overline{i'}$ then $ \keyw{val} M = M_1(\overline{M_2}); trans(\overline{i'})$\\
\indent if $\keyw{require} \tau~\keyw{as} M; \overline{i'}$ then $ M : \tau,~trans(\overline{i'})$\\
\\
$filterImportsAndInstantiates(\overline{i}) = $\\
\indent if $\keyw{import} M_1~\keyw{as} M; \overline{i'}$ then return $\keyw{import} M_1~\keyw{as} M; filterImportsAndInstantiates(\overline{i'})$\\
\indent if $\keyw{instantiate} M_1(\overline{M_2})~\keyw{as} M; \overline{i'}$ then return $\keyw{instantiate} M_1(\overline{M_2})~\keyw{as} M; filterImportsAndInstantiates(\overline{i'})$\\
\indent if $\keyw{require} \tau~\keyw{as} M; \overline{i'}$ then return $filterImportsAndInstantiates(\overline{i'})$\\
\indent if $\overline{i}$ is empty then return $\varnothing$ \\

\noindent$filterRequires(\overline{i}) = $\\
\indent if $\keyw{import} M_1~\keyw{as} M; \overline{i'}$ then return $filterRequires(\overline{i'})$\\
\indent if $\keyw{instantiate} M_1(\overline{M_2})~\keyw{as} M; \overline{i'}$ then return $filterRequires(\overline{i'})$\\
\indent if $\keyw{require} \tau~\keyw{as} M; \overline{i'}$ then return $\keyw{require} \tau~\keyw{as} M; filterRequires(\overline{i'})$\\
\indent if $\overline{i}$ is empty then return $\keyw{require} \keyw{Unit} \keyw{as} \_$ \\

\noindent$getTypes(\overline{x : \tau}) = $ if $\overline{x : \tau}$ is empty return $\keywadj{Unit}$, otherwise return $\overline{\tau}$\\

\noindent$trans(\keyw{resource} \keyw{module} M : \tau; \overline{i}; d) \equiv$

$\keyw{val} M : getTypes(trans(filterRequires(\overline{i}))) \rightarrow \tau = $

$\keyw{fn} trans(filterRequires(\overline{i})) \Rightarrow~\keyw{bind} trans(filterImportsAndInstantiates(\overline{i})~\keyw{in} \keywadj{new}(M \Rightarrow d)$\\

\noindent If the module does not require or instantiate any modules, then its translation is simplified as follows:
\\\\
\noindent$trans(\keyw{resource} \keyw{module} M : \tau; d) \equiv \keyw{val} M : \keyw{Unit} \rightarrow \tau = \keyw{fn} \_ : \keyw{Unit} \Rightarrow~\keyw{bind} \keyw{in} \keywadj{new}(M \Rightarrow d)$


\newpage

\section{Authority}

\[
\begin{array}{c}
\infer[\textsc{(auth-config)}]
  {auth(l,\mu \circ e) = auth_{store}(l,\mu) \cup auth_{stack}(l,e,\mu)}
  {}\\[5ex]

\infer[\textsc{(auth-store)}]
  {auth_{store}(l,\mu) = pointsto(d,\mu)}
  {\mu(l) = \{ x \Rightarrow d \}_{s}}\\[5ex]
  
\infer[\textsc{(auth-stack)}]
  {auth_{stack}(l,E[l.m(l') \rhd e'],\mu) = pointsto(e',\mu) \cup auth_{stack}(l,e',\mu)}
  {l.m(l'') \rhd E' \not\in E}\\[5ex]
  
\infer[\textsc{(auth-stack-nocall)}]
  {auth_{stack}(l,e,\mu) = \varnothing}
  {l.m(l') \rhd e' \not\in e}\\[5ex]
  
\infer[\textsc{(pointsto-def)}]
  {pointsto(\keyw{def} m(x:\tau):\tau = e; d,\mu) = pointsto(e,\mu) \cup pointsto(d,\mu)}
  {}\\[5ex]
  
\infer[\textsc{(pointsto-var)}]
  {pointsto(\keyw{var} f:\tau = e; d,\mu) = pointsto(e,\mu) \cup pointsto(d,\mu)}
  {}\\[5ex]
  
\infer[\textsc{(pointsto-principal)}]
  {pointsto(l,\mu) = \{ l \}}
  {\mu(l) = \{ x \Rightarrow d \}_\keywadj{stateful}}\\[5ex]
  
\infer[\textsc{(pointsto-data)}]
  {pointsto(l,\mu) = \varnothing }
  {\mu(l) = \{ x \Rightarrow d \}_\keywadj{pure}}\\[5ex]

\infer[\textsc{(pointsto-call-principal)}]
  {pointsto(l.m(l') \rhd e,\mu) = \{l\} }
  {\mu(l) = \{ x \Rightarrow d \}_\keywadj{stateful}}\\[5ex]
  
\infer[\textsc{(pointsto-call-data)}]
  {pointsto(l.m(l') \rhd e,\mu) = pointsto(e,\mu)}
  {\mu(l) = \{ x \Rightarrow d \}_\keywadj{pure}}\\[5ex]
  
\infer[\textsc{(pointsto-new)}]
  {pointsto(\keywadj{new}_{s}(x \Rightarrow d),\mu) = pointsto(d,\mu) }
  {}\\[5ex]
  
\infer[\textsc{(pointsto-otherexp)}]
  {pointsto(e,\mu) = pointsto(subexprs(e),\mu) }
  {\textit{when $e$ is not one of the expression forms defined above}}\\[5ex]

\infer[Ben-Ex]{ l_1.f = v | \mu \longrightarrow v | \mu' }{\mu' = \mu[l_1 \mapsto \{\keywadj{self} \Rightarrow d_v[\keyw{var} f:\tau = v]\}]}

\end{array}
\]

note: the last rule is a shorthand; the rest of the \textit{pointsto} rules for expressions are just congruence - look inside the e's

To Do:

\begin{itemize}
\item add a textual description to give intuition
\item write some examples (all but Darya will do)
\end{itemize}

\newpage

\subsection{Theory}

\textbf{Theorem [Authority Safety].}  If $\Sigma \vdash \mu$, $\bullet, \Sigma \vdash e : \tau$, and $\mu \circ e \stepsto \mu' \circ e'$ then $\forall l \in domain(\mu)$ if $auth(l,\mu' \circ e') \supset auth(l,\mu \circ e)$ then one of the following holds:

\begin{itemize}
  \item \textbf{Creation.} $e = E[\keywadj{new}_p(x \Rightarrow d)]$, $e' = E[l']$, and $auth(l,\mu' \circ e') = auth(l,\mu \circ e) \cup \{ l' \}$.  \textit{check that the creator was $l$}
  \item \textbf{Call.} $e = E[l.m(l')]$, $e' = E[l.m(l') \rhd e]$, and $auth(l,\mu' \circ e') = auth(l,\mu \circ e) \cup \{ pointsto(l',\mu) \}$  \textit{check that the caller previously had the authority that $l$ gained}
  \item \textbf{Return.}
\end{itemize}

Here, the Authority Safety theorem follows the definition in Maffeis et al. 2010.


\end{document}