\documentclass{llncs}

\usepackage{listings}
\usepackage{proof}
\usepackage{amssymb}
\usepackage[margin=.9in]{geometry}
\usepackage{amsmath}
\usepackage[english]{babel}
\usepackage[utf8]{inputenc}
\usepackage{enumitem}
\usepackage{filecontents}
 
\usepackage{fancyhdr}
\renewcommand{\headrulewidth}{0pt}
\pagestyle{fancy}
\fancyhf{}
\rhead{\thepage}

\lstset{tabsize=3, basicstyle=\ttfamily\small, commentstyle=\itshape\rmfamily, numbers=left, numberstyle=\tiny, language=java,moredelim=[il][\sffamily]{?},mathescape=true,showspaces=false,showstringspaces=false,columns=fullflexible,xleftmargin=5pt,escapeinside={(@}{@)}, morekeywords=[1]{objtype,module,import,let,in,fn,var,type,rec,fold,unfold,letrec,alloc,ref,application,policy,external,component,connects,to,meth,val,where,return,group,by,within,count,connect,with,attr,html,head,title,style,body,div,keyword,unit,def}}
\lstloadlanguages{Java,VBScript,XML,HTML}

\newcommand{\keywadj}[1]{\mathtt{#1}}
\newcommand{\keyw}[1]{\keywadj{#1}~}
\newcommand{\reftt}{\mathtt{ref}~}
\newcommand{\Reftt}{\mathtt{Ref}~}
\newcommand{\inttt}{\mathtt{int}~}
\newcommand{\Inttt}{\mathtt{Int}~}
\newcommand{\stepsto}{\leadsto}
\newcommand{\todo}[1]{\textbf{[#1]}}

\newlist{pcases}{enumerate}{1}
\setlist[pcases]{
  label=\textit{Case}\protect\thiscase\textit{:}~,
  ref=\arabic*,
  align=left,
  labelsep=0pt,
  leftmargin=0pt,
  labelwidth=0pt,
  parsep=0pt
}
\newcommand{\pcase}[1][]{
  \if\relax\detokenize{#1}\relax
    \def\thiscase{}
  \else
    \def\thiscase{~#1}
  \fi
  \item
}

\begin{document}

\section{Syntax}

\[
\begin{array}{lll}
\begin{array}{lllr}
e & ::= & x & expressions \\
& | & \keywadj{new}_{s}(x \Rightarrow d) \\
& | & e.m(e)\\
& | & e.f \\
& | & e.f = e \\
& | & \keyw{bind} x = e~\keyw{in} e \\
& | & l & (run\mbox{-}time~forms)\\
%& | & \keywadj{new}_{s}(x \Rightarrow d_v)\\
& | & l.m(l) \rhd e \\
&&\\
s & ::= & \keyw{stateful} | ~\keyw{pure} \\
&&\\
d & ::= & \epsilon & declarations \\
  & |   & \keyw{def} m(x:\tau):\tau = e; d \\
  & |   & \keyw{var} f:\tau = x; d \\
  & |   & \keyw{var} f:\tau = l; d & (run\mbox{-}time~form)\\
&&\\
%d_v & ::= & \epsilon & decl.~values \\
%  & |   & \keyw{def} m(x:\tau):\tau = e; d_v \\
%  & |   & \keyw{var} f:\tau = l; d_v \\
%&&\\
\tau & ::= & \{ \sigma \}_{s} & types \\
&&\\
\sigma & ::= & \epsilon & decl.~ types \\
       & |   & \keyw{def} m:\tau \rightarrow \tau; \sigma \\
       & |   & \keyw{var} f:\tau; \sigma \\
&&\\
\end{array}
& ~~~~~~
&
\begin{array}{lllr}
\Gamma & :: = & \varnothing & contexts\\
& | & \Gamma,~x : \tau\\
&&\\
\mu & :: = & \varnothing & store\\
%& | & \mu,~l \mapsto \{ x \Rightarrow d_v \}_{s}\\
& | & \mu,~l \mapsto \{ x \Rightarrow d \}_{s}\\
&&\\
\Sigma & :: = & \varnothing & store~type\\
& | & \Sigma,~l : \tau\\
&&\\
E & ::= & [~] & evaluation~ contexts\\
%  & |   & \keywadj{new}_{s}(x \Rightarrow D) \\
%  & |   & \keywadj{new}_{s}(x \Rightarrow d) \\
  & |   & E.m(e)\\
  & |   & l.m(E)\\
  & |   & E.f \\
  & |   & E.f = e \\
  & |   & \keyw{bind} x = E~\keyw{in} e \\
  & |   & l.f = E \\
  & |   & l.m(l) \rhd E \\
&&\\
%D & ::= & [~] & decl.~eval.~ contexts\\
%  & |   & \keyw{def} m(x:\tau):\tau = e; D \\
%  & |   & \keyw{var} f:\tau = E; d \\
%  & |   & \keyw{var} f:\tau = l; D \\
%&&\\
\end{array}
\end{array}
\]

\noindent \textbf{Notes:}

\begin{itemize}
\item The $s$ tag indicates whether an object is stateful (i.e. captures mutable state) or pure (i.e. captures no mutable state).
\item Example of how to initialize and use object declarations (and potentially encode modules):
\[
\keywadj{new}_{s}(x \Rightarrow \keyw{def} m(y : \tau) : \tau' = \keywadj{new}_{s}(x \Rightarrow \keyw{var} f_1 : \tau = y;~\keyw{var} f_2 : \tau = y;~\keyw{var} f_3 : \tau = y)).m(\keywadj{new}_{s} (x \Rightarrow \epsilon))
\]
which in memory looks like 
\[
l_2 \mapsto \{l_1,l_1,l_1\},~l_1 \mapsto \varnothing
\]
\item In the \textsc{DT-DefPure} rule, the argument $x$ may be stateful, but because all other variables in $\Gamma$ are pure, $x$ cannot be used (e.g. be assigned to a variable) inside $e$.
\item In the preservation of types under substitution lemma, $y$ and $z$ have to be values because we don't have $\keyw{var} f:\tau = e; d$ and instead have a more restrictive $\keyw{var} f:\tau = l;d$.
\item Substitution rules for $\keyw{bind}$ and $\keyw{let}$:
\[
    [l/x](\keyw{bind} y = e_1~\keyw{in} e_2) = \keyw{bind} y = [l/x]e_1~\keyw{in} e_2
\]
\[
    [l/x](\keyw{let} y = e_1~\keyw{in} e_2) = 
\begin{cases}
    \keyw{let} y = [l/x]e_1~\keyw{in} [l/x]e_2, & \text{if } x \not= y\\
    \keyw{let} y = [l/x]e_1~\keyw{in} e_2, & \text{if } x = y
\end{cases}
\]

Although the two constructs look similar, there is an important semantical difference between them. The $\keywadj{bind}$ construct is more restrictive as its body $e_2$ is type checked in the context that contains only the preamble $e_1$ (see \textsc{T-Bind}). Then in both cases, substitution of $l$ for $x$ happens only in the preamble $e_1$ and not in the body $e_2$. In contrast, in $\keywadj{let}$, when $x \not= y$, $l$ is substituted for $x$ in both the preamble $e_1$ and the body $e_2$, and when $x = y$, $l$ is substituted for $x$ in the preamble $e_1$, and then $y$ shadows $x$ in the body $e_2$, and thus $e_2$ does not change. 

\item Encoding of $\keyw{let}$:
\[
\keyw{let} y = e_1~\keyw{in} e_2~~~\equiv~~~\keywadj{new}_s(x \Rightarrow \keyw{def} m(y) = e_2).m(e_1)
\]

\end{itemize}


\newpage

\section{Semantics}

$\fbox{$\Gamma~|~\Sigma \vdash e : \tau$}$
\[
\begin{array}{c}
\infer[\textsc{(T-Var)}]
  {\Gamma~|~\Sigma \vdash x : \tau}
  {x : \tau \in \Gamma}%\\[5ex]
~~~~~~~~~~
\infer[\textsc{(T-New)}]
	{\Gamma~|~\Sigma \vdash \keywadj{new}_{s}(x \Rightarrow d) : \{ \sigma \}_{s}}
	{x : \{ \sigma \}_{s}~|~\Sigma \vdash_s d : \sigma} \\[5ex]

\infer[\textsc{(T-Meth)}]
	{\Gamma~|~\Sigma \vdash e_1.m(e_2) : \tau_1} 
	{\Gamma~|~\Sigma \vdash e_1 : \{\sigma\}_s  & \keyw{def}~ m : \tau_2 \rightarrow \tau_1 \in \sigma & \Gamma~|~\Sigma \vdash e_2 : \tau_2}\\[5ex]

\infer[\textsc{(T-Field)}]
	{\Gamma~|~\Sigma \vdash  e.f : \tau} 
	{\Gamma~|~\Sigma \vdash e : \{\sigma\}_s & \keyw{var}~ f : \tau \in \sigma}\\[5ex]
	
\infer[\textsc{(T-Assign)}]
	{\Gamma~|~\Sigma \vdash  e_1.f=e_2 : \tau} 
	{\Gamma~|~\Sigma \vdash e_1 : \{\sigma\}_s & \keyw{var}~ f : \tau \in \sigma & \Gamma~|~\Sigma \vdash e_2 : \tau}\\[5ex]

\infer[\textsc{(T-Bind)}]
  {\Gamma~|~\Sigma \vdash \keyw{bind} x = e_1~\keyw{in} e_2 : \tau_2}
  {\Gamma~|~\Sigma \vdash e_1 : \tau_1 & x : \tau_1~|~\Sigma \vdash e_2 : \tau_2}%\\[5ex]
~~~~~~~~~~
\infer[\textsc{(T-Loc)}]
  {\Gamma~|~\Sigma \vdash l : \tau}
  {l : \tau \in \Sigma}\\[5ex]

\infer[\textsc{(T-StackFrames)}]
	{\Gamma~|~\Sigma \vdash l_1.m(l_2) \rhd e : \tau_1}
	{\Gamma~|~\Sigma \vdash l_1 : \{\sigma\}_s & \keyw{def}~ m : \tau_2 \rightarrow \tau_1 \in \sigma & \Gamma~|~\Sigma \vdash l_2 : \tau_2 & \Gamma~|~\Sigma \vdash e : \tau_1} \\[5ex]

\infer[\textsc{(T-Sub)}]
  {\Gamma~|~\Sigma \vdash e : \tau_2}
  {\Gamma~|~\Sigma \vdash e : \tau_1 & \tau_1 <: \tau_2}\\[5ex]

\end{array}
\]

$\fbox{$\Gamma~|~\Sigma \vdash_s d : \sigma$}$
\[
\begin{array}{c}

\infer[\textsc{(DT-Empty)}]
  {\Gamma~|~\Sigma \vdash_s \epsilon~; d~:~\epsilon~; \sigma}
  {\Gamma~|~\Sigma \vdash_s d : \sigma}\\[3.5ex]

\infer[\textsc{(DT-DefPure)}]
  {\Gamma~|~\Sigma \vdash_{\keywadj{pure}} \keyw{def} m(y : \tau_1) : \tau_2 = e; d~:~\keyw{def} m : \tau_1 \rightarrow \tau_2; \sigma}
  {\def\arraystretch{1.5}
  \begin{array}{c}
\Gamma_{stateful} = \{x : \{ \sigma \}_{\keywadj{stateful}} ~|~ x : \{ \sigma \}_{\keywadj{stateful}} \in \Gamma\} \\
\Gamma_{pure} = \Gamma \setminus \Gamma_{stateful}~~~~~~~\Gamma_{pure},~y : \tau_1~|~\Sigma \vdash e : \tau_2~~~~~~~\Gamma~|~\Sigma \vdash_{\keywadj{pure}} d : \sigma
  \end{array}}\\[5ex]

\infer[\textsc{(DT-DefStateful)}]
  {\Gamma~|~\Sigma \vdash_{\keywadj{stateful}} \keyw{def} m(x : \tau_1) : \tau_2 = e; d~:~\keyw{def} m : \tau_1 \rightarrow \tau_2; \sigma}
  {\Gamma,~x : \tau_1~|~\Sigma \vdash e : \tau_2 & \Gamma~|~\Sigma \vdash_s d : \sigma}\\[5ex]

\infer[\textsc{(DT-Varx)}]
  {\Gamma~|~\Sigma \vdash_{\keywadj{stateful}} \keyw{var} f : \tau = x; d~:~\keyw{var} f : \tau; \sigma}
  {\Gamma~|~\Sigma \vdash x : \tau & \Gamma~|~\Sigma \vdash_s d : \sigma}\\[5ex]

\infer[\textsc{(DT-Varl)}]
  {\Gamma~|~\Sigma \vdash_{\keywadj{stateful}} \keyw{var} f : \tau = l; d~:~\keyw{var} f : \tau; \sigma}
  {\Gamma~|~\Sigma \vdash l : \tau & \Gamma~|~\Sigma \vdash_s d : \sigma}\\[5ex]

\end{array}
\]

$\fbox{$\tau <: \tau'$}$
\[
\begin{array}{c}
\infer[\textsc{(S-Refl)}]
  {\tau <: \tau}
  {}%\\[5ex]
~~~~~~~~~~
\infer[\textsc{(S-Trans)}]
  {\tau_1 <: \tau_3}
  {\tau_1 <: \tau_2 & \tau_2 <: \tau_3}\\[5ex]

\infer[\textsc{(S-State)}]
  {\{ \sigma \}_{\keyw{pure}} <: \{ \sigma' \}_{\keyw{stateful}}}
  {\sigma <: \sigma'}%\\[5ex]  
~~~~~~~~~~
\infer[\textsc{(S-Obj)}]
  {\{ \sigma \}_{s} <: \{ \sigma' \}_{s}}
  {\sigma <: \sigma'}\\[5ex]  

\end{array}
\]

$\fbox{$\sigma <: \sigma'$}$
\[
\begin{array}{c}
\infer[\textsc{(S-Refl)}]
  {\sigma <: \sigma}
  {}%\\[5ex]
~~~~~~~~~~
\infer[\textsc{(S-Trans)}]
  {\sigma_1 <: \sigma_3}
  {\sigma_1 <: \sigma_2 & \sigma_2 <: \sigma_3}%\\[5ex]
~~~~~~~~~~
\infer[\textsc{(S-Eps)}]
  {\sigma <: \epsilon}
  {}\\[5ex]

\infer[\textsc{(S-Def)}]
  {\keyw{def} m:\tau_1 \rightarrow \tau_2; \sigma <: \keyw{def} m:\tau_1' \rightarrow \tau_2'; \sigma'}
  {\tau_1' <: \tau_1 & \tau_2 <: \tau_2' & \sigma <: \sigma'}%\\[5ex]  
~~~~~~~~~~
\infer[\textsc{(S-Var)}]
  {\keyw{var} f:\tau; \sigma <: \keyw{var} f:\tau; \sigma'}
  {\sigma <: \sigma'}\\[5ex]  

\end{array}
\]

$\fbox{$\mu : \Sigma$}$
\[
\begin{array}{c}

\infer[\textsc{(T-StoreEmpty)}]
  {\varnothing : \varnothing}
  {}%\\[5ex]
~~~~~~~~~~
\infer[\textsc{(T-Store)}]
  {\mu,~l \mapsto \{ x \Rightarrow d \}_s~:~\Sigma,~l : \{ \sigma \}_s}
  {\mu : \Sigma & x : \{ \sigma \}_s~|~\Sigma \vdash_s d : \sigma}\\[5ex]

\end{array}
\]

$\fbox{$E[e]~|~\mu \longrightarrow E[e']~|~\mu'$}$
\[
\begin{array}{c}
\infer[\textsc{(E-Congruence)}]
  {E[e]~|~\mu \longrightarrow E[e']~|~\mu'}
  {e~|~\mu \longrightarrow e'~|~\mu'}\\[5ex]  
\end{array}
\]


$\fbox{$e~|~\mu \longrightarrow e'~|~\mu'$}$
\[
\begin{array}{c}
\infer[\textsc{(E-New)}]
  {\keywadj{new}_{s}(x \Rightarrow d)~|~\mu \longrightarrow l~|~\mu, l \mapsto \{ x \Rightarrow d \}_{s}}
  {l \not\in dom(\mu) & \keywadj{new}_{s}(x \Rightarrow d)~is~closed}\\[5ex]

\infer[\textsc{(E-Meth)}]
  {l_1.m(l_2)~|~\mu \longrightarrow l_1.m(l_2) \rhd [l_2/y][l_1/x]e~|~\mu}
  {l_1 \mapsto \{ x \Rightarrow d \}_{s} \in \mu & \keyw{def} m(y : \tau_1) : \tau_2 = e \in d}\\[5ex]

\infer[\textsc{(E-Field)}]
  {l.f~|~\mu \longrightarrow l_1~|~\mu}
  {l \mapsto \{ x \Rightarrow d \}_{s} \in \mu & \keyw{var} f:\tau = l_1 \in d}\\[5ex]

\infer[\textsc{(E-Assign)}]
  {l_1.f = l_2~|~\mu \longrightarrow l_2~|~\mu'}
  {\def\arraystretch{1.6}
  \begin{array}{c}
l_1 \mapsto \{ x \Rightarrow d \}_{s} \in \mu~~~~~~~~~~~\keyw{var} f:\tau = l \in d \\
d' = [\keyw{var} f:\tau = l_2/\keyw{var} f:\tau = l]d~~~~~\mu' = [l_1 \mapsto \{ x \Rightarrow d' \}_{s}/l_1 \mapsto \{ x \Rightarrow d \}_{s}]\mu
  \end{array}}\\[5ex]

\infer[\textsc{(E-Bind)}]
  {\keyw{bind} x = l_1~\keyw{in} e_2~|~\mu \longrightarrow [l_1/x]e_2~|~\mu}
  {}%\\[5ex]
~~~~~~~~~~
\infer[\textsc{(E-StackFrame)}]
  {l.m(l_1) \rhd l_2~|~\mu \longrightarrow l_2~|~\mu}
  {}\\[5ex]
  
\end{array}
\]

\newpage

\section{Theorems}

\subsection{Preservation}

\begin{lemma}[Permutation]
If $\Gamma~|~\varnothing \vdash e : \tau$ and $\Delta$ is a permutation of $\Gamma$, then $\Delta~|~\varnothing \vdash e : \tau$, and the latter derivation has the same depth as the former.
\end{lemma}

\begin{proof}
Straightforward induction on typing derivations. \qed
\end{proof}

\vspace{8pt}

\begin{lemma}[Weakening]
If $\Gamma~|~\varnothing \vdash e : \tau$ and $x \not\in dom(\Gamma)$, then $\Gamma,~x : \tau'~|~\varnothing \vdash e : \tau$, and the latter derivation has the same depth as the former.
\end{lemma}

\begin{proof}
Straightforward induction on typing derivations. \qed
\end{proof}

\vspace{8pt}

\begin{lemma}[Preservation of types under substitution]
If $\Gamma,~z : \tau'~|~\Sigma \vdash e : \tau$ and $\Gamma~|~\Sigma \vdash e' : \tau'$, then $\Gamma~|~\Sigma \vdash [e'/z]e : \tau$. Furthermore, if $\Gamma,~z : \tau'~|~\Sigma \vdash_s d : \sigma$ and $\Gamma~|~\Sigma \vdash e' : \tau'$, then $\Gamma~|~\Sigma \vdash_s [e'/z]d : \sigma$.
\end{lemma}

\begin{proof} The proof is by simultaneous induction on a derivation of $\Gamma,~z : \tau'~|~\Sigma \vdash e : \tau$ and $\Gamma,~z : \tau'~|~\Sigma \vdash_s d : \sigma$. For a given derivation, we proceed by cases on the final typing rule used in the proof:

\begin{pcases}
\pcase[\textsc{T-Var}]
$e = x$, and from the premise, we get $x : \tau \in (\Gamma,~z : \tau')$. There are two subcases to consider, depending on whether $x$ is $z$ or another variable. If $x = z$, then $[e'/z]x = e'$. The required result is then $\Gamma~|~\Sigma \vdash e' : \tau'$, which is among the assumptions of the lemma. Otherwise, $[e'/z]x = x$, and the desired result is immediate.
\\
\pcase[\textsc{T-New}]
$e = \keywadj{new}_{s}(x \Rightarrow d)$, and from the premise, we get
\[
x : \{ \sigma \}_{s},~z : \tau'~|~\Sigma \vdash_s d : \sigma
\]
By the induction hypothesis, $x : \{ \sigma \}_{s}~|~\Sigma \vdash_s [e'/z]d : \sigma$. Then, by \textsc{T-New}, $\Gamma~|~\Sigma \vdash \keywadj{new}_{s}(x \Rightarrow [e'/z]d) : \{ \sigma \}_{s}$, i.e. $\Gamma~|~\Sigma \vdash [e'/z](\keywadj{new}_{s}(x \Rightarrow d)) : \{ \sigma \}_{s}$.
\\
\pcase[\textsc{T-Meth}]
$e = e_1.m(e_2)$, and from the premise, we get
\[
\Gamma,~z : \tau'~|~\Sigma \vdash e_1 : \{\sigma\}_s~~~~~\keyw{def}~ m : \tau_2 \rightarrow \tau_1 \in \sigma~~~~~\Gamma,~z : \tau'~|~\Sigma \vdash e_2 : \tau_2
\]
By the induction hypothesis, $\Gamma~|~\Sigma \vdash [e'/z]e_1 : \{\sigma\}_s$ and $\Gamma~|~\Sigma \vdash [e'/z]e_2 : \tau_2$. Then, by \textsc{T-Meth}, $\Gamma~|~\Sigma \vdash [e'/z]e_1.m([e'/z]e_2) : \tau_1$, i.e. $\Gamma~|~\Sigma \vdash [e'/z](e_1.m(e_2)) : \tau_1$.
\\
\pcase[\textsc{T-Field}]
$e = e.f$, and from the premise, we get
\[
\Gamma,~z : \tau'~|~\Sigma \vdash e : \{\sigma\}_s~~~~~\keyw{var}~ f : \tau \in \sigma
\]
By the induction hypothesis, $\Gamma~|~\Sigma \vdash [e'/z]e : \{\sigma\}_s$. Then, by \textsc{T-Field}, $\Gamma~|~\Sigma \vdash ([e'/z]e).f : \tau$, i.e. $\Gamma~|~\Sigma \vdash [e'/z](e.f) : \tau$.
\\
\pcase[\textsc{T-Assign}]
$e = e_1.f=e_2$, and from the premise, we get
\[
\Gamma,~z : \tau'~|~\Sigma \vdash e_1 : \{\sigma\}_s~~~~~\keyw{var}~ f : \tau \in \sigma~~~~~\Gamma,~z : \tau'~|~\Sigma \vdash e_2 : \tau
\]
By the induction hypothesis, $\Gamma~|~\Sigma \vdash [e'/z]e_1 : \{\sigma\}_s$ and $\Gamma~|~\Sigma \vdash [e'/z]e_2 : \tau$. Then, by \textsc{T-Assign}, $\Gamma~|~\Sigma \vdash  [e'/z]e_1.f = [e'/z]e_2 : \tau$, i.e. $\Gamma~|~\Sigma \vdash [e'/z](e_1.f=e_2) : \tau$.
\\
\pcase[\textsc{T-Bind}]
$e = \keyw{bind} x = e_1~\keyw{in} e_2 : \tau_2$, and $[e'/z](\keyw{bind} x = e_1~\keyw{in} e_2) = \keyw{bind} x = [e'/z]e_1~\keyw{in} e_2$. From the premise, we get $\Gamma,~z : \tau'~|~\Sigma \vdash e_1 : \tau_1$, and by the induction hypothesis, $\Gamma~|~\Sigma \vdash [e'/z]e_1 : \tau_1$. Then, by \textsc{T-Bind}, $\Gamma~|~\Sigma \vdash \keyw{bind} x = [e'/z]e_1~\keyw{in} e_2 : \tau_2$, i.e. $\Gamma~|~\Sigma \vdash [e'/z](\keyw{bind} x = e_1~\keyw{in} e_2) : \tau_2$.
\\
\pcase[\textsc{T-Loc}]
$e = l$, $[e'/z]l = l$, and the desired result is immediate.
\\
\pcase[\textsc{T-StackFrame}]
$e = l_1.m(l_2) \rhd e$, and from the premise, we get
\[
\Gamma,~z : \tau'~|~\Sigma \vdash l_1 : \{\sigma\}_s~~~~~\keyw{def}~ m : \tau_2 \rightarrow \tau_1 \in \sigma~~~~~\Gamma,~z : \tau'~|~\Sigma \vdash l_2 : \tau_2~~~~~\Gamma,~z : \tau'~|~\Sigma \vdash e : \tau_1
\]
Locations are not affected by the substitution, and by the induction hypothesis, $\Gamma~|~\Sigma \vdash [e'/z]e : \tau_1$. Then, by \textsc{T-StackFrame}, $\Gamma~|~\Sigma \vdash l_1.m(l_2) \rhd [e'/z]e : \tau_1$, i.e. $\Gamma~|~\Sigma \vdash [e'/z](l_1.m(l_2) \rhd e) : \tau_1$.
\\
\pcase[\textsc{T-Sub}]
$e = e$, and from the premise, we get
\[
\Gamma,~z : \tau'~|~\Sigma \vdash e : \tau_1~~~~~\tau_1 <: \tau_2
\]
By the induction hypothesis, $\Gamma~|~\Sigma \vdash [e'/z]e : \tau_1$ and $\tau_1 <: \tau_2$. Then, by \textsc{T-Sub}, $\Gamma~|~\Sigma \vdash [e'/z]e : \tau_2$.
\\

\pcase[\textsc{DT-Empty}]
$d = \epsilon$, i.e. an empty declaration. This is not affected by the substitution, and the desired result is immediate.
\\
\pcase[\textsc{DT-DefPure}]
$d = \keyw{def} m(x : \tau_1) : \tau_2 = e$. There are two subcases depending on whether $z$ is in $\Gamma_{pure}$:
\\
\begin{itemize}
\item[] \textit{Subcase $z \in \Gamma_{pure}$}: From the premise, we get
\[
\Gamma_{stateful} = \{x : \{ \sigma \}_{\keyw{stateful}} ~|~ x : \{ \sigma \}_{\keyw{stateful}} \in \Gamma\}~~~~~\Gamma_{pure} = \Gamma \setminus \Gamma_{stateful}
\]\[
\Gamma_{pure},~x : \tau_1~|~\Sigma \vdash e : \tau_2~~~~~~~~~~\Gamma,~z : \tau'~|~\Sigma \vdash_{\keywadj{pure}} d : \sigma
\]
By the induction hypothesis, $\Gamma~|~\Sigma \vdash_{\keywadj{pure}} [e'/z]d : \sigma$. Then, by \textsc{DT-DefPure}, $\Gamma~|~\Sigma \vdash_{\keywadj{pure}} \keyw{def} m(x : \tau_1) : \tau_2 = e; [e'/z]d~:~\keyw{def} m : \tau_1 \rightarrow \tau_2; \sigma$, i.e. $\Gamma~|~\Sigma \vdash_{\keywadj{pure}} [e'/z](\keyw{def} m(x : \tau_1) : \tau_2 = e; d)~:~\keyw{def} m : \tau_1 \rightarrow \tau_2; \sigma$.
\\
\item[] \textit{Subcase $z \not\in \Gamma_{pure}$}: From the premise, we get
\[
\Gamma_{stateful} = \{x : \{ \sigma \}_{\keyw{stateful}} ~|~ x : \{ \sigma \}_{\keyw{stateful}} \in \Gamma\}~~~~~\Gamma_{pure} = \Gamma \setminus \Gamma_{stateful}
\]\[
\Gamma_{pure},~x : \tau_1,~z : \tau'~|~\Sigma \vdash e : \tau_2~~~~~~~~~~\Gamma,~z : \tau'~|~\Sigma \vdash_{\keywadj{pure}} d : \sigma
\]
By the induction hypothesis, $\Gamma_{pure},~x : \tau_1~|~\Sigma \vdash [e'/z]e : \tau_2$ and $\Gamma~|~\Sigma \vdash_{\keywadj{pure}} [e'/z]d : \sigma$. Then, by \textsc{DT-DefPure}, $\Gamma~|~\Sigma \vdash_{\keywadj{pure}} \keyw{def} m(x : \tau_1) : \tau_2 = [e'/z]e; [e'/z]d~:~\keyw{def} m : \tau_1 \rightarrow \tau_2; \sigma$, i.e. $\Gamma~|~\Sigma \vdash_{\keyw{pure}} [e'/z](\keyw{def} m(x : \tau_1) : \tau_2 = e; d)~:~\keyw{def} m : \tau_1 \rightarrow \tau_2; \sigma$.
\\
\end{itemize}

Thus, in both cases, the type of $d$ is preserved under substitution.
\\
\pcase[\textsc{DT-DefStateful}]
$d = \keyw{def} m(x : \tau_1) : \tau_2 = e$, and from the premise, we get
\[
\Gamma, x : \tau_1,~z : \tau'~|~\Sigma \vdash e : \tau_2~~~~~\Gamma,~z : \tau'~|~\Sigma \vdash_s d : \sigma
\]
By the induction hypothesis, $\Gamma, x : \tau_1~|~\Sigma \vdash [e'/z]e : \tau_2$ and $\Gamma~|~\Sigma \vdash_s [e'/z]d : \sigma$. Then, by \textsc{DT-DefStateful}, $\Gamma~|~\Sigma \vdash_{\keyw{stateful}} \keyw{def} m(x : \tau_1) : \tau_2 = [e'/z]e; [e'/z]d~:~\keyw{def} m : \tau_1 \rightarrow \tau_2; \sigma$, i.e. $\Gamma~|~\Sigma \vdash_{\keyw{stateful}} [e'/z](\keyw{def} m(x : \tau_1) : \tau_2 = e; d)~:~\keyw{def} m : \tau_1 \rightarrow \tau_2; \sigma$.
\\
\pcase[\textsc{DT-Varx}]
$d = \keyw{var} f : \tau = x$, and from the premise, we get
\[
\Gamma,~z : \tau'~|~\Sigma \vdash x : \tau~~~~~\Gamma,~z : \tau'~|~\Sigma \vdash_s d : \sigma
\]
There are two subcases to consider, depending on whether $x$ is $z$ or another variable. If $x = z$, then
$\Gamma,~z : \tau'~|~\Sigma \vdash [e'/z]x : \tau$ yields $\Gamma,~z : \tau'~|~\Sigma \vdash e' : \tau$ and $\tau = \tau'$. By the induction hypothesis, $\Gamma~|~\Sigma \vdash_s [e'/z]d : \sigma$. Thus, $\Gamma~|~\Sigma \vdash_{\keyw{stateful}} \keyw{var} f : \tau = e'; d~:~\keyw{var} f : \tau; \sigma$ as required. If $x \not = z$, then $\Gamma,~z : \tau'~|~\Sigma \vdash [e'/z]x : \tau$ yields $\Gamma,~z : \tau'~|~\Sigma \vdash x : \tau$, and the desired result is immediate.
\\
\pcase[\textsc{DT-Varl}]
$d = \keyw{var} f : \tau = l$, i.e. the field is resolved to a location $l$. This is not affected by the substitution, and the desired result is immediate.

\end{pcases}

\noindent Thus, substituting terms in a well-typed expression preserves the typing. \qed
\end{proof}

\newpage

\begin{theorem}[Preservation]
If
\begin{enumerate}
\item $\Gamma~|~\Sigma \vdash e : \tau$,
\item $\mu : \Sigma$, and
\item $e~|~\mu \longrightarrow e'~|~\mu'$,
\end{enumerate}
then
\begin{enumerate}
\item $\exists \Sigma' \supseteq \Sigma$,
\item $\mu' : \Sigma'$, and
\item $\Gamma~|~\Sigma' \vdash e' : \tau$.
\end{enumerate}

\end{theorem}

\begin{proof} The proof is by induction on a derivation of $\Gamma~|~\Sigma \vdash e : \tau$. At each step of the induction, we assume that the desired property holds for all subderivations and proceed by case analysis on the final rule in the derivation. The cases when $e$ is a variable (\textsc{T-Var}) or a location (\textsc{T-Loc}) cannot arise, since we assumed $e \longrightarrow e'$, and there are no evaluation rules corresponding to variables or locations. For the other cases, we argue as follows:

\begin{pcases}
\pcase[\textsc{T-New}]
$e = \keywadj{new}_{s}(x \Rightarrow d)$, and by inversion on \textsc{T-New}, we get $x : \{ \sigma \}_{s}~|~\Sigma \vdash d : \sigma$. The store changes from $\mu$ to $\mu' = \mu,~l \mapsto \{ x \Rightarrow d \}_s$, i.e. the new store is the old store augmented with a new mapping for the location $l$, which was not in the old store. From the premise of the theorem, we know that $\mu : \Sigma$, and then by \textsc{T-Store}, we have $\mu,~l \mapsto \{ x \Rightarrow d \}_s~:~\Sigma,~l : \{ \sigma \}_s$, which implies that $\Sigma' = \Sigma,~l : \{ \sigma \}_s$. Finally, by \textsc{T-Loc}, $\Gamma~|~\Sigma \vdash l : \{ \sigma \}_s$. Hence, the right-hand side is well typed.
\\
\pcase[\textsc{T-Meth}]
$e = e_1.m(e_2)$, and by the definition of the evaluation relation, there are two subcases:
\\
\begin{itemize}
\item[]  \textit{Subcase} \textsc{E-Congruence}\textit{:} In this case, either $e_1 \longrightarrow e_1'$ or $e_1$ is a value and $e_2 \longrightarrow e_2'$. Then, the result follows from the induction hypothesis and \textsc{T-Meth}.
\\
\item[]  \textit{Subcase} \textsc{E-Meth}\textit{:} In this case, both $e_1$ and $e_2$ are values, namely locations $l_1$ and $l_2$ respectively. Then, by inversion on \textsc{T-Meth}, we get that $\Gamma~|~\Sigma \vdash l_1 : \{\sigma\}_s$, $\keyw{def}~ m : \tau_2 \rightarrow \tau_1 \in \sigma$, and $\Gamma~|~\Sigma \vdash l_2 : \tau_2$. The store $\mu$ does not change, and since \textsc{T-Store} has been applied throughout, the store is well typed, and thus, \mbox{$\Gamma~|~\Sigma \vdash_s \keyw{def}~ m(l_2 : \tau_2) : \tau_1 = e; d'~:~\keyw{def}~ m : \tau_2 \rightarrow \tau_1; \sigma'$}. Then, by inversion on both \textsc{DT-DefPure} and \textsc{DT-DefStateful}, we know that $\Gamma~|~\Sigma \vdash e : \tau_1$, and by \textsc{T-StackFrame}, we have $\Gamma~|~\Sigma \vdash l_1.m(l_2) \rhd e : \tau_1$. Finally, by the preservation under subsumption lemma, substituting locations for variables in $e$ preserve its type, and therefore, the right-hand side is well typed.
\\
\end{itemize}

\pcase[\textsc{T-Field}]
$e = e.f$, and by the definition of the evaluation relation, there are two subcases:
\\
\begin{itemize}
\item[]  \textit{Subcase} \textsc{E-Congruence}\textit{:} In this case, $e \longrightarrow e'$, and the result follows from the induction hypothesis and \textsc{T-Field}.
\\
\item[]  \textit{Subcase} \textsc{E-Field}\textit{:} In this case, $e$ is a value, i.e. a location $l$. Then, by inversion on  \textsc{T-Field}, we have $\Gamma~|~\Sigma \vdash l : \{\sigma\}_s$ and $\keyw{var}~ f : \tau \in \sigma$. The store $\mu$ does not change, and since \textsc{T-Store} has been applied throughout, the store is well typed, and thus, \mbox{$\Gamma~|~\Sigma \vdash_s \keyw{var}~ f : \tau = l_1; d'~:~\keyw{var}~ f : \tau; \sigma'$}. Then, by inversion on \textsc{DT-Varl}, we know that $\Gamma~|~\Sigma \vdash l_1 : \tau$, and the right-hand side is well typed.
\\
\end{itemize}

\pcase[\textsc{T-Assign}]
$e = e_1.f=e_2$, and by the definition of the evaluation relation, there are two subcases:
\\
\begin{itemize}
\item[]  \textit{Subcase} \textsc{E-Congruence}\textit{:} In this case, either $e_1 \longrightarrow e_1'$ or $e_1$ is a value and $e_2 \longrightarrow e_2'$. Then, the result follows from the induction hypothesis and \textsc{T-Assign}.
\\
\item[]  \textit{Subcase} \textsc{E-Assign}\textit{:} In this case, both $e_1$ and $e_2$ are values, namely locations $l_1$ and $l_2$ respectively. Then, by inversion on \textsc{T-Assign}, we get that $\Gamma~|~\Sigma \vdash l_1 : \{\sigma\}_s$, $\keyw{var}~ f : \tau \in \sigma$, and $\Gamma~|~\Sigma \vdash l_2 : \tau$. The store changes as follows: $\mu' = [l_1 \mapsto \{ x \Rightarrow d' \}_{s}/l_1 \mapsto \{ x \Rightarrow d \}_{s}]\mu$, where $d' = [\keyw{var} f:\tau = l_2/\keyw{var} f:\tau = l]d$. However, since \textsc{T-Store} has been applied throughout and the substituted location has the type expected by \textsc{T-Store}, the new store is well typed (as well as the old store), and thus, \mbox{$\Gamma~|~\Sigma \vdash_s \keyw{var}~ f : \tau = l_2; d'~:~\keyw{var}~ f : \tau; \sigma'$}. Then, by inversion on \textsc{DT-Varl}, we know that $\Gamma~|~\Sigma \vdash l_2 : \tau$, and the right-hand side is well typed.
\\
\end{itemize}

\pcase[\textsc{T-Bind}]
$e = \keyw{bind} x = e_1~\keyw{in} e_2$, and by the definition of the evaluation relation, there are two subcases:
\\
\begin{itemize}
\item[]  \textit{Subcase} \textsc{E-Congruence}\textit{:} In this case, $e_1 \longrightarrow e_1'$, and the result follows from the induction hypothesis and \textsc{T-Bind}.
\\
\item[]  \textit{Subcase} \textsc{E-Bind}\textit{:} In this case, $e_1$ is a value, namely a locations $l_1$, and the result follows directly from the inversion on \textsc{T-Bind} and the preservation of types under substitution lemma.
\\
\end{itemize}

\pcase[\textsc{T-StackFrame}]
$e = l.m(l_1) \rhd e_2$, and by the definition of the evaluation relation, there are two subcases:
\\
\begin{itemize}
\item[]  \textit{Subcase} \textsc{E-Congruence}\textit{:} In this case, $e \longrightarrow e'$, and the result follows from the induction hypothesis and \textsc{T-StackFrame}.
\\
\item[]  \textit{Subcase} \textsc{E-StackFrame}\textit{:} In this case, $e_2$ is a value, i.e. a location $l_2$, and the result follows directly from the inversion on \textsc{T-StackFrame}.
\\
\end{itemize}

\pcase[\textsc{T-Sub}]
The result follows directly from the induction hypothesis.
\\
\end{pcases}

\noindent Thus, the program written in this language is always well typed. \qed

\end{proof}


\subsection{Progress}

\begin{theorem}[Progress]
If $\varnothing~|~\Sigma \vdash e : \tau$ (i.e. $e$ is a closed, well-typed expression), then either
\begin{enumerate}
\item $e$ is a value (i.e. a location) or
\item $\forall \mu$ such that $\mu : \Sigma$,
   $\exists e', \mu'$ such that $e~|~\mu \longrightarrow e'~|~\mu'$.
\end{enumerate}
\end{theorem}
\begin{proof} The proof is by induction on the derivation of $\Gamma~|~\Sigma \vdash e : \tau$, with a case analysis on the last typing rule used. The case when $e$ is a variable (\textsc{T-Var}) cannot occur, and the case when $e$ is a location (\textsc{T-Loc}) is immediate, since in that case $e$ is a value. For the other cases, we argue as follows:

\begin{pcases}
\pcase[\textsc{T-New}]
$e = \keywadj{new}_{s}(x \Rightarrow d)$, and by \textsc{E-New}, $e$ can make a step of evaluation if the $\keyw{new}$ expression is closed and there is a location available that is not in the current store $\mu$. From the theorem premise, we know that the expression is closed, and there are infinitely many available new locations, and therefore, $e$ indeed can take a step and become a value (i.e. a location $l$). The new store $\mu'$ then is $\mu, l \mapsto \{ x \Rightarrow d \}_{s}$, and all the declarations $d$ inside the new objects are mapped in the new store.
\\
\pcase[\textsc{T-Meth}]
$e = e_1.m(e_2)$, and by the induction hypothesis applied to $\Gamma~|~\Sigma \vdash e_1 : \{\sigma\}_s$, either $e_1$ is a value or else it can make a step of evaluation, and, similarly, by the induction hypothesis applied to $\Gamma~|~\Sigma \vdash e_2 : \tau_2$, either $e_2$ is a value or else it can make a step of evaluation. Then, there are two subcases:
\\
\begin{itemize}
\item[]  \textit{Subcase $e_1 \longrightarrow e_1'$ or $e_1$ is a value and $e_2 \longrightarrow e_2'$:} If $e_1$ can take a step or if $e_1$ is a value and $e_2$ can take a step, then rule \textsc{E-Congruence} applies to $e$, and $e$ can take a step.
\\
\item[]  \textit{Subcase $e_1$ and $e_2$ are values:} If both $e_1$ and $e_2$ are values, i.e. they are locations $l_1$ and $l_2$ respectively, then by inversion on \textsc{T-Meth}, we have $\Gamma~|~\Sigma \vdash l_1 : \{\sigma\}_s$ and $\keyw{def}~ m : \tau_2 \rightarrow \tau_1 \in \sigma$. By inversion on \textsc{T-Loc}, we know that the store contains an appropriate mapping for the location $l_1$, and since \textsc{T-Store} has been applied throughout, the store is well typed and $l_1 \mapsto \{ x \Rightarrow d \}_{s} \in \mu$ with $\keyw{def} m(y : \tau_1) : \tau_2 = e \in d$. Therefore, the rule \textsc{E-Meth} applies to $e$, $e$ can take a step, and $\mu' = \mu$.
\\
\end{itemize}

\pcase[\textsc{T-Field}]
$e = e_1.f$, and by the induction hypothesis, either $e_1$ can make a step of evaluation or it is a value. Then, there are two subcases:
\\
\begin{itemize}
\item[]  \textit{Subcase $e_1 \longrightarrow e_1'$:} If $e_1$ can take a step, then rule \textsc{E-Congruence} applies to $e$, and $e$ can take a step.
\\
\item[]  \textit{Subcase $e_1$ is a value:} If $e_1$ is a value, i.e. a location $l$, then by inversion on \textsc{T-Field}, we have $\Gamma~|~\Sigma \vdash l : \{\sigma\}_s$ and $\keyw{var}~ f : \tau \in \sigma
$. By inversion on \textsc{T-Loc}, we know that the store contains an appropriate mapping for the location $l$, and since \textsc{T-Store} has been applied throughout, the store is well typed and $l \mapsto \{ x \Rightarrow d \}_{s} \in \mu$ with $\keyw{var} f : \tau = l_1 \in d$. Therefore, the rule \textsc{E-Field} applies to $e$, $e$ can take a step, and $\mu' = \mu$.
\\
\end{itemize}

\pcase[\textsc{T-Assign}]
$e = e_1.f=e_2$, and by the induction hypothesis, either $e_1$ is a value or else it can make a step of evaluation, and likewise $e_2$. Then, there are two subcases:
\\
\begin{itemize}
\item[]  \textit{Subcase $e_1 \longrightarrow e_1'$ or $e_1$ is a value and $e_2 \longrightarrow e_2'$:} If $e_1$ can take a step or if $e_1$ is a value and $e_2$ can take a step, then rule \textsc{E-Congruence} applies to $e$, and $e$ can take a step.
\\
\item[]  \textit{Subcase $e_1$ and $e_2$ are values:} If both $e_1$ and $e_2$ are values, i.e. they are locations $l_1$ and $l_2$ respectively, then by inversion on \textsc{T-Assign}, we have $\Gamma~|~\Sigma \vdash l_1 : \{\sigma\}_s$, $\keyw{var}~ f:\tau \in \sigma$, and $\Gamma~|~\Sigma \vdash l_2 : \tau$. By inversion on \textsc{T-Loc}, we know that the store contains an appropriate mapping for the locations $l_1$ and $l_2$, and 
since \textsc{T-Store} has been applied throughout, the store is well typed and $l_1 \mapsto \{ x \Rightarrow d \}_{s} \in \mu$ with $\keyw{var} f:\tau = l \in d$. A new well-typed store can be created as follows: $\mu' = [l_1 \mapsto \{ x \Rightarrow d' \}_{s}/l_1 \mapsto \{ x \Rightarrow d \}_{s}]\mu$, where $d' = [\keyw{var} f:\tau = l_2/\keyw{var} f:\tau = l]d$. Then, the rule \textsc{E-Assign} applies to $e$, and $e$ can take a step.
\\
\end{itemize}

\pcase[\textsc{T-Bind}]
$e =  \keyw{bind} x = e_1~\keyw{in} e_2$, and by the induction hypothesis, either $e_1$ can make a step of evaluation or it is a value. Then, there are two subcases:
\\
\begin{itemize}
\item[]  \textit{Subcase $e_1 \longrightarrow e_1'$:} If $e_1$ can take a step, then rule \textsc{E-Congruence} applies to $e$, and $e$ can take a step.
\\
\item[]  \textit{Subcase $e_1$ is a value:} If $e_1$ is a value, i.e. a location $l_1$, the rule \textsc{E-Bind} applies, and $e$ can take a step.
\\
\end{itemize}

\pcase[\textsc{T-StackFrame}]
$e = l.m(l_1) \rhd e_2$, and by the induction hypothesis, either $e_2$ can make a step of evaluation or it is a value. Then, there are two subcases:
\\
\begin{itemize}
\item[]  \textit{Subcase $e_2 \longrightarrow e_2'$:} If $e_2$ can take a step, then rule \textsc{E-Congruence} applies to $e$, and $e$ can take a step.
\\
\item[]  \textit{Subcase $e_2$ is a value:} If $e_2$ is a value, i.e. a location $l_2$, the rule \textsc{E-StackFrame} applies, and $e$ can take a step.
\\
\end{itemize}

\pcase[\textsc{T-Sub}]
The result follows directly from the induction hypothesis.

\end{pcases}

\noindent Thus, the program written in this language never gets stuck. \qed

\end{proof}

\newpage

\section{Module Translation}

\subsection{Code Example 1}

\begin{tabular}{p{0.53\textwidth}p{0.47\textwidth}}
\begin{minipage}[t]{\textwidth}
\begin{lstlisting}
resource signature sigA =
  var f : Int
  def m(y : Int) : Int

resource signature sigB =
  var k : Int

resource signature sigC =
  def n(Unit) : Int

resource signature sigD =
  def m(y : Int) : Int

resource module A : sigA
  var f : Int = 3
  def m(y : Int) : Int = y

module D : sigD
  def m(y : Int) : Int = 42

resource module B : sigB
  import D as myD
  require sigA as myA
  var k : Int = myA.m(3)

resource module C : sigC
  require sigA as myA
  def n(Unit) : Int = myA.m(5)

resource module Main
  instantiate A() as copyOfAForB 
  instantiate A() as copyOfAForC 
  instantiate B(copyOfAForB)
  instantiate C(copyOfAForC)

\end{lstlisting}
\end{minipage}
&
\hspace{-10ex}
\begin{minipage}[t]{\textwidth}
\begin{lstlisting}
resource signature sigA =
  var f : Int
  def m : Int -> Int

resource signature sigB =
  var k : Int

resource signature sigC =
  def n : Unit -> Int

resource signature sigD =
  def m : Int -> Int
  
val A : Unit -> sigA = fn _ : Unit =>
    bind in
    new (A =>
        var f : Int = 3
        def m(y : Int) : Int = y
    )
    
val D : sigD =
    bind in
    new (D =>
        def m(y : Int) : Int = 42
    )

val B : sigA -> sigB = fn myA : sigA =>
  bind
     val myD : sigD = D
  in
  new(B =>
    var k : Int = myA.m(3)
  )

val C : sigA -> sigC = fn myA : sigA =>
  bind in
  new(C =>
    def n (Unit) : Int = myA.m(5)
  )

val Main = fn _ : Unit =>
    bind
        val copyOfAForB = A()
        val copyOfAForC = A()
        val B = B(copyOfAForB)
        val C = C(copyOfAForC)
    in
        // nothing
        
\end{lstlisting}
\end{minipage}
\end{tabular}

\newpage

\subsection{Code Example 2}

\begin{tabular}{p{0.6\textwidth}p{0.6\textwidth}}
\begin{minipage}[t]{\textwidth}
\begin{lstlisting}
signature sigA = 
  def a(Unit) : Int

signature sigB =
  def b(y : Int) : Int

resource signature sigC =
  var c : Int

resource signature sigD =
  var d : Int

resource signature sigE =
  var e : Int
  
resource signature sigF =
  def f(y : Int) : Int
  
resource signature sigG =
  def g(Unit) : Int

resource signature sigX =
  def sumX(y : Int) : Int


module A : sigA
  def a(Unit) : Int = 42

module B : sigB
  def b(y : Int) : Int = y

resource module C : sigC
  var c : Int = 2

resource module D : sigD
  var d : Int = 0
  
\end{lstlisting}
\end{minipage}
&
\hspace{-10ex}
\begin{minipage}[t]{\textwidth}
\begin{lstlisting}
resource module X : sigX

  resource module E : sigE
    var e : Int = 5

  resource module F : sigF
    require sigE as myE
    def f(y : Int) = y + myE.e

  resource module G : sigG
    require sig E as myE
    def g(Unit) = myE.e

  import sigA as myA
  import sigB as myB
  require sigC as myC
  require sigD as myD
  instantiate E() as myE
  instantiate E() as eForF
  instantiate E() as eForG
  instantiate F(eForF) as myF
  instantiate G(eForG) as myG

  def sumX(y : Int) : Int = 
      myA.a() + myB.b(y) + myC.c + myD.d
      + myE.E + myF.f(y) + myG.g()


resource module Main
  instantiate C() as cForX
  instantiate D() as dForX
  instantiate X(cForX, dForX) as x
// instantiate X(instantiate C(), instantiate D())



\end{lstlisting}
\end{minipage}
\end{tabular}


\newpage

\subsection{Abstract Grammar}
\[
\begin{array}{cllr}
p & ::= & e & program \\
&&\\
m & ::= & h~\overline{i}~\overline{d} & module \\
&&\\
h & ::= & [\keywadj{resource}]~\keyw{module} x : t & module~header \\
&&\\
i & ::= & \keyw{import} x~[\keyw{as} y] \\
  & |   & \keyw{instantiate} x(\overline{y})~[\keyw{as} y] \\
  & |   & \keyw{require} t~[\keyw{as} y]\\
&&\\
sm & ::= & [\keywadj{resource}]~\keyw{signature} x = \tau; sm &~~~signatures~module \\
&&\\
d & ::= & ... & declarations \\
&&\\
e & ::= & ... & expressions \\
&&\\
\end{array}
\]

\noindent\textbf{Notes:}
\begin{itemize}
\item The program is an expression $e$, which is syntactic sugar for

$\keywadj{new}_s(x \Rightarrow \keyw{def} m(y : Unit) : \tau = e).m(\keywadj{new}_{\keywadj{pure}}(x \Rightarrow \epsilon))$.
\item Declarations and expressions (static forms, except for $\keywadj{bind}$) are defined by regular Wyvern.
\item $sm$ is a module that defines only module signatures.
\item Signatures are translated to top-level type abbreviations.
\item All module signatures are declared in the signatures module (even if the modules are nested).
\item Module names ($x$s in the grammar) are specified by URIs.
\item $t$ is a type variable.
\item $\keywadj{import}$ is for non-resource modules whose implementation you specify directly.
\item $\keywadj{instantiate}$ is for specifying the parameter of some parameterized module and for creating a resource module. (It is assumed that modules can call native code only if they are an FFI module.)
\item In $\keywadj{instantiate}$, $\overline{y}$ must come either from $\keywadj{import}$s or $\keywadj{require}$s.
\item $\keyw{instantiate} X(\keyw{instantiate} Y_1(), \keyw{instantiate} Y_2())$ can be used as a shorthand.
\item $\keywadj{instantiate}$ can be used not only in the main module, but also in other modules, which provides a higher level of abstraction.
\item $\keywadj{require}$ is used to specify module parameters; the $t$ is the module signature (type variable).

\end{itemize}


\newpage

\subsection{Translation Rules}

$\textit{trans}(\keyw{def} m(x : \tau_1) : \tau_2; \sigma) \equiv \keyw{def} m : \tau_1 \rightarrow \tau_2; trans(\sigma)$\\
\\
$\textit{trans}(\keyw{signature} M = \sigma) \equiv \keyw{signature} M = trans(\sigma)$ \\
$\textit{trans}(\keyw{signature} \keyw{resource} M = \sigma) \equiv \keyw{signature} \keyw{resource} M = trans(\sigma)$ \\

\noindent$trans(\keyw{module} M : \tau; d) \equiv \keyw{val} M : \tau = \keyw{bind} \keyw{in} \keywadj{new}(M \Rightarrow d)$
\\\\
\noindent$\textit{trans}(\overline{i}) \equiv$\\
\indent if $\keyw{import} M_1~\keyw{as} M; \overline{i'}$ then $\keyw{val} M = M_1; trans(\overline{i'})$\\
\indent if $\keyw{instantiate} M_1(\overline{M_2})~\keyw{as} M; \overline{i'}$ then $ \keyw{val} M = M_1(\overline{M_2}); trans(\overline{i'})$\\
\indent if $\keyw{require} \tau~\keyw{as} M; \overline{i'}$ then $ M : \tau,~trans(\overline{i'})$\\
\\
$filterImportsAndInstantiates(\overline{i}) = $\\
\indent if $\keyw{import} M_1~\keyw{as} M; \overline{i'}$ then return $\keyw{import} M_1~\keyw{as} M; filterImportsAndInstantiates(\overline{i'})$\\
\indent if $\keyw{instantiate} M_1(\overline{M_2})~\keyw{as} M; \overline{i'}$ then return $\keyw{instantiate} M_1(\overline{M_2})~\keyw{as} M; filterImportsAndInstantiates(\overline{i'})$\\
\indent if $\keyw{require} \tau~\keyw{as} M; \overline{i'}$ then return $filterImportsAndInstantiates(\overline{i'})$\\
\indent if $\overline{i}$ is empty then return $\varnothing$ \\

\noindent$filterRequires(\overline{i}) = $\\
\indent if $\keyw{import} M_1~\keyw{as} M; \overline{i'}$ then return $filterRequires(\overline{i'})$\\
\indent if $\keyw{instantiate} M_1(\overline{M_2})~\keyw{as} M; \overline{i'}$ then return $filterRequires(\overline{i'})$\\
\indent if $\keyw{require} \tau~\keyw{as} M; \overline{i'}$ then return $\keyw{require} \tau~\keyw{as} M; filterRequires(\overline{i'})$\\
\indent if $\overline{i}$ is empty then return $\keyw{require} \keyw{Unit} \keyw{as} \_$ \\

\noindent$getTypes(\overline{x : \tau}) = $ if $\overline{x : \tau}$ is empty return $\keywadj{Unit}$, otherwise return $\overline{\tau}$\\

\noindent$trans(\keyw{resource} \keyw{module} M : \tau; \overline{i}; d) \equiv$

$\keyw{val} M : getTypes(trans(filterRequires(\overline{i}))) \rightarrow \tau = $

$\keyw{fn} trans(filterRequires(\overline{i})) \Rightarrow~\keyw{bind} trans(filterImportsAndInstantiates(\overline{i})~\keyw{in} \keywadj{new}(M \Rightarrow d)$\\

\noindent If the module does not require or instantiate any modules, then its translation is simplified as follows:
\\\\
\noindent$trans(\keyw{resource} \keyw{module} M : \tau; d) \equiv \keyw{val} M : \keyw{Unit} \rightarrow \tau = \keyw{fn} \_ : \keyw{Unit} \Rightarrow~\keyw{bind} \keyw{in} \keywadj{new}(M \Rightarrow d)$


\newpage

\section{Authority}

\subsection{Rules}

\vspace{1pt}
$\fbox{$~pointsto(e, \mu)~$}$
\[
\begin{array}{c}
\infer[\textsc{(pointsto-var)}]
  {pointsto(x,\mu) = \varnothing }
  {}%\\[5ex]
~~~~~~~~~~
\infer[\textsc{(pointsto-new)}]
  {pointsto(\keywadj{new}_{s}(x \Rightarrow d),\mu) = pointsto(d,\mu) }
  {}\\[5ex]

\infer[\textsc{(pointsto-principal)}]
  {pointsto(l,\mu) = \{ l \}}
  {l \mapsto \{ x \Rightarrow d \}_\keywadj{stateful} \in \mu}%\\[5ex]
~~~~~~~~~~
\infer[\textsc{(pointsto-data)}]
  {pointsto(l,\mu) = \varnothing }
  {l \mapsto \{ x \Rightarrow d \}_\keywadj{pure} \in \mu}\\[5ex]

\infer[\textsc{(pointsto-call-principal)}]
  {pointsto(l.m(l') \rhd e,\mu) = \{l\} }
  {l \mapsto \{ x \Rightarrow d \}_\keywadj{stateful} \in \mu}\\[5ex]

\infer[\textsc{(pointsto-call-data)}]
  {pointsto(l.m(l') \rhd e,\mu) = pointsto(e,\mu)}
  {l \mapsto \{ x \Rightarrow d \}_\keywadj{pure} \in \mu}\\[5ex]

\infer[\textsc{(pointsto-otherexp)}]
  {pointsto(e,\mu) = \bigcup pointsto(subexprs(e),\mu) }
  {\textit{$e$ is not one of the expression forms defined above}}\\[5ex]

\end{array}
\]

\vspace{-6pt}
$\fbox{$~pointsto(d, \mu)~$}$
\[
\begin{array}{c}
\infer[\textsc{(pointsto-emptydecl)}]
  {pointsto(\epsilon,\mu) = \varnothing}
  {}\\[5ex]

\infer[\textsc{(pointsto-def)}]
  {pointsto(\keyw{def} m(x:\tau):\tau = e; d,\mu) = pointsto(e,\mu) \cup pointsto(d,\mu)}
  {}\\[5ex]
  
\infer[\textsc{(pointsto-varx)}]
  {pointsto(\keyw{var} f:\tau = x; d,\mu) = pointsto(x,\mu) \cup pointsto(d,\mu)}
  {}\\[5ex]

\infer[\textsc{(pointsto-varl)}]
  {pointsto(\keyw{var} f:\tau = l; d,\mu) = pointsto(l,\mu) \cup pointsto(d,\mu)}
  {}\\[5ex]

\end{array}
\]

\vspace{-6pt}
$\fbox{$~auth()~$}$
\[
\begin{array}{c}
\infer[\textsc{(auth-store)}]
  {auth_{store}(l,\mu) = pointsto(l, \mu) \cup pointsto(d,\mu)}
  {l \mapsto \{ x \Rightarrow d \}_{s} \in \mu}\\[5ex]

\infer[\textsc{(auth-stack-nocall)}]
  {auth_{stack}(l,e,\mu) = \varnothing}
  {l.m(l') \rhd e' \not\in e}\\[5ex]
%~~~~~~~~~~
\infer[\textsc{(auth-stack)}]
  {auth_{stack}(l, e, \mu) = pointsto(e',\mu) \cup auth_{stack}(l,e',\mu)}
  {e = E[l.m(l') \rhd e'] & l.m'(l'') \rhd E' \not\in E}\\[5ex]
  
\infer[\textsc{(auth-config)}]
  {auth(l,e, \mu) = auth_{store}(l,\mu) \cup auth_{stack}(l,e,\mu)}
  {}\\[5ex]

\end{array}
\]

$\fbox{$~lookupCaller(E)~$}$
\[
\begin{array}{c}
\infer[\textsc{(lookupCaller)}]
  {lookupCaller(e) = l}
  {l.m(l') \rhd E \in e & l''.m'(l''') \rhd E' \not\in E}\\[5ex]

\end{array}
\]

%\newpage

\subsection{Definitions}
\label{sec:definitions}

\begin{itemize}
\item A \textit{principal} is a stateful object.
\item An object can \emph{directly access} a principal if the object either has a reference to the principal or can access the principal via methods (e.g. when the principal is passed in as a parameter).
\item ``$pointsto(e, \mu) = \{ ... \}$'' means that, in the context of memory $\mu$, the expression $e$ can directly access principals identified (pointed to) by locations in $\{ ... \}$.
\item ``$auth_{store}(l, \mu) = \{ ... \}$'' means that, in the context of memory $\mu$, the principal identified by $l$ can directly access the other principals identified by locations in $\{ ... \}$, by virtue of $l$'s (static) state.
\item ``$auth_{stack}(l, e, \mu) = \{ ... \}$'' means that, in the context of memory $\mu$, the principal identified by $l$ can directly access the other principals identified by locations in $\{ ... \}$, by virtue of the execution state of methods of $l$ executing in $e$.
\item The difference in meaning between $auth_{store}(l, \mu)$ and $pointsto(l, \mu)$ is that, in the latter, $l$ is treated as an expression, not a principal, and so the only location it can access is $l$.
\item The difference between $auth_{store}()$ and $auth_{stack}()$ is that, in the latter: 1) arguments are concrete, and 2) the method body may have been re-written to something else that may have new principals (due to acquiring access to a new location pointing to a principal via a return, creating new temporary variables with such new locations (three address code), etc.).
\item The expression $l.m(l_1) \rhd e$ means that we are currently executing a method $m$ of the receiver (object) $l$ having given it an argument $l_1$, and the method body is $e$. This expression was introduced to to keep track of that the LHS expression has access to.
\item The \textsc{pointsto-otherexp} rule is a shorthand; the rest of the \textit{pointsto} rules for expressions are just congruence---look inside the $e$s.
\item In the \textsc{auth-stack} rule, $l''$ represents the concrete argument that was passed in, and $l'$ can be arbitrary since it is substituted with $l''$ right away.  It is likely that $l$ is part of $e'$. In the conclusion, it is possible that the argument was not used, or it was used but is not being used any more, in which case, $l$ no longer has authority over $l'$. The premise makes sure that, as we go down the stack, we do not miss any calls to methods of $l$. (To see why, one could try applying the rule to a stack that has two calls and try to make it match the inner rather than the outer one. This is possible without the premise, but impossible with it.)
\item Specification for $lookupCaller(E)$:
\item[] \emph{Input:} evaluation context $E$
\item[] \emph{Output:} Let $E''$ be the innermost evaluation context of $E$ that has the form $l.m(l') \rhd E'$. Return $l$.

\end{itemize}


\subsubsection{Authority safety}

\begin{itemize}
\item Authority can be possessed by either a principal (a stateful object) or an expression. (See $pointsto()$ rules.)

\item Our definition of an \emph{object} comes directly from object-oriented programming and means an encapsulation of state and functionality. Then, having \emph{authority over a principal} means having access to the principal's state and functionality.

\item \emph{Authority} is the upper bound on the set of principals (stateful objects) that a subject (another principal or an expression) can access and act upon.

\item The \emph{authority of an object or an expression} is the union of the set of all principals (stateful objects) the current principal or expression has access to.

\item The authority of an object can change due to actions performed by other objects. For example:
\begin{itemize}
\item If an object A creates a new object B, A gains authority over B.
\item If a principal A, authority for which an object B does not have, is passed in as an argument to the object B's method by some caller object C, which has authority over A, then B (perhaps only temporarily) gains authority over A.
\item If a principal A, authority for which an object B does not have, was returned from a method call that B initiated, then B gains authority over A.
\end{itemize}

\item The authority of an expression can change as the expression is being executed.

\item The authority of an object (with respect to the store) can be statically determined by the code stored in the object. (See the \textsc{auth-store} rule.)

\item A principal can lose authority and become a pure object, e.g. on a reassignment of a principal's field (if the field ceased to hold a principal) or on a return of a principal's method (if the method received some authority via arguments and did not save the authority within the principal).

\end{itemize}


\newpage

\subsection{Theory}

\begin{lemma}
\label{lemma-four}
For an expressions $e$, such that $lookupCaller(e) = l$,
\[
auth(l, l.m(l') \rhd E[e], \mu) = pointsto(l, \mu) \cup pointsto(d, \mu) \cup pointsto(E, \mu) \cup pointsto(e, \mu).
\]
\end{lemma}

\begin{proof} By \textsc{lookupCaller}, there are no method calls in $E$. Let us consider the authority of $l$:

\vspace{8pt}

$auth(l, l.m(l') \rhd E[e], \mu)$

\vspace{-17pt}

\begin{align*}
&= auth_{store}(l, \mu) \cup auth_{stack}(l, l.m(l') \rhd E[e], \mu) & (\textsc{auth-config})\\
&= pointsto(l, \mu) \cup pointsto(d, \mu) \cup auth_{stack}(l, l.m(l') \rhd E[e], \mu) & (\textsc{auth-store})\\
&= pointsto(l, \mu) \cup pointsto(d, \mu) \cup pointsto(E[e], \mu) \cup auth_{stack}(l, E[e], \mu) & (\textsc{auth-stack})\\
&= pointsto(l, \mu) \cup pointsto(d, \mu) \cup pointsto(E[e], \mu) & (\textsc{auth-stack-nocall})\\
&= pointsto(l, \mu) \cup pointsto(d, \mu) \cup pointsto(E, \mu) \cup pointsto(e, \mu) & (\textsc{pointsto-otherexp})
\end{align*}\qed

\end{proof}


\begin{lemma}
\label{lemma-four}
For any two expressions $e$ and $e'$, such that
\begin{enumerate}
\item $e~|~\mu \longrightarrow e'~|~\mu'$,
\item $lookupCaller(e) = l$, and
\item in a broader context, $l.m(l') \rhd E[e]~|~\mu \longrightarrow l.m(l') \rhd E[e']~|~\mu'$,
\end{enumerate}
$auth(l, l.m(l') \rhd E[e], \mu)$ and $auth(l, l.m(l') \rhd E[e'], \mu')$ differ only by the values of $pointsto(e, \mu)$ and $pointsto(e', \mu')$.
\end{lemma}

\begin{proof} By \textsc{lookupCaller}, there are no method calls in $E$. Let us consider the authority of $l$ in the two states. In the initial state, we have:

\vspace{8pt}

$auth(l, l.m(l') \rhd E[e], \mu)$

\vspace{-17pt}

\begin{align*}
&= auth_{store}(l, \mu) \cup auth_{stack}(l, l.m(l') \rhd E[e], \mu) & (\textsc{auth-config})\\
&= pointsto(l, \mu) \cup pointsto(d, \mu) \cup auth_{stack}(l, l.m(l') \rhd E[e], \mu) & (\textsc{auth-store})\\
&= \{ l \} \cup pointsto(d, \mu) \cup auth_{stack}(l, l.m(l') \rhd E[e], \mu) & (\textsc{pointsto-principal})\\
&= \{ l \} \cup pointsto(d, \mu) \cup pointsto(E[e], \mu) \cup auth_{stack}(l, E[e], \mu) & (\textsc{auth-stack})\\
&= \{ l \} \cup pointsto(d, \mu) \cup pointsto(E[e], \mu) & (\textsc{auth-stack-nocall})\\
&= \{ l \} \cup pointsto(d, \mu) \cup pointsto(E, \mu) \cup pointsto(e, \mu) & (\textsc{pointsto-otherexp})
\end{align*}

In the final state, we have:

\vspace{8pt}

$auth(l, l.m(l') \rhd E[e'], \mu')$

\vspace{-17pt}

\begin{align*}
&= auth_{store}(l, \mu') \cup auth_{stack}(l, l.m(l') \rhd E[e'], \mu') & (\textsc{auth-config})\\
&= pointsto(l, \mu') \cup pointsto(d, \mu') \cup auth_{stack}(l, l.m(l') \rhd E[e'], \mu') & (\textsc{auth-store})\\
&= \{ l \} \cup pointsto(d, \mu') \cup auth_{stack}(l, l.m(l') \rhd E[e'], \mu') & (\textsc{pointsto-principal})\\
&= \{ l \} \cup pointsto(d, \mu') \cup pointsto(E[e'], \mu') \cup auth_{stack}(l, E[e'], \mu') & (\textsc{auth-stack})\\
&= \{ l \} \cup pointsto(d, \mu') \cup pointsto(E[e'], \mu') & (\textsc{auth-stack-nocall})\\
&= \{ l \} \cup pointsto(d, \mu') \cup pointsto(E, \mu') \cup pointsto(e', \mu') & (\textsc{pointsto-otherexp})
\end{align*}

Since we are considering small-step semantics, within this evaluation step, there were no changes to $d$ and $E$, and thus $pointsto(d, \mu) = pointsto(d, \mu')$ and $pointsto(E, \mu) = pointsto(E, \mu')$. Therefore, $auth(l, l.m(l') \rhd E[e'], \mu')$ differ only by the values of $pointsto(e, \mu)$ and $pointsto(e', \mu')$. \qed

\end{proof}

\begin{theorem}[Authority Safety] If
\begin{enumerate}
\item $\mu : \Sigma$,
\item $\varnothing~|~\Sigma \vdash e : \tau$, {\emph{~~~~~~~~~~~~[The program $e$ is well-formed]}}
\item $e~|~\mu \longrightarrow e'~|~\mu'$, and
\item for some principal $l$, $auth(l, e', \mu') \supset auth(l, e, \mu)$,
\end{enumerate}
then one of the following must hold:
\begin{itemize}
  \item \textbf{Object creation:}
  \begin{enumerate}
  \item $e = E[\keywadj{new}_{\keywadj{stateful}}(x \Rightarrow d')]$, $e' = E[l']$,
  \item $lookupCaller(E) = l$, {\emph{~~~~~~~~~~[The creator was $l$]}}
  \item $auth(l, e', \mu') = auth(l, e, \mu) \cup \{ l' \}$.\\
  \end{enumerate}
  \item \textbf{Method call:}
  \begin{enumerate}
  \item $e = E[l.m(l')]$, $e' = E[l.m(l') \rhd e'']$,
  \item $pointsto(l',\mu) = \{ l' \}$, {\emph{~~~~~~~~~~[The argument $l'$ is a principal]}}
  \item $\{ l' \} \not\in auth(l, e, \mu)$,
  \item $auth(l, e', \mu') = auth(l, e, \mu) \cup \{ l' \}$.\\
  \end{enumerate}
  \item \textbf{Method return:}
  \begin{enumerate}
  \item $e = E[l'.m(l'') \rhd l''']$, $e' = E[l''']$,
  \item $pointsto(l''',\mu) = \{ l''' \}$, {\emph{~~~~~~~~[The returned $l'''$ is a principal]}}
  \item $lookupCaller(E) = l$, {\emph{~~~~~~~~~~~[The caller was $l$]}}
  \item $\{ l''' \} \not\in auth(l, e, \mu)$,
  \item $auth(l, e', \mu') = auth(l, e, \mu) \cup \{ l''' \}$.\\
  \end{enumerate}
\end{itemize}

\end{theorem}

(The Authority Safety theorem follows the definition in Maffeis et al.~\cite{Maffeis:2010}.)

\begin{proof} The proof is by induction on a derivation of $e~|~\mu \longrightarrow e'~|~\mu'$. For a given derivation, we proceed by cases on the last evaluation rule used:

\sloppy 

\begin{pcases}
\pcase[\textsc{E-New}]
$e = E[\keywadj{new}_{\keywadj{stateful}}(x \Rightarrow d')]$; $e' = E[l']$; and $\mu' = \mu, l' \mapsto \{ x \Rightarrow d' \}_{\keywadj{stateful}}$. Without loss of generality, let $lookupCaller(E) = l$, such that $l \mapsto \{ x \Rightarrow d \}_{\keywadj{stateful}} \in \mu$. (If $l$ can create a principle, it must be a principal itself.) By \textsc{lookupCaller}, in a broader context, the following evaluation step is being made: \mbox{$l.m(l'') \rhd E'[\keywadj{new}_{\keywadj{stateful}}(x \Rightarrow d')]~|~\mu \longrightarrow l.m(l'') \rhd E'[l']~|~\mu'$}, where there are no method calls in $E'$. Since we are concerned only with \emph{direct} access (see definition in section~\ref{sec:definitions}), this is the \emph{only} relevant evaluation context.\\

Let us consider $l$'s authority for the two states. In the initial state, we have:\\

$auth(l, l.m(l'') \rhd E'[\keywadj{new}_{\keywadj{stateful}}(x \Rightarrow d')], \mu)$

\vspace{-17pt}

\begin{align*}
&= auth_{store}(l, \mu) \cup auth_{stack}(l, l.m(l'') \rhd E'[\keywadj{new}_{\keywadj{stateful}}(x \Rightarrow d')], \mu) & (\textsc{auth-config})\\
&= pointsto(l, \mu) \cup pointsto(d, \mu) \cup auth_{stack}(l, l.m(l'') \rhd E'[\keywadj{new}_{\keywadj{stateful}}(x \Rightarrow d')], \mu) & (\textsc{auth-store})\\
&= \{ l \} \cup pointsto(d, \mu) \cup auth_{stack}(l, l.m(l'') \rhd E'[\keywadj{new}_{\keywadj{stateful}}(x \Rightarrow d')], \mu) & (\textsc{pointsto-principal})\\
&= \{ l \} \cup pointsto(d, \mu) \cup pointsto(E'[\keywadj{new}_{\keywadj{stateful}}(x \Rightarrow d')], \mu) & (\textsc{auth-stack})\\
&\cup auth_{stack}(l, E'[\keywadj{new}_{\keywadj{stateful}}(x \Rightarrow d')], \mu) & \\
&= \{ l \} \cup pointsto(d, \mu) \cup pointsto(E'[\keywadj{new}_{\keywadj{stateful}}(x \Rightarrow d')], \mu) & (\textsc{auth-stack-nocall})\\
&= \{ l \} \cup pointsto(d, \mu) \cup pointsto(E', \mu) \cup pointsto(\keywadj{new}_{\keywadj{stateful}}(x \Rightarrow d'), \mu) & (\textsc{pointsto-otherexp})\\
&= \{ l \} \cup pointsto(d, \mu) \cup pointsto(E', \mu) \cup pointsto(d', \mu) & (\textsc{pointsto-new})
\end{align*}

\todo{Prove that $pointsto(d', \mu)$ does not increase authority of $l$.}

According to the $pointsto()$ rules for declarations, $pointsto(d, \mu)$ and $pointsto(d', \mu)$ equal to the $pointsto()$ function applied to the left-hand sides of their declarations. Since it was $l$ that created $l'$ and it could initialize the declarations of $l'$ only with expressions it had access to, $pointsto(d', \mu) \subset pointsto(d, \mu)$, and
\[
auth(l, l.m(l'') \rhd E[\keywadj{new}_{\keywadj{stateful}}(x \Rightarrow d')], \mu) = \{ l \} \cup pointsto(d, \mu) \cup pointsto(E', \mu)
\]

In the final state, we have:\\

$auth(l, l.m(l'') \rhd E'[l'], \mu')$

\vspace{-17pt}

\begin{align*}
&= auth_{store}(l, \mu') \cup auth_{stack}(l, l.m(l'') \rhd E'[l'], \mu') & (\textsc{auth-config})\\
&= pointsto(l, \mu') \cup pointsto(d, \mu') \cup auth_{stack}(l, l.m(l'') \rhd E'[l'], \mu') & (\textsc{auth-store})\\
&= \{ l \} \cup pointsto(d, \mu') \cup auth_{stack}(l, l.m(l'') \rhd E'[l'], \mu') & (\textsc{pointsto-principal})\\
&= \{ l \} \cup pointsto(d, \mu') \cup pointsto(E'[l'], \mu') \cup auth_{stack}(l, E'[l'], \mu') & (\textsc{auth-stack})\\
&= \{ l \} \cup pointsto(d, \mu') \cup pointsto(E'[l'], \mu') & (\textsc{auth-stack-nocall})\\
&= \{ l \} \cup pointsto(d, \mu') \cup pointsto(E', \mu') \cup pointsto(l', \mu') & (\textsc{pointsto-otherexp})\\
&= \{ l,~l' \} \cup pointsto(d, \mu') \cup pointsto(E', \mu') & (\textsc{pointsto-principle})
\end{align*}

Since we are considering small-step semantics, within this evaluation step, there were no changes to $d$ and $E'$, and thus $pointsto(d, \mu) = pointsto(d, \mu')$ and $pointsto(E', \mu) = pointsto(E', \mu')$.\\

Thus, $auth(l, e', \mu') = auth(l, e, \mu) \cup \{ l' \}$ in compliance with the \textbf{\textit{object creation}} case of the theorem.\\

\todo{Consider the case when $l'$ is pure.}

\todo{Consider the case when the $\keywadj{new}$ is top-level, and thus there is no creator.}
\\
\pcase[\textsc{E-Field}]
$e = l.f$, $e' = l'$, $\mu' = \mu$, and from the premise, we get $l \mapsto \{ x \Rightarrow d \}_{s} \in \mu$ and $\keyw{var} f:\tau = l_1 \in d$.

\hspace{12pt} Let us consider $auth(l, e, \mu)$. By \textsc{auth-config}, $auth(l, e, \mu) = auth_{store}(l, \mu) \cup auth_{stack}(l,e,\mu)$. By \textsc{auth-stack-nocall}, $auth_{stack}(l, e, \mu) = \varnothing$. Then, $auth(l, e, \mu) = auth_{store}(l, \mu)$.

\hspace{12pt} Now let us consider $auth(l, e', \mu')$. By \textsc{auth-config}, $auth(l, e', \mu') = auth_{store}(l, \mu') \cup auth_{stack}(l, e',\mu')$. By \textsc{auth-stack-nocall}, $auth_{stack}(l, e', \mu') = \varnothing$. Then, $auth(l, e', \mu') = auth_{store}(l, \mu')$.

\hspace{12pt} Since $\mu' = \mu$, $auth_{store}(l, \mu') = auth_{store}(l, \mu)$ and thus $auth(l, e', \mu') = auth(l, e, \mu)$, i.e., no gain in authority occurred.
\\
\pcase[\textsc{E-Assign}]
$e = l_1.f = l_2$, $e' = l_2$, $\mu' = [l_1 \mapsto \{ x \Rightarrow d' \}_{s}/l_1 \mapsto \{ x \Rightarrow d \}_{s}]\mu$, where $d' = [\keyw{var} f:\tau = l_2/\keyw{var} f:\tau = l]d$, and from the premise, we get $l_1 \mapsto \{ x \Rightarrow d \}_{s} \in \mu$ and $\keyw{var} f:\tau = l \in d$.

\hspace{12pt} Let us consider $auth(l, e, \mu)$. By \textsc{auth-config}, $auth(l, e, \mu) = auth_{store}(l, \mu) \cup auth_{stack}(l,e,\mu)$. By \textsc{auth-stack-nocall}, $auth_{stack}(l, e, \mu) = \varnothing$. Then, $auth(l, e, \mu) = auth_{store}(l, \mu)$.

\hspace{12pt} Now let us consider $auth(l, e', \mu')$. By \textsc{auth-config}, $auth(l, e', \mu') = auth_{store}(l, \mu') \cup auth_{stack}(l, e',\mu')$. By \textsc{auth-stack-nocall}, $auth_{stack}(l, e', \mu') = \varnothing$. Then, $auth(l, e', \mu') = auth_{store}(l, \mu')$.
...
\\
\pcase[\textsc{E-Bind}]
$e = \keyw{bind} x = l_1~\keyw{in} e_2$, $e' = [l_1/x]e_2$, $\mu' = \mu$.

\hspace{12pt} Let us consider $auth(l, e, \mu)$. By \textsc{auth-config}, $auth(l, e, \mu) = auth_{store}(l, \mu) \cup auth_{stack}(l,e,\mu)$. By \textsc{auth-stack-nocall}, $auth_{stack}(l, e, \mu) = \varnothing$. Then, $auth(l, e, \mu) = auth_{store}(l, \mu)$.

\hspace{12pt} Now let us consider $auth(l, e', \mu')$. By \textsc{auth-config}, $auth(l, e', \mu') = auth_{store}(l, \mu') \cup auth_{stack}(l, e',\mu')$. By \textsc{auth-stack-nocall}, $auth_{stack}(l, e', \mu') = \varnothing$. Then, $auth(l, e', \mu') = auth_{store}(l, \mu')$.

\hspace{12pt} Since $\mu' = \mu$, $auth_{store}(l, \mu') = auth_{store}(l, \mu)$ and thus $auth(l, e', \mu') = auth(l, e, \mu)$, i.e., no gain in authority occurred.
\\
\pcase[\textsc{E-Meth}]
$e = l_1.m(l_2)$, $e' = l_1.m(l_2) \rhd [l_2/y][l_1/x]e$, $\mu' = \mu$, and from the premise, we get $l_1 \mapsto \{ x \Rightarrow d \}_{s} \in \mu$ and $\keyw{def} m(y : \tau_1) : \tau_2 = e \in d$.

\hspace{12pt} Let us consider $auth(l, e, \mu)$. By \textsc{auth-config}, $auth(l, e, \mu) = auth_{store}(l, \mu) \cup auth_{stack}(l,e,\mu)$. By \textsc{auth-stack-nocall}, $auth_{stack}(l, e, \mu) = \varnothing$. Then, $auth(l, e, \mu) = auth_{store}(l, \mu)$.

\hspace{12pt} Now let us consider $auth(l, e', \mu')$. By \textsc{auth-config}, $auth(l, e', \mu') = auth_{store}(l, \mu') \cup auth_{stack}(l, e',\mu')$. Since $\mu' = \mu$, $auth(l, e', \mu') = auth_{store}(l, \mu) \cup auth_{stack}(l, e',\mu)$. By \textsc{auth-stack}, $auth_{stack}(l, e', \mu) = pointsto([l_2/y][l_1/x]e, \mu) \cup auth_{stack}(l, [l_2/y][l_1/x]e, \mu)$. Then, $auth(l, e', \mu') = auth_{store}(l, \mu) \cup pointsto([l_2/y][l_1/x]e, \mu) \cup auth_{stack}(l, [l_2/y][l_1/x]e, \mu)$.
...
\\
\pcase[\textsc{E-StackFrame}]
$e = l'.m(l_1) \rhd l_2$, $e' = l_2$, $\mu' = \mu$.

\hspace{12pt} Let us consider $auth(l, e, \mu)$. By \textsc{auth-config}, $auth(l, e, \mu) = auth_{store}(l, \mu) \cup auth_{stack}(l,e,\mu)$. By \textsc{auth-stack}, $auth_{stack}(l, e, \mu) = pointsto(l_2, \mu) \cup auth_{stack}(l, l_2, \mu)$. By \textsc{auth-stack-nocall}, $auth_{stack}(l, l_2, \mu) = \varnothing$. Then, $auth(l, e, \mu) = auth_{store}(l, \mu) \cup pointsto(l_2, \mu)$.

\hspace{12pt} Now let us consider $auth(l, e', \mu')$. By \textsc{auth-config}, $auth(l, e', \mu') = auth_{store}(l, \mu') \cup auth_{stack}(l, e',\mu')$. Since $\mu' = \mu$, $auth(l, e', \mu') = auth_{store}(l, \mu) \cup auth_{stack}(l, e',\mu)$. By \textsc{auth-stack-nocall}, $auth_{stack}(l, e', \mu') = \varnothing$. Then, $auth(l, e', \mu') = auth_{store}(l, \mu')$.
...
\\
\end{pcases}

\noindent...
\end{proof}


\newpage

\subsection{Sample Scenarios}

\noindent\underline{\smash{\textbf{\textit{Object creation:}}}}

\vspace{16pt}

\noindent Program code:
\vspace{-6pt}
\begin{lstlisting}[xleftmargin=20pt]
new($l_p$ =>
  def m(y : typeOfD) : typeOfE =
    new($l_D$ => var xd : typeOfE = y)
).m(new($l_E$ => var xe : Int = 5))
\end{lstlisting}

\vspace{12pt}
\noindent Evaluation step 1 (form before: $E.m(e)$, rule applied: \textsc{E-New}):
\vspace{-6pt}
\begin{lstlisting}[xleftmargin=20pt]
$l_p \rightarrow \{$ def m(y : typeOfD) : typeOfE =
          new($l_D$ => var xd : typeOfE = y) $\}$

$l_p$.m(new($l_E$ => var xe : Int = 5)
\end{lstlisting}

\vspace{12pt}
\noindent Evaluation step 2 (form before: $l.m(E)$, rule applied: \textsc{E-New}):
\vspace{-6pt}
\begin{lstlisting}[xleftmargin=20pt]
$l_p \rightarrow \{$ def m(y : typeOfD) : typeOfE =
          new($l_D$ => var xd : typeOfE = y) $\}$
$l_E \rightarrow \{$ var xe : Int = 5 } // assume 5 is a location in memory representing the integer value

$l_p$.m($l_E$)
\end{lstlisting}

\vspace{12pt}
\noindent Evaluation step 3 (form before: $l.m(l)$, rule applied: \textsc{E-Meth}):\\
\vspace{-6pt}
\indent\textit{Memory same as above}\\
\vspace{-6pt}
\begin{lstlisting}[xleftmargin=20pt]
$l_p$.m($l_E$) $\rhd$ new($l_D$ => var xd : typeOfE = $l_E$)
\end{lstlisting}

\vspace{12pt}
\noindent Evaluation step 4 (form before: $l.m(l) \rhd E$, rule applied: \textsc{E-New}):
\vspace{-6pt}
\begin{lstlisting}[xleftmargin=20pt]
$l_p \rightarrow \{$ def m(y : typeOfD) : typeOfE =
          new($l_D$ => var xd : typeOfE = y) $\}$
$l_E \rightarrow \{$ var xe : Int = 5 $\}$
$l_D \rightarrow \{$ var xd : typeOfE = $l_E$ $\}$

$l_p$.m($l_E$) $\rhd$ $l_D$
\end{lstlisting}

\vspace{12pt}
\noindent Evaluation step 5 (form before: $l.m(l) \rhd l$, rule applied: \textsc{E-StackFrame}):\\
\vspace{-6pt}
\indent\textit{Memory same as above}\\
\vspace{-6pt}
\begin{lstlisting}[xleftmargin=20pt]
$l_D$
\end{lstlisting}

\newpage

\noindent\textbf{$l_p$ gains authority over $l_D$ in evaluation step 4 when \textsc{E-New} is applied to create $l_D$:}\\

\noindent$e = E[\keywadj{new}_{\keywadj{stateful}}(l_D \Rightarrow \keyw{var} x_d : typeOfE = l_E)]$, $e' = E[l_D]$\\

\noindent$e = l_p.m(l_E) \rhd \keywadj{new}_{\keywadj{stateful}}(l_D \Rightarrow \keyw{var} x_d : typeOfE = l_E)$, $e' = l_p.m(l_E) \rhd l_D$\\

\noindent$auth(l_p, e', \mu') = auth(l_p, e, \mu) \cup \{ l_D \}$

\begin{align*}
auth(l_p, e', \mu') &= auth_{store}(l_p, \mu') \cup auth_{stack}(l_p, e', \mu') & (\textsc{auth-config})\\
auth_{store}(l_p, \mu') &= pointsto(l_p, \mu') \cup pointsto(\keyw{def} m~...~, \mu') & (\textsc{auth-store})\\
 &= \{ l_p \} \cup pointsto(\keyw{def} m(...) : typeOfE = \keywadj{new}(...), \mu') & (\textsc{pointsto-principal})\\
 &= \{ l_p \} \cup pointsto(\keywadj{new}(l_D \Rightarrow \keyw{var} x_d~...), \mu') & (\textsc{pointsto-def, -emptydecl})\\
 &= \{ l_p \} \cup pointsto(\keyw{var} x_d : typeOfE = l_E), \mu') & (\textsc{pointsto-new})\\
 &= \{ l_p \} \cup pointsto(l_E, \mu') & (\textsc{pointsto-varl, -emptydecl})\\ 
 &= \{ l_p,~l_E \} & (\textsc{pointsto-principal})\\
auth_{stack}(l_p, e', \mu') &= pointsto(l_D, \mu') \cup auth_{stack}(l_p, l_D, \mu') & (\textsc{auth-stack})\\
 &= \{ l_D \} \cup auth_{stack}(l_p, l_D, \mu') & (\textsc{pointsto-principal})\\
 &= \{ l_D \} & (\textsc{auth-stack-nocall})\\
auth(l_p, e', \mu') &= \{ l_p,~l_E,~l_D \} &\\
\end{align*}
\begin{align*}
auth(l_p, e, \mu) &= auth_{store}(l_p, \mu) \cup auth_{stack}(l_p, e, \mu) & (\textsc{auth-config})\\
auth_{store}(l_p, \mu) &= pointsto(l_p, \mu) \cup pointsto(\keyw{def} m~...~, \mu) & (\textsc{auth-store})\\
 &= \{ l_p \} \cup pointsto(\keyw{def} m(...) : typeOfE = \keywadj{new}(...), \mu) & (\textsc{pointsto-principal})\\
 &= \{ l_p \} \cup pointsto(\keywadj{new}(l_D \Rightarrow \keyw{var} x_d~...), \mu) & (\textsc{pointsto-def, -emptydecl})\\
 &= \{ l_p \} \cup pointsto(\keyw{var} x_d : typeOfE = l_E), \mu) & (\textsc{pointsto-new})\\
 &= \{ l_p \} \cup pointsto(l_E, \mu) & (\textsc{pointsto-varl, -emptydecl})\\ 
 &= \{ l_p,~l_E \} & (\textsc{pointsto-principal})\\
auth_{stack}(l_p, e, \mu) &= pointsto(\keywadj{new}(l_D \Rightarrow \keyw{var} x_d : typeOfE = l_E), \mu) &\\
 &\cup auth_{stack}(l_p, \keywadj{new}(l_D \Rightarrow \keyw{var} x_d : typeOfE = l_E), \mu) & (\textsc{auth-stack})\\
 &= pointsto(\keywadj{new}(l_D \Rightarrow \keyw{var} x_d : typeOfE = l_E), \mu) & (\textsc{auth-stack-nocall})\\
 &= pointsto(\keyw{var} x_d : typeOfE = l_E), \mu') & (\textsc{pointsto-new})\\
 &= pointsto(l_E, \mu') & (\textsc{pointsto-varl, -emptydecl})\\ 
 &= \{ l_E \} & (\textsc{pointsto-principal})\\
auth(l_p, e, \mu) &=  \{ l_p,~l_E\}&\\
\end{align*}


\newpage

\noindent\underline{\smash{\textbf{\textit{Method call:}}}}

\vspace{16pt}

\noindent Program code:
\vspace{-6pt}
\begin{lstlisting}[xleftmargin=20pt]
new($l_p$ =>
  def mp(yi : typeOfI, yh : typeOfH) : typeOfJ =
    new($l_J$ =>
      var xji : typeOfI = yi
      var xjh : typeOfH = yh
      def mj(Unit) : Int =
        xjh.mh(xji)))
.mp(new($l_I$ => var xi : Int = 3), new($l_H$ => def mh(y : typeOfI) : Int = y.xi))
.mj()
\end{lstlisting}

\vspace{12pt}
\noindent Evaluation step 1 (form before: $E.m(e)$, going deeper):\\
\indent\textit{Code same as above}

\vspace{12pt}
\noindent Evaluation step 2 (form before: $E.m(e)$, rule applied: \textsc{E-New}):
\vspace{-6pt}
\begin{lstlisting}[xleftmargin=20pt]
$l_p \rightarrow \{$   def mp(yi : typeOfI, yh : typeOfH) : typeOfJ =
         new($l_J$ =>
           var xji : typeOfI = yi
           var xjh : typeOfH = yh
           def mj(Unit) : Int =
             xjh.mh(xji)) $\}$

$l_p$.mp(new($l_I$ => var xi : Int = 3), new($l_H$ => def mh(y : typeOfI) : Int = y.xi)).mj()
\end{lstlisting}

\vspace{12pt}
\noindent Evaluation step 3 (form before: $l.m(E, E)$, rule applied: \textsc{E-New} x2):
\vspace{-6pt}
\begin{lstlisting}[xleftmargin=20pt]
$l_p \rightarrow \{$   def mp(yi : typeOfI, yh : typeOfH) : typeOfJ =
         new($l_J$ =>
           var xji : typeOfI = yi
           var xjh : typeOfH = yh
           def mj(Unit) : Int =
             xjh.mh(xji)) $\}$
$l_I \rightarrow \{$ var xi : Int = 3 } // assume 3 is a location in memory representing the integer value
$l_H \rightarrow \{$ def mh(y : typeOfI) : Int = y.xi $\}$

$l_p$.mp($l_I$, $l_H$).mj()
\end{lstlisting}

\vspace{12pt}
\noindent Evaluation step 4 (form before: $l.m(l, l)$, rule applied: \textsc{E-Meth}):\\
\vspace{-6pt}
\indent\textit{Memory same as above}\\
\vspace{-6pt}
\begin{lstlisting}[xleftmargin=20pt]
($l_p$.mp($l_I$, $l_H$) $\rhd$ new($l_J$ =>
                    var xji : typeOfI = $l_I$
                    var xjh : typeOfH = $l_H$
                    def mj(Unit) : Int =
                      xjh.mh(xji)))
.mj()
\end{lstlisting}

\vspace{12pt}
\noindent Evaluation step 5 (form before: $l.m(l, l) \rhd E$, rule applied: \textsc{E-New}):
\vspace{-6pt}
\begin{lstlisting}[xleftmargin=20pt]
$l_p \rightarrow \{$   def mp(yi : typeOfI, yh : typeOfH) : typeOfJ =
         new($l_J$ =>
           var xji : typeOfI = yi
           var xjh : typeOfH = yh
           def mj(Unit) : Int =
             xjh.mh(xji)) $\}$
$l_I \rightarrow \{$ var xi : Int = 3 }
$l_H \rightarrow \{$ def mh(y : typeOfI) : Int = y.xi $\}$
$l_J \rightarrow \{$ var xji : typeOfI = $l_I$
        var xjh : typeOfH = $l_H$
        def mj(Unit) : Int =
          xjh.mh(xji) $\}$

($l_p$.mp($l_I$, $l_H$) $\rhd$ $l_J$).mj()
\end{lstlisting}

\vspace{12pt}
\noindent Evaluation step 6 (form before: $l.m(l) \rhd l$, rule applied: \textsc{E-StackFrame}):\\
\vspace{-6pt}
\indent\textit{Memory same as above}\\
\vspace{-6pt}
\begin{lstlisting}[xleftmargin=20pt]
$l_J$.mj()
\end{lstlisting}

\vspace{12pt}
\noindent Evaluation step 7 (form before: $E.m(e)$, $E$ is already evaluated to an object):\\
\indent\textit{Code same as above}

\vspace{12pt}
\noindent Evaluation step 8 (form before: $l.m(E)$, a \lstinline{Unit} argument passed, $E$ is already fully evaluated):\\
\indent\textit{Code same as above}

\vspace{12pt}
\noindent Evaluation step 9 (form before: $l.m(l)$, rule applied: \textsc{E-Meth}):\\
\vspace{-6pt}
\indent\textit{Memory same as above}\\
\vspace{-6pt}
\begin{lstlisting}[xleftmargin=20pt]
$l_J$.mj() $\rhd$ xjh.mh(xji)
\end{lstlisting}

\vspace{12pt}
\noindent Evaluation step 10 (form before: $l.m(l) \rhd E$, going deeper):\\
\textit{Code same as above}

\vspace{12pt}
\noindent Evaluation step 11 (forms before: $E.m(e)$, $l.m(E)$, substitution of concrete objects for variables x2):\\
\vspace{-6pt}
\indent\textit{Memory same as above}\\
\vspace{-6pt}
\begin{lstlisting}[xleftmargin=20pt]
$l_J$.mj() $\rhd$ $l_H$.mh($l_I$)
\end{lstlisting}

\vspace{12pt}
\noindent Evaluation step 12 (form before: $l.m(l)$, rule applied: \textsc{E-Meth}):\\
\vspace{-6pt}
\indent\textit{Memory same as above}\\
\vspace{-6pt}
\begin{lstlisting}[xleftmargin=20pt]
$l_J$.mj() $\rhd$ ($l_H$.mh($l_I$) $\rhd$ $l_I$.xi)
\end{lstlisting}

\vspace{12pt}
\noindent Evaluation step 13 (form before: $l.m(l) \rhd E$, rule applied: \textsc{E-Field}):\\
\vspace{-6pt}
\indent\textit{Memory same as above}\\
\vspace{-6pt}
\begin{lstlisting}[xleftmargin=20pt]
$l_J$.mj() $\rhd$ ($l_H$.mh($l_I$) $\rhd$ 3)
\end{lstlisting}

\vspace{12pt}
\noindent Evaluation step 14 (form before: $l.m(l) \rhd l$, rule applied: \textsc{E-StackFrame}):\\
\vspace{-6pt}
\indent\textit{Memory same as above}\\
\vspace{-6pt}
\begin{lstlisting}[xleftmargin=20pt]
$l_J$.mj() $\rhd$ 3
\end{lstlisting}

\vspace{12pt}
\noindent Evaluation step 15 (form before: $l.m(l) \rhd l$, rule applied: \textsc{E-StackFrame}):\\
\vspace{-6pt}
\indent\textit{Memory same as above}\\
\vspace{-6pt}
\begin{lstlisting}[xleftmargin=20pt]
3
\end{lstlisting}


\newpage

\noindent\textbf{$l_H$ gains authority over $l_I$ through $l_J$ when $l_I$ is passed in as an argument to $m_h()$ in evaluation step 12 when \textsc{E-Meth} is applied:}\\

\noindent ($l_H$ is pure; $l_I$ and $l_J$ are stateful)\\

\noindent$e = E[l_H.m_h(l_I)]$, $e' = E[l_H.m_h(l_I) \rhd l_I.x_i]$\\

\noindent$e = l_J.m_j() \rhd l_H.m_h(l_I)$, $e' = l_J.m_j() \rhd (l_H.m_h(l_I) \rhd l_I.x_i)$\\

\noindent$auth(l_H, e', \mu') = auth(l_H, e, \mu) \cup \{ l_I \}$

\begin{align*}
auth(l_H, e', \mu') &= auth_{store}(l_H, \mu') \cup auth_{stack}(l_H, e', \mu') & (\textsc{auth-config})\\
auth_{store}(l_H, \mu') &= pointsto(l_H, \mu') \cup pointsto(\keyw{def} m_h~...~, \mu') & (\textsc{auth-store})\\
 &= pointsto(\keyw{def} m_h(y : typeOfI) : Int = y.x_i, \mu') & (\textsc{pointsto-data})\\
 &= pointsto(y.x_i, \mu') & (\textsc{pointsto-def, -emptydecl})\\
 &= pointsto(y, \mu') & (\textsc{pointsto-otherexp})\\
 &= \varnothing & (\textsc{pointsto-var})\\
auth_{stack}(l_H, e', \mu') &= pointsto(l_I.x_i, \mu') \cup auth_{stack}(l_H, l_I.x_i, \mu') & (\textsc{auth-stack})\\
 &= pointsto(l_I.x_i, \mu') & (\textsc{auth-stack-nocall})\\
 &= pointsto(l_I, \mu') & (\textsc{pointsto-otherexp})\\
 &= \{ l_I \} & (\textsc{pointsto-principal})\\
auth(l_H, e', \mu') &= \{ l_I \} &\\
\end{align*}
\begin{align*}
auth(l_H, e, \mu) &= auth_{store}(l_H, \mu) \cup auth_{stack}(l_H, e, \mu) & (\textsc{auth-config})\\
auth_{store}(l_H, \mu) &= pointsto(l_H, \mu) \cup pointsto(\keyw{def} m_h~...~, \mu) & (\textsc{auth-store})\\
 &= pointsto(\keyw{def} m_h(y : typeOfI) : Int = y.x_i, \mu) & (\textsc{pointsto-data})\\
 &= pointsto(y.x_i, \mu) & (\textsc{pointsto-def, -emptydecl})\\
 &= pointsto(y, \mu) & (\textsc{pointsto-otherexp})\\
 &= \varnothing & (\textsc{pointsto-var})\\
auth_{stack}(l_H, e, \mu) &= \varnothing & (\textsc{auth-stack-nocall})\\
auth(l_H, e, \mu) &=  \varnothing &\\
\end{align*}


\newpage

\noindent\underline{\smash{\textbf{\textit{Method return:}}}}

\vspace{16pt}

\noindent Program code:
\vspace{-6pt}
\begin{lstlisting}[xleftmargin=20pt]
new($l_p$ =>
  def mp(y : typeOfA) : typeOfC =
    new($l_C$ =>
      var xca : typeOfA = y
      def mc() : typeOfB =
        xca.ma()))
.mp(new($l_A$ => def ma() : typeOfB =
                  new($l_B$ => var xb : Int = 42)))
.mc()
\end{lstlisting}

\vspace{12pt}
\noindent Evaluation step 1 (form before: $E.m(e)$, going deeper):\\
\indent\textit{Code same as above}

\vspace{12pt}
\noindent Evaluation step 2 (form before: $E.m(e)$, rule applied: \textsc{E-New}):
\vspace{-6pt}
\begin{lstlisting}[xleftmargin=20pt]
$l_p \rightarrow \{$ def mp(y : typeOfA) : typeOfC =
         new($l_C$ =>
           var xca : typeOfA = y
           def mc() : typeOfB =
             xca.ma()) $\}$
$l_p$.mp(new($l_A$ => def ma() : typeOfB = new($l_B$ => var xb : Int = 42))).mc()
\end{lstlisting}

\vspace{12pt}
\noindent Evaluation step 3 (form before: $l.m(E)$, rule applied: \textsc{E-New}):
\vspace{-6pt}
\begin{lstlisting}[xleftmargin=20pt]
$l_p \rightarrow \{$ def mp(y : typeOfA) : typeOfC =
         new($l_C$ =>
           var xca : typeOfA = y
           def mc() : typeOfB =
             xca.ma()) $\}$
$l_A \rightarrow \{$ def ma() : typeOfB =
          new($l_B$ => var xb : Int = 42) $\}$

$l_p$.mp($l_A$).mc()
\end{lstlisting}

\vspace{12pt}
\noindent Evaluation step 4 (form before: $l.m(l)$, rule applied: \textsc{E-Meth}):\\
\vspace{-6pt}
\indent\textit{Memory same as above}\\
\vspace{-6pt}
\begin{lstlisting}[xleftmargin=20pt]
($l_p$.mp($l_A$) $\rhd$ new($l_C$ =>
                var xca : typeOfA = $l_A$
                def mc() : typeOfB =
                  xca.ma())
.mc()
\end{lstlisting}

\vspace{12pt}
\noindent Evaluation step 5 (form before: $l.m(l) \rhd E$, rule applied: \textsc{E-New}):
\vspace{-6pt}
\begin{lstlisting}[xleftmargin=20pt]
$l_p \rightarrow \{$ def mp(y : typeOfA) : typeOfC =
         new($l_C$ =>
           var xca : typeOfA = y
           def mc() : typeOfB =
             xca.ma()) $\}$
$l_A \rightarrow \{$ def ma() : typeOfB =
          new($l_B$ => var xb : Int = 42) $\}$
$l_C \rightarrow \{$ var xca : typeOfA = $l_A$
        def mc() : typeOfB =
          xca.ma()) $\}$

($l_p$.mp($l_A$) $\rhd$ $l_C$).mc()
\end{lstlisting}

\vspace{12pt}
\noindent Evaluation step 6 (form before: $l.m(l) \rhd l$, rule applied: \textsc{E-StackFrame}):\\
\vspace{-6pt}
\indent\textit{Memory same as above}\\
\vspace{-6pt}
\begin{lstlisting}[xleftmargin=20pt]
$l_C$.mc()
\end{lstlisting}

\vspace{12pt}
\noindent Evaluation step 7 (form before: $E.m(e)$, $E$ is already evaluated to an object):\\
\indent\textit{Code same as above}

\vspace{12pt}
\noindent Evaluation step 8 (form before: $l.m(E)$, a \lstinline{Unit} argument passed, $E$ is already fully evaluated):\\
\indent\textit{Code same as above}

\vspace{12pt}
\noindent Evaluation step 9 (form before: $l.m(l)$, rule applied: \textsc{E-Meth}):\\
\vspace{-6pt}
\indent\textit{Memory same as above}\\
\vspace{-6pt}
\begin{lstlisting}[xleftmargin=20pt]
$l_C$.mc() $\rhd$ xca.ma()
\end{lstlisting}

\vspace{12pt}
\noindent Evaluation step 10 (form before: $l.m(l) \rhd E$, going deeper):\\
\textit{Code same as above}

\vspace{12pt}
\noindent Evaluation step 11 (form before: $E.m(e)$, substitution of concrete object for variable):\\
\vspace{-6pt}
\indent\textit{Memory same as above}\\
\vspace{-6pt}
\begin{lstlisting}[xleftmargin=20pt]
$l_C$.mc() $\rhd$ $l_A$.ma()
\end{lstlisting}

\vspace{12pt}
\noindent Evaluation step 12 (form before: $l.m(E)$, a \lstinline{Unit} argument passed, $E$ is already fully evaluated):\\
\indent\textit{Code same as above}

\vspace{12pt}
\noindent Evaluation step 13 (form before: $l.m(l)$, rule applied: \textsc{E-Meth}):\\
\vspace{-6pt}
\indent\textit{Memory same as above}\\
\vspace{-6pt}
\begin{lstlisting}[xleftmargin=20pt]
$l_C$.mc() $\rhd$ ($l_A$.ma() $\rhd$ new($l_B$ => var xb : Int = 42))
\end{lstlisting}

\vspace{12pt}
\noindent Evaluation step 14 (form before: $l.m(l) \rhd E$, rule applied: \textsc{E-New}):
\vspace{-6pt}
\begin{lstlisting}[xleftmargin=20pt]
$l_p \rightarrow \{$ def mp(y : typeOfA) : typeOfC =
         new($l_C$ =>
           var xca : typeOfA = y
           def mc() : typeOfB =
             xca.ma()) $\}$
$l_A \rightarrow \{$ def ma() : typeOfB =
          new($l_B$ => var xb : Int = 42) $\}$
$l_C \rightarrow \{$ var xca : typeOfA = $l_A$
        def mc() : typeOfB =
          xca.ma()) $\}$
$l_B \rightarrow \{$ var xb : Int = 42 $\}$

$l_C$.mc() $\rhd$ ($l_A$.ma() $\rhd$ $l_B$)
\end{lstlisting}

\vspace{12pt}
\noindent Evaluation step 15 (form before: $l.m(l) \rhd l$, rule applied: \textsc{E-StackFrame}):\\
\vspace{-6pt}
\indent\textit{Memory same as above}\\
\vspace{-6pt}
\begin{lstlisting}[xleftmargin=20pt]
$l_C$.mc() $\rhd$ $l_B$
\end{lstlisting}

\vspace{12pt}
\noindent Evaluation step 16 (form before: $l.m(l) \rhd l$, rule applied: \textsc{E-StackFrame}):\\
\vspace{-6pt}
\indent\textit{Memory same as above}\\
\vspace{-6pt}
\begin{lstlisting}[xleftmargin=20pt]
$l_B$
\end{lstlisting}


\newpage

\noindent\textbf{$l_C$ gains authority over $l_B$ due to a return from $l_A$'s method $m_a()$ in evaluation step 15 when \textsc{E-StackFrame} is applied:}\\

\noindent ($l_A$, $l_B$ and $l_C$ are stateful)\\

\noindent$e = E[l_A.m_a() \rhd l_B]$, $e' = E[l_B]$\\

\noindent$e = l_C.m_c() \rhd (l_A.m_a() \rhd l_B)$, $e' = l_C.m_c() \rhd l_B$\\

\noindent$auth(l_C, e', \mu') = auth(l_C, e, \mu) \cup \{ l_B \}$

\begin{align*}
auth(l_C, e', \mu') &= auth_{store}(l_C, \mu') \cup auth_{stack}(l_C, e', \mu') & (\textsc{auth-config})\\
auth_{store}(l_C, \mu') &= pointsto(l_C, \mu') \cup pointsto(\keyw{var} x_{ca}~...~, \mu') & (\textsc{auth-store})\\
 &= \{ l_C \} \cup pointsto(\keyw{var} x_{ca} : typeOfA = l_A;~\keyw{def} m_c~...~, \mu') & (\textsc{pointsto-principal})\\
 &= \{ l_C \} \cup pointsto(l_A, \mu) \cup pointsto(\keyw{def} m_c~...~, \mu') & (\textsc{pointsto-varl})\\
 &= \{ l_C, l_A \} \cup pointsto(\keyw{def} m_c() : typeOfB = x_{ca}.m_a(), \mu') & (\textsc{pointsto-principal})\\
 &= \{ l_C, l_A \} \cup pointsto(x_{ca}.m_a(), \mu') & (\textsc{pointsto-def, -emptydecl})\\
 &= \{ l_C, l_A \} \cup pointsto(x_{ca}, \mu') & (\textsc{pointsto-otherexp, -data})\\
 &= \{ l_C,~l_A \} & (\textsc{pointsto-var})\\
auth_{stack}(l_C, e', \mu') &= pointsto(l_B, \mu') \cup auth_{stack}(l_C, l_B, \mu') & (\textsc{auth-stack})\\
 &= \{ l_B \} \cup auth_{stack}(l_C, l_B, \mu') & (\textsc{pointsto-principal})\\
 &= \{ l_B \} & (\textsc{auth-stack-nocall})\\
auth(l_C, e', \mu') &= \{ l_C,~l_A,~l_B \} &\\
\end{align*}
\begin{align*}
auth(l_C, e, \mu) &= auth_{store}(l_C, \mu) \cup auth_{stack}(l_C, e, \mu) & (\textsc{auth-config})\\
auth_{store}(l_C, \mu) &= pointsto(l_C, \mu) \cup pointsto(\keyw{var} x_{ca}~...~, \mu) & (\textsc{auth-store})\\
 &= \{ l_C \} \cup pointsto(\keyw{var} x_{ca} : typeOfA = l_A;~\keyw{def} m_c~...~, \mu) & (\textsc{pointsto-principal})\\
 &= \{ l_C \} \cup pointsto(l_A, \mu) \cup pointsto(\keyw{def} m_c~...~, \mu) & (\textsc{pointsto-varl})\\
 &= \{ l_C, l_A \} \cup pointsto(\keyw{def} m_c() : typeOfB = x_{ca}.m_a(), \mu) & (\textsc{pointsto-principal})\\
 &= \{ l_C, l_A \} \cup pointsto(x_{ca}.m_a(), \mu) & (\textsc{pointsto-def, -emptydecl})\\
 &= \{ l_C, l_A \} \cup pointsto(x_{ca}, \mu) & (\textsc{pointsto-otherexp, -data})\\
 &= \{ l_C,~l_A \} & (\textsc{pointsto-var})\\
auth_{stack}(l_C, e, \mu) &= pointsto(l_A.m_a() \rhd l_B, \mu) \cup auth_{stack}(l_C, l_A.m_a() \rhd l_B, \mu) & (\textsc{auth-stack})\\
 &= \{ l_A \} \cup auth_{stack}(l_C, l_A.m_a() \rhd l_B, \mu) & (\textsc{pointsto-call-principal})\\
 &= \{ l_A \} & (\textsc{auth-stack-nocall})\\
auth(l_C, e, \mu) &=  \{ l_C,~l_A\}&\\
\end{align*}



\bibliographystyle{plain}
\bibliography{bibtex}

\end{document}