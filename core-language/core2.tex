\section{A Method-Based Language}

This language is the lambda calculus with iso-recursive types and mutable objects.  It enforces the uniform access principle.  It also encapsulates state within objects.  The differences vs. the core language are as follows:

Contribution: OO language that is closest to lambda calculus, while enforcing the uniform access principle and encapsulating state.

\begin{itemize}

 \item Distinguishes the ``internal'' type of a receiver that is known within a method, which contains both methods and mutable fields, from types that can be written in the external language, which only contain methods.
 
 \item Encodes references as mutable fields.
 
 \item Uses different operations for field dereference vs. method call

\end{itemize}

justify lazy eval: otherwise m acts as a field, we want computation and the uniform access principle

In the rules below, the context $\Delta$ contains types and their signatures. The context $\Gamma$ contains variables and their types. 
The judgement $\Delta; \Gamma; \sigma \vdash e:\tau $ states that from the type context $\Delta$, the variable context $\Gamma$ and the surrounding type $\sigma$, we deduce that $e$ is of type $\tau$.

\begin{figure}
\centering
\[
\begin{array}{lll}

e    & \bnfdef & x \\
     & \bnfalt & \boldsymbol\lambda x{:}\tau . e \\
     & \bnfalt & e(e) \\
     & \bnfalt & \keyw{new}~ \{ \overline{d} \}\\
     & \bnfalt & e.f \\
     & \bnfalt & e.f = e \\
     & \bnfalt & e.m \\
\\[1ex]

\tau & \bnfdef & t\\
     & \bnfalt & \tau \rightarrow \tau \\
\\[1ex]
	 
\end{array}
\begin{array}{lll}
~~~~
\end{array}
\begin{array}{lll}
	 
d    & \bnfdef & \keyw{var}~ f:\tau = e \\
     & \bnfalt & \keyw{meth}~ m:\tau = e \\
     & \bnfalt & \keyw{type}~ t~ = \{ \overline{\tau_d}, \keyw{attributes}=e \} \\
     & \bnfalt & \keyw{type}~ t~ = \{ \tau \}  \\
\\[1ex]

\tau_d   & \bnfdef & \keyw{meth}~ m:\tau \\
\\[1ex]

\sigma & \bnfdef & \tau \\
       & \bnfalt & \{ \overline{\sigma_d} \} \\
\\[1ex]

\sigma_d & \bnfdef & \keyw{var}~ f:\tau \\
         & \bnfalt & \tau_d \\

\end{array}
\]
\caption{Featherweight Wyvern Syntax}
\label{fig:core2-syntax}
\end{figure}


\begin{figure}
\centering
\[
\begin{array}{c}

\infer[\textit{T-varx}]
	{\Delta,\Gamma \vdash x:\tau } 
	{x:\tau \in \Gamma & \tau= \{...\} \in \Delta}\\[3ex]

\infer[\textit{T-abs}]
	{\Delta; \Gamma \vdash  \boldsymbol\lambda x{:}\tau . e :: \tau \rightarrow \tau_1} 
	{\Delta; (\Gamma, x:\tau) \vdash e:\tau_1 }\\[3ex]

\infer[\textit{T-appl}]
	{\Delta; \Gamma \vdash  e(e_1):\tau_2} 
	{\Delta; \Gamma \vdash e:\tau_1 \rightarrow \tau_2 & \Gamma \vdash e_1:\tau_1}\\[3ex]

\infer[\textit{T-new}]
	{\Delta; \Gamma \vdash \keyw{new}~\sigma \{ \overline{d} \} : \{ \overline{\tau_d} \}}
	{\Delta; \Gamma; \sigma \vdash \overline{d} :: \overline{\sigma_d} & \sigma = \{ \overline{\sigma_d} \} & \overline{\tau_d} \subseteq \overline{\sigma_d}} \\[3ex]

\infer[\textit{T-field}]
	{\Delta; \Gamma \vdash  e.f:\tau_1} 
	{\Delta; \Gamma \vdash e:\sigma & \sigma= \{\keyw{var}~ f:\tau_1,... \} }\\[3ex]

\infer[\textit{T-assign }]
	{\Delta; \Gamma \vdash  e.f=e_2:\tau_1} 
	{\Delta; \Gamma \vdash e:\sigma & \sigma= \{\keyw{var}~ f:\tau_1=e_1,... \} & \Delta; \Gamma\vdash e_2:\tau_1 }\\[3ex]

\infer[\textit{T-meth2 }]
	{\Delta; \Gamma \vdash  e.m:\tau_1} 
	{\Delta; \Gamma \vdash e:\tau & \tau=\{...\keyw{meth}~ m:\tau_1=e_1,... \} }\\[3ex]

\infer[\textit{DT-var}]
	{\Delta; \Gamma; \sigma \vdash \keyw{var}~ f:\tau = e :: \keyw{var}~ f:\tau; d }
	{\Delta; \Gamma \vdash e : \tau & \Delta; \Gamma \vdash d} \\[3ex]
	
\infer[\textit{DT-meth}]
	{\Delta; \Gamma; \sigma \vdash \keyw{meth}~ m:\tau = e :: \keyw{meth}~ m:\tau; d }
	{\Delta; (\Gamma, \textit{this}:\sigma) \vdash e : \tau & \Delta; \Gamma \vdash d } \\[3ex]

\infer[\textit{DT-type}]
	{\Delta; \Gamma \vdash  \keyw{type}~ t~=\{\overline{\tau_d}, \keyw{attributes}=e\}; d} 
	{\Delta; \Gamma \vdash e:\tau ~~~~~ \Delta; \Gamma; t:\{\overline{\tau_d},\tau\} \vdash d ~~~~~ \tau: \{\overline{\tau_d}\}  }\\[3ex]

\infer[\textit{DT-getattrib}]
        {\Delta; \Gamma \vdash e:\tau   }
	{\Delta; \Gamma \vdash  \keyw{type}~ t~=\{\overline{\tau_d}, \keyw{attributes}=e\} : \{\overline{\tau_d},\tau\}} \\[3ex]
	

\infer[\textit{DT-typeabbrev}]
	{\Delta; \Gamma \vdash  \keyw{type}~ t~=\{\tau\}; d} 
	{\Delta; \Gamma \vdash [\tau/t]d }\\[3ex]


\infer[\textit{Type-meth}]
	{\Delta; \Gamma \vdash  t.m:\tau} 
	{t : \{\overline{\tau_d}, t1\} \in \Delta ~~~~~ t1 : \{\overline{\tau'_d},\_\} \in \Delta ~~~~~ \keyw{meth}:\tau \in \overline{\tau'_d}   }\\[3ex]
	

\end{array}
\]
\caption{Static Semantics Rules Core 2}
\end{figure}

\begin{figure}
\centering
\[
\begin{array}{c}
\infer[\textit{T-refl}]
	{\tau<:\tau}
	{} \\[3ex]

\infer[\textit{T-trans}]
	{\tau_1<:\tau_3}
	{\tau_1<:\tau_2 & \tau_2<:\tau_3} \\[3ex]	

\infer[\textit{T-RcdWidth}]
	{\{ \keyw{meth}~m_i{:}\tau_i ^{i \in 1..n+k} \} <: \{\keyw{meth}~m_i{:}\tau_i ^{i \in 1..n} \}}
	{}\\[3ex]

\infer[\textit{T-RcdDepth}]
	{\{ \keyw{meth}~m_i{:}\tau_i ^{i \in 1..n} \} <: \{ \keyw{meth}~m_i{:}{\tau'}_i ^{i \in 1..n} \}}
	{\textrm{for each \textit{i}} ~ \tau_i <: {\tau'}_i}\\[3ex]

\infer[\textit{T-RcdPerm}]
	{\{ \keyw{meth}~m_j{:}\tau_j ^{j \in 1..n} \} <: \{ \keyw{meth}~{m'}_i{:}{\tau'}_i ^{i \in 1..n} \}}
	{\{ \keyw{meth}~m_j{:}\tau_j ^{j \in 1..n} \} \textrm{ is a permutation of } \{ \keyw{meth}~{m'}_i{:}{\tau'}_i ^{i \in 1..n} \}}\\[3ex]

\infer[\textit{T-Arrow}]
	{\tau_1 \rightarrow \tau_2 <: \tau_3 \rightarrow \tau_4}
	{\tau_3 <: \tau_1 & \tau_2 <: \tau_4}\\[3ex]

\infer[\textit{T-SubMeth}]
	{\keyw{meth}~m:\tau_1 <: \keyw{meth}~ m:\tau_2}
	{\tau_1 <: \tau_2}\\[3ex]

\infer[\textit{T-SubVar}]
	{\keyw{var} f:\tau_1 <: \keyw{var} f:\tau_1}
	{\tau_1 <: \tau_1}\\[3ex]

\infer[\textit{T-SubNew}]
	{\keyw{new} \sigma_1\{\overline{d_1}\} <: \keyw{new} \sigma_2\{\overline{d_2}\}}
	{\sigma_1 <: \sigma_2}\\[3ex]

\end{array}
\]
\caption{Subtyping Rules Core 2}
\end{figure}


\begin{figure}
\centering
\begin{tabular}{ r c l }
$\textit{trans}(\keyw{meth}~ m:\tau)$ & $\equiv$ & $m:\keyw{unit} \rightarrow \tau$ \\

$\textit{trans}(\keyw{var}~ f:\tau)$ & $\equiv$ & $f:\keyw{ref}~ \tau$ \\

$\textit{trans}(\keyw{meth}~ m:\tau = e; \overline{d})$ & $\equiv$ & $m:\tau = \boldsymbol\lambda \_:\keyw{unit}.\textit{trans}(e);$ \\
&& $\textit{trans}(\overline{d})$\\

$\textit{trans}(\keyw{var}~ f:\tau = e; \overline{d})$ & $\equiv$ & $f:\tau := \keyw{alloc}~ \textit{trans}(e);$ \\
&& $\textit{trans}(\overline{d})$\\

$\textit{trans}(e.m)$ & $\equiv$ & $(\keyw{unfold}~ \textit{trans}(e)).m()$ \\

$\textit{trans}(\keyw{new}~ \{\overline{d}\})$ & $\equiv$ & $\keyw{letrec}~this:\sigma =$\\
&&$ ~~~~ \{ \textit{trans}(\overline{d}) \}$\\
&&$ \keyw{in}~ \keyw{fold}~ this$ \\
&& \textit{where} $~ \Gamma,\sigma \vdash \{ \overline{d} \} : \sigma$ \\

$\textit{trans}(e.f=e_1)$ & $\equiv$ & $\textit{trans}(e).f=\textit{trans}(e_1)$ \\

$\textit{trans}(e.f)$ & $\equiv$ &  $!\textit{trans}(e).f$ \\

$\textit{trans}(\keyw{type}~ t = \{\overline{\tau_d}\}; \overline{d})$ & $\equiv$ & $\textit{trans}([\boldsymbol\mu t.\textit{trans}(\overline{\tau_d})/t]\overline{d})$ \\

\end{tabular}
\caption{Translation from Featherweight Wyvern to the Extended Lambda Calculus}
\label{fig:cw-translate}
\end{figure}



T-new probably needs to change to fold the result type.


Note: recursively bound occurrences in an internal type ($\sigma$) are of an external type.  That is, the language does not support a ``thistype''.

\todo{Ligia: Give translation rules to core 1.  Need T-type rule.  Need store and store type.  Need subtyping. Where is $\mu$ used in the typing rules? show off: classes are first class(but point out distinction from smalltalk)- write one example, enforces uniform access principle; related work}

\newpage 

\subsection{Example Program in the Method-Based Language}

The code below uses \lstinline!val! for readability; assume \lstinline!val x = e1; e2! is equivalent to \lstinline!(fn x => e2)(e1)!.  We also use \lstinline!fn x : type => e! in place of $\boldsymbol\lambda x{:}\tau . e$, and we write \lstinline!rec! for $\boldsymbol\mu$.

\begin{lstlisting}
type t { meth add : int -> t }
type ti { var f : int
	  type t3 { int -> t }
	  meth add : t3 }
val o : t = new ti
	var f : int = 1
	type t3 { int -> t }
	meth add : t3 =
		fn x : int =>
			this.f = this.f + x
			this
val o3 : t = o.add(2)
\end{lstlisting}


\subsection{Translation of the Program to the Core Language}

We assume global type abbreviations because the translated code would be unintelligible without thenm.  They can be eliminated by capture-avoiding substitution.  We also assume val declarations, which can be encoded with functions and function calls.

\begin{lstlisting}
type unit = {} // standard prelude
type t { add : int -> t }
	
type ti { f : ref int
	  add : int -> t }
	
val o =
	letrec this : ti = {
		add = fn _ : unit => fn x : int =>
			this.f = !this.f + x
			fold[t] this
		f = alloc 1
	} in fold[t] this

val o3 = (unfold[t] o).add()(2)
\end{lstlisting}

Tasks:

\begin{itemize}

 \item write a rewriting rule for new expressions, translating methods to lambdas, translating types appropriately, and translating vars to refs
 \item give complete rewriting rules (R*) to the core language
 \item prove that well-typed source programs translate to well-typed core programs.  Is it possible to prove a property related to the uniform access principle and/or state encapsulation?
 \item argue this is OO in the sense of Cook.  Since the body of a method is evaluated on every access to an object, it seems to qualify.
\end{itemize}

\section{Example Factorial}
\begin{lstlisting}
type t { meth factorial : int -> t }
type ti { var f : int
	  meth factorial : int -> t }
val o : t = new ti
	var f : int = 1
	type t3 { int -> t }  
	meth factorial : t3 =
		fn x : int =>
                        if x>1 
			      this.f = x * (factorial(x-1)).f
                        else
                              this.f = 1
			this
val o3 : t = o.factorial(10)
\end{lstlisting}


