% !TEX root = plas14.tex

\section{Introduction / Motivation}

Weaknesses categorized as \emph{injection vulnerabilities} have topped
rankings of the most critical web application vulnerabilities for
several years in a row amongst industry groups \cite{owasp, cwsans}.
Web application vulnerabilities achieve top rankings by being quite
onerous. Injection vulnerabilities are tedious and
difficult to prevent, and are often easily exploitable with
severe technical consequences.

\todo{handlebars stuff here or close to this?}

Injection vulnerabilities can occur anywhere that user input may be erroneously
executed as code. In essence, an injection attack on a web application
occurs when a user---anyone with access to the application and a
vulnerable input field---submits maliciously crafted input that has
been made to appear and execute as code in a specific browser context.
The term \emph{browser context} refers to the browser's parsing state
when processing a string. If the browser is parsing a string as HTML,
it is in an HTML context. Likewise, if the browser has determined
a string is CSS and is processing it using its CSS engine, we call
the context the CSS context.

To ensure user inputs are not erroneously executed as a code, the inputs
must be properly encoded before they are placed in a vulnerable context. Using 
the correct encoding for a given context ensures the input will be recognized as 
data and not code. Traditionally, developers have had to create sanitizers to perform 
this encoding process themselves and then must place each sanitizer manually within
their application's code to prevent injection vulnerabilities. Many researchers have
noted that developers have significant trouble making and placing sanitizers manually \todo{cite these studies}. While the problem of creating correct sanitizers has largely been solved by the use of mainstream security libraries and frameworks, there is no accepted solution for sanitizer placement.

To prevent mistakes in placing sanitizers, several researchers have presented approaches 
that automatically place sanitizers in the correct locations for existing programs. 
The earliest fully automatic approach was presented by Samuel \emph{et al}., which
generates type constraints for contexts and solves those constraints to automatically place
sanitizers in small Google Closure programs \cite{Samuel:2011}. This 
approach may not scale to large, complicated applications due to its dependency on 
a custom constraint solver. Livshits \emph{et al}. instead use a dataflow analysis that
statically places sanitizers in existing code where possible \cite{Livshits:2013}. Where the 
type of the input cannot be determined statically, this approach must dynamically instrument
the program to determine the proper sanitizer outside of the program's existing logic. This
approach also cannot currently handle cases where inputs of different types are concatenated.

Both approaches support source sensitivity and sink sensitivity. \emph{Sink sensitivity} encodes user inputs based on the sink that consumes them, but cannot on its own allow different security policies to be enforced based on the source of input. \emph{Source sensitivity} allows for policies to be enforced based on the source of the input. 

We present a scalable approach that utilizes Type-Specific Languages (TSLs) \todo{cite ecoop paper} to allow developers using the Wyvern programming language \todo{cite something more for Wyvern?}
to safely splice user inputs into their webpages. TSLs allow the developer to
specify their webpages using a \todo{fill in a high level TSL and injection prevention description}.

Our approach makes the following contributions:

\todo{My wording here sucks...}
\begin{itemize}
\item We safely handle the concatenation of inputs of multiple types by requiring the developer to express their intent through statically checkable type conversion annotations. A failure to do so is a compile error.
\item We do not introduce dynamic checks into programs aside from those inherent in the program's structure at implementation time.
\item Existing programs developed using popular template systems may be integrated without change.
\item Our approach utilizes sink sensitivity by default. As a result, Wyvern applications implement the most secure tactic for preventing injection vulnerabilities in a given context unless the developer intentionally weakens the outcome. The ability to weaken the outcome is how we implement source sensitivity.
\end{itemize}
\subsection{Motivating Example}

\todo{Revise, rewrite, add.}

Web blogs are a convenient
example of such an application. A web blog, or simply a blog, is a
website that promotes discussion and information sharing on a specific
topic or a set of topics and that consists of discrete entries called posts.
Typically, a blog author will write posts by typing up the message
and formatting it in a text field that is available only to the blog's
authors. This message is later published for public consumption. The
post author formats their post by using HTML and CSS. However, the
post author does not need access to the full subset of HTML, and there
is in fact a lot of incentive for blog hosting providers (e.g. Blogger
\cite{blogger}) to limit the HTML/CSS tags that are available. In other words, sometimes the developer wants to override the default of non-executability, but control execution by limiting the set of tags that are interpreted by the browser.

Due to their relative simplicity and ubiquity, we use XSS and SQLi
vulnerabilities as the platform for illustrating our approach. The
approach generalizes to other web application injection vulnerabilities.
Figure \ref{fig:Motivating-Example-Code} shows a simple example that
contains an XSS and SQLi vulnerability in a PHP-like psuedocode. This
sample presents the user with a prompt to enter her username. When
the user enters her username and hits the submit button, a SQL query
is executed to get her last login time from a database table named
UserLog. Her last login time is then printed in a message below
the form, along with the username she entered. 

A SQLi vulnerability exists because the username the user entered
is concatenated into the SQL statement exactly as it was entered.
If the user had entered SQL code as their username instead, the extra
SQL statements they entered would be executed by the database. For
example, entering \texttt{';DROP TABLE UserLog; -{}- }as the username
would cause the UserLog table to be deleted from the database. The
attacker could have just as easily injected SQL to perform operations
such as adding fake data to the database, dumping the contents of
the database to try to steal password hashes, etc. 

The XSS vulnerability is equally simple.
The XSS vulnerability exists because the user name is being directly
written to a context the attacker can leverage to execute Javascript.
An attacker could enter the username \texttt{<script> alert(``XSS!''); </script>},
thus causing a Javascript pop-up to appear. The XSS exploit could
have also read the user's cookie and sent it to a different website,
defaced the website to turn it into a phishing form, or set-up a \texttt{document.onkeypress}
event handler to keylog the webpage. These exploits don't need to
be entered into the form directly.

The form uses the HTTP GET method, which allows the attacker to enter
his exploit into the URL using the \texttt{usr} parameter. As a
result, the attacker could craft a URL that contains the exploit and
trick a user into clicking the malicious link, thus causing the user's
browser to execute the exploit instead.


\begin{figure}
\begin{lstlisting}
let user_input_s = readline_user
let author_input_s = readline_author
let user_input : HTML = parse_user(user_input_s)
let author_input : HTML_A = parse_author(author_input_s)
let page : HTML = ~
   >html
      >title Best Author's Blog
      >body
      >h1 Post Title
      >div
         < admin_input : HTML_A (* htmla_to_html(author_input)*)
      >h2 Comments
      >div
         < user_input
      ... 
\end{lstlisting}
\caption{A Wyvern Example}
\label{fig:wyvern-example}
\end{figure}


\begin{figure}
\begin{lstlisting}
casetype HTML_A
   IFrame of HTMLA
   Div of HTMLA
   ...

casetype HTML
   Div of HTML
   ...
\end{lstlisting}
\caption{Wyvern HTML Type}
\label{fig:wyvern-example}
\end{figure}
