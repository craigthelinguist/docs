\documentclass{llncs}

\usepackage{listings}
\usepackage{amssymb}
\usepackage[margin=.9in]{geometry}
%\usepackage{amsmath}
%\usepackage{amsthm}
\usepackage{mathpartir}




\lstdefinestyle{custom_lang}{
  xleftmargin=\parindent,
  showstringspaces=false,
  basicstyle=\ttfamily,
  keywordstyle=\bfseries
}

\lstset{emph={%  
    var, def, type, new, z%
    },emphstyle={\bfseries \tt}%
}

\begin{document}





\section{Type Members}

\section{Counter Examples to Preservation}
	\label{s:examples}

\subsection{Term Membership Restriction}
	\label{s:term_mem}
Figure \ref{f:mem} gives the \emph{Membership} judgement. 
$\Gamma \vdash e \ni \sigma$ says that an expression $e$ 
has $\sigma_i$ as a member of its type in environment $\Gamma$. 
There are two rules for membership. By \textsc{M-Path} a 
variable $x$ has a member of type $[x/\texttt{z}]\sigma_i$ 
if $\sigma_i$ is a member of the type of $x$. By \textsc{M-Exp}
an expression $e$ has member $\sigma_i$ if $\sigma_i$ is 
a member of $e$'s type, and \texttt{z} does not occur 
within $\sigma_i$.

This is reasonable since we cannot substitute a non-value 
expression into a selection type such as $\texttt{z}.L$. 
This does however present a counter-example to preservation. 
Given two types $X$ and $Y$,

\begin{lstlisting}[mathescape, style=custom_lang]
Y = {z $\Rightarrow$
     var l : $\top$
     def m : Y(y:Y){
       var a = new {};
       y
     }
    }
\end{lstlisting}
\begin{lstlisting}[mathescape, style=custom_lang]
X = {z $\Rightarrow$
     type L : $\top$ .. $\top$
     var l : z.L
     def m : Y(y:Y){
       y
     }
    }
\end{lstlisting}

the following expression is well typed.
\begin{lstlisting}[mathescape, style=custom_lang]
var x = new X(l = z);
var y = new Y(l = (var z = new {}; z));
y.m(x).l
\end{lstlisting}
We can see that this is well-typed, and in particular that membership 
holds for \texttt{y.m(x)} and \texttt{l}. This reduces to 
\begin{lstlisting}[mathescape, style=custom_lang]
(var a = new {}; x).l
\end{lstlisting}
which is not well-typed since \texttt{var a = new {}; x} has type 
\texttt{X}, and \texttt{X.l} contains a \texttt{z} reference 
\texttt{z.L}.

After patching we allow expressions to maintain original type. 
Expressions are then typed with respect to the original type.
This is given in Figure \ref{f:syntax} as $e \unlhd T$ where 
$T$ is the original type. The patched reduction is given in 
Figure \ref{f:red}. The amended method reduction \textsc{R-Meth} is 
shown below.
\begin{mathpar}
\inferrule
  {\mu(path(\mu, v_1)) = \{\texttt{z} \Rightarrow ...,m:T(x:S)=e,...\}}
  {\mu \; | \; v_1.m(v_2) \;\rightarrow \mu \; | \; [v_1/\texttt{z},v_2 \unlhd S/x]e \unlhd T}
  \quad (\textsc {R-Meth})
\end{mathpar}
This is very similar to the standard method reduction. The two main 
differences are the introduction of the $path$ function, and the 
inclusion of the original types in the returned expression. For now 
ignore the $path$ function, this is explained in \ref{s:patheq}.

We retrieve the method for the receiver from the store, and substitute 
the method parameter into the body. The method parameter retains it's 
original type ($v_2 \unlhd S$) as does the entire returned body 
$[v_1/\texttt{z},v_2 \unlhd S/x]e \unlhd T$. Using this rule we 
can attempt to evaluate our original example.
\begin{lstlisting}[mathescape, style=custom_lang]
var x = new X(l = z);
var y = new Y(l = (var z = new {}; z));
y.m(x).l
\end{lstlisting}
reduces to
\begin{lstlisting}[mathescape, style=custom_lang]
((var a = new {}; (x $\unlhd$ Y))$\unlhd$ Y).l
\end{lstlisting}
Now we can treat the method body as having type \texttt{Y} and 
we can determine membership of \texttt{l} for \texttt{Y} rather 
than \texttt{X}.

\subsection{Expansion Lost}
For preservation to hold, we need to ensure that types are expandable 
to lists of declarations. This is captured in Figure \ref{f:wfe}. It is 
possible for expansion to be lost during environment narrowing. This is 
shown in \cite{Amin:2012}, but is briefly covered again below.
\begin{lstlisting}[mathescape, style=custom_lang]
X = {z $\Rightarrow$
     type A : $\bot$ .. z.B
     type B : $\bot$ .. $\top$
    }
Y = {z $\Rightarrow$
     type A : $\bot$ .. $\top$
     type B : $\bot$ .. z.A
    }
\end{lstlisting}
While both these types are expandable, their union is not.
\begin{lstlisting}[mathescape, style=custom_lang]
X $\wedge$ Y = {z $\Rightarrow$
         type A : $\bot$ .. z.B
         type B : $\bot$ .. z.A
        }
\end{lstlisting}
While we could not type this normally, we can create this 
scenario during environmental narrowing that results in 
an a type without an expansion, providing a counter-example 
to preservation.
\begin{lstlisting}[mathescape, style=custom_lang]
W = {z $\Rightarrow$
     type A : $\bot$ .. $\top$
     type B : $\bot$ .. $\top$
    }
\end{lstlisting}
Looking at the following expression, we can see it is well-type, 
and the type \texttt{i.L} $\wedge$ \texttt{x.K} in particular is 
well-formed and expanding.
\begin{lstlisting}[mathescape, style=custom_lang]
var x = new {z $\Rightarrow$ type K : $\bot$ .. X};
var y = new {z $\Rightarrow$ type L : $\bot$ .. Y};
var w = new {z $\Rightarrow$ type L : $\bot$ .. W};
var c = new {z $\Rightarrow$
         def meth1 : $\top$ (i : {z $\Rightarrow$ type L : $\bot$ .. W}){
           var d = new {z $\Rightarrow$
                        def meth2 : $\top$ (j : i.L $\wedge$ x.K){
                          j
                        }
                       };
           d
         }
        };
c.meth1(y)
\end{lstlisting}
Evaluating the above expression results in the following.
\begin{lstlisting}[mathescape, style=custom_lang]
var d = new {z $\Rightarrow$
             def meth2 : $\top$ (j : y.L $\wedge$ x.K){
               j
             }
            };
d
\end{lstlisting}
Now the type \texttt{y.L} $\wedge$ \texttt{x.K} gives us the loss of 
expansion we were attempting to construct. In our new patched calculus, 
this problem does not occur because the original type of the \texttt{i} 
parameter is maintained. So if we re-evaluate the original expression 
with our type system, we get the following.
\begin{lstlisting}[mathescape, style=custom_lang]
var d = new {z $\Rightarrow$
             def meth2 : $\top$ (j : (y $\unlhd$ {z $\Rightarrow$ type L : $\bot$ .. W}).L $\wedge$ x.K){
               j
             }
            };
d
\end{lstlisting}
The type \texttt{(y} $\unlhd$ \texttt{\{z} $\Rightarrow$ 
\texttt{type L :} $\bot$ \texttt{.. W\}).L} $\wedge$ \texttt{x.K} is now 
both well-formed and expanding because it is the original type 
from our original expression.



\subsection{Well-Formedness Lost}



\subsection{Path Equality}
\label{s:patheq}
The example below illustrates the problem with path equality.
\begin{lstlisting}[mathescape, style=custom_lang]
var b = new {z $\Rightarrow$
             type L : $\top$ .. $\top$
             var l : z.L = b};
var a = new {z $\Rightarrow$
             var i : {z $\Rightarrow$
                      type L : $\bot$ .. $\top$
                      var l : z.L} = b
             def meth : $\top$ (x : z.i.L){x}};
a.meth(a.i.l)
\end{lstlisting}
\texttt{a.i.l} reduces to \texttt{b.l} which has type \texttt{b.L}. 
There is no way for us to ensure that \texttt{b.L} <: \texttt{a.i.L}.
This presents a path equality problem for preservation.

Typing paths correctly requires that we maintain some information 
about the original paths. For this reason we don't evaluate paths 
in the calculus. We still however need to retrieve the correct object 
the path is pointing to during method reduction and object initialization. 
This is done by the \emph{path} function given in Figure \ref{f:path}. 
With a path and a store, we can find the object referenced by that path.
Below is the evaluation of the above example without evaluating paths.
\begin{lstlisting}[mathescape, style=custom_lang]
var b = new {z $\Rightarrow$
             type L : $\top$ .. $\top$
             var l : z.L = b};
var a = new {z $\Rightarrow$
             var i : {z $\Rightarrow$
                      type L : $\bot$ .. $\top$
                      var l : z.L} = b
             def meth : $\top$ (x : z.i.L){x}};
a.meth(a.i.l)
\end{lstlisting}
Reduces to \texttt{a.meth(a.i.l)} which type checks. \texttt{a.meth(a.i.l)}
reduces to \texttt{a.i.l}

\subsection{Path Equality Redux}
\begin{lstlisting}[mathescape, style=custom_lang]
var n = new {z $\Rightarrow$ type X : $\top$ .. $\top$};
var a = new {z $\Rightarrow$
             type Y : {z $\Rightarrow$ type X : $\bot$ .. $\top$} .. {z $\Rightarrow$ type X : $\bot$ .. $\top$}
             var i : z.Y = n};
var b = new {z $\Rightarrow$
             type L : $\bot$ .. {z $\Rightarrow$ type Y : {z $\Rightarrow$ type X : $\bot$ .. $\top$} .. 
                                          {z $\Rightarrow$ type X : $\bot$ .. $\top$}
                                 var i : z.Y}
             var j : z.L = a};
var c = new {z $\Rightarrow$
             var k : {z $\Rightarrow$
                      type L : $\bot$ .. {z $\Rightarrow$ type Y : {z $\Rightarrow$ type X : $\bot$ .. $\top$} .. 
                                                   {z $\Rightarrow$ type X : $\bot$ .. $\top$}
                                          var i : z.Y}
                      var j : z.L} = b
             def m : $\top$ (x : z.k.L){
                 var y = new {z $\Rightarrow$ var l : x.Y = x.i 
                            var m : x.i.X = new {z $\Rightarrow$}}; y
             }};
c.m(b.j)
\end{lstlisting}
%\begin{lstlisting}[mathescape, style=custom_lang]
%c.m(c.k.j) $\rightarrow$ c.m(b.j)
%\end{lstlisting} 
\texttt{b.j} has type \texttt{b.L}, therefore we need 
$\texttt{b.L} <: \texttt{c.k.L}$
\begin{lstlisting}[mathescape, style=custom_lang]
c.m(b.j) $\rightarrow$ c.m(a)
c.m(a) $\rightarrow$ (var y = new {z $\Rightarrow$ var l : (a $\unlhd$ c.k.L).Y = (a $\unlhd$ c.k.L).i 
                            var m : (a $\unlhd$ c.k.L).i.X = new {z $\Rightarrow$}}; y) $\unlhd$ $\top$
\end{lstlisting}

%\subsection{Patched Counter Examples}
%\subsubsection{Term Membership Restriction}
%\subsubsection{Expansion Lost}
%\subsubsection{Well-formedness Lost}
%\subsubsection{Path Equality}
\section{Type System}
	\label{s:type_sys}


\subsection{Syntax}


\begin{figure}[h]
\[
\begin{array}{lll}
\begin{array}{lllr}
e & ::= & x & expression \\
& | & \texttt{var} \; x = \texttt{new} \; \{\texttt{z} \Rightarrow \overline{d}\};&\\
%& | & \lambda x:T = e\\
%& | & e(e)\\
& | & e.m(e) &\\
& | & e.f &\\
%& | & e.f = e&\\
& | & e \unlhd T&\\
%& | & l &\\
&&\\
p & ::= & x & paths \\
& | & p.f &\\
& | & p \unlhd T &\\
&&\\
v & ::= & x & value \\
& | & v.f &\\
& | & v \unlhd T &\\
&&\\
d & ::= & \texttt{var} \; f : T = x & declaration \\
  & |   & \texttt{def} \; m : T(x:T) = e &\\
  & |   & \texttt{type} \; L : T .. T = T &\\
&&\\
d_v & ::= & \texttt{var} \; f : T = x & declaration \; value \\
  & |   & \texttt{def} \; m : T(x:T) = e &\\
  & |   & \texttt{type} \; L : T .. T = T &\\
&&\\
%p & ::= & x & path \\
%  & | & l & \\
&&\\
\Gamma & :: = & \varnothing \; | \; \Gamma, \; x : T & context \\
&&\\
\mu & :: = & \varnothing \; | \; \mu,\; x \mapsto \{\texttt{z} \Rightarrow \overline{d}\} & store \\
\end{array}
& ~~~~~~
&
\begin{array}{lllr}
%T & ::= & \{\texttt{z} \Rightarrow \overline{\sigma}\} & type \\
T & ::= & \{\texttt{z} \Rightarrow \overline{\sigma}\} & type \\
& | & p.L &\\
& | & T \wedge T & \\
& | & \top & \\
& | & \bot & \\
&&\\
\sigma & ::= & \texttt{var} \; f:T & decl \; type\\
       & |   & \texttt{def} \; m:T \rightarrow T \\
		 & |   & \texttt{type} \; L : T .. T &\\
&&\\
E & :: = & \bigcirc & eval \; context\\
       & | & E.m(e)\\
       & | & v.m(E)\\
       & | & E.f\\
%       & | & E.f = e\\
%       & | & v.f = E\\
       & | & E \unlhd T\\
       & | & \texttt{var} \; x = \texttt{new} \; \{\texttt{z} \Rightarrow \overline{D}\}; \; e & eval \; context \\
       & | & \texttt{var} \; x = \texttt{new} \; \{\texttt{z} \Rightarrow \overline{d_v}\}; \; E& \\
&&\\
D & :: = & \texttt{var} \; f : T = E & decl \; eval \; context \\
  & |   & \texttt{def} \; m : T(x:T) = e &\\
  & |   & \texttt{type} \; L : T .. T = T &\\
%&&\\
%K & :: = & \varnothing \; | \; K, \; E & evaluation \; stack \\
%\Delta & :: = & \varnothing \; | \; \Delta,\; x \mapsto l & substitution \; context \\
%&&\\
%\Sigma & :: = & \varnothing \; | \; \Sigma,\; l : T & store \; context \\
\end{array}
\end{array}
\]
\caption{Syntax}
\label{f:syntax}
\end{figure}


\begin{figure}[h]
\begin{mathpar}
\inferrule
  {}
  {path(\mu, x) = \mu(x)}
	\and
\inferrule
  {}
  {path(\mu, p \unlhd T) = path(\mu, p)}
	\and
\inferrule
  {\mu(path(\mu, p)) = \{z \Rightarrow \overline{d}\} \\
	\texttt{var} \; f : T = x \in \overline{d}}
  {path(\mu, p.f) = x}
\end{mathpar}
\caption{Path Function}
\label{f:path}
\end{figure}
%\begin{figure}[h]
%\begin{mathpar}
%\inferrule
%  {}
%  {narrow(x) = x \\ narrow(v \unlhd T) = narrow(v)}
%\end{mathpar}
%\caption{Narrow Function}
%\label{f:narrow}
%\end{figure}

\subsection{Semantics}
%\subsubsection{Subtyping}

\begin{figure}[h]
\hfill \fbox{$A; \Sigma; \Gamma_1 \vdash S <: T \dashv \Gamma_2$}
\begin{mathpar}
\inferrule
  {}
  {A; \Sigma; \Gamma \vdash T\; \texttt{<:}\; T \dashv \Gamma}
  \quad (\textsc {S-Refl})
	\and
\inferrule
  {(S <: T) \in A}
  {A; \Sigma; \Gamma_1 \vdash S\; \texttt{<:}\; T \dashv \Gamma_2}
  \quad (\textsc {S-Assume})
	\and
\inferrule
  {A' = A, (\{z \Rightarrow \overline{\sigma}\} <: \{z \Rightarrow \overline{\sigma}'\}) \\
  	A'; \Sigma; \Gamma_1, z : \{z \Rightarrow \overline{\sigma}\} \vdash \overline{\sigma} <:\; \overline{\sigma}'  \dashv \Gamma_2, z : \{z \Rightarrow \overline{\sigma}'\}}
  {A; \Sigma; \Gamma_1 \vdash \{z \Rightarrow \overline{\sigma}\}\; <:\; \{z \Rightarrow \overline{\sigma}'\}\dashv \Gamma_2}
  \quad (\textsc {S-Rec})
	\and
\inferrule
  {A; \Sigma; \Gamma_1 \vdash p \ni \texttt{type} \; L : S_1 .. U_1\\
  	A; \Sigma; \Gamma_2 \vdash p \ni \texttt{type} \; L : S_2 .. U_2\\
  	A; \Sigma; \Gamma_2 \vdash S_2 \; \texttt{<:}\; S_1 \dashv \Gamma_1 \\
  	A; \Sigma; \Gamma_1 \vdash U_1 \; \texttt{<:}\; U_2 \dashv \Gamma_2}
  {A; \Sigma; \Gamma_1 \vdash p.L \; \texttt{<:}\; p.L \dashv \Gamma_2}
  \quad (\textsc {S-Select-Refl})
	\and
\inferrule
  {A; \Sigma; \Gamma_1 \vdash p \ni \texttt{type} \; L : S .. U\\
  	A; \Sigma; \Gamma_1 \vdash S <: U \dashv \Gamma_1 \\
  	A; \Sigma; \Gamma_1 \vdash U <: U' \dashv \Gamma_2}
  {A; \Sigma; \Gamma_1 \vdash p.L\; <:\; U' \dashv \Gamma_2}
  \quad (\textsc {S-Select-Upper})
	\and
\inferrule
  {A; \Sigma; \Gamma_2 \vdash p \ni \texttt{type} \; L : S .. U \\
  	A; \Sigma; \Gamma_2 \vdash S <: U \dashv \Gamma_2 \\
  	A; \Sigma; \Gamma_1 \vdash S' <: S \dashv \Gamma_2}
  {A; \Sigma; \Gamma_1 \vdash S'\; <:\; p.L \dashv \Gamma_2}
  \quad (\textsc {S-Select-Lower})
	\and
\inferrule
  {}
  {A; \Sigma; \Gamma_1 \vdash T\; \texttt{<:}\; \top \dashv \Gamma_2}
  \quad (\textsc {S-Top})
	\and
\inferrule
  {}
  {A; \Sigma; \Gamma_1 \vdash \bot\; \texttt{<:}\; T \dashv \Gamma_2}
  \quad (\textsc {S-Bottom})
\end{mathpar}
\hfill \fbox{$A; \Sigma; \Gamma_1 \vdash \sigma <: \sigma' \dashv \Gamma_2$}
\begin{mathpar}
\inferrule
  {}
  {A; \Sigma; \Gamma_1 \vdash \texttt{val} \; f:T <: \texttt{val} \; f:T \dashv \Gamma_2}
  \quad (\textsc {S-Decl-Val})
	\and
\inferrule
  {A; \Sigma; \Gamma_2 \vdash S' <: S \dashv \Gamma_1 \\
  	A; \Sigma; \Gamma_1 \vdash T <: T' \dashv \Gamma_2}
  {A; \Sigma; \Gamma_1 \vdash \texttt{def} \; m:S \rightarrow T <: \texttt{def} \; m:S' \rightarrow T' \dashv \Gamma_2}
  \quad (\textsc {S-Decl-Def})
	\and
\inferrule
  {A; \Sigma; \Gamma_2 \vdash S' <: S \dashv \Gamma_1 \\
  	A; \Sigma; \Gamma_1 \vdash U <: U' \dashv \Gamma_2}
  {A; \Sigma; \Gamma_1 \vdash \texttt{type} \; L : S .. U \; <:\; \texttt{type} \; L : S' .. U' \dashv \Gamma_2}
  \quad (\textsc {S-Decl-Type})
\end{mathpar}
\caption{Subtyping}
\label{f:subtype}
\end{figure}

\begin{figure}[h]
\hfill \fbox{$A; \Sigma; \Gamma \vdash T \;  \textbf{wf}$}
\begin{mathpar}
\inferrule
  {A; \Sigma; \Gamma \vdash p \ni \texttt{type} \; L : S .. U \\
  	A; \Sigma; \Gamma \vdash \texttt{type} \; L : S .. U \; \textbf{wf} }
  {A; \Sigma; \Gamma \vdash p.L \; \textbf{wf}}
  \quad (\textsc {WF-Sel})
	\and
\inferrule
  {A; \Sigma; \Gamma,z:\{z \Rightarrow \overline{\sigma}\} \vdash \overline{\sigma} \; \textbf{wf} \\
  	\forall j \neq i, \; dom(\sigma_j) \neq dom(\sigma_i)}
  {A; \Gamma; \Sigma \vdash \{z \Rightarrow \overline{\sigma}\} \; \textbf{wf}}
  \quad (\textsc {WF-Rec})
%	\and
%\inferrule
%  {\Gamma \vdash S \;  \textbf{wf} \\
%  	\Gamma \vdash T \;  \textbf{wf}}
%  {\Gamma \vdash S \rightarrow T \;  \textbf{wf}}
%  \quad (\textsc {WF-Func})
	\and
\inferrule
  {}
  {A; \Sigma; \Gamma \vdash \top \;  \textbf{wf}}
  \quad (\textsc {WF-Top})
	\and
\inferrule
  {}
  {A; \Sigma; \Gamma \vdash \bot \;  \textbf{wf}}
  \quad (\textsc {WF-Bot})
\end{mathpar}
\hfill \fbox{$A; \Sigma; \Gamma \vdash \sigma \;  \textbf{wf}$}
\begin{mathpar}
\inferrule
  {A; \Sigma; \Gamma \vdash T : \textbf{wf}}
  {A; \Sigma; \Gamma \vdash \texttt{val} \; f:T \;  \textbf{wf}}
  \quad (\textsc {WF-Val})
	\and
\inferrule
  {A; \Sigma; \Gamma \vdash T : \textbf{wf} \\
  	A; \Sigma; \Gamma \vdash S : \textbf{wf}}
  {A; \Sigma; \Gamma \vdash \texttt{def} \; m:S \rightarrow T \;  \textbf{wf}}
  \quad (\textsc {WF-Def})
	\and
\inferrule
  {A; \Sigma; \Gamma \vdash S : \textbf{wfe} \; \vee \; S = \bot\\
  	A; \Sigma; \Gamma \vdash U : \textbf{wfe} \\
  	A; \Sigma; \Gamma \vdash S <: U}
  {A; \Sigma; \Gamma \vdash \texttt{type} \; L : S .. U \; \textbf{wf}}
  \quad (\textsc {WF-Type})
\end{mathpar}
\hfill \fbox{$A; \Sigma \vdash \Gamma \;  \textbf{wf}$}
\begin{mathpar}
\inferrule
  {\forall x \in dom(\Gamma), A; \Sigma; \Gamma \vdash \Gamma(x) \; \textbf{wf}}
  {\Sigma \vdash \Gamma \;  \textbf{wf}}
  \quad (\textsc {WF-Environment})
\end{mathpar}
\hfill \fbox{$\Sigma \;  \textbf{wf}$}
\begin{mathpar}
\inferrule
  {\forall l \in dom(\Sigma), \varnothing; \Sigma; \varnothing \vdash \Sigma(l) \; \textbf{wf}}
  {\Sigma \;  \textbf{wf}}
  \quad (\textsc {WF-Store-Context})
\end{mathpar}
\begin{mathpar}
\inferrule
  {\forall l \in dom(\mu), \varnothing; \Sigma; \varnothing \vdash \mu(l) : \Sigma(l)}
  {\mu : \Sigma}
  \quad (\textsc {WF-Store})
\end{mathpar}
\caption{Well-Formedness}
\label{f:wf}
\end{figure}

\begin{figure}[h]
\hfill \fbox{$A; \Sigma; \Gamma \vdash T \;  \textbf{wfe}$}
\begin{mathpar}
\inferrule
  {A; \Sigma; \Gamma \vdash T \; \textbf{wf} \\
  	A; \Sigma; \Gamma \vdash T \prec \overline{\sigma}}
  {A; \Sigma; \Gamma \vdash T \; \textbf{wfe}}
  \quad (\textsc {WFE})
\end{mathpar}
\caption{Well-Formed and Expanding Types}
\label{f:wfe}
\end{figure}

\begin{figure}[h]
\hfill \fbox{$\Sigma; \Gamma \vdash T \prec \overline{\sigma}$}
\begin{mathpar}
\inferrule
  {}
  {\Sigma; \Gamma \vdash \{z \Rightarrow \overline{\sigma}\} \prec_z \overline{\sigma}}
  \quad (\textsc {E-Rec})
	\and
\inferrule
  {\Sigma; \Gamma \vdash p \ni \texttt{type} \; L : S..U \\
  	\Sigma; \Gamma \vdash U \prec_z \overline{\sigma}}
  {\Sigma; \Gamma \vdash p.L \prec_z \overline{\sigma}}
  \quad (\textsc {E-Sel})
	\and
\inferrule
  {}
  {\Sigma; \Gamma \vdash \top \prec_z \varnothing}
  \quad (\textsc {E-Top})
\end{mathpar}
\caption{Expansion}
\label{f:exp}
\end{figure}
\begin{figure}[h]
\hfill \fbox{$\Sigma; \Gamma \vdash e \ni \sigma$}
\begin{mathpar}
\inferrule
  {\Sigma; \Gamma \vdash p : T \\
  	\Sigma; \Gamma \vdash T \prec_z \overline{\sigma}\\
  	\sigma_i \in \overline{\sigma}}
  {\Sigma; \Gamma \vdash p \ni [p/z]\sigma_i}
  \quad (\textsc {M-Path})
	\and
\inferrule
  {\Sigma; \Gamma \vdash e : T \\
  	\Sigma; \Gamma \vdash T \prec_z \overline{\sigma}\\
  	\sigma_i \in \overline{\sigma} \\
  	z \notin \sigma_i}
  {\Sigma; \Gamma \vdash e \ni \sigma_i}
  \quad (\textsc {M-Exp})
\end{mathpar}
\caption{Membership}
\label{f:mem}
\end{figure}

%\subsubsection{Typing}

\begin{figure}[h]
\hfill \fbox{$\Gamma; \Sigma \vdash e:T$}
\begin{mathpar}
\inferrule
  {x \in dom(\Gamma)}
  {	\Gamma; \Sigma \vdash x : \Gamma(x)}
  \quad (\textsc {T-Var})
	\and
\inferrule
  {	l \in dom(\Sigma)}
  {	\Gamma; \Sigma \vdash l : \Sigma(l)}
  \quad (\textsc {T-Loc})
	\and
\inferrule
  {\Gamma, z : \{z \Rightarrow \overline{\sigma}\}; \Sigma 
  \vdash \overline{d} : \overline{\sigma} \\
  	y \notin dom(\Gamma)}
  {	\Gamma; \Sigma\vdash \texttt{new} \; \{z \Rightarrow \overline{d}\} : 
  \{z \Rightarrow \overline{\sigma}\}}
  \quad (\textsc {T-New})
	\and
\inferrule
  {\Gamma; \Sigma \vdash e_0 \ni \texttt{def} \; m:S \rightarrow T \\
  	\Gamma; \Sigma \vdash e_1 : S \\
  	\Gamma; \Sigma \vdash T <: U}
  {	\Gamma; \Sigma \vdash e_0.m_U(e_1) : U}
  \quad (\textsc {T-Meth})
	\and
\inferrule
  {	\Gamma; \Sigma \vdash e : S \\
  	\Gamma; \Sigma \vdash e \ni \texttt{val} \; f:T}
  {	\Gamma; \Sigma \vdash e.f : T}
  \quad (\textsc {T-Acc})
%	\and
%\inferrule
%  {	\Gamma; E \vdash e_0 : T_0 \\
%  	T_0 \ni \texttt{def} \; f:T = e \\
%  	\Gamma; E \vdash e_1 : T}
%  {	\Gamma; E \vdash e_0.f = e_1 : T}
%  \quad (\textsc {T-Assign})
%	\and
	\and
\inferrule
  {	\Gamma; \Sigma \vdash e : T}
  {	\Gamma; \Sigma \vdash e \unlhd T : T}
  \quad (\textsc {T-Type})
	\and
\inferrule
  {	\Gamma; \Sigma \vdash e : S \\
  	\Gamma; \Sigma \vdash S <: T}
  {	\Gamma; \Sigma \vdash e : T}
  \quad (\textsc {T-Sub})
\end{mathpar}
\caption{Expression Typing}
\label{f:e_typ}
\end{figure}
\begin{figure}[h]
\hfill \fbox{$\Gamma; E \vdash d:\sigma$}
\begin{mathpar}
\inferrule
  {\Gamma; \Sigma \vdash e : T}
  {\Gamma; \Sigma \vdash \texttt{def} \; f:T = e : \texttt{def} \; f:T}
  \quad (\textsc {T-Decl-Var})
	\and
\inferrule
  {\Gamma, x : S; \Sigma \vdash e_0 : T}
  {\Gamma; \Sigma \vdash \texttt{def} \; m(x:S) = e : T : \texttt{def} \; m:S \rightarrow T}
  \quad (\textsc {T-Decl-Def})
	\and
\inferrule
  {\Gamma; \Sigma \vdash \texttt{type} \; L : S .. U \; \textbf{wf} }
  {\Gamma; \Sigma \vdash \texttt{type} \; L : S .. U : \texttt{type} \; L : S .. U}
  \quad (\textsc {T-Decl-Type})
\end{mathpar}
\caption{Declaration Typing}
\label{f:d_typ}
\end{figure}
%\begin{figure}[h]
%\hfill \fbox{$\Gamma \vdash \mu:\Sigma$}
%\begin{mathpar}
%\inferrule
%  {\forall x \in dom(\mu), \; \mu(x)= \{\texttt{z} \Rightarrow \overline{d}\} \\
%  	\Gamma(x) = \{\texttt{z} \Rightarrow \overline{\sigma}\} \\
%  	\Gamma \vdash \overline{d} : \overline{\sigma}}
%  {\Gamma \vdash \mu}
%  \quad (\textsc {T-Store})
%\end{mathpar}
%\caption{Store Typing}
%\label{f:s_typ}
%\end{figure}

\begin{figure}[h]
\hfill \fbox{$\mu \; | \; e \; \rightarrow \mu' \; | \; e'$}
\begin{mathpar}
\inferrule
  {l \notin dom(\mu) \\
  	\mu' = \mu, l \mapsto \{\texttt{z} \Rightarrow \overline{d_v}\}}
  {\mu \; | \; \texttt{new} \; \{\texttt{z} \Rightarrow \overline{d_v}\} \; \rightarrow \mu' \; | \; l}
  \quad (\textsc {R-New})
  \and
\inferrule
  {\mu : \Sigma \\
   \mu; \Sigma \vdash v_1 \leadsto_{m} e}
  {\mu \; | \; v_1.m(v_2) \;\rightarrow \mu \; | \; [l/\texttt{z},v_2 \unlhd S/x]e}
  \quad (\textsc {R-Meth})
  \and
\inferrule
  {	\mu \; | \; e \; \rightarrow \; \mu' \; | \; e'}
  {\mu \; | \; E[e] \; \rightarrow \mu' \; | \; E[e']}
  \quad (\textsc {R-Context})
\end{mathpar}
\caption{Reduction}
\label{f:red}
\end{figure}


\section{Type Safety}

\begin{lemma}[Subtype Definition]\label{lem:sub}
If $\Gamma \vdash S \prec_z \overline{\sigma_1}$,
	$\Gamma \vdash T \prec_z \overline{\sigma_2}$ and
	$\forall \sigma_i \in \overline{\sigma_2} \;
	 \exists \sigma_j \in \overline{\sigma_1}:
	 \Gamma, z : \{z \Rightarrow \overline{\sigma_1}\} 
	 \vdash \sigma_j <: \sigma_i$ then
	$\Gamma \vdash S <: T$
\end{lemma}
\begin{proof}
Can't be shown.
\end{proof}

%\begin{theorem}[Subtype Reflexivity]
% 	If $\Gamma \vdash T \; \textbf{\tt{wfe}}$ then $\Gamma \vdash T <: T$
%\end{theorem}
%\begin{proof}
%This is easy to demonstrate using \textsc{S-Struct}.
%\end{proof}

\begin{lemma}[Type Equality]\label{lem:sub_eq}
If $\Gamma \vdash S <: T$ and
	$\Gamma \vdash T <: S$ then
	$S = T$
\end{lemma}
\begin{proof}
By induction on the derivation of $\Gamma \vdash S <: T$.
\begin{case}[\textsc{S-Refl}]
The result follows immediately. 
\end{case}
\begin{case}[\textsc{S-Rec}]
\begin{mathpar}
\inferrule
  {S = \{z \Rightarrow \overline{\sigma}\} \\
	T = \{z \Rightarrow \overline{\sigma'}\} \\
	\forall \sigma_i' \in \overline{\sigma'}, \; \exists \; \sigma_i \in \overline{\sigma} \; : \; \Gamma, z : \{z \Rightarrow \overline{\sigma}\} \vdash \sigma_i <:\; \sigma_i'}
  {}
\end{mathpar}
Since $\Gamma \vdash \{z \Rightarrow \overline{\sigma}\} <: \{z \Rightarrow \overline{\sigma'}\}$, by case analysis on this 
derivation, it follows that either $S = T$ by \textsc{S-Refl} in which 
the result is immediate, or by \textsc{S-Rec} the following holds.
\begin{mathpar}
\inferrule
  {\forall \sigma_i \in \overline{\sigma}, \; \exists \; \sigma_i' \in \overline{\sigma'} \; : \; \Gamma, z : \{z \Rightarrow \overline{\sigma'}\} \vdash \sigma'_i <:\; \sigma_i}
  {}
\end{mathpar}
We can now apply our induction hypothesis to each $\sigma_i$ 
and $\sigma_i'$ pair, and infer that $\sigma_i = \sigma_i'$ 
%We can further say that the subtype relation is bijective from 
%$\overline{\sigma}$ to $\overline{\sigma'}$ due to the uniqueness 
%of declaration label names
and consequently that $\overline{\sigma} = \overline{\sigma'}$. 
i.e. that $\{z \Rightarrow \overline{\sigma}\} = \{z \Rightarrow \overline{\sigma'}\}$,
or $S = T$.
\end{case}
\begin{case}[\textsc{S-Select-Upper}]
\begin{mathpar}
\inferrule
  {S = p.L \\
	\Gamma \vdash p \ni \texttt{type} \; L : S' .. U'\\
  	\Gamma \vdash S' <: U' \\
  	\Gamma \vdash U' <: T}
  {}
\end{mathpar}
\end{case}
\begin{case}[\textsc{S-Selcet-Lower}]
\begin{mathpar}
\inferrule
  {T = p.L \\
	\Gamma \vdash p \ni \texttt{type} \; L : S' .. U'\\
  	\Gamma \vdash S' <: U' \\
  	\Gamma \vdash S <: S'}
  {}
\end{mathpar}
\end{case}
\begin{case}[\textsc{S-Top}]
\begin{mathpar}
\inferrule
  {T = \top}
  {}
\end{mathpar}
We need to know show that $S = \top$, or otherwise forms a 
contradiction. Case analysis on the derivation of $\Gamma \vdash T <: S$.
\begin{itemize}
\item[]  \textit{Subcase 1} (\textsc{S-Refl}): 
Trivial.
\\
\item[]  \textit{Subcase 2} (\textsc{S-Selecet-Lower}):
\begin{mathpar}
\inferrule
  {S = p.L\\
	\Gamma \vdash p \ni \texttt{type} \; L : S' .. U'\\
  	\Gamma \vdash S' <: U' \\
  	\Gamma \vdash \top <: S'}
  {}
\end{mathpar} 
\\
\item[]  \textit{Subcase 3} (\textsc{S-Top}):
Trivial.
\\
\end{itemize}
\end{case}
\begin{case}[\textsc{S-Bottom}]
\begin{mathpar}
\inferrule
  {S = \bot \\}
  {}
\end{mathpar}
\end{case}

\end{proof}

\begin{theorem}[Subtype Transitivity]
If $\Gamma \vdash S, T, U \; \textbf{\tt{wfe}}$,
	$\Gamma \vdash S <: T$ and 
   	$\Gamma \vdash T <: U$ then
	$\Gamma \vdash S <: U$
\end{theorem}
\begin{proof}
Proceed by induction on the derivation of $\Gamma \vdash S <: T$.
\begin{case}[\textsc{S-Refl}]
\begin{mathpar}
\inferrule
  {S = T \\
	\Gamma \vdash T\; \texttt{<:}\; U}
  {}
\end{mathpar}
The result follows from our initial assumptions.
\end{case}
\begin{case}[\textsc{S-Rec}]
\begin{mathpar}
\inferrule
  {S = \{z \Rightarrow \overline{\sigma}\} \\
  	T = \{z \Rightarrow \overline{\sigma'}\} \\
  	\forall \sigma'_i \in \overline{\sigma'}, \; 
  	\exists \sigma_i \in \overline{\sigma}, \; s.t. \;
  	\Gamma \vdash \sigma_i \; <: \; \sigma'_i \\
	\Gamma \vdash T <: U }
  {}
\end{mathpar}
With our induction hypothesis:
\begin{mathpar}
\inferrule
  {\forall \sigma, \; \Gamma \vdash \sigma'_i <: \sigma \; \Rightarrow
	\Gamma \vdash \sigma_i <: \sigma}
  {}
\end{mathpar}
Case analysis on the derivation of $\Gamma \vdash T <: U$.
\begin{itemize}
\item[]  \textit{Subcase 1} (\textsc{S-Refl}): 
\begin{mathpar}
\inferrule
  {U = T = \{z \Rightarrow \overline{\sigma'}\}}
  {}
\end{mathpar}
The result follows from our initial assumptions.
\\
\item[]  \textit{Subcase 2} (\textsc{S-Rec}): 
\begin{mathpar}
\inferrule
  {U = \{z \Rightarrow \overline{\sigma''}\} \\
  	\forall \sigma''_i \in \overline{\sigma''}, \; 
  	\exists \sigma'_i \in \overline{\sigma'}, \; s.t. \;
  	\Gamma \vdash \sigma'_i \; <: \; \sigma''_i}
  {}
\end{mathpar}
Let $\sigma''_i \in \overline{\sigma''}$. 
It follows from above that there exists a 
$\sigma'_i \in \overline{\sigma'}$ such that 
$\Gamma \vdash \sigma'_i \; <: \; \sigma''_i$.
By our original assumption there exists some 
$\sigma_i \in \overline{\sigma}$ such that 
$\Gamma \vdash \sigma_i \; <: \; \sigma'_i$. 
It follows by our induction hypothesis that 
$\Gamma \vdash \sigma_i \; <: \; \sigma''_i$. 
Therefore by \textsc{S-Rec} $\Gamma \vdash S <: U$.
\\
\item[]  \textit{Subcase 3} (\textsc{S-Select-Lower}): 
\begin{mathpar}
\inferrule
  {U = p.L \\
	\Gamma \vdash p \ni \texttt{type} \; L : S' .. U' \\
	\Gamma \vdash \{z \Rightarrow \overline{\sigma'}\} <: S}
  {}
\end{mathpar}
???????????????????????????????????????????????
\\
\item[]  \textit{Subcase 4} (\textsc{S-Top}): 
\begin{mathpar}
\inferrule
  {U = \top}
  {}
\end{mathpar}
$\Gamma \vdash S <: \top$ follows from \textsc{S-Top}.
\\
\end{itemize}
\end{case}
%\begin{case}[\textsc{S-Refl}]
%\begin{mathpar}
%\inferrule
%  {}
%  {S = T \\
%	\Gamma \vdash T\; <: \; T \\
%	\Gamma \vdash T <: U}
%\end{mathpar}
%This follows automatically from our assumption.
%\end{case}
%\begin{case}[\textsc{S-Rec}]
%\begin{mathpar}
%\inferrule
%  {S = \{\texttt{z} \Rightarrow \overline{\sigma}\} \\
%	T = \{\texttt{z} \Rightarrow \overline{\sigma'}\} \\
%	\forall \sigma_i' \in \overline{\sigma}', \; \exists \; \sigma_i \in \overline{\sigma} \; 
%	st \; \Gamma, \texttt{z} : \{\texttt{z} \Rightarrow \overline{\sigma}\} \vdash 
%	\sigma_i <:\; \sigma_i' \\
%	\Gamma \vdash T <: U}
%  {}
%\end{mathpar}
%\end{case}
%\begin{case}[\textsc{S-Select-Upper}]
%\end{case}
\begin{case}[\textsc{S-Select-Lower}]
\begin{mathpar}
\inferrule
  {T = p.L \\
	\Gamma \vdash p \ni \texttt{type} \; L : S' .. U' \\
  	\Gamma \vdash S <: S' \\
	\Gamma \vdash p.L <: U}
  {}
\end{mathpar}
With our induction hypothesis:
\begin{mathpar}
\inferrule
  {\forall U'', \Gamma \vdash S' <: U'' \Rightarrow \Gamma \vdash S <: U''}
  {}
\end{mathpar}
Case analysis on  the derivation of $\Gamma \vdash p.L <: U$ gives us the following cases.

\begin{itemize}
\item[]  \textit{Subcase 1} (\textsc{S-Refl}):
\begin{mathpar}
\inferrule
  {U = T = p.L}
  {}
\end{mathpar}
The result follows by assumption.
\\
\item[]  \textit{Subcase 2} (\textsc{S-Select-Upper}):
\begin{mathpar}
\inferrule
  {T = p.L \\
	\Gamma \vdash p \ni \texttt{type} \; L : S' .. U' \\
	\Gamma \vdash S' <: U' \\
	\Gamma \vdash U' <: U}
  {}
\end{mathpar}
By our existing premise $\Gamma \vdash S <: S'$.
We also know that $\Gamma \vdash S' <: U'$ thus it follows 
from our induction hypothesis that $\Gamma \vdash S <: U'$.
\\
\item[]  \textit{Subcase 3} (\textsc{S-Top}):
\begin{mathpar}
\inferrule
  {U = \top}
  {}
\end{mathpar}
The result follows from \textsc{S-Top}
\\
\end{itemize}
\end{case}
%Complete proof for S-Select-Lower
%\begin{case}[\textsc{S-Top}]
%\begin{mathpar}
%\inferrule
%  {T = \top \\
%	\Gamma \vdash \top <: U}
%  {}
%\end{mathpar}
%\end{case}
\begin{case}[\textsc{S-Top}]
\begin{mathpar}
\inferrule
  {T = \top\\
	\Gamma \vdash T <: U}
  {}
\end{mathpar}
It follows from Lemma \ref{lem:sub_eq} that $U = \top$, and subsequently 
that $S <: U$ by \textsc{S-Top}.
\end{case}
\begin{case}[\textsc{S-Bottom}]
\begin{mathpar}
\inferrule
  {S = \bot\\
	\Gamma \vdash T <: U}
  {}
\end{mathpar}
The result follows from \textsc{S-Bottom}
\end{case}
\end{proof}

\subsection{Subject Reduction}

\begin{theorem}[Subject Reduction]
If $\Gamma \vdash e : T$, 
   	$\mu \; | \; e \; \rightarrow \mu' \; | \; e'$ where
	$\Gamma \vdash \mu$ then 
 	$\exists \Gamma', T'$ s.t. 
	$\Gamma'$ extends $\Gamma$, 
	$\Gamma' \vdash \mu'$, 
	$\Gamma' \vdash e' : T'$ 
	and $\Gamma' \vdash T'<:T$.
\end{theorem}
\begin{proof}
By induction on the derivation of $\Delta; \; \mu \; | \; e \rightarrow \Delta'; \; \mu' \; | \; e'$.
\begin{case}[\textsc{R-New}]
\begin{mathpar}
\inferrule
  {e = (\texttt{var} \; y = \texttt{new} \; \{\texttt{z} \Rightarrow \overline{d_v}\}) \\
	e' = y \\
	\mu' = \mu, y \mapsto \{\texttt{z} \Rightarrow \overline{d_v}\}\\
	\forall d_i \in \overline{d_v}, \; 
  \Gamma, z : \{z \Rightarrow \overline{\sigma}\} 
  \vdash d_i : \sigma_i\\
  	y \notin dom(\Gamma)}
  {}
\end{mathpar}
Let $\Gamma' = \Gamma, y : \{z \Rightarrow \overline{\sigma}\}$. 
\begin{mathpar}
\inferrule
  {\forall d_i \in \overline{d_v}, \; 
  \Gamma, z : \{z \Rightarrow \overline{\sigma}\} 
  \vdash d_i : \sigma_i\\
  	y \notin dom(\Gamma)}
  {}
\end{mathpar}
\end{case}

\begin{case}[\textsc{R-Meth}]
\begin{mathpar}
\inferrule
  {e = v_1.m(v_2) \\
	e' = v_1/\texttt{z},v_2 \unlhd S/x]e \unlhd T \\
	\mu' = \mu \\
	\mu(path(\mu, v_1)) = \{\texttt{z} \Rightarrow ...,m:T(x:S)=e_0,...\}}
  {}
\end{mathpar}
\end{case}
\begin{case}[\textsc{R-Context}]
\begin{mathpar}
\inferrule
  {	\mu \; | \; e \; \rightarrow \; \mu' \; | \; e'}
  {\mu \; | \; E[e] \; \rightarrow \mu' \; | \; E[e']}
  \quad (\textsc {R-Context})
\end{mathpar}
\end{case}
\end{proof}

\subsection{Progress}
\begin{theorem}[Progress]
If $\Gamma \vdash e : T$, then either
\begin{enumerate}
\item e is a value, or
\item $\forall \mu$ s.t.
		   $\Gamma \vdash \mu$,
         $\exists e'$ and $\mu'$ s.t. 
         $\mu \; | \; e \; \rightarrow \mu' \; | \; e'$
\end{enumerate}
\end{theorem}



\bibliographystyle{plain}
\bibliography{bib}

\end{document}