\documentclass{llncs}

\usepackage{listings}
\usepackage{amssymb}
\usepackage[margin=.9in]{geometry}
%\usepackage{amsmath}
\usepackage{mathpartir}

\lstdefinestyle{custom_lang}{
  xleftmargin=\parindent,
  showstringspaces=false,
  basicstyle=\ttfamily,
  keywordstyle=\bfseries
}

\lstset{emph={%  
    var, def, type, new, z%
    },emphstyle={\bfseries \tt}%
}

\begin{document}





\section{Type Members}

\section{Counter Examples to Preservation}
	\label{s:examples}

\subsection{Term Membership Restriction}
Figure \ref{f:mem} gives the \emph{Membership} judgement. 
$\Gamma \vdash e \ni \sigma$ says that an expression $e$ 
has $\sigma_i$ as a member of its type in environment $\Gamma$. 
There are two rules for membership. By \textsc{M-Path} a 
variable $x$ has a member of type $[x/\texttt{z}]\sigma_i$ 
if $\sigma_i$ is a member of the type of $x$. By \textsc{M-Exp}
an expression $e$ has member $\sigma_i$ if $\sigma_i$ is 
a member of $e$'s type, and \texttt{z} does not occur 
within $\sigma_i$.

This is reasonable since we cannot substitute a non-value 
expression into a selection type such as $\texttt{z}.L$. 
This does however present a counter-example to preservation. 
Given two types $X$ and $Y$,

\begin{lstlisting}[mathescape, style=custom_lang]
Y = {z $\Rightarrow$
     var l : $\top$
     def m : Y(y:Y){
       var a = new {};
       y
     }
    }
\end{lstlisting}
\begin{lstlisting}[mathescape, style=custom_lang]
X = {z $\Rightarrow$
     type L : $\top$ .. $\top$
     var l : z.L
     def m : Y(y:Y){
       y
     }
    }
\end{lstlisting}

the following expression is well typed.
\begin{lstlisting}[mathescape, style=custom_lang]
var x = new X(l = z);
var y = new Y(l = (var z = new {}; z));
y.m(x).l
\end{lstlisting}
We can see that this is well-typed, and in particular that membership 
holds for \texttt{y.m(x)} and \texttt{l}. This reduces to 
\begin{lstlisting}[mathescape, style=custom_lang]
(var a = new {}; x).l
\end{lstlisting}
which is not well-typed since \texttt{var a = new {}; x} has type 
\texttt{X}, and \texttt{X.l} contains a \texttt{z} reference 
\texttt{z.L}.

\subsection{Expansion Lost}
\subsection{Well-Formedness Lost}

\subsection{Path Equality}
The example below illustrates the problem with path equality.
\begin{lstlisting}[mathescape, style=custom_lang]
var b = new {z $\Rightarrow$
             type L : $\top$ .. $\top$
             var l : z.L = b};
var a = new {z $\Rightarrow$
             var i : {z $\Rightarrow$
                      type L : $\bot$ .. $\top$
                      var l : z.L} = b
             def meth : $\top$ (x : z.i.L)z.i};
a.meth(b.l)
\end{lstlisting}
Even though we initialize a.i as b, we are unable to determine if
a.i and b are equivalent paths, hence we cannot ensure that b.L subtypes 
a.i.L.

\subsection{Path Equality Redux}
\begin{lstlisting}[mathescape, style=custom_lang]
var n = new {z $\Rightarrow$ type X : $\top$ .. $\top$};
var a = new {z $\Rightarrow$
             type Y : {z $\Rightarrow$ type X : $\bot$ .. $\top$} .. {z $\Rightarrow$ type X : $\bot$ .. $\top$}
             var i : z.Y = n};
var b = new {z $\Rightarrow$
             type L : $\bot$ .. {z $\Rightarrow$ type Y : {z $\Rightarrow$ type X : $\bot$ .. $\top$} .. 
                                          {z $\Rightarrow$ type X : $\bot$ .. $\top$}
                                 var i : z.X}
             var j : z.L = a};
var c = new {z $\Rightarrow$
             var k : {z $\Rightarrow$
                      type L : $\bot$ .. {z $\Rightarrow$ type Y : {z $\Rightarrow$ type X : $\bot$ .. $\top$} .. 
                                                   {z $\Rightarrow$ type X : $\bot$ .. $\top$}
                                          var i : z.X}
                      var j : z.L} = b
             def m : $\top$ (x : z.k.L){
                 var y = new {z $\Rightarrow$ var l : x.Y = x.i
                                   var m : x.i.X = new {z $\Rightarrow$}}; y
             }};
c.m(c.k.j)
\end{lstlisting}
\begin{lstlisting}[mathescape, style=custom_lang]
c.m(c.k.j) $\rightarrow$ c.m(b.j)
\end{lstlisting} 
\texttt{b.j} has type \texttt{b.L}, therefore we need 
$\texttt{b.L} <: \texttt{c.k.L}$
\begin{lstlisting}[mathescape, style=custom_lang]
c.m(b.j) $\rightarrow$ c.m(a)
c.m(a) $\rightarrow$ (var y = new {z $\Rightarrow$ var l : (a $\unlhd$ c.k.L).X = (a $\unlhd$ c.k.L).i 
                            var m : x.i.X = new {z $\Rightarrow$}}; y) $\unlhd$ $\top$
\end{lstlisting}
\section{Type System}
	\label{s:type_sys}


\subsection{Syntax}


\begin{figure}[h]
\[
\begin{array}{lll}
\begin{array}{lllr}
e & ::= & x & expression \\
& | & \texttt{var} \; x = \texttt{new} \; \{\texttt{z} \Rightarrow \overline{d}\}; \; e&\\
%& | & \lambda x:T = e\\
%& | & e(e)\\
& | & e.m(e) &\\
& | & e.f &\\
%& | & e.f = e&\\
& | & e \unlhd T&\\
%& | & l &\\
&&\\
p & ::= & x & paths \\
& | & p.f &\\
& | & p \unlhd T &\\
&&\\
v & ::= & x & value \\
& | & v \unlhd T &\\
&&\\
d & ::= & \texttt{var} \; f : T = e & declaration \\
  & |   & \texttt{def} \; m : T(x:T) = e &\\
  & |   & \texttt{type} \; L : T .. T = T &\\
&&\\
d_v & ::= & \texttt{var} \; f : T = v & declaration \; value \\
  & |   & \texttt{def} \; m : T(x:T) = e &\\
  & |   & \texttt{type} \; L : T .. T = T &\\
&&\\
%p & ::= & x & path \\
%  & | & l & \\
&&\\
\Gamma & :: = & \varnothing \; | \; \Gamma, \; x : T & context \\
&&\\
\mu & :: = & \varnothing \; | \; \mu,\; x \mapsto \{\texttt{z} \Rightarrow \overline{d}\} & store \\
\end{array}
& ~~~~~~
&
\begin{array}{lllr}
%T & ::= & \{\texttt{z} \Rightarrow \overline{\sigma}\} & type \\
T & ::= & \{\texttt{z} \Rightarrow \overline{\sigma}\} & type \\
& | & p.L &\\
& | & T \wedge T & \\
& | & \top & \\
& | & \bot & \\
&&\\
\sigma & ::= & \texttt{var} \; f:T & decl \; type\\
       & |   & \texttt{def} \; m:T \rightarrow T \\
		 & |   & \texttt{type} \; L : T .. T &\\
&&\\
E & :: = & \bigcirc & eval \; context\\
       & | & E.m(e)\\
       & | & v.m(E)\\
       & | & E.f\\
%       & | & E.f = e\\
%       & | & v.f = E\\
       & | & E \unlhd T\\
       & | & \texttt{var} \; x = \texttt{new} \; \{\texttt{z} \Rightarrow \overline{D}\}; \; e & eval \; context \\
       & | & \texttt{var} \; x = \texttt{new} \; \{\texttt{z} \Rightarrow \overline{d_v}\}; \; E& \\
&&\\
D & :: = & \texttt{var} \; f : T = E & decl \; eval \; context \\
  & |   & \texttt{def} \; m : T(x:T) = e &\\
  & |   & \texttt{type} \; L : T .. T = T &\\
%&&\\
%K & :: = & \varnothing \; | \; K, \; E & evaluation \; stack \\
%\Delta & :: = & \varnothing \; | \; \Delta,\; x \mapsto l & substitution \; context \\
%&&\\
%\Sigma & :: = & \varnothing \; | \; \Sigma,\; l : T & store \; context \\
\end{array}
\end{array}
\]
\caption{Syntax}
\label{f:syntax}
\end{figure}


\begin{figure}[h]
\begin{mathpar}
\inferrule
  {}
  {narrow(x) = x \\ narrow(v \unlhd T) = narrow(v)}
\end{mathpar}
\caption{Narrow Function}
\label{f:narrow}
\end{figure}

\subsection{Semantics}
%\subsubsection{Subtyping}

\begin{figure}[h]
\hfill \fbox{$A; \Sigma; \Gamma_1 \vdash S <: T \dashv \Gamma_2$}
\begin{mathpar}
\inferrule
  {}
  {A; \Sigma; \Gamma \vdash T\; \texttt{<:}\; T \dashv \Gamma}
  \quad (\textsc {S-Refl})
	\and
\inferrule
  {(S <: T) \in A}
  {A; \Sigma; \Gamma_1 \vdash S\; \texttt{<:}\; T \dashv \Gamma_2}
  \quad (\textsc {S-Assume})
	\and
\inferrule
  {A' = A, (\{z \Rightarrow \overline{\sigma}\} <: \{z \Rightarrow \overline{\sigma}'\}) \\
  	A'; \Sigma; \Gamma_1, z : \{z \Rightarrow \overline{\sigma}\} \vdash \overline{\sigma} <:\; \overline{\sigma}'  \dashv \Gamma_2, z : \{z \Rightarrow \overline{\sigma}'\}}
  {A; \Sigma; \Gamma_1 \vdash \{z \Rightarrow \overline{\sigma}\}\; <:\; \{z \Rightarrow \overline{\sigma}'\}\dashv \Gamma_2}
  \quad (\textsc {S-Rec})
	\and
\inferrule
  {A; \Sigma; \Gamma_1 \vdash p \ni \texttt{type} \; L : S_1 .. U_1\\
  	A; \Sigma; \Gamma_2 \vdash p \ni \texttt{type} \; L : S_2 .. U_2\\
  	A; \Sigma; \Gamma_2 \vdash S_2 \; \texttt{<:}\; S_1 \dashv \Gamma_1 \\
  	A; \Sigma; \Gamma_1 \vdash U_1 \; \texttt{<:}\; U_2 \dashv \Gamma_2}
  {A; \Sigma; \Gamma_1 \vdash p.L \; \texttt{<:}\; p.L \dashv \Gamma_2}
  \quad (\textsc {S-Select-Refl})
	\and
\inferrule
  {A; \Sigma; \Gamma_1 \vdash p \ni \texttt{type} \; L : S .. U\\
  	A; \Sigma; \Gamma_1 \vdash S <: U \dashv \Gamma_1 \\
  	A; \Sigma; \Gamma_1 \vdash U <: U' \dashv \Gamma_2}
  {A; \Sigma; \Gamma_1 \vdash p.L\; <:\; U' \dashv \Gamma_2}
  \quad (\textsc {S-Select-Upper})
	\and
\inferrule
  {A; \Sigma; \Gamma_2 \vdash p \ni \texttt{type} \; L : S .. U \\
  	A; \Sigma; \Gamma_2 \vdash S <: U \dashv \Gamma_2 \\
  	A; \Sigma; \Gamma_1 \vdash S' <: S \dashv \Gamma_2}
  {A; \Sigma; \Gamma_1 \vdash S'\; <:\; p.L \dashv \Gamma_2}
  \quad (\textsc {S-Select-Lower})
	\and
\inferrule
  {}
  {A; \Sigma; \Gamma_1 \vdash T\; \texttt{<:}\; \top \dashv \Gamma_2}
  \quad (\textsc {S-Top})
	\and
\inferrule
  {}
  {A; \Sigma; \Gamma_1 \vdash \bot\; \texttt{<:}\; T \dashv \Gamma_2}
  \quad (\textsc {S-Bottom})
\end{mathpar}
\hfill \fbox{$A; \Sigma; \Gamma_1 \vdash \sigma <: \sigma' \dashv \Gamma_2$}
\begin{mathpar}
\inferrule
  {}
  {A; \Sigma; \Gamma_1 \vdash \texttt{val} \; f:T <: \texttt{val} \; f:T \dashv \Gamma_2}
  \quad (\textsc {S-Decl-Val})
	\and
\inferrule
  {A; \Sigma; \Gamma_2 \vdash S' <: S \dashv \Gamma_1 \\
  	A; \Sigma; \Gamma_1 \vdash T <: T' \dashv \Gamma_2}
  {A; \Sigma; \Gamma_1 \vdash \texttt{def} \; m:S \rightarrow T <: \texttt{def} \; m:S' \rightarrow T' \dashv \Gamma_2}
  \quad (\textsc {S-Decl-Def})
	\and
\inferrule
  {A; \Sigma; \Gamma_2 \vdash S' <: S \dashv \Gamma_1 \\
  	A; \Sigma; \Gamma_1 \vdash U <: U' \dashv \Gamma_2}
  {A; \Sigma; \Gamma_1 \vdash \texttt{type} \; L : S .. U \; <:\; \texttt{type} \; L : S' .. U' \dashv \Gamma_2}
  \quad (\textsc {S-Decl-Type})
\end{mathpar}
\caption{Subtyping}
\label{f:subtype}
\end{figure}

\begin{figure}[h]
\hfill \fbox{$A; \Sigma; \Gamma \vdash T \;  \textbf{wf}$}
\begin{mathpar}
\inferrule
  {A; \Sigma; \Gamma \vdash p \ni \texttt{type} \; L : S .. U \\
  	A; \Sigma; \Gamma \vdash \texttt{type} \; L : S .. U \; \textbf{wf} }
  {A; \Sigma; \Gamma \vdash p.L \; \textbf{wf}}
  \quad (\textsc {WF-Sel})
	\and
\inferrule
  {A; \Sigma; \Gamma,z:\{z \Rightarrow \overline{\sigma}\} \vdash \overline{\sigma} \; \textbf{wf} \\
  	\forall j \neq i, \; dom(\sigma_j) \neq dom(\sigma_i)}
  {A; \Gamma; \Sigma \vdash \{z \Rightarrow \overline{\sigma}\} \; \textbf{wf}}
  \quad (\textsc {WF-Rec})
%	\and
%\inferrule
%  {\Gamma \vdash S \;  \textbf{wf} \\
%  	\Gamma \vdash T \;  \textbf{wf}}
%  {\Gamma \vdash S \rightarrow T \;  \textbf{wf}}
%  \quad (\textsc {WF-Func})
	\and
\inferrule
  {}
  {A; \Sigma; \Gamma \vdash \top \;  \textbf{wf}}
  \quad (\textsc {WF-Top})
	\and
\inferrule
  {}
  {A; \Sigma; \Gamma \vdash \bot \;  \textbf{wf}}
  \quad (\textsc {WF-Bot})
\end{mathpar}
\hfill \fbox{$A; \Sigma; \Gamma \vdash \sigma \;  \textbf{wf}$}
\begin{mathpar}
\inferrule
  {A; \Sigma; \Gamma \vdash T : \textbf{wf}}
  {A; \Sigma; \Gamma \vdash \texttt{val} \; f:T \;  \textbf{wf}}
  \quad (\textsc {WF-Val})
	\and
\inferrule
  {A; \Sigma; \Gamma \vdash T : \textbf{wf} \\
  	A; \Sigma; \Gamma \vdash S : \textbf{wf}}
  {A; \Sigma; \Gamma \vdash \texttt{def} \; m:S \rightarrow T \;  \textbf{wf}}
  \quad (\textsc {WF-Def})
	\and
\inferrule
  {A; \Sigma; \Gamma \vdash S : \textbf{wfe} \; \vee \; S = \bot\\
  	A; \Sigma; \Gamma \vdash U : \textbf{wfe} \\
  	A; \Sigma; \Gamma \vdash S <: U}
  {A; \Sigma; \Gamma \vdash \texttt{type} \; L : S .. U \; \textbf{wf}}
  \quad (\textsc {WF-Type})
\end{mathpar}
\hfill \fbox{$A; \Sigma \vdash \Gamma \;  \textbf{wf}$}
\begin{mathpar}
\inferrule
  {\forall x \in dom(\Gamma), A; \Sigma; \Gamma \vdash \Gamma(x) \; \textbf{wf}}
  {\Sigma \vdash \Gamma \;  \textbf{wf}}
  \quad (\textsc {WF-Environment})
\end{mathpar}
\hfill \fbox{$\Sigma \;  \textbf{wf}$}
\begin{mathpar}
\inferrule
  {\forall l \in dom(\Sigma), \varnothing; \Sigma; \varnothing \vdash \Sigma(l) \; \textbf{wf}}
  {\Sigma \;  \textbf{wf}}
  \quad (\textsc {WF-Store-Context})
\end{mathpar}
\begin{mathpar}
\inferrule
  {\forall l \in dom(\mu), \varnothing; \Sigma; \varnothing \vdash \mu(l) : \Sigma(l)}
  {\mu : \Sigma}
  \quad (\textsc {WF-Store})
\end{mathpar}
\caption{Well-Formedness}
\label{f:wf}
\end{figure}

\begin{figure}[h]
\hfill \fbox{$A; \Sigma; \Gamma \vdash T \;  \textbf{wfe}$}
\begin{mathpar}
\inferrule
  {A; \Sigma; \Gamma \vdash T \; \textbf{wf} \\
  	A; \Sigma; \Gamma \vdash T \prec \overline{\sigma}}
  {A; \Sigma; \Gamma \vdash T \; \textbf{wfe}}
  \quad (\textsc {WFE})
\end{mathpar}
\caption{Well-Formed and Expanding Types}
\label{f:wfe}
\end{figure}

\begin{figure}[h]
\hfill \fbox{$\Sigma; \Gamma \vdash T \prec \overline{\sigma}$}
\begin{mathpar}
\inferrule
  {}
  {\Sigma; \Gamma \vdash \{z \Rightarrow \overline{\sigma}\} \prec_z \overline{\sigma}}
  \quad (\textsc {E-Rec})
	\and
\inferrule
  {\Sigma; \Gamma \vdash p \ni \texttt{type} \; L : S..U \\
  	\Sigma; \Gamma \vdash U \prec_z \overline{\sigma}}
  {\Sigma; \Gamma \vdash p.L \prec_z \overline{\sigma}}
  \quad (\textsc {E-Sel})
	\and
\inferrule
  {}
  {\Sigma; \Gamma \vdash \top \prec_z \varnothing}
  \quad (\textsc {E-Top})
\end{mathpar}
\caption{Expansion}
\label{f:exp}
\end{figure}
\begin{figure}[h]
\hfill \fbox{$\Sigma; \Gamma \vdash e \ni \sigma$}
\begin{mathpar}
\inferrule
  {\Sigma; \Gamma \vdash p : T \\
  	\Sigma; \Gamma \vdash T \prec_z \overline{\sigma}\\
  	\sigma_i \in \overline{\sigma}}
  {\Sigma; \Gamma \vdash p \ni [p/z]\sigma_i}
  \quad (\textsc {M-Path})
	\and
\inferrule
  {\Sigma; \Gamma \vdash e : T \\
  	\Sigma; \Gamma \vdash T \prec_z \overline{\sigma}\\
  	\sigma_i \in \overline{\sigma} \\
  	z \notin \sigma_i}
  {\Sigma; \Gamma \vdash e \ni \sigma_i}
  \quad (\textsc {M-Exp})
\end{mathpar}
\caption{Membership}
\label{f:mem}
\end{figure}

%\subsubsection{Typing}

\begin{figure}[h]
\hfill \fbox{$\Gamma; \Sigma \vdash e:T$}
\begin{mathpar}
\inferrule
  {x \in dom(\Gamma)}
  {	\Gamma; \Sigma \vdash x : \Gamma(x)}
  \quad (\textsc {T-Var})
	\and
\inferrule
  {	l \in dom(\Sigma)}
  {	\Gamma; \Sigma \vdash l : \Sigma(l)}
  \quad (\textsc {T-Loc})
	\and
\inferrule
  {\Gamma, z : \{z \Rightarrow \overline{\sigma}\}; \Sigma 
  \vdash \overline{d} : \overline{\sigma} \\
  	y \notin dom(\Gamma)}
  {	\Gamma; \Sigma\vdash \texttt{new} \; \{z \Rightarrow \overline{d}\} : 
  \{z \Rightarrow \overline{\sigma}\}}
  \quad (\textsc {T-New})
	\and
\inferrule
  {\Gamma; \Sigma \vdash e_0 \ni \texttt{def} \; m:S \rightarrow T \\
  	\Gamma; \Sigma \vdash e_1 : S \\
  	\Gamma; \Sigma \vdash T <: U}
  {	\Gamma; \Sigma \vdash e_0.m_U(e_1) : U}
  \quad (\textsc {T-Meth})
	\and
\inferrule
  {	\Gamma; \Sigma \vdash e : S \\
  	\Gamma; \Sigma \vdash e \ni \texttt{val} \; f:T}
  {	\Gamma; \Sigma \vdash e.f : T}
  \quad (\textsc {T-Acc})
%	\and
%\inferrule
%  {	\Gamma; E \vdash e_0 : T_0 \\
%  	T_0 \ni \texttt{def} \; f:T = e \\
%  	\Gamma; E \vdash e_1 : T}
%  {	\Gamma; E \vdash e_0.f = e_1 : T}
%  \quad (\textsc {T-Assign})
%	\and
	\and
\inferrule
  {	\Gamma; \Sigma \vdash e : T}
  {	\Gamma; \Sigma \vdash e \unlhd T : T}
  \quad (\textsc {T-Type})
	\and
\inferrule
  {	\Gamma; \Sigma \vdash e : S \\
  	\Gamma; \Sigma \vdash S <: T}
  {	\Gamma; \Sigma \vdash e : T}
  \quad (\textsc {T-Sub})
\end{mathpar}
\caption{Expression Typing}
\label{f:e_typ}
\end{figure}
\begin{figure}[h]
\hfill \fbox{$\Gamma; E \vdash d:\sigma$}
\begin{mathpar}
\inferrule
  {\Gamma; \Sigma \vdash e : T}
  {\Gamma; \Sigma \vdash \texttt{def} \; f:T = e : \texttt{def} \; f:T}
  \quad (\textsc {T-Decl-Var})
	\and
\inferrule
  {\Gamma, x : S; \Sigma \vdash e_0 : T}
  {\Gamma; \Sigma \vdash \texttt{def} \; m(x:S) = e : T : \texttt{def} \; m:S \rightarrow T}
  \quad (\textsc {T-Decl-Def})
	\and
\inferrule
  {\Gamma; \Sigma \vdash \texttt{type} \; L : S .. U \; \textbf{wf} }
  {\Gamma; \Sigma \vdash \texttt{type} \; L : S .. U : \texttt{type} \; L : S .. U}
  \quad (\textsc {T-Decl-Type})
\end{mathpar}
\caption{Declaration Typing}
\label{f:d_typ}
\end{figure}
%\begin{figure}[h]
%\hfill \fbox{$\Gamma \vdash \mu:\Sigma$}
%\begin{mathpar}
%\inferrule
%  {\forall x \in dom(\mu), \; \mu(x)= \{\texttt{z} \Rightarrow \overline{d}\} \\
%  	\Gamma(x) = \{\texttt{z} \Rightarrow \overline{\sigma}\} \\
%  	\Gamma \vdash \overline{d} : \overline{\sigma}}
%  {\Gamma \vdash \mu}
%  \quad (\textsc {T-Store})
%\end{mathpar}
%\caption{Store Typing}
%\label{f:s_typ}
%\end{figure}

\begin{figure}[h]
\hfill \fbox{$\mu \; | \; e \; \rightarrow \mu' \; | \; e'$}
\begin{mathpar}
\inferrule
  {l \notin dom(\mu) \\
  	\mu' = \mu, l \mapsto \{\texttt{z} \Rightarrow \overline{d_v}\}}
  {\mu \; | \; \texttt{new} \; \{\texttt{z} \Rightarrow \overline{d_v}\} \; \rightarrow \mu' \; | \; l}
  \quad (\textsc {R-New})
  \and
\inferrule
  {\mu : \Sigma \\
   \mu; \Sigma \vdash v_1 \leadsto_{m} e}
  {\mu \; | \; v_1.m(v_2) \;\rightarrow \mu \; | \; [l/\texttt{z},v_2 \unlhd S/x]e}
  \quad (\textsc {R-Meth})
  \and
\inferrule
  {	\mu \; | \; e \; \rightarrow \; \mu' \; | \; e'}
  {\mu \; | \; E[e] \; \rightarrow \mu' \; | \; E[e']}
  \quad (\textsc {R-Context})
\end{mathpar}
\caption{Reduction}
\label{f:red}
\end{figure}


\section{Type Safety}

\subsection{Subject Reduction}


\begin{theorem}[Subject Reduction]
If $\Gamma; E \vdash e : T$, 
   	$\mu \; | \; e \; | \; E \rightarrow \mu' \; | \; e' \; | \; E'$ where
	$\Gamma \vdash \mu$ then 
 	$\exists \Gamma', T'$ s.t. 
	$\Gamma'$ extends $\Gamma$, 
	$\Gamma' \vdash \mu'$, 
	$\Gamma'; E' \vdash e' : T'$ 
	and $\Gamma'; E' \vdash T'<:T$.
\end{theorem}
%\begin{proof}
%By induction on 	$\Delta; \; \mu \; | \; e \rightarrow \Delta'; \; \mu' \; | \; e'$.
%\begin{case}[\textsc{R-New}]
%%\\ \fbox{
%\begin{mathpar}
%\inferrule
%  {}
%  {e := \texttt{var} \; y = \texttt{new} \; \{\texttt{z} \Rightarrow \overline{d}\}; e \\
%	e' := e\\
%	\mu' = \mu, l \mapsto \{\texttt{z} \Rightarrow \overline{d}\} \\
%  	\Delta' = \Delta, y \mapsto l\\
%	\Delta; \; \Sigma; \; \Gamma, \texttt{z} : \{\texttt{z} \Rightarrow \overline{\sigma}\} \vdash \overline{d} : \overline{\sigma}\\
%	\Sigma; \; \Gamma, y : \{\texttt{z} \Rightarrow \overline{\sigma}\} \vdash e : T}
%\end{mathpar}
%
%Let $\Sigma' = \Sigma, l \mapsto \{\texttt{z} \Rightarrow \overline{\sigma}\}$ and 
%$\Gamma' = \Gamma, y : \{\texttt{z} \Rightarrow \overline{\sigma}\}$. 
%$\therefore \Sigma'; \Gamma';  \vdash e' : T$ and
%$\Gamma' \vdash T<:T$ by reflexivity.
%%}
%\end{case}
%\begin{case}[\textsc{R-Meth}]
%\begin{mathpar}
%\inferrule
%  {}
%  {e := y.m(z)\\
%	e' := [y/\texttt{z},z'/x]e\\
%	\mu(\Delta(y)) = \{\texttt{z} \Rightarrow ...,m:T(x:S)=e,...\}\\
%   \Delta(z) = l' \\
%  	z' \notin dom(\Delta)\\
%  	\Delta' = \Delta,z' \mapsto l'}
%\end{mathpar}
%
%Let $\Gamma' = \Gamma, z' : S$, $\Sigma' = \Sigma$ and $T' = T$. Since 
%$\Sigma; \Gamma;  \vdash y.m(z) : T$, the following hold (\textsc{T-Meth}):
%\begin{mathpar}
%\inferrule
%  {}
%  {\Sigma; \; \Gamma \vdash z : S \\
%  	\Sigma; \; \Gamma \vdash y : T_0 \\
%  	T_0 \ni \texttt{def} \; m:S \rightarrow T}
%\end{mathpar}
%
%Since $\Gamma \vdash \mu : \Sigma$ and $\Sigma \; \textbf{wf}$ it
%follows that $\Gamma \vdash T : \textbf{wf}$, and thus that 
%$\Gamma, z' : S \vdash T : \textbf{wf}$. 
%
%\end{case}
%\begin{case}[\textsc{R-Var}]
%\end{case}
%\begin{case}[\textsc{R-Var}]
%\end{case}
%\begin{case}[\textsc{R-Assign}]
%\end{case}
%\begin{case}[\textsc{R-Context}]
%\end{case}
%\end{proof}
%
\subsection{Progress}
\begin{theorem}[Progress]
If $\Gamma; E \vdash e : T$, then either
\begin{enumerate}
\item e is a value, or
\item $\forall \mu$ s.t.
		   $\Gamma \vdash \mu$,
         $\exists e', \mu', E'$ s.t. 
         $\mu \; | \; e \; | \; E \rightarrow \mu' \; | \; e' \; | \; E'$
\end{enumerate}
\end{theorem}



\bibliographystyle{plain}
\bibliography{bib}

\end{document}