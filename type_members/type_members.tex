\documentclass{llncs}

\usepackage{listings}
\usepackage{amssymb}
\usepackage[margin=.9in]{geometry}
%\usepackage{amsmath}
%\usepackage{amsthm}
\usepackage{mathpartir}




\lstdefinestyle{custom_lang}{
  xleftmargin=\parindent,
  showstringspaces=false,
  basicstyle=\ttfamily,
  keywordstyle=\bfseries
}

\lstset{emph={%  
    val, def, type, new, z%
    },emphstyle={\bfseries \tt}%
}

\begin{document}





\section{Type Members}
Type members are object members that describe types. 
In the same way that an object can have a member that describes 
a method or a field, with type members we can give
objects the ability to define types, and subsequently use types. 
Type members attempt to solve a similar problem to that of Generic type 
parameters in a language like Java. Below is a simple example of 
Java Generics.
\begin{lstlisting}[mathescape, style=custom_lang]
public class Foo<T>{
  T bar;
  public Foo(T bar){
    this.bar = bar;
  }

  public void setBar(T bar){
    this.bar = bar;
  }
  public T getBar(){
    return this.bar;
  }
}
\end{lstlisting}
If we introduced type members, this example could be rewritten as 
follows.
\begin{lstlisting}[mathescape, style=custom_lang]
public class Foo{
  type T;
  T bar;
  public Foo(T bar){
    this.bar = bar;
  }

  public void setBar(T bar){
    this.bar = bar;
  }
  public T getBar(){
    return this.bar;
  }
}
\end{lstlisting}
While \emph{Java} does not support Type Members, \emph{Scala} does.
This can be seen as an extension of the mechanisms already used in 
a language such as Java. In Java we can define abstract method members, but 
are unable to pass a method as a parameter. We can pass values as 
parameters, but are unable to define abstract field members. We can pass 
types as parameters, but are unable to define abstract type members.
In Scala, all three can either be defined as members of a type, or 
can be passed as parameters. 

In this document we present a a structurally typed language with type 
members. In a structural type system, types are defined by their structure 
as opposed to their name in nominal type systems. Normally this means the 
field and method definitions they contain. When adding type members 
we extend this to include types. This work is based on and extends 
similar work done on type members in Scala \cite{Amin:2012}.
While Amin et. al.\cite{Amin:2014} developed a big step semantics and 
proved it sound, we hope to develop an altered type system in order to maintain 
a small step semantics while providing sound type members. 

Using Java Generics, we can define an upper bound in order to restrict 
possible types used. We could easily conceive a similar bound being 
used for type members. Below is an example using the syntax of the language 
presented in this document.
\begin{lstlisting}[mathescape, style=custom_lang]
var o = new {z => type L : S .. U}
\end{lstlisting}
We have created a new object \texttt{o} that has a single member, a type member 
\texttt{L}. \texttt{L} is restricted by two bounds, a lower bound \texttt{S} 
and an upper bound \texttt{U} (Where \texttt{S} and \texttt{U} are some types). 
Using an access on \texttt{o} we can now refer to 
\texttt{L}: \texttt{o.L}.
We could allow type member accesses to not only be made on  
an object, but a path \texttt{p} constructed from a series of field accesses.
Below we can see that the type member \texttt{L} refers to 
depends entirely on the path taken to retrieve it.
\begin{lstlisting}[mathescape, style=custom_lang]
var o1 = new {z => type L : S .. U}
var o2 = new {z => type L : S' .. U'}
\end{lstlisting}
Clearly \texttt{o1.L} $\neq$ \texttt{o2.L}


\subsection{Type Members in Wyvern}
Wyvern is a structurally typed programming language being 
developed concurrently with this work. Wyvern puts an emphasis 
on secure web applications. The type members described here 
is intended to be integrated into the wider development of the 
Wyvern Programming language.


\section{Counter Examples to Preservation}
	\label{s:examples}

\subsection{Term Membership Restriction}
	\label{s:term_mem}
Figure \ref{f:mem} gives the \emph{Membership} judgement. 
$\Gamma \vdash e \ni \sigma$ says that an expression $e$ 
has $\sigma_i$ as a member of its type in environment $\Gamma$. 
There are two rules for membership. By \textsc{M-Path} a 
variable $x$ has a member of type $[x/\texttt{z}]\sigma_i$ 
if $\sigma_i$ is a member of the type of $x$. By \textsc{M-Exp}
an expression $e$ has member $\sigma_i$ if $\sigma_i$ is 
a member of $e$'s type, and \texttt{z} does not occur 
within $\sigma_i$.

This is reasonable since we cannot substitute a non-value 
expression into a selection type such as $\texttt{z}.L$. 
This does however present a counter-example to preservation. 
Given two types $X$ and $Y$,

\begin{lstlisting}[mathescape, style=custom_lang]
Y = {z $\Rightarrow$
     val l : $\top$
     def m : Y(y:Y){
       val a = new {};
       y
     }
    }
\end{lstlisting}
\begin{lstlisting}[mathescape, style=custom_lang]
X = {z $\Rightarrow$
     type L : $\top$ .. $\top$
     val l : z.L
     def m : Y(y:Y){
       y
     }
    }
\end{lstlisting}

the following expression is well typed.
\begin{lstlisting}[mathescape, style=custom_lang]
val x = new X(l = z);
val y = new Y(l = (val z = new {}; z));
y.m(x).l
\end{lstlisting}
We can see that this is well-typed, and in particular that membership 
holds for \texttt{y.m(x)} and \texttt{l}. This reduces to 
\begin{lstlisting}[mathescape, style=custom_lang]
(val a = new {}; x).l
\end{lstlisting}
which is not well-typed since \texttt{val a = new {}; x} has type 
\texttt{X}, and \texttt{X.l} contains a \texttt{z} reference 
\texttt{z.L}.

After patching we allow expressions to maintain original type. 
Expressions are then typed with respect to the original type.
This is given in Figure \ref{f:syntax} as $e \unlhd T$ where 
$T$ is the original type. The patched reduction is given in 
Figure \ref{f:red}. The amended method reduction \textsc{R-Meth} is 
shown below.
\begin{mathpar}
\inferrule
  {\mu(path(\mu, v_1)) = \{\texttt{z} \Rightarrow ...,m:T(x:S)=e,...\}}
  {\mu \; | \; v_1.m(v_2) \;\rightarrow \mu \; | \; [v_1/\texttt{z},v_2 \unlhd S/x]e \unlhd T}
  \quad (\textsc {R-Meth})
\end{mathpar}
This is very similar to the standard method reduction. The two main 
differences are the introduction of the $path$ function, and the 
inclusion of the original types in the returned expression. For now 
ignore the $path$ function, this is explained in \ref{s:patheq}.

We retrieve the method for the receiver from the store, and substitute 
the method parameter into the body. The method parameter retains it's 
original type ($v_2 \unlhd S$) as does the entire returned body 
$[v_1/\texttt{z},v_2 \unlhd S/x]e \unlhd T$. Using this rule we 
can attempt to evaluate our original example.
\begin{lstlisting}[mathescape, style=custom_lang]
val x = new X(l = z);
val y = new Y(l = (val z = new {}; z));
y.m(x).l
\end{lstlisting}
reduces to
\begin{lstlisting}[mathescape, style=custom_lang]
((val a = new {}; (x $\unlhd$ Y))$\unlhd$ Y).l
\end{lstlisting}
Now we can treat the method body as having type \texttt{Y} and 
we can determine membership of \texttt{l} for \texttt{Y} rather 
than \texttt{X}.

\subsection{Expansion Lost}
For preservation to hold, we need to ensure that types are expandable 
to lists of declarations. This is captured in Figure \ref{f:wfe}. It is 
possible for expansion to be lost during environment narrowing. This is 
shown in \cite{Amin:2012}, but is briefly covered again below.
\begin{lstlisting}[mathescape, style=custom_lang]
X = {z $\Rightarrow$
     type A : $\bot$ .. z.B
     type B : $\bot$ .. $\top$
    }
Y = {z $\Rightarrow$
     type A : $\bot$ .. $\top$
     type B : $\bot$ .. z.A
    }
\end{lstlisting}
While both these types are expandable, their union is not.
\begin{lstlisting}[mathescape, style=custom_lang]
X $\wedge$ Y = {z $\Rightarrow$
         type A : $\bot$ .. z.B
         type B : $\bot$ .. z.A
        }
\end{lstlisting}
While we could not type this normally, we can create this 
scenario during environmental narrowing that results in 
an a type without an expansion, providing a counter-example 
to preservation.
\begin{lstlisting}[mathescape, style=custom_lang]
W = {z $\Rightarrow$
     type A : $\bot$ .. $\top$
     type B : $\bot$ .. $\top$
    }
\end{lstlisting}
Looking at the following expression, we can see it is well-type, 
and the type \texttt{i.L} $\wedge$ \texttt{x.K} in particular is 
well-formed and expanding.
\begin{lstlisting}[mathescape, style=custom_lang]
val x = new {z $\Rightarrow$ type K : $\bot$ .. X};
val y = new {z $\Rightarrow$ type L : $\bot$ .. Y};
val w = new {z $\Rightarrow$ type L : $\bot$ .. W};
val c = new {z $\Rightarrow$
         def meth1 : $\top$ (i : {z $\Rightarrow$ type L : $\bot$ .. W}){
           val d = new {z $\Rightarrow$
                        def meth2 : $\top$ (j : i.L $\wedge$ x.K){
                          j
                        }
                       };
           d
         }
        };
c.meth1(y)
\end{lstlisting}
Evaluating the above expression results in the following.
\begin{lstlisting}[mathescape, style=custom_lang]
val d = new {z $\Rightarrow$
             def meth2 : $\top$ (j : y.L $\wedge$ x.K){
               j
             }
            };
d
\end{lstlisting}
Now the type \texttt{y.L} $\wedge$ \texttt{x.K} gives us the loss of 
expansion we were attempting to construct. In our new patched calculus, 
this problem does not occur because the original type of the \texttt{i} 
parameter is maintained. So if we re-evaluate the original expression 
with our type system, we get the following.
\begin{lstlisting}[mathescape, style=custom_lang]
val d = new {z $\Rightarrow$
             def meth2 : $\top$ (j : (y $\unlhd$ {z $\Rightarrow$ type L : $\bot$ .. W}).L $\wedge$ x.K){
               j
             }
            };
d
\end{lstlisting}
The type \texttt{(y} $\unlhd$ \texttt{\{z} $\Rightarrow$ 
\texttt{type L :} $\bot$ \texttt{.. W\}).L} $\wedge$ \texttt{x.K} is now 
both well-formed and expanding because it is the original type 
from our original expression.



\subsection{Expansion Lost Redux}
	\label{s:term_mem2}
First we define the following types.
\begin{lstlisting}[mathescape, style=custom_lang]
X = {z $\Rightarrow$
     def m(x : $\top$){var y = new {z $\Rightarrow$}}:$\top$
    }
Y = {z $\Rightarrow$
     type L : $\bot$ .. $\top$
     def m(x : $\top$){var y = new {z $\Rightarrow$}}:z.L
    }
\end{lstlisting}
Then we attempt to evaluate the following expression.
\begin{lstlisting}[mathescape, style=custom_lang]
var a = new {z $\Rightarrow$
              def meth(x:$\top$){val b = new Y}:X
            };
val c = new {z $\Rightarrow$};
a.meth(c).m(c)
\end{lstlisting}
This reduces to \texttt{a.meth(c).m(c)} which has type $\top$. 
Applying \textsc{R-Meth} we get \texttt{(val b = new Y $\unlhd$ X).m(c)}
which has type $\top$ and so is still well typed. This eventually reduces 
to \texttt{[z/b $\unlhd$ X](var y = new \{z $\Rightarrow$\}) $\unlhd$ z.L}.
Since during reduction on a method call of the form $v \unlhd T$
we retrieve the return type of the object ($v$) and 
not the type ($T$), the reduced expression has type \texttt{b $\unlhd$ X.L}. 
This type is allowed since \texttt{b $\unlhd$ X} is a path, but 
it does present a case of narrowing.

\subsection{Well-Formedness Lost}



\subsection{Path Equality}
\label{s:patheq}
The example below illustrates the problem with path equality.
\begin{lstlisting}[mathescape, style=custom_lang]
val b = new {z $\Rightarrow$
             type L : $\top$ .. $\top$
             val l : z.L = b};
val a = new {z $\Rightarrow$
             val i : {z $\Rightarrow$
                      type L : $\bot$ .. $\top$
                      val l : z.L} = b
             def meth : $\top$ (x : z.i.L){x}};
a.meth(a.i.l)
\end{lstlisting}
\texttt{a.i.l} reduces to \texttt{b.l} which has type \texttt{b.L}. 
There is no way for us to ensure that \texttt{b.L} <: \texttt{a.i.L}.
This presents a path equality problem for preservation.

Typing paths correctly requires that we maintain some information 
about the original paths. For this reason we don't evaluate paths 
in the calculus. We still however need to retrieve the correct object 
the path is pointing to during method reduction and object initialization. 
This is done by the \emph{path} function given in Figure \ref{f:path}. 
With a path and a store, we can find the object referenced by that path.
Below is the evaluation of the above example without evaluating paths.
\begin{lstlisting}[mathescape, style=custom_lang]
val b = new {z $\Rightarrow$
             type L : $\top$ .. $\top$
             val l : z.L = b};
val a = new {z $\Rightarrow$
             val i : {z $\Rightarrow$
                      type L : $\bot$ .. $\top$
                      val l : z.L} = b
             def meth : $\top$ (x : z.i.L){x}};
a.meth(a.i.l)
\end{lstlisting}
Reduces to \texttt{a.meth(a.i.l)} which type checks. \texttt{a.meth(a.i.l)}
reduces to \texttt{a.i.l}

\subsection{Path Equality Redux}
\begin{lstlisting}[mathescape, style=custom_lang]
val n = new {z $\Rightarrow$ type X : $\top$ .. $\top$};
val a = new {z $\Rightarrow$
             type Y : {z $\Rightarrow$ type X : $\bot$ .. $\top$} .. {z $\Rightarrow$ type X : $\bot$ .. $\top$}
             val i : z.Y = n};
val b = new {z $\Rightarrow$
             type L : $\bot$ .. {z $\Rightarrow$ type Y : {z $\Rightarrow$ type X : $\bot$ .. $\top$} .. 
                                          {z $\Rightarrow$ type X : $\bot$ .. $\top$}
                                 val i : z.Y}
             val j : z.L = a};
val c = new {z $\Rightarrow$
             val k : {z $\Rightarrow$
                      type L : $\bot$ .. {z $\Rightarrow$ type Y : {z $\Rightarrow$ type X : $\bot$ .. $\top$} .. 
                                                   {z $\Rightarrow$ type X : $\bot$ .. $\top$}
                                          val i : z.Y}
                      val j : z.L} = b
             def m : $\top$ (x : z.k.L){
                 val y = new {z $\Rightarrow$ val l : x.Y = x.i 
                            val m : x.i.X = new {z $\Rightarrow$}}; y
             }};
c.m(b.j)
\end{lstlisting}
%\begin{lstlisting}[mathescape, style=custom_lang]
%c.m(c.k.j) $\rightarrow$ c.m(b.j)
%\end{lstlisting} 
\texttt{b.j} has type \texttt{b.L}, therefore we need 
$\texttt{b.L} <: \texttt{c.k.L}$
\begin{lstlisting}[mathescape, style=custom_lang]
c.m(b.j) $\rightarrow$ c.m(a)
c.m(a) $\rightarrow$ (val y = new {z $\Rightarrow$ val l : (a $\unlhd$ c.k.L).Y = (a $\unlhd$ c.k.L).i 
                            val m : (a $\unlhd$ c.k.L).i.X = new {z $\Rightarrow$}}; y) $\unlhd$ $\top$
\end{lstlisting}

\section{Type System}
	\label{s:type_sys}


\subsection{Syntax}


\begin{figure}[h]
\[
\begin{array}{lll}
\begin{array}{lllr}
e & ::= & x & expression \\
& | & \texttt{new} \; \{z \Rightarrow \overline{d}\}&\\
%& | & \lambda x:T = e\\
%& | & e(e)\\
& | & e.m_T(e) &\\
& | & e.f &\\
%& | & e.f = e&\\
& | & e \unlhd T&\\
& | & l &\\
&&\\
p & ::= & x & paths \\
& | & l &\\
& | & p.f &\\
& | & p \unlhd T &\\
&&\\
v & ::= & l & value \\
& | & v.f &\\
& | & v \unlhd T &\\
&&\\
d & ::= & \texttt{val} \; f : T = p & declaration \\
  & |   & \texttt{def} \; m(x:T) = e : T &\\
  & |   & \texttt{type} \; L : T .. T&\\
&&\\
\Gamma & :: = & \varnothing \; | \; \Gamma,\; x : T & Environment \\
 \end{array}
& ~~~~~~
&
\begin{array}{lllr}
%T & ::= & \{\texttt{z} \Rightarrow \overline{\sigma}\} & type \\
T & ::= & \{z \Rightarrow \overline{\sigma}\} & type \\
& | & p.L &\\
& | & T \wedge T & \\
& | & \top & \\
& | & \bot & \\
&&\\
\sigma & ::= & \texttt{val} \; f:T & decl \; type\\
       & |   & \texttt{def} \; m:T \rightarrow T \\
		 & |   & \texttt{type} \; L : T .. T &\\
&&\\
E & :: = & \bigcirc & eval \; context\\
       & | & E.m(e)\\
       & | & p.m(E)\\
       & | & E.f\\
%       & | & E.f = e\\
%       & | & v.f = E\\
       & | & E \unlhd T\\
&&\\
d_v & ::= & \texttt{val} \; f : T = v & declaration \; value \\
  & |   & \texttt{def} \; m(x:T) = e : T &\\
  & |   & \texttt{type} \; L : T .. T = T &\\
%       & | & \texttt{var} \; x = \texttt{new} \; \{z \Rightarrow \overline{D}\} & eval \; context \\
%&&\\
%D & :: = & \texttt{val} \; f : T = E & decl \; eval \; context \\
%  & |   & \texttt{def} \; m(x:T) = e : T &\\
%  & |   & \texttt{type} \; L : T .. T &\\
%&&\\
%K & :: = & \varnothing \; | \; K, \; E & evaluation \; stack \\
%\Delta & :: = & \varnothing \; | \; \Delta,\; x \mapsto l & substitution \; context \\
%&&\\
%\Sigma & :: = & \varnothing \; | \; \Sigma,\; l : T & store \; context \\
%&&\\
%p & ::= & x & path \\
%  & | & l & \\
&&\\
\mu & :: = & \varnothing \; | \; \mu,\; l \mapsto \{z \Rightarrow \overline{d}\} & store \\
\Sigma & :: = & \varnothing \; | \; \Sigma,\; l : \{\texttt{z} \Rightarrow \overline{\sigma}\} & store \; type \\
\end{array}
\end{array}
\]
\caption{Syntax}
\label{f:syntax}
\end{figure}


%\begin{figure}[h]
%\begin{mathpar}
%\end{mathpar}
%\caption{Declaration Path Function}
%\label{f:path}
%\end{figure}
%\begin{figure}[h]
%\begin{mathpar}
%\inferrule
%  {}
%  {narrow(x) = x \\ narrow(v \unlhd T) = narrow(v)}
%\end{mathpar}
%\caption{Narrow Function}
%\label{f:narrow}
%\end{figure}

\subsection{Semantics}

\begin{figure}[h]
\hfill \fbox{$\mu \vdash v \leadsto l$}
\begin{mathpar}
\inferrule
  {}
  {\mu \vdash l \leadsto l}
  \quad (\textsc {L-Loc})
	\and
\inferrule
  {\mu \vdash v \leadsto l}
  {\mu \vdash v \unlhd T \leadsto l}
  \quad (\textsc {L-Type})
	\and
\inferrule
  {\mu \vdash v \leadsto l' \\
	\mu(l') = \{z \Rightarrow ..., \texttt{val} f : T = l, ...\}}
  {\mu \vdash v.f \leadsto l}
  \quad (\textsc {L-Path})
\end{mathpar}
\hfill \fbox{$\mu \vdash d_v \leadsto d$}
\begin{mathpar}
\inferrule
  {\mu \vdash v \leadsto l}
  {\mu \vdash \texttt{val} \; f : T = v \leadsto \texttt{val} \; f : T = l}
  \quad (\textsc {L-Val})
	\and
\inferrule
  {}
  {\mu \vdash \texttt{def} \; m : S(x:T) = e \leadsto \texttt{def} \; m(x:S) = e : T}
  \quad (\textsc {L-Def})
	\and
\inferrule
  {}
  {\mu \vdash \texttt{type} \; L : S .. U \leadsto \texttt{type} \; L : S .. U}
  \quad (\textsc {L-Type})
\end{mathpar}
\caption{Path Leads-to Judgement}
\label{f:path}
\end{figure}

\begin{figure}[h]
\hfill \fbox{$A; \Sigma; \Gamma_1 \vdash S <: T \dashv \Gamma_2$}
\begin{mathpar}
\inferrule
  {}
  {A; \Sigma; \Gamma \vdash T\; \texttt{<:}\; T \dashv \Gamma}
  \quad (\textsc {S-Refl})
	\and
\inferrule
  {(S <: T) \in A}
  {A; \Sigma; \Gamma_1 \vdash S\; \texttt{<:}\; T \dashv \Gamma_2}
  \quad (\textsc {S-Assume})
	\and
\inferrule
  {A' = A, (\{z \Rightarrow \overline{\sigma}\} <: \{z \Rightarrow \overline{\sigma}'\}) \\
  	A'; \Sigma; \Gamma_1, z : \{z \Rightarrow \overline{\sigma}\} \vdash \overline{\sigma} <:\; \overline{\sigma}'  \dashv \Gamma_2, z : \{z \Rightarrow \overline{\sigma}'\}}
  {A; \Sigma; \Gamma_1 \vdash \{z \Rightarrow \overline{\sigma}\}\; <:\; \{z \Rightarrow \overline{\sigma}'\}\dashv \Gamma_2}
  \quad (\textsc {S-Rec})
	\and
\inferrule
  {A; \Sigma; \Gamma_1 \vdash p \ni \texttt{type} \; L : S_1 .. U_1\\
  	A; \Sigma; \Gamma_2 \vdash p \ni \texttt{type} \; L : S_2 .. U_2\\
  	A; \Sigma; \Gamma_2 \vdash S_2 \; \texttt{<:}\; S_1 \dashv \Gamma_1 \\
  	A; \Sigma; \Gamma_1 \vdash U_1 \; \texttt{<:}\; U_2 \dashv \Gamma_2}
  {A; \Sigma; \Gamma_1 \vdash p.L \; \texttt{<:}\; p.L \dashv \Gamma_2}
  \quad (\textsc {S-Select-Refl})
	\and
\inferrule
  {A; \Sigma; \Gamma_1 \vdash p \ni \texttt{type} \; L : S .. U\\
  	A; \Sigma; \Gamma_1 \vdash S <: U \dashv \Gamma_1 \\
  	A; \Sigma; \Gamma_1 \vdash U <: U' \dashv \Gamma_2}
  {A; \Sigma; \Gamma_1 \vdash p.L\; <:\; U' \dashv \Gamma_2}
  \quad (\textsc {S-Select-Upper})
	\and
\inferrule
  {A; \Sigma; \Gamma_2 \vdash p \ni \texttt{type} \; L : S .. U \\
  	A; \Sigma; \Gamma_2 \vdash S <: U \dashv \Gamma_2 \\
  	A; \Sigma; \Gamma_1 \vdash S' <: S \dashv \Gamma_2}
  {A; \Sigma; \Gamma_1 \vdash S'\; <:\; p.L \dashv \Gamma_2}
  \quad (\textsc {S-Select-Lower})
	\and
\inferrule
  {}
  {A; \Sigma; \Gamma_1 \vdash T\; \texttt{<:}\; \top \dashv \Gamma_2}
  \quad (\textsc {S-Top})
	\and
\inferrule
  {}
  {A; \Sigma; \Gamma_1 \vdash \bot\; \texttt{<:}\; T \dashv \Gamma_2}
  \quad (\textsc {S-Bottom})
\end{mathpar}
\hfill \fbox{$A; \Sigma; \Gamma_1 \vdash \sigma <: \sigma' \dashv \Gamma_2$}
\begin{mathpar}
\inferrule
  {}
  {A; \Sigma; \Gamma_1 \vdash \texttt{val} \; f:T <: \texttt{val} \; f:T \dashv \Gamma_2}
  \quad (\textsc {S-Decl-Val})
	\and
\inferrule
  {A; \Sigma; \Gamma_2 \vdash S' <: S \dashv \Gamma_1 \\
  	A; \Sigma; \Gamma_1 \vdash T <: T' \dashv \Gamma_2}
  {A; \Sigma; \Gamma_1 \vdash \texttt{def} \; m:S \rightarrow T <: \texttt{def} \; m:S' \rightarrow T' \dashv \Gamma_2}
  \quad (\textsc {S-Decl-Def})
	\and
\inferrule
  {A; \Sigma; \Gamma_2 \vdash S' <: S \dashv \Gamma_1 \\
  	A; \Sigma; \Gamma_1 \vdash U <: U' \dashv \Gamma_2}
  {A; \Sigma; \Gamma_1 \vdash \texttt{type} \; L : S .. U \; <:\; \texttt{type} \; L : S' .. U' \dashv \Gamma_2}
  \quad (\textsc {S-Decl-Type})
\end{mathpar}
\caption{Subtyping}
\label{f:subtype}
\end{figure}

\begin{figure}[h]
\hfill \fbox{$A; \Sigma; \Gamma \vdash T \;  \textbf{wf}$}
\begin{mathpar}
\inferrule
  {A; \Sigma; \Gamma \vdash p \ni \texttt{type} \; L : S .. U \\
  	A; \Sigma; \Gamma \vdash \texttt{type} \; L : S .. U \; \textbf{wf} }
  {A; \Sigma; \Gamma \vdash p.L \; \textbf{wf}}
  \quad (\textsc {WF-Sel})
	\and
\inferrule
  {A; \Sigma; \Gamma,z:\{z \Rightarrow \overline{\sigma}\} \vdash \overline{\sigma} \; \textbf{wf} \\
  	\forall j \neq i, \; dom(\sigma_j) \neq dom(\sigma_i)}
  {A; \Gamma; \Sigma \vdash \{z \Rightarrow \overline{\sigma}\} \; \textbf{wf}}
  \quad (\textsc {WF-Rec})
%	\and
%\inferrule
%  {\Gamma \vdash S \;  \textbf{wf} \\
%  	\Gamma \vdash T \;  \textbf{wf}}
%  {\Gamma \vdash S \rightarrow T \;  \textbf{wf}}
%  \quad (\textsc {WF-Func})
	\and
\inferrule
  {}
  {A; \Sigma; \Gamma \vdash \top \;  \textbf{wf}}
  \quad (\textsc {WF-Top})
	\and
\inferrule
  {}
  {A; \Sigma; \Gamma \vdash \bot \;  \textbf{wf}}
  \quad (\textsc {WF-Bot})
\end{mathpar}
\hfill \fbox{$A; \Sigma; \Gamma \vdash \sigma \;  \textbf{wf}$}
\begin{mathpar}
\inferrule
  {A; \Sigma; \Gamma \vdash T : \textbf{wf}}
  {A; \Sigma; \Gamma \vdash \texttt{val} \; f:T \;  \textbf{wf}}
  \quad (\textsc {WF-Val})
	\and
\inferrule
  {A; \Sigma; \Gamma \vdash T : \textbf{wf} \\
  	A; \Sigma; \Gamma \vdash S : \textbf{wf}}
  {A; \Sigma; \Gamma \vdash \texttt{def} \; m:S \rightarrow T \;  \textbf{wf}}
  \quad (\textsc {WF-Def})
	\and
\inferrule
  {A; \Sigma; \Gamma \vdash S : \textbf{wfe} \; \vee \; S = \bot\\
  	A; \Sigma; \Gamma \vdash U : \textbf{wfe} \\
  	A; \Sigma; \Gamma \vdash S <: U}
  {A; \Sigma; \Gamma \vdash \texttt{type} \; L : S .. U \; \textbf{wf}}
  \quad (\textsc {WF-Type})
\end{mathpar}
\hfill \fbox{$A; \Sigma \vdash \Gamma \;  \textbf{wf}$}
\begin{mathpar}
\inferrule
  {\forall x \in dom(\Gamma), A; \Sigma; \Gamma \vdash \Gamma(x) \; \textbf{wf}}
  {\Sigma \vdash \Gamma \;  \textbf{wf}}
  \quad (\textsc {WF-Environment})
\end{mathpar}
\hfill \fbox{$\Sigma \;  \textbf{wf}$}
\begin{mathpar}
\inferrule
  {\forall l \in dom(\Sigma), \varnothing; \Sigma; \varnothing \vdash \Sigma(l) \; \textbf{wf}}
  {\Sigma \;  \textbf{wf}}
  \quad (\textsc {WF-Store-Context})
\end{mathpar}
\begin{mathpar}
\inferrule
  {\forall l \in dom(\mu), \varnothing; \Sigma; \varnothing \vdash \mu(l) : \Sigma(l)}
  {\mu : \Sigma}
  \quad (\textsc {WF-Store})
\end{mathpar}
\caption{Well-Formedness}
\label{f:wf}
\end{figure}

\begin{figure}[h]
\hfill \fbox{$A; \Sigma; \Gamma \vdash T \;  \textbf{wfe}$}
\begin{mathpar}
\inferrule
  {A; \Sigma; \Gamma \vdash T \; \textbf{wf} \\
  	A; \Sigma; \Gamma \vdash T \prec \overline{\sigma}}
  {A; \Sigma; \Gamma \vdash T \; \textbf{wfe}}
  \quad (\textsc {WFE})
\end{mathpar}
\caption{Well-Formed and Expanding Types}
\label{f:wfe}
\end{figure}

\begin{figure}[h]
\hfill \fbox{$\Sigma; \Gamma \vdash T \prec \overline{\sigma}$}
\begin{mathpar}
\inferrule
  {}
  {\Sigma; \Gamma \vdash \{z \Rightarrow \overline{\sigma}\} \prec_z \overline{\sigma}}
  \quad (\textsc {E-Rec})
	\and
\inferrule
  {\Sigma; \Gamma \vdash p \ni \texttt{type} \; L : S..U \\
  	\Sigma; \Gamma \vdash U \prec_z \overline{\sigma}}
  {\Sigma; \Gamma \vdash p.L \prec_z \overline{\sigma}}
  \quad (\textsc {E-Sel})
	\and
\inferrule
  {}
  {\Sigma; \Gamma \vdash \top \prec_z \varnothing}
  \quad (\textsc {E-Top})
\end{mathpar}
\caption{Expansion}
\label{f:exp}
\end{figure}
\begin{figure}[h]
\hfill \fbox{$\Sigma; \Gamma \vdash e \ni \sigma$}
\begin{mathpar}
\inferrule
  {\Sigma; \Gamma \vdash p : T \\
  	\Sigma; \Gamma \vdash T \prec_z \overline{\sigma}\\
  	\sigma_i \in \overline{\sigma}}
  {\Sigma; \Gamma \vdash p \ni [p/z]\sigma_i}
  \quad (\textsc {M-Path})
	\and
\inferrule
  {\Sigma; \Gamma \vdash e : T \\
  	\Sigma; \Gamma \vdash T \prec_z \overline{\sigma}\\
  	\sigma_i \in \overline{\sigma} \\
  	z \notin \sigma_i}
  {\Sigma; \Gamma \vdash e \ni \sigma_i}
  \quad (\textsc {M-Exp})
\end{mathpar}
\caption{Membership}
\label{f:mem}
\end{figure}

%\subsubsection{Typing}

\begin{figure}[h]
\hfill \fbox{$\Gamma; \Sigma \vdash e:T$}
\begin{mathpar}
\inferrule
  {x \in dom(\Gamma)}
  {	\Gamma; \Sigma \vdash x : \Gamma(x)}
  \quad (\textsc {T-Var})
	\and
\inferrule
  {	l \in dom(\Sigma)}
  {	\Gamma; \Sigma \vdash l : \Sigma(l)}
  \quad (\textsc {T-Loc})
	\and
\inferrule
  {\Gamma, z : \{z \Rightarrow \overline{\sigma}\}; \Sigma 
  \vdash \overline{d} : \overline{\sigma} \\
  	y \notin dom(\Gamma)}
  {	\Gamma; \Sigma\vdash \texttt{new} \; \{z \Rightarrow \overline{d}\} : 
  \{z \Rightarrow \overline{\sigma}\}}
  \quad (\textsc {T-New})
	\and
\inferrule
  {\Gamma; \Sigma \vdash e_0 \ni \texttt{def} \; m:S \rightarrow T \\
  	\Gamma; \Sigma \vdash e_1 : S \\
  	\Gamma; \Sigma \vdash T <: U}
  {	\Gamma; \Sigma \vdash e_0.m_U(e_1) : U}
  \quad (\textsc {T-Meth})
	\and
\inferrule
  {	\Gamma; \Sigma \vdash e : S \\
  	\Gamma; \Sigma \vdash e \ni \texttt{val} \; f:T}
  {	\Gamma; \Sigma \vdash e.f : T}
  \quad (\textsc {T-Acc})
%	\and
%\inferrule
%  {	\Gamma; E \vdash e_0 : T_0 \\
%  	T_0 \ni \texttt{def} \; f:T = e \\
%  	\Gamma; E \vdash e_1 : T}
%  {	\Gamma; E \vdash e_0.f = e_1 : T}
%  \quad (\textsc {T-Assign})
%	\and
	\and
\inferrule
  {	\Gamma; \Sigma \vdash e : T}
  {	\Gamma; \Sigma \vdash e \unlhd T : T}
  \quad (\textsc {T-Type})
	\and
\inferrule
  {	\Gamma; \Sigma \vdash e : S \\
  	\Gamma; \Sigma \vdash S <: T}
  {	\Gamma; \Sigma \vdash e : T}
  \quad (\textsc {T-Sub})
\end{mathpar}
\caption{Expression Typing}
\label{f:e_typ}
\end{figure}
\begin{figure}[h]
\hfill \fbox{$\Gamma; E \vdash d:\sigma$}
\begin{mathpar}
\inferrule
  {\Gamma; \Sigma \vdash e : T}
  {\Gamma; \Sigma \vdash \texttt{def} \; f:T = e : \texttt{def} \; f:T}
  \quad (\textsc {T-Decl-Var})
	\and
\inferrule
  {\Gamma, x : S; \Sigma \vdash e_0 : T}
  {\Gamma; \Sigma \vdash \texttt{def} \; m(x:S) = e : T : \texttt{def} \; m:S \rightarrow T}
  \quad (\textsc {T-Decl-Def})
	\and
\inferrule
  {\Gamma; \Sigma \vdash \texttt{type} \; L : S .. U \; \textbf{wf} }
  {\Gamma; \Sigma \vdash \texttt{type} \; L : S .. U : \texttt{type} \; L : S .. U}
  \quad (\textsc {T-Decl-Type})
\end{mathpar}
\caption{Declaration Typing}
\label{f:d_typ}
\end{figure}
%\begin{figure}[h]
%\hfill \fbox{$\Gamma \vdash \mu:\Sigma$}
%\begin{mathpar}
%\inferrule
%  {\forall x \in dom(\mu), \; \mu(x)= \{\texttt{z} \Rightarrow \overline{d}\} \\
%  	\Gamma(x) = \{\texttt{z} \Rightarrow \overline{\sigma}\} \\
%  	\Gamma \vdash \overline{d} : \overline{\sigma}}
%  {\Gamma \vdash \mu}
%  \quad (\textsc {T-Store})
%\end{mathpar}
%\caption{Store Typing}
%\label{f:s_typ}
%\end{figure}

\begin{figure}[h]
\hfill \fbox{$\mu \; | \; e \; \rightarrow \mu' \; | \; e'$}
\begin{mathpar}
\inferrule
  {l \notin dom(\mu) \\
  	\mu' = \mu, l \mapsto \{\texttt{z} \Rightarrow \overline{d_v}\}}
  {\mu \; | \; \texttt{new} \; \{\texttt{z} \Rightarrow \overline{d_v}\} \; \rightarrow \mu' \; | \; l}
  \quad (\textsc {R-New})
  \and
\inferrule
  {\mu : \Sigma \\
   \mu; \Sigma \vdash v_1 \leadsto_{m} e}
  {\mu \; | \; v_1.m(v_2) \;\rightarrow \mu \; | \; [l/\texttt{z},v_2 \unlhd S/x]e}
  \quad (\textsc {R-Meth})
  \and
\inferrule
  {	\mu \; | \; e \; \rightarrow \; \mu' \; | \; e'}
  {\mu \; | \; E[e] \; \rightarrow \mu' \; | \; E[e']}
  \quad (\textsc {R-Context})
\end{mathpar}
\caption{Reduction}
\label{f:red}
\end{figure}





\bibliographystyle{plain}
\bibliography{bib}

\end{document}