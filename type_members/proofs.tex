\documentclass{llncs}

\usepackage{listings}
\usepackage{amssymb}
\usepackage[margin=.9in]{geometry}
%\usepackage{amsmath}
%\usepackage{amsthm}
\usepackage{mathpartir}
\usepackage{color,soul}




\lstdefinestyle{custom_lang}{
  xleftmargin=\parindent,
  showstringspaces=false,
  basicstyle=\ttfamily,
  keywordstyle=\bfseries
}

\lstset{emph={%  
    val, def, type, new, z%
    },emphstyle={\bfseries \tt}%
}

\begin{document}
\hl{*Note highlighted text implies more work is needed.}
\section{Type Safety}

\subsection{Subtype Transitivity and Environment Narrowing}

%The proof of \emph{Subtype Transitivity} in $F_{<:}$ requires 
%the simultaneous proof \emph{Environment Narrowing} \cite{Ayedemir:2005}. This is 
%done by induction on the middle type.


%\begin{lemma}[Subtype Definition]\label{lem:sub}
%If $\Gamma \vdash S \prec_z \overline{\sigma_1}$,
%	$\Gamma \vdash T \prec_z \overline{\sigma_2}$ and
%	$\forall \sigma_i \in \overline{\sigma_2} \;
%	 \exists \sigma_j \in \overline{\sigma_1}:
%	 \Gamma, z : \{z \Rightarrow \overline{\sigma_1}\} 
%	 \vdash \sigma_j <: \sigma_i$ then
%	$\Gamma \vdash S <: T$
%\end{lemma}
%\begin{proof}
%Can't be shown.
%\end{proof}

%\begin{theorem}[Subtype Reflexivity]
% 	If $\Gamma \vdash T \; \textbf{\tt{wfe}}$ then $\Gamma \vdash T <: T$
%\end{theorem}
%\begin{proof}
%This is easy to demonstrate using \textsc{S-Struct}.
%\end{proof}

%\begin{lemma}[Type Equality]\label{lem:sub_eq}
%If $\Gamma \vdash S <: T$ and
%	$\Gamma \vdash T <: S$ then
%	$S = T$
%\end{lemma}
%\begin{proof}
%By induction on the derivation of $\Gamma \vdash S <: T$.
%\begin{case}[\textsc{S-Refl}]
%The result follows immediately. 
%\end{case}
%\begin{case}[\textsc{S-Rec}]
%\begin{mathpar}
%\inferrule
%  {S = \{z \Rightarrow \overline{\sigma}\} \\
%	T = \{z \Rightarrow \overline{\sigma'}\} \\
%	\forall \sigma_i' \in \overline{\sigma'}, \; \exists \; \sigma_i \in \overline{\sigma} \; : \; \Gamma, z : \{z \Rightarrow \overline{\sigma}\} \vdash \sigma_i <:\; \sigma_i'}
%  {}
%\end{mathpar}
%Since $\Gamma \vdash \{z \Rightarrow \overline{\sigma}\} <: \{z \Rightarrow \overline{\sigma'}\}$, by case analysis on this 
%derivation, it follows that either $S = T$ by \textsc{S-Refl} in which 
%the result is immediate, or by \textsc{S-Rec} the following holds.
%\begin{mathpar}
%\inferrule
%  {\forall \sigma_i \in \overline{\sigma}, \; \exists \; \sigma_i' \in \overline{\sigma'} \; : \; \Gamma, z : \{z \Rightarrow \overline{\sigma'}\} \vdash \sigma'_i <:\; \sigma_i}
%  {}
%\end{mathpar}
%We can now apply our induction hypothesis to each $\sigma_i$ 
%and $\sigma_i'$ pair, and infer that $\sigma_i = \sigma_i'$ 
%We can further say that the subtype relation is bijective from 
%$\overline{\sigma}$ to $\overline{\sigma'}$ due to the uniqueness 
%of declaration label names
%and consequently that $\overline{\sigma} = \overline{\sigma'}$. 
%i.e. that $\{z \Rightarrow \overline{\sigma}\} = \{z \Rightarrow \overline{\sigma'}\}$,
%or $S = T$.
%\end{case}
%\begin{case}[\textsc{S-Select-Upper}]
%\begin{mathpar}
%\inferrule
%  {S = p.L \\
%	\Gamma \vdash p \ni \texttt{type} \; L : S' .. U'\\
%  	\Gamma \vdash S' <: U' \\
%  	\Gamma \vdash U' <: T}
%  {}
%\end{mathpar}
%\end{case}
%\begin{case}[\textsc{S-Selcet-Lower}]
%\begin{mathpar}
%\inferrule
%  {T = p.L \\
%	\Gamma \vdash p \ni \texttt{type} \; L : S' .. U'\\
%  	\Gamma \vdash S' <: U' \\
%  	\Gamma \vdash S <: S'}
%  {}
%\end{mathpar}
%\end{case}
%\begin{case}[\textsc{S-Top}]
%\begin{mathpar}
%\inferrule
%  {T = \top}
%  {}
%\end{mathpar}
%We need to know show that $S = \top$, or otherwise forms a 
%contradiction. Case analysis on the derivation of $\Gamma \vdash T <: S$.
%\begin{itemize}
%\item[]  \textit{Subcase 1} (\textsc{S-Refl}): 
%Trivial.
%\\
%\item[]  \textit{Subcase 2} (\textsc{S-Selecet-Lower}):
%\begin{mathpar}
%\inferrule
%  {S = p.L\\
%	\Gamma \vdash p \ni \texttt{type} \; L : S' .. U'\\
%  	\Gamma \vdash S' <: U' \\
%  	\Gamma \vdash \top <: S'}
%  {}
%\end{mathpar} 
%\\
%\item[]  \textit{Subcase 3} (\textsc{S-Top}):
%Trivial.
%\\
%\end{itemize}
%\end{case}
%\begin{case}[\textsc{S-Bottom}]
%\begin{mathpar}
%\inferrule
%  {S = \bot \\}
%  {}
%\end{mathpar}
%\end{case}
%
%\end{proof}

%---------------------- Narrowing ----------------------%
First we define the following \emph{Subtype Transitivity} judgement 
              \begin{mathpar}
\inferrule
  {}
  {\Gamma \vdash S \; <:^* \; S}
  \quad (\textsc {S\textsuperscript{*}-Refl})
	\and
\inferrule
  {\Gamma \vdash S \; <:^* \; T \\
	\Gamma \vdash T \; <: \; U}
  {\Gamma \vdash S \; <:^* \; U}
  \quad (\textsc {S\textsuperscript{*}-Trans})
\end{mathpar}
This is used to admit transitivity when proving a relaxed version of
Environment Narrowing below, making Narrowing independent of 
Subtype Transitivity.
\begin{theorem}[Environment Narrowing*]
If $\Gamma_a, (x : U), \Gamma_b \vdash T <: T'$ and 
   	$\Gamma_a \vdash S <: U$ then
	$\Gamma_a, (x : S), \Gamma_b \vdash T <:^* T'$
\end{theorem}
\begin{proof}
By induction on the derivation of $\Gamma_a, (z : U), \Gamma_b \vdash T <: T'$.
\begin{case}[\textsc{S-Refl}]
\begin{mathpar}
\inferrule
  {T = T'}
  {}
\end{mathpar}
Trivial.
\end{case}
\begin{case}[\textsc{S-Rec}]
\begin{mathpar}
\inferrule
  {T = \{z \Rightarrow \overline{\sigma}\} \\
  	T' = \{z \Rightarrow \overline{\sigma}'\} \\
  	\forall \sigma_i' \in \overline{\sigma}', \; \exists \; \sigma_i \in \overline{\sigma} \; st \; \Gamma_a, (x : U), \Gamma_b,(z : \{z \Rightarrow \overline{\sigma}\}) \vdash \sigma_i <:\; \sigma_i'}
  {}
\end{mathpar}
Applying our induction hypotheses to the smaller derivation of 
$\Gamma_a, (x : U), \Gamma_b,(z : \{z \Rightarrow \overline{\sigma}\}) 
\vdash \sigma_i <:\; \sigma_i'$ for each $\sigma_i$ and $\sigma_i'$, 
we can show that 
$\Gamma_a, (x : S), \Gamma_b,(z : \{z \Rightarrow \overline{\sigma}\}) 
\vdash \sigma_i <:^*\; \sigma_i'$.
We can use this and \textsc{S-Rec} to construct a series of record types 
such that $\Gamma_a, (x : S), \Gamma_b \vdash 
\{z \Rightarrow \overline{\sigma}\} <: 
\{z \Rightarrow \overline{\sigma}_0\} <: ...
<: \{z \Rightarrow \overline{\sigma}_n\} <:
\{z \Rightarrow \overline{\sigma}'\}$.
i.e that 
$\Gamma_a, (x : S), \Gamma_b \vdash 
\{z \Rightarrow \overline{\sigma}\}\; <:^*\; 
\{z \Rightarrow \overline{\sigma}'\}$.
\end{case}
\begin{case}[\textsc{S-Select-Upper}]
\begin{mathpar}
\inferrule
  {T = p.L \\
  	\Gamma_a, (x : U), \Gamma_b \vdash p \ni \texttt{type} \; L : S' .. U' \\
  	\Gamma_a, (x : U), \Gamma_b \vdash S' <: U' \\
  	\Gamma_a, (x : U), \Gamma_b \vdash U' <: T'}
  {}
\end{mathpar}
\hl{It can be shown that 
$\Gamma_a, (x : S), \Gamma_b \vdash p \ni \texttt{type} \; L : S'' .. U''$
where 
$\Gamma_a, (x : S), \Gamma_b \vdash S' <: S'' <: U'' <: U'$}.
Further, by applying our induction hypothesis to the smaller 
derivation of $\Gamma_a, (x : U), \Gamma_b \vdash U' <: T'$ 
to get $\Gamma_a, (x : S), \Gamma_b \vdash U' <: T'$ we 
we can construct the chain 
$\Gamma_a, (x : S), \Gamma_b \vdash U'' <: U' <: T'$ 
and thus by \textsc{S-Select-Upper} and 
\textsc {S\textsuperscript{*}-Trans} that 
$\Gamma_a, (x : S), \Gamma_b \vdash p.L <:^* T'$ completing the case.
\end{case}
\begin{case}[\textsc{S-Select-Lower}]
\begin{mathpar}
\inferrule
  {T' = p.L \\
  	\Gamma_a, (x : U), \Gamma_b \vdash p \ni \texttt{type} \; L : S' .. U' \\
  	\Gamma_a, (x : U), \Gamma_b \vdash S' <: U' \\
  	\Gamma_a, (x : U), \Gamma_b \vdash T <: S'}
  {}
\end{mathpar}
\hl{TODO: Complete reasoning in similar manner as \textsc{S-Select-Upper}}
\end{case}
\begin{case}[\textsc{S-Top}]
\begin{mathpar}
\inferrule
  {T' = \top}
  {}
\end{mathpar}
Trivial.
\end{case}
\begin{case}[\textsc{S-Bottom}]
\begin{mathpar}
\inferrule
  {T = \bot}
  {}
\end{mathpar}
Trivial.
\end{case}
\begin{case}[\textsc{S-Decl-Var}]
\begin{mathpar}
\inferrule
  {\sigma = \texttt{val} \; f:T \\
  	\sigma' = \texttt{val} \; f:T}
  {}
\end{mathpar}
Trivial.
\end{case}
\begin{case}[\textsc{S-Decl-Meth}]
\begin{mathpar}
\inferrule
  {\sigma = \texttt{def} \; m:S \rightarrow T \\
  	\sigma' = \texttt{def} \; m:S' \rightarrow T' \\
  	\Gamma, (x : U), \Gamma_b \vdash T <: T' \\
  	\Gamma, (x : U), \Gamma_b \vdash S' <: S}
  {}
\end{mathpar}
Applying the induction hypothesis to the smaller derivations of 
$\Gamma, (x : U), \Gamma_b \vdash T <: T'$ and
$\Gamma, (x : U), \Gamma_b \vdash S' <: S$ we get 
\begin{mathpar}
\inferrule
  {\Gamma, (x : S), \Gamma_b \vdash T <:^* T' \\
  	\Gamma, (x : S), \Gamma_b \vdash S' <:^* S}
  {}
\end{mathpar}
This means we can construct two subtype chains: 
\begin{mathpar}
\inferrule
  {\Gamma, (x : S), \Gamma_b \vdash T <: T_0 <: ... <: T_m <: T' \\
  	\Gamma, (x : S), \Gamma_b \vdash S' <: S_0 <: ... <: S_n <: S}
  {}
\end{mathpar}
Using these we can construct a similar subtype chain 
\begin{mathpar}
\inferrule
  {\Gamma, (x : S), \Gamma_b \vdash 
\texttt{def} \; m:S \rightarrow T <: \texttt{def} \; m:S_n \rightarrow T_0 
<: ... <: \texttt{def} \; m:S_0 \rightarrow T_m <: 
\texttt{def} \; m:S' \rightarrow T'}
  {}
\end{mathpar}
i.e. $\Gamma, (x : S), \Gamma_b \vdash 
\texttt{def} \; m:S \rightarrow T <:^* 
\texttt{def} \; m:S' \rightarrow T'$
\end{case}
\begin{case}[\textsc{S-Decl-Type}]
\begin{mathpar}
\inferrule
  {\sigma = \texttt{type} \; L : S .. U \\
  	\sigma' = \texttt{type} \; L : S' .. U' \\
  	\Gamma, (x : U), \Gamma_b \vdash S' <: S \\
  	\Gamma, (x : U), \Gamma_b \vdash U <: U'}
  {}
\end{mathpar}
In a similar manner to the \emph{Case} for \textsc{S-Decl-Meth}, we can use 
the induction hypothesis to derive the following.
\begin{mathpar}
\inferrule
  {\Gamma, (x : S), \Gamma_b \vdash S' <:^* S \\
  	\Gamma, (x : S), \Gamma_b \vdash U <:^* U'}
  {}
\end{mathpar}
Similarly we can derive the following subtype chains,
\begin{mathpar}
\inferrule
  {\Gamma, (x : S), \Gamma_b \vdash S' <: S_0 <: ... <: S_m <: S \\
  	\Gamma, (x : S), \Gamma_b \vdash U <: U_0 <: ... <: U_n <: U'}
  {}
\end{mathpar}
and subsequently the following chain of declaration subtypes.
\begin{mathpar}
\inferrule
  {\Gamma, (x : S), \Gamma_b \vdash 
\texttt{type} \; L : S .. U <: \texttt{type} \; L : S_m .. U_0 
<: ... <: \texttt{type} \; L : S_0 .. U_n <: 
\texttt{type} \; L : S' .. U'}
  {}
\end{mathpar}
This gives us the desired result: 
$\Gamma, (x : S), \Gamma_b \vdash 
\texttt{type} \; L : S .. U <:^* \texttt{type} \; L : S' .. U'$.
\end{case}
\end{proof}
\qed

%---------------- Subtype Chain Construction ----------------%
We now show that for any chain of \texttt{wfe} types, $S <:^* U$,
it's possible to construct a chain that contains no selection types.
\begin{lemma} \label{lem:subtype_chain}
If $\Gamma \vdash S, U \; \textbf{\tt{wfe}}$ where
	$\Gamma \vdash S <:^* U$ then we can construct a subtype sequence
	$\Gamma \vdash S <: T'_0 <: ... <: T'_m <: U$ such that 
	$\forall i \in [0,m], T'_i \neq p.L$.
\end{lemma}
\begin{proof}
By induction on the derivation of $\Gamma \vdash S <:^* U$.
\begin{case}[\textsc {S\textsuperscript{*}-Refl}]
\begin{mathpar}
\inferrule
  {U = S}
  {}
\end{mathpar}
Trivial.
\end{case}
\begin{case}[\textsc {S\textsuperscript{*}-Trans}]
\begin{mathpar}
\inferrule
  {\Gamma \vdash S \; <:^* \; T \\
	\Gamma \vdash T \; <: \; U}
  {}
\end{mathpar}
Applying the induction hypothesis to the smaller 
derivation of $\Gamma \vdash S \; <:^* \; T$ we 
can show that there exists a subtype sequence 
$\Gamma \vdash S <: T_0 <: ... <: T_n <: T$ where 
for all $i \in [0,n]$, $T_i \neq p.L$ for any $p$ and $L$.

Now we need to show that we can derive a similar 
sequence $\Gamma \vdash T_n <: T_{n+1} <: ... <: T_m <: U$ 
to complete the sequence and replace $T$. To do this we 
do a case analysis on the structure of $T$.
\begin{itemize}
\item[]  \textit{Subcase 1} ($\{z \Rightarrow \overline{\sigma}\}$):
\begin{mathpar}
\inferrule
  {T = \{z \Rightarrow \overline{\sigma}\}}
  {}
\end{mathpar}
Trivial.
\item[]  \textit{Subcase 2} ($p.L$):
\begin{mathpar}
\inferrule
  {T = p.L}
  {}
\end{mathpar}
\hl{TODO}
\item[]  \textit{Subcase 3} ($\top$):
\begin{mathpar}
\inferrule
  {T = \top}
  {}
\end{mathpar}
Trivial.
\item[]  \textit{Subcase 4} ($\bot$):
\begin{mathpar}
\inferrule
  {T = \bot}
  {}
\end{mathpar}
\hl{TODO: Solve issue with expanding types and $\bot$}
\end{itemize} 
\end{case}
\end{proof}
\qed

%---------------------- Transitivity ----------------------%
Subtype Transitivity 

Poplmark \cite{Ayedemir:2005}

\begin{theorem}[Subtype Transitivity]
If $\Gamma \vdash S <:^* U$ then
	$\Gamma \vdash S <: U$
\end{theorem}
\begin{proof}
Expand $\Gamma \vdash S <:^* U$ to the following chain:
\begin{mathpar}
\inferrule
  {\Gamma \vdash S <: T_0 <: ... <: T_n <: U}
  {}
\end{mathpar}
Now proceed by breaking the proof into two parts. First the case where 
$\forall i \in [0,n], T_i \neq p.L$. Secondly we show that if the chain 
does contain a selection type, we can construct another chain that contains 
no such selection type.
\begin{case}
Begin by induction on the derivation of $\Gamma \vdash S <:^* U$.
\begin{itemize}
\item[]  \textit{Subcase 1} (\textsc {S\textsuperscript{*}-Refl}):
Trivial.
\item[]  \textit{Subcase 2} (\textsc {S\textsuperscript{*}-Trans}):
\begin{mathpar}
\inferrule
  {\Gamma \vdash S <:^* T_n \\
  	\Gamma \vdash T_n <: U}
  {}
\end{mathpar}
Applying the induction hypothesis to the smaller derivation 
of $\Gamma \vdash S <:^* T$ we get $\Gamma \vdash S <: T$. 
This gives us the more traditional form of \emph{Subtype Transitivity}.
\begin{mathpar}
\inferrule
  {\Gamma \vdash S <: T \\
   	\Gamma \vdash T <: U}
  {\Gamma \vdash S <: U}
\end{mathpar}
We can now proceed by induction on the derivation of 
$\Gamma \vdash S <: T$ keeping in mind that we have restricted 
$T$ to only non-selection types (meaning \textsc{S-Select-Lower} 
is not considered).
\begin{itemize}
\item[]  \textit{Subsubcase 1} (\textsc {S-Refl}):
\begin{mathpar}
\inferrule
  {S = T}
  {}
\end{mathpar}
Trivial.
\item[]  \textit{Subsubcase 2} (\textsc {S-Rec}):
\begin{mathpar}
\inferrule
  {S =  \{z \Rightarrow \overline{\sigma}\} \\
  	T =  \{z \Rightarrow \overline{\sigma}'\} \\
   	\forall \sigma_i' \in \overline{\sigma}', \; \exists \; \sigma_i \in \overline{\sigma} \; st \; \Gamma, z : \{z \Rightarrow \overline{\sigma}\} \vdash \sigma_i <:\; \sigma_i'}
  {}
\end{mathpar}
\hl{TODO: complete}
\item[]  \textit{Subsubcase 3} (\textsc {S-Select-Upper}):
\begin{mathpar}
\inferrule
  {S =  p.L \\
  	\Gamma \vdash p \ni \texttt{type} \; L : S' .. U'\\
  	\Gamma \vdash S' <: U' \\
  	\Gamma \vdash U' <: T}
  {}
\end{mathpar}
Applying the induction hypothesis to the smaller derivations 
$\Gamma \vdash U' <: T$ and $\Gamma \vdash T <: U$ we get $\Gamma \vdash U' <: U$. 
By \textsc{S-Select-Upper} is follows that $\Gamma \vdash S <: U$.
\item[]  \textit{Subsubcase 4} (\textsc {S-Top}):
\begin{mathpar}
\inferrule
  {T = \top}
  {}
\end{mathpar}
\hl{TODO: Show that if $\Gamma \vdash S <: T, S \prec \overline{\sigma}, 
T \prec \overline{\sigma}'$ then 
$\Gamma \vdash \overline{\sigma} <: \overline{\sigma}'$}\\
$\Gamma \vdash U \; \texttt{wfe} \Rightarrow
\exists \overline{\sigma}: \; \Gamma \vdash U \prec \overline{\sigma}$. 
Since $\Gamma \vdash \top \prec \varnothing$, it follows that 
for every declaration $\sigma \in \overline{\sigma}$, 
$\exists \sigma' \in \varnothing$. This implies that 
$\overline{\sigma} = \varnothing$, and subsequently that 
$U = \top$. The desired result follows immediately from \textsc{S-Top}.
\item[]  \textit{Subsubcase 5} (\textsc {S-Bottom}):
\begin{mathpar}
\inferrule
  {S = \bot}
  {}
\end{mathpar}
Trivial.
\end{itemize}
\end{itemize}
\end{case}
\begin{case}
In the case where our subtype chain $\Gamma \vdash S <:^* U$ 
contains a selection type, we can use Lemma \ref{lem:subtype_chain}
to construct one that does not. We can now apply our reasoning from the 
previous case to the new subtype chain to show that $\Gamma \vdash S <: U$.
\end{case}
\end{proof}
\qed

\subsection{Subject Reduction}

%---------------------- Preservation ----------------------%
\begin{theorem}[Preservation]
If $\Gamma; \Sigma \vdash e : T$, 
   	$\mu \; | \; e \; \rightarrow \mu' \; | \; e'$ where
	$\Sigma \vdash \mu \; \tt{\bf{wf}}$ then 
 	$\exists \Sigma'$ s.t. 
	$\Sigma'$ extends $\Sigma$, 
	$\Sigma' \vdash \mu' \; \tt{\bf{wf}}$, 
	$\Gamma; \Sigma' \vdash e' : T$.
\end{theorem}
\begin{proof}
By induction on the derivation of $\Gamma \vdash e : T$.
\begin{case}[\textsc{T-Var}]
\begin{mathpar}
\inferrule
  {x \in dom(\Gamma)}
  {	\Gamma; \Sigma \vdash x : \Gamma(x)}
\end{mathpar}
Variables are not reducible, and therefore contradict 
the initial premise 	$\mu \; | \; e \; \rightarrow \mu' \; | \; e'$.
\end{case}
\begin{case}[\textsc{T-Loc}]
\begin{mathpar}
\inferrule
  {	l \in dom(\Sigma)}
  {	\Gamma; \Sigma \vdash l : \Sigma(l)}
  \quad (\textsc {T-Loc})
\end{mathpar}
Again, there is no reduction for locations, and so contradicts 
the initial premise.
\end{case}
\begin{case}[\textsc{T-New}]
\begin{mathpar}
\inferrule
  {\Gamma, z : \{z \Rightarrow \overline{\sigma}\}; \Sigma 
  \vdash \overline{d} : \overline{\sigma} \\
  	y \notin dom(\Gamma)}
  {	\Gamma; \Sigma\vdash \texttt{new} \; \{z \Rightarrow \overline{d}\} : 
  \{z \Rightarrow \overline{\sigma}\}}
\end{mathpar}
By \textsc{R-New}, we get the following reduction: 
\begin{mathpar}
\inferrule
  {l \notin dom(\mu) \\
  	\mu' = \mu, l \mapsto \{\texttt{z} \Rightarrow \overline{d}_v\}}
  {\mu \; | \; \texttt{new} \; \{\texttt{z} \Rightarrow \overline{d}_v\} \; \rightarrow \mu' \; | \; l}
\end{mathpar}
$\texttt{new} \; \{z \Rightarrow \overline{d}\}$
reduces to $l$.
Let $\Sigma' = \Sigma,(l:\{z \Rightarrow \overline{\sigma}\})$.
Now, by \textsc{T-Loc} $\Gamma; \Sigma' \vdash l : \{\texttt{z} \Rightarrow \overline{d}\}$.
\end{case}
\begin{case}[\textsc{T-Meth}]
\begin{mathpar}
\inferrule
  {\Gamma; \Sigma \vdash e_0 \ni \texttt{def} \; m:S \rightarrow T \\
  	\Gamma; \Sigma \vdash e_1 : S \\
  	\Gamma; \Sigma \vdash T <: U}
  {	\Gamma; \Sigma \vdash e_0.m_U(e_1) : U}
\end{mathpar}
Case analysis on the reduction judgement gives the following subcases
\begin{itemize}
\item[]  \textit{Subcase 1} (R-Meth):
\item[]  \textit{Subcase 2} (R-Context):
\begin{itemize}
\item[]  \textit{Subsubcase 1} (Receiver Reduction):
\item[]  \textit{Subsubcase 2} (Argument Reduction):
\end{itemize}
\end{itemize}
\end{case}
\begin{case}[\textsc{T-Acc}]
\begin{mathpar}
\inferrule
  {	\Gamma; \Sigma \vdash e : S \\
  	\Gamma; \Sigma \vdash e \ni \texttt{val} \; f:T}
  {	\Gamma; \Sigma \vdash e.f : T}
\end{mathpar}
Since there is no \textsc{R-Field} reduction rule, 
for $e.f$ to reduce, it must be as a result of a context 
reduction of $e$.
\begin{mathpar}
\inferrule
  {	\mu \; | \; e \; \rightarrow \mu' \; | \; e'}
  {	\mu \; | \; e.f \; \rightarrow \mu' \; | \; e'.f}
\end{mathpar}
Applying the induction hypothesis to
$\Gamma \vdash e : S$, it follows that 
$\exists \Gamma': \;\Gamma' \vdash e : S$, $\Gamma'$ extends 
$\Gamma$ and $\Gamma' \vdash \mu'$. \hl{We 
can then show that $\Gamma' \vdash e' \ni \texttt{val} \; f : T$} 
It follows from \textsc{T-Acc} that $\Gamma' \vdash e'.f : T$.
\end{case}
\begin{case}[\textsc{T-Type}]
\begin{mathpar}
\inferrule
  {	\Gamma; \Sigma \vdash e : T}
  {	\Gamma; \Sigma \vdash e \unlhd T : T}
\end{mathpar}
As with \textsc{T-Acc}, there is no \textsc{R-Type} reduction rule,
implying reduction is as a result of a context reduction.
\begin{mathpar}
\inferrule
  {	\mu \; | \; e \; \rightarrow \mu' \; | \; e'}
  {	\mu \; | \; e \unlhd T \; \rightarrow \mu' \; | \; e' \unlhd T}
\end{mathpar}
We can then apply the induction hypothesis to $\Gamma \vdash e : T$ 
to get $\Gamma' \vdash e' : T$ where $\Gamma'$ extends $\Gamma$ and 
$\Gamma' \vdash \mu'$. Thus by \textsc{T-Type}, 
$\Gamma' \vdash e' \unlhd T : T$.

\end{case}
\begin{case}[\textsc{T-Sub}]
\begin{mathpar}
\inferrule
  {	\Gamma; \Sigma \vdash e : S \\
  	\Gamma; \Sigma \vdash S <: T}
  {	\Gamma; \Sigma \vdash e : T}
  \and
  {	\mu \; | \; e \; \rightarrow \mu' \; | \; e'}
  {}
\end{mathpar}
We can apply the induction hypothesis to $\Gamma \vdash e : S$ 
to get $\Gamma' \vdash e' : S$ where $\Gamma'$ extends $\Gamma$ 
and $\Gamma' \vdash \mu'$. Since $\Gamma'$ extends $\Gamma$, 
and $\Gamma \vdash S <: T$ it follows that 
$\Gamma' \vdash S <: T$. Therefore, $\Gamma' \vdash e : T$ by \textsc{T-Sub}.
\end{case}
\end{proof}
\qed


%By induction on the derivation of $\mu \; | \; e \; \rightarrow \mu' \; | \; e'$.
%\begin{case}[\textsc{R-New}]
% \begin{mathpar}
%\inferrule
%  {\mu' = \mu, y \mapsto \{\texttt{z} \Rightarrow value_d(\mu,\overline{d})\}}
%  {\mu \; | \; \texttt{var} \; y = \texttt{new} \; \{\texttt{z} \Rightarrow \overline{d}\} \; \rightarrow \mu' \; | \; y}
%\end{mathpar}
%
%\end{case}
%\begin{case}[\textsc{R-Meth}]
%\begin{mathpar}
%\inferrule
%  {\mu(path(\mu, p_1)) = \{\texttt{z} \Rightarrow ...,m:T(x:S)=e,...\}}
%  {\mu \; | \; p_1.m_U(p_2) \;\rightarrow \mu \; | \; [p_1/\texttt{z},p_2 \unlhd S/x]e \unlhd U}
%\end{mathpar}
%\end{case}
%\begin{case}[\textsc{R-Context}]
%\begin{mathpar}
%\inferrule
%  {	\mu \; | \; e \; \rightarrow \; \mu' \; | \; e'}
%  {\mu \; | \; E[e] \; \rightarrow \mu' \; | \; E[e']}
%\end{mathpar}
%\end{case}

%---------------------- Progress ----------------------%
\subsection{Progress}
\begin{theorem}[Progress]
If $\Gamma \vdash e : T$, then either
\begin{enumerate}
\item e is a value, or
\item $\forall \mu$ s.t.
		   $\Gamma \vdash \mu$,
         $\exists e'$ and $\mu'$ s.t. 
         $\mu \; | \; e \; \rightarrow \mu' \; | \; e'$
\end{enumerate}
\end{theorem}
\qed 






\bibliographystyle{plain}
\bibliography{bib}

\end{document}