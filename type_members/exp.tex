\begin{figure}[h]
\hfill \fbox{$\Gamma \vdash T \prec \overline{\sigma}$}
\begin{mathpar}
\inferrule
  {}
  {\Gamma \vdash \{\texttt{z} \Rightarrow \overline{\sigma}\} \prec \overline{\sigma}}
  \quad (\textsc {E-Rec})
	\and
\inferrule
  {\Gamma \vdash p \ni \texttt{type} \; L : S..U \\
  	\Gamma \vdash U \prec \overline{\sigma}}
  {\Gamma \vdash p.L \prec \overline{\sigma}}
  \quad (\textsc {E-Sel})
	\and
\inferrule
  {}
  {\Gamma \vdash \top \prec \varnothing}
  \quad (\textsc {E-Top})
\end{mathpar}
\caption{Expansion}
\label{f:exp}
\end{figure}
\begin{figure}[h]
\hfill \fbox{$\Gamma \vdash e \ni \sigma$}
\begin{mathpar}
\inferrule
  {\Gamma \vdash p : T \\
  	\Gamma \vdash T \prec \overline{\sigma}\\
  	\sigma_i \in \overline{\sigma}}
  {\Gamma \vdash p \ni [p/\texttt{z}]\sigma_i}
  \quad (\textsc {M-Path})
	\and
\inferrule
  {\Gamma \vdash e : T \\
  	\Gamma \vdash T \prec \overline{\sigma}\\
  	\sigma_i \in \overline{\sigma} \\
  	\texttt{z} \notin \sigma_i}
  {\Gamma \vdash e \ni \sigma_i}
  \quad (\textsc {M-Exp})
\end{mathpar}
\caption{Membership}
\label{f:mem}
\end{figure}
%\begin{figure}[h]
%\hfill \fbox{$K \vdash p \equiv p'$}
%\begin{mathpar}
%\inferrule
%  {K \vdash p_1 \equiv p_2}
%  {K \vdash p_2 \equiv p_1}
%  \quad (\textsc {EQ-Sym})
%	\and
%\inferrule
%  {K \vdash p_1 \equiv p_2 \\
%   K \vdash p_3 \equiv p_3}
%  {K \vdash p_1 \equiv p_3}
%  \quad (\textsc {EQ-Trans})
%	\and
%\inferrule
%  {K \vdash p_1 \equiv p_2}
%  {K \vdash p_1.f \equiv p_2.f}
%  \quad (\textsc {EQ-Field})
%	\and
%\inferrule
%  {K = K', (\texttt{var} \; x = \texttt{new} \; 
%            \{\texttt{z} \Rightarrow 
%            \; ..., \texttt{var} f : T = p, ... \;\}; \; \bigcirc)}
%  {K \vdash x.f \equiv p}
%  \quad (\textsc {EQ-Head})
%	\and
%\inferrule
%  {K' \vdash p \equiv p'}
%  {K',E \vdash p \equiv p'}
%  \quad (\textsc {EQ-Tail})
%\end{mathpar}
%\caption{Path Equality}
%\label{f:p-eq}
%\end{figure}
%\begin{figure}[h]
%\hfill \fbox{$E \vdash p \equiv p'$}
%\begin{mathpar}
%\inferrule
%  {}
%  {E \vdash p \equiv p}
%  \quad (\textsc {EQ-Refl})
%	\and
%\inferrule
%  {E \vdash p_1 \equiv p_2}
%  {E \vdash p_2 \equiv p_1}
%  \quad (\textsc {EQ-Sym})
%	\and
%\inferrule
%  {E \vdash p_1 \equiv p_2 \\
%   E \vdash p_3 \equiv p_3}
%  {E \vdash p_1 \equiv p_3}
%  \quad (\textsc {EQ-Trans})
%	\and
%\inferrule
%  {E \vdash p_1 \equiv p_2}
%  {E \vdash p_1.f \equiv p_2.f}
%  \quad (\textsc {EQ-Sel})
%	\and
%\inferrule
%  {}
%  {E \vdash v \unlhd T \equiv v}
%  \quad (\textsc {EQ-Val-Wide})
%	\and
%\inferrule
%  {E = (\texttt{var} \; x = \texttt{new} \; \{\texttt{z} 
%  			\Rightarrow ..., \texttt{var} f : T = p, ...\}; \; E')}
%  {E \vdash x.f \equiv p}
%  \quad (\textsc {EQ-New})
%	\and
%\inferrule
%  {E = E'.m(e) \\
%   E' \vdash p_1 \equiv p_2}
%  {E \vdash p_1 \equiv p_2}
%  \quad (\textsc {EQ-Meth-Recv})
%	\and
%\inferrule
%  {E = e.m(E') \\
%   E' \vdash p_1 \equiv p_2}
%  {E \vdash p_1 \equiv p_2}
%  \quad (\textsc {EQ-Meth-Arg})
%	\and
%\inferrule
%  {E = E'.f \\
%   E' \vdash p_1 \equiv p_2}
%  {E \vdash p_1 \equiv p_2}
%  \quad (\textsc {EQ-Var})
%	\and
%\inferrule
%  {E = E' \unlhd T \\
%   E' \vdash p_1 \equiv p_2}
%  {E \vdash p_1 \equiv p_2}
%  \quad (\textsc {EQ-Wide})
%\end{mathpar}
%\caption{Path Equality}
%\label{f:p-eq}
%\end{figure}