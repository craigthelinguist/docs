\documentclass{llncs}

\usepackage{listings}
\usepackage{amssymb}
\usepackage[margin=.9in]{geometry}
\usepackage{amsmath}
%\usepackage{amsthm}
\usepackage{mathpartir}
\usepackage{color,soul}


\newtheorem{subcase}{SubCase}
\numberwithin{subcase}{case}
\numberwithin{case}{theorem}
\numberwithin{case}{lemma}




\lstdefinestyle{custom_lang}{
  xleftmargin=\parindent,
  showstringspaces=false,
  basicstyle=\ttfamily,
  keywordstyle=\bfseries
}

\lstset{emph={%  
    val, def, type, new, z%
    },emphstyle={\bfseries \tt}%
}

\begin{document}

\section{Syntax}

\begin{figure}[h]
\[
\begin{array}{lll}
\begin{array}{lllr}
e & ::= & x & expression \\
& | & \texttt{new} \; \{z \Rightarrow \overline{d}\}&\\
%& | & \lambda x:T = e\\
%& | & e(e)\\
& | & e.m_T(e) &\\
& | & e.f &\\
%& | & e.f = e&\\
& | & e \unlhd T&\\
& | & l &\\
&&\\
p & ::= & x & paths \\
& | & l &\\
& | & p.f &\\
& | & p \unlhd T &\\
&&\\
v & ::= & l & value \\
& | & v.f &\\
& | & v \unlhd T &\\
&&\\
d & ::= & \texttt{val} \; f : T = p & declaration \\
  & |   & \texttt{def} \; m(x:T) = e : T &\\
  & |   & \texttt{type} \; L : T .. T&\\
&&\\
\Gamma & :: = & \varnothing \; | \; \Gamma,\; x : T & Environment \\
&&\\
A & :: = & \varnothing \; | \; A,\;  <: T & Assumption \; Context \\
 \end{array}
& ~~~~~~
&
\begin{array}{lllr}
%T & ::= & \{\texttt{z} \Rightarrow \overline{\sigma}\} & type \\
T & ::= & \{z \Rightarrow \overline{\sigma}\} & type \\
& | & p.L &\\
%& | & T \wedge T & \\
& | & \top & \\
& | & \bot & \\
&&\\
\sigma & ::= & \texttt{val} \; f:T & decl \; type\\
       & |   & \texttt{def} \; m:T \rightarrow T \\
		 & |   & \texttt{type} \; L : T .. T &\\
&&\\
E & :: = & \bigcirc & eval \; context\\
       & | & E.m(e)\\
       & | & p.m(E)\\
       & | & E.f\\
%       & | & E.f = e\\
%       & | & v.f = E\\
       & | & E \unlhd T\\
&&\\
d_v & ::= & \texttt{val} \; f : T = v & declaration \; value \\
  & |   & \texttt{def} \; m(x:T) = e : T &\\
  & |   & \texttt{type} \; L : T .. T = T &\\
%       & | & \texttt{var} \; x = \texttt{new} \; \{z \Rightarrow \overline{D}\} & eval \; context \\
%&&\\
%D & :: = & \texttt{val} \; f : T = E & decl \; eval \; context \\
%  & |   & \texttt{def} \; m(x:T) = e : T &\\
%  & |   & \texttt{type} \; L : T .. T &\\
%&&\\
%K & :: = & \varnothing \; | \; K, \; E & evaluation \; stack \\
%\Delta & :: = & \varnothing \; | \; \Delta,\; x \mapsto l & substitution \; context \\
%&&\\
%\Sigma & :: = & \varnothing \; | \; \Sigma,\; l : T & store \; context \\
%&&\\
%p & ::= & x & path \\
%  & | & l & \\
&&\\
\mu & :: = & \varnothing \; | \; \mu,\; l \mapsto \{z \Rightarrow \overline{d}\} & store \\
\Sigma & :: = & \varnothing \; | \; \Sigma,\; l : \{\texttt{z} \Rightarrow \overline{\sigma}\} & store \; type \\
\end{array}
\end{array}
\]
\caption{Syntax}
\label{f:syntax}
\end{figure}


%\begin{figure}[h]
%\begin{mathpar}
%\end{mathpar}
%\caption{Declaration Path Function}
%\label{f:path}
%\end{figure}
%\begin{figure}[h]
%\begin{mathpar}
%\inferrule
%  {}
%  {narrow(x) = x \\ narrow(v \unlhd T) = narrow(v)}
%\end{mathpar}
%\caption{Narrow Function}
%\label{f:narrow}
%\end{figure}

\newpage

\section{Semantics}

\begin{figure}[h]
\hfill \fbox{$\mu; \Sigma \vdash v \leadsto l$}
\begin{mathpar}
\inferrule
  {}
  {\mu; \Sigma \vdash l \leadsto l }
  \quad (\textsc {P-Loc})
	\and
\inferrule
  {\mu; \Sigma \vdash v \leadsto v'}
  {\mu; \Sigma \vdash v \unlhd T \leadsto v' \unlhd T}
  \quad (\textsc {L-Type})
	\and
\inferrule
  {\mu; \Sigma \vdash v \leadsto v' \\
   \mu; \Sigma \vdash v' \leadsto_{f} v_f}
  {\mu; \Sigma \vdash v.f \leadsto v_f}
  \quad (\textsc {L-Path})
\end{mathpar}
\hfill \fbox{$\mu; \Sigma \vdash d_v \leadsto d$}
\begin{mathpar}
\inferrule
  {\mu; \Sigma \vdash v \leadsto v'}
  {\mu; \Sigma \vdash \texttt{val} \; f : T = v \leadsto \texttt{val} \; f : T = v'}
  \quad (\textsc {L-Val})
	\and
\inferrule
  {}
  {\mu; \Sigma \vdash \texttt{def} \; m : S(x:T) = e \leadsto \texttt{def} \; m(x:S) = e : T}
  \quad (\textsc {L-Def})
	\and
\inferrule
  {}
  {\mu; \Sigma \vdash \texttt{type} \; L : S .. U \leadsto \texttt{type} \; L : S .. U}
  \quad (\textsc {L-Type})
\end{mathpar}
\caption{Path Leads-to Judgement}
\label{f:path}
\end{figure}

\begin{figure}[h]
\hfill \fbox{$\mu; \Sigma \vdash v.f \leadsto_{f} v$}
\begin{mathpar}
\inferrule
  {\mu(l) = \{z \Rightarrow ..., \texttt{val} f : T_f = v_f, ...\}}
  {\mu; \Sigma \vdash l \leadsto_{f} v_f \unlhd [l/z] T_f}
  \quad (\textsc {L\textsubscript{$f$}-Loc})
	\and
\inferrule
  {\mu; \Sigma \vdash v \leadsto_{f} v_f \\
  	\varnothing; \Sigma; \varnothing \vdash v \unlhd T \ni \texttt{val} f : T_f}
  {\mu; \Sigma \vdash v \unlhd T \leadsto_{f} v_f \unlhd T_f}
  \quad (\textsc {L\textsubscript{$f$}-Type})
\end{mathpar}
\caption{Field Leadsto Judgement}
\label{f:path_field}
\end{figure}

\begin{figure}[h]
\hfill \fbox{$v \leadsto_{\unlhd} l$}
\begin{mathpar}
\inferrule
  {}
  {l \leadsto_{\unlhd} l}
  \quad (\textsc {L\textsubscript{$\unlhd$}-Loc})
	\and
\inferrule
  {v \leadsto_{\unlhd} l}
  {v \unlhd T \leadsto_{\unlhd} l}
  \quad (\textsc {L\textsubscript{$\unlhd$}-Type})
\end{mathpar}
\caption{Cast Leadsto Judgement}
\label{f:path_cast}
\end{figure}

%\hl{Notes:} Can we define subtyping as simply subtype of expansions? i.e.
%\begin{mathpar}
%\inferrule
%  {}
%  {\Gamma \vdash \bot <: T}
%  \quad (\textsc {S-Bottom})
%	\and
%\inferrule
%  {\Gamma \vdash S \prec_z \overline{\sigma}, T \prec_z \overline{\sigma}' \\
%   \Gamma, (z : S) \vdash \overline{\sigma} <: \overline{\sigma}'}
%  {\Gamma \vdash S <: T}
%  \quad (\textsc {S-Exp})
%\end{mathpar}
%No, because potentially 
%\begin{mathpar}
%\inferrule
%  {}
%  {\Gamma \vdash S <: S' \\ 
%  	\Gamma \vdash S' \prec_z \overline{\sigma}\\
%  	\Gamma \vdash U \prec_z \overline{\sigma}'\\
%  	}
%\end{mathpar}

\newpage


\begin{figure}[h]
\hfill \fbox{$A; \Sigma; \Gamma_1 \vdash S <: T \dashv \Gamma_2$}
\begin{mathpar}
\inferrule
  {}
  {A; \Sigma; \Gamma \vdash T\; \texttt{<:}\; T \dashv \Gamma}
  \quad (\textsc {S-Refl})
	\and
\inferrule
  {(S <: T) \in A}
  {A; \Sigma; \Gamma_1 \vdash S\; \texttt{<:}\; T \dashv \Gamma_2}
  \quad (\textsc {S-Assume})
	\and
\inferrule
  {A' = A, (\{z \Rightarrow \overline{\sigma}\} <: \{z \Rightarrow \overline{\sigma}'\}) \\
  	A'; \Sigma; \Gamma_1, z : \{z \Rightarrow \overline{\sigma}\} \vdash \overline{\sigma} <:\; \overline{\sigma}'  \dashv \Gamma_2, z : \{z \Rightarrow \overline{\sigma}'\}}
  {A; \Sigma; \Gamma_1 \vdash \{z \Rightarrow \overline{\sigma}\}\; <:\; \{z \Rightarrow \overline{\sigma}'\}\dashv \Gamma_2}
  \quad (\textsc {S-Rec})
	\and
\inferrule
  {A; \Sigma; \Gamma_1 \vdash p \ni \texttt{type} \; L : S_1 .. U_1\\
  	A; \Sigma; \Gamma_2 \vdash p \ni \texttt{type} \; L : S_2 .. U_2\\
  	A; \Sigma; \Gamma_2 \vdash S_2 \; \texttt{<:}\; S_1 \dashv \Gamma_1 \\
  	A; \Sigma; \Gamma_1 \vdash U_1 \; \texttt{<:}\; U_2 \dashv \Gamma_2}
  {A; \Sigma; \Gamma_1 \vdash p.L \; \texttt{<:}\; p.L \dashv \Gamma_2}
  \quad (\textsc {S-Select-Refl})
	\and
\inferrule
  {A; \Sigma; \Gamma_1 \vdash p \ni \texttt{type} \; L : S .. U\\
  	A; \Sigma; \Gamma_1 \vdash S <: U \dashv \Gamma_1 \\
  	A; \Sigma; \Gamma_1 \vdash U <: U' \dashv \Gamma_2}
  {A; \Sigma; \Gamma_1 \vdash p.L\; <:\; U' \dashv \Gamma_2}
  \quad (\textsc {S-Select-Upper})
	\and
\inferrule
  {A; \Sigma; \Gamma_2 \vdash p \ni \texttt{type} \; L : S .. U \\
  	A; \Sigma; \Gamma_2 \vdash S <: U \dashv \Gamma_2 \\
  	A; \Sigma; \Gamma_1 \vdash S' <: S \dashv \Gamma_2}
  {A; \Sigma; \Gamma_1 \vdash S'\; <:\; p.L \dashv \Gamma_2}
  \quad (\textsc {S-Select-Lower})
	\and
\inferrule
  {}
  {A; \Sigma; \Gamma_1 \vdash T\; \texttt{<:}\; \top \dashv \Gamma_2}
  \quad (\textsc {S-Top})
	\and
\inferrule
  {}
  {A; \Sigma; \Gamma_1 \vdash \bot\; \texttt{<:}\; T \dashv \Gamma_2}
  \quad (\textsc {S-Bottom})
\end{mathpar}
\hfill \fbox{$A; \Sigma; \Gamma_1 \vdash \sigma <: \sigma' \dashv \Gamma_2$}
\begin{mathpar}
\inferrule
  {}
  {A; \Sigma; \Gamma_1 \vdash \texttt{val} \; f:T <: \texttt{val} \; f:T \dashv \Gamma_2}
  \quad (\textsc {S-Decl-Val})
	\and
\inferrule
  {A; \Sigma; \Gamma_2 \vdash S' <: S \dashv \Gamma_1 \\
  	A; \Sigma; \Gamma_1 \vdash T <: T' \dashv \Gamma_2}
  {A; \Sigma; \Gamma_1 \vdash \texttt{def} \; m:S \rightarrow T <: \texttt{def} \; m:S' \rightarrow T' \dashv \Gamma_2}
  \quad (\textsc {S-Decl-Def})
	\and
\inferrule
  {A; \Sigma; \Gamma_2 \vdash S' <: S \dashv \Gamma_1 \\
  	A; \Sigma; \Gamma_1 \vdash U <: U' \dashv \Gamma_2}
  {A; \Sigma; \Gamma_1 \vdash \texttt{type} \; L : S .. U \; <:\; \texttt{type} \; L : S' .. U' \dashv \Gamma_2}
  \quad (\textsc {S-Decl-Type})
\end{mathpar}
\caption{Subtyping}
\label{f:subtype}
\end{figure}

\begin{figure}[h]
\hfill \fbox{$A; \Sigma; \Gamma \vdash T \;  \textbf{wf}$}
\begin{mathpar}
\inferrule
  {A; \Sigma; \Gamma \vdash p \ni \texttt{type} \; L : S .. U \\
  	A; \Sigma; \Gamma \vdash \texttt{type} \; L : S .. U \; \textbf{wf} }
  {A; \Sigma; \Gamma \vdash p.L \; \textbf{wf}}
  \quad (\textsc {WF-Sel})
	\and
\inferrule
  {A; \Sigma; \Gamma,z:\{z \Rightarrow \overline{\sigma}\} \vdash \overline{\sigma} \; \textbf{wf} \\
  	\forall j \neq i, \; dom(\sigma_j) \neq dom(\sigma_i)}
  {A; \Gamma; \Sigma \vdash \{z \Rightarrow \overline{\sigma}\} \; \textbf{wf}}
  \quad (\textsc {WF-Rec})
%	\and
%\inferrule
%  {\Gamma \vdash S \;  \textbf{wf} \\
%  	\Gamma \vdash T \;  \textbf{wf}}
%  {\Gamma \vdash S \rightarrow T \;  \textbf{wf}}
%  \quad (\textsc {WF-Func})
	\and
\inferrule
  {}
  {A; \Sigma; \Gamma \vdash \top \;  \textbf{wf}}
  \quad (\textsc {WF-Top})
	\and
\inferrule
  {}
  {A; \Sigma; \Gamma \vdash \bot \;  \textbf{wf}}
  \quad (\textsc {WF-Bot})
\end{mathpar}
\hfill \fbox{$A; \Sigma; \Gamma \vdash \sigma \;  \textbf{wf}$}
\begin{mathpar}
\inferrule
  {A; \Sigma; \Gamma \vdash T : \textbf{wf}}
  {A; \Sigma; \Gamma \vdash \texttt{val} \; f:T \;  \textbf{wf}}
  \quad (\textsc {WF-Val})
	\and
\inferrule
  {A; \Sigma; \Gamma \vdash T : \textbf{wf} \\
  	A; \Sigma; \Gamma \vdash S : \textbf{wf}}
  {A; \Sigma; \Gamma \vdash \texttt{def} \; m:S \rightarrow T \;  \textbf{wf}}
  \quad (\textsc {WF-Def})
	\and
\inferrule
  {A; \Sigma; \Gamma \vdash S : \textbf{wfe} \; \vee \; S = \bot\\
  	A; \Sigma; \Gamma \vdash U : \textbf{wfe} \\
  	A; \Sigma; \Gamma \vdash S <: U}
  {A; \Sigma; \Gamma \vdash \texttt{type} \; L : S .. U \; \textbf{wf}}
  \quad (\textsc {WF-Type})
\end{mathpar}
\hfill \fbox{$A; \Sigma \vdash \Gamma \;  \textbf{wf}$}
\begin{mathpar}
\inferrule
  {\forall x \in dom(\Gamma), A; \Sigma; \Gamma \vdash \Gamma(x) \; \textbf{wf}}
  {\Sigma \vdash \Gamma \;  \textbf{wf}}
  \quad (\textsc {WF-Environment})
\end{mathpar}
\hfill \fbox{$\Sigma \;  \textbf{wf}$}
\begin{mathpar}
\inferrule
  {\forall l \in dom(\Sigma), \varnothing; \Sigma; \varnothing \vdash \Sigma(l) \; \textbf{wf}}
  {\Sigma \;  \textbf{wf}}
  \quad (\textsc {WF-Store-Context})
\end{mathpar}
\begin{mathpar}
\inferrule
  {\forall l \in dom(\mu), \varnothing; \Sigma; \varnothing \vdash \mu(l) : \Sigma(l)}
  {\mu : \Sigma}
  \quad (\textsc {WF-Store})
\end{mathpar}
\caption{Well-Formedness}
\label{f:wf}
\end{figure}

\begin{figure}[h]
\hfill \fbox{$A; \Sigma; \Gamma \vdash T \;  \textbf{wfe}$}
\begin{mathpar}
\inferrule
  {A; \Sigma; \Gamma \vdash T \; \textbf{wf} \\
  	A; \Sigma; \Gamma \vdash T \prec \overline{\sigma}}
  {A; \Sigma; \Gamma \vdash T \; \textbf{wfe}}
  \quad (\textsc {WFE})
\end{mathpar}
\caption{Well-Formed and Expanding Types}
\label{f:wfe}
\end{figure}

\begin{figure}[h]
\hfill \fbox{$\Sigma; \Gamma \vdash T \prec \overline{\sigma}$}
\begin{mathpar}
\inferrule
  {}
  {\Sigma; \Gamma \vdash \{z \Rightarrow \overline{\sigma}\} \prec_z \overline{\sigma}}
  \quad (\textsc {E-Rec})
	\and
\inferrule
  {\Sigma; \Gamma \vdash p \ni \texttt{type} \; L : S..U \\
  	\Sigma; \Gamma \vdash U \prec_z \overline{\sigma}}
  {\Sigma; \Gamma \vdash p.L \prec_z \overline{\sigma}}
  \quad (\textsc {E-Sel})
	\and
\inferrule
  {}
  {\Sigma; \Gamma \vdash \top \prec_z \varnothing}
  \quad (\textsc {E-Top})
\end{mathpar}
\caption{Expansion}
\label{f:exp}
\end{figure}
\begin{figure}[h]
\hfill \fbox{$\Sigma; \Gamma \vdash e \ni \sigma$}
\begin{mathpar}
\inferrule
  {\Sigma; \Gamma \vdash p : T \\
  	\Sigma; \Gamma \vdash T \prec_z \overline{\sigma}\\
  	\sigma_i \in \overline{\sigma}}
  {\Sigma; \Gamma \vdash p \ni [p/z]\sigma_i}
  \quad (\textsc {M-Path})
	\and
\inferrule
  {\Sigma; \Gamma \vdash e : T \\
  	\Sigma; \Gamma \vdash T \prec_z \overline{\sigma}\\
  	\sigma_i \in \overline{\sigma} \\
  	z \notin \sigma_i}
  {\Sigma; \Gamma \vdash e \ni \sigma_i}
  \quad (\textsc {M-Exp})
\end{mathpar}
\caption{Membership}
\label{f:mem}
\end{figure}

%\subsubsection{Typing}

\begin{figure}[h]
\hfill \fbox{$\Gamma; \Sigma \vdash e:T$}
\begin{mathpar}
\inferrule
  {x \in dom(\Gamma)}
  {	\Gamma; \Sigma \vdash x : \Gamma(x)}
  \quad (\textsc {T-Var})
	\and
\inferrule
  {	l \in dom(\Sigma)}
  {	\Gamma; \Sigma \vdash l : \Sigma(l)}
  \quad (\textsc {T-Loc})
	\and
\inferrule
  {\Gamma, z : \{z \Rightarrow \overline{\sigma}\}; \Sigma 
  \vdash \overline{d} : \overline{\sigma} \\
  	y \notin dom(\Gamma)}
  {	\Gamma; \Sigma\vdash \texttt{new} \; \{z \Rightarrow \overline{d}\} : 
  \{z \Rightarrow \overline{\sigma}\}}
  \quad (\textsc {T-New})
	\and
\inferrule
  {\Gamma; \Sigma \vdash e_0 \ni \texttt{def} \; m:S \rightarrow T \\
  	\Gamma; \Sigma \vdash e_1 : S \\
  	\Gamma; \Sigma \vdash T <: U}
  {	\Gamma; \Sigma \vdash e_0.m_U(e_1) : U}
  \quad (\textsc {T-Meth})
	\and
\inferrule
  {	\Gamma; \Sigma \vdash e : S \\
  	\Gamma; \Sigma \vdash e \ni \texttt{val} \; f:T}
  {	\Gamma; \Sigma \vdash e.f : T}
  \quad (\textsc {T-Acc})
%	\and
%\inferrule
%  {	\Gamma; E \vdash e_0 : T_0 \\
%  	T_0 \ni \texttt{def} \; f:T = e \\
%  	\Gamma; E \vdash e_1 : T}
%  {	\Gamma; E \vdash e_0.f = e_1 : T}
%  \quad (\textsc {T-Assign})
%	\and
	\and
\inferrule
  {	\Gamma; \Sigma \vdash e : T}
  {	\Gamma; \Sigma \vdash e \unlhd T : T}
  \quad (\textsc {T-Type})
	\and
\inferrule
  {	\Gamma; \Sigma \vdash e : S \\
  	\Gamma; \Sigma \vdash S <: T}
  {	\Gamma; \Sigma \vdash e : T}
  \quad (\textsc {T-Sub})
\end{mathpar}
\caption{Expression Typing}
\label{f:e_typ}
\end{figure}
\begin{figure}[h]
\hfill \fbox{$\Gamma; E \vdash d:\sigma$}
\begin{mathpar}
\inferrule
  {\Gamma; \Sigma \vdash e : T}
  {\Gamma; \Sigma \vdash \texttt{def} \; f:T = e : \texttt{def} \; f:T}
  \quad (\textsc {T-Decl-Var})
	\and
\inferrule
  {\Gamma, x : S; \Sigma \vdash e_0 : T}
  {\Gamma; \Sigma \vdash \texttt{def} \; m(x:S) = e : T : \texttt{def} \; m:S \rightarrow T}
  \quad (\textsc {T-Decl-Def})
	\and
\inferrule
  {\Gamma; \Sigma \vdash \texttt{type} \; L : S .. U \; \textbf{wf} }
  {\Gamma; \Sigma \vdash \texttt{type} \; L : S .. U : \texttt{type} \; L : S .. U}
  \quad (\textsc {T-Decl-Type})
\end{mathpar}
\caption{Declaration Typing}
\label{f:d_typ}
\end{figure}
%\begin{figure}[h]
%\hfill \fbox{$\Gamma \vdash \mu:\Sigma$}
%\begin{mathpar}
%\inferrule
%  {\forall x \in dom(\mu), \; \mu(x)= \{\texttt{z} \Rightarrow \overline{d}\} \\
%  	\Gamma(x) = \{\texttt{z} \Rightarrow \overline{\sigma}\} \\
%  	\Gamma \vdash \overline{d} : \overline{\sigma}}
%  {\Gamma \vdash \mu}
%  \quad (\textsc {T-Store})
%\end{mathpar}
%\caption{Store Typing}
%\label{f:s_typ}
%\end{figure}

\begin{figure}[h]
\hfill \fbox{$\mu \; | \; e \; \rightarrow \mu' \; | \; e'$}
\begin{mathpar}
\inferrule
  {l \notin dom(\mu) \\
  	\mu' = \mu, l \mapsto \{\texttt{z} \Rightarrow \overline{d_v}\}}
  {\mu \; | \; \texttt{new} \; \{\texttt{z} \Rightarrow \overline{d_v}\} \; \rightarrow \mu' \; | \; l}
  \quad (\textsc {R-New})
  \and
\inferrule
  {\mu : \Sigma \\
   \mu; \Sigma \vdash v_1 \leadsto_{m} e}
  {\mu \; | \; v_1.m(v_2) \;\rightarrow \mu \; | \; [l/\texttt{z},v_2 \unlhd S/x]e}
  \quad (\textsc {R-Meth})
  \and
\inferrule
  {	\mu \; | \; e \; \rightarrow \; \mu' \; | \; e'}
  {\mu \; | \; E[e] \; \rightarrow \mu' \; | \; E[e']}
  \quad (\textsc {R-Context})
\end{mathpar}
\caption{Reduction}
\label{f:red}
\end{figure}

\bibliographystyle{plain}
\bibliography{bib}

\end{document} 
