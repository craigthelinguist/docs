\documentclass[a4paper,UKenglish]{lipics}
%This is a template for producing LIPIcs articles. 
%See lipics-manual.pdf for further information.
%for A4 paper format use option "a4paper", for US-letter use option "letterpaper"
%for british hyphenation rules use option "UKenglish", for american hyphenation rules use option "USenglish"
% for section-numbered lemmas etc., use "numberwithinsect"
 
\usepackage{microtype}%if unwanted, comment out or use option "draft"

\usepackage{listings}
\usepackage{amssymb}
%\usepackage[margin=.9in]{geometry}
\usepackage{amsmath}
%\usepackage{amsthm}
\usepackage{mathpartir}
\usepackage{color,soul}
\usepackage{graphicx}

%\theoremstyle{definition}
%%\newtheorem{case1}{Case1}
%\spnewtheorem{casethm}{Case}[theorem]{\itshape}{\rmfamily}
%\spnewtheorem{subcase}{Subcase}{\itshape}{\rmfamily}
%\numberwithin{subcase}{casethm}
%\numberwithin{casethm}{theorem}
%\numberwithin{casethm}{lemma}





\lstdefinestyle{custom_lang}{
  xleftmargin=\parindent,
  showstringspaces=false,
  basicstyle=\ttfamily,
  keywordstyle=\bfseries
}

\lstset{emph={%  
    val, def, type, new, z%
    },emphstyle={\bfseries \tt}%
}

%\graphicspath{{./graphics/}}%helpful if your graphic files are in another directory

\bibliographystyle{plain}% the recommended bibstyle

% Author macros::begin %%%%%%%%%%%%%%%%%%%%%%%%%%%%%%%%%%%%%%%%%%%%%%%%
\title{A Sample Article for the LIPIcs series\footnote{This work was partially supported by someone.}}
\titlerunning{A Sample LIPIcs Article} %optional, in case that the title is too long; the running title should fit into the top page column

\author[1]{John Q. Open}
\author[2]{Joan R. Access}
\affil[1]{Dummy University Computing Laboratory\\
  Address, Country\\
  \texttt{open@dummyuni.org}}
\affil[2]{Department of Informatics, Dummy College\\
  Address, Country\\
  \texttt{access@dummycollege.org}}
\authorrunning{J.\,Q. Open and J.\,R. Access} %mandatory. First: Use abbreviated first/middle names. Second (only in severe cases): Use first author plus 'et. al.'

\Copyright{John Q. Open and Joan R. Access}%mandatory, please use full first names. LIPIcs license is "CC-BY";  http://creativecommons.org/licenses/by/3.0/

\subjclass{Dummy classification -- please refer to \url{http://www.acm.org/about/class/ccs98-html}}% mandatory: Please choose ACM 1998 classifications from http://www.acm.org/about/class/ccs98-html . E.g., cite as "F.1.1 Models of Computation". 
\keywords{Dummy keyword -- please provide 1--5 keywords}% mandatory: Please provide 1-5 keywords
% Author macros::end %%%%%%%%%%%%%%%%%%%%%%%%%%%%%%%%%%%%%%%%%%%%%%%%%

%Editor-only macros:: begin (do not touch as author)%%%%%%%%%%%%%%%%%%%%%%%%%%%%%%%%%%
\serieslogo{}%please provide filename (without suffix)
\volumeinfo%(easychair interface)
  {Billy Editor and Bill Editors}% editors
  {2}% number of editors: 1, 2, ....
  {Conference title on which this volume is based on}% event
  {1}% volume
  {1}% issue
  {1}% starting page number
\EventShortName{}
\DOI{10.4230/LIPIcs.xxx.yyy.p}% to be completed by the volume editor
% Editor-only macros::end %%%%%%%%%%%%%%%%%%%%%%%%%%%%%%%%%%%%%%%%%%%%%%%

\begin{document}

\maketitle

\begin{abstract}
Lorem ipsum dolor sit amet, consectetur adipiscing elit. Praesent convallis orci arcu, eu mollis dolor. Aliquam eleifend suscipit lacinia. Maecenas quam mi, porta ut lacinia sed, convallis ac dui. Lorem ipsum dolor sit amet, consectetur adipiscing elit. Suspendisse potenti. 
 \end{abstract}
 
\section{Introduction}

\section{Related Work}


\section{Generic Wyvern}
Wyvern is a new statically typed programming language being developed for secure web applications. There are two popular and well-documented methods \cite{Virtual Types stuff, generic java etc} for implementing generic types.
\begin{itemize}
\item \emph{Type Parameters}: Types are parameterized by generic type names that allow programmers to reuse generic code using bounded types. Many popular languages use generic type parameters such as \emph{Java}, \emph{C\#} and \emph{Scala}.
\item \emph{Type Members}: Types and objects may contain type members in the same manner as normal field or method members. These can be subtyped by more precise types to provide generic behavior.
\end{itemize}
There have long been attempts to formalize a sound small-step semantics for type members in a structural setting for Scala, which have as yet all been unsuccessful due to issues where well-typed expressions are lost during subject reduction. In 2014 Amin et.al. were able to formalize a big step semantics for Scala style type members. While this demonstrates the soundness of the kinds of programs we wish to type check, a small-step semantics offers powerful advantages when reasoning about the behavior of programs. For this reason we formalize type members with a small step semantics.
We build upon the work of Amin et. al. \cite{Scala stuff} to formalize type members in structurally typed languages using a small step semantics. 
 
\subsection{Formalizing Type Members}
In Figure \ref{f:min_syntax} we begin by developing a small syntax for a type system containing type members. Expressions ($e$) are variables, new expressions, method calls, field accesses and locations in memory. Types ($T$) are structural types, type member selections on paths, the top and bottom types. Declarations ($d$) are fields, methods and type members. Declaration types ($\sigma$) are field types, method types and type members. Paths ($p$) are variables, memory locations and field accesses.
\begin{figure}[h]
\[
\begin{array}{lll}
\begin{array}{lllr}
e & ::= & x & expression \\
& | & \texttt{new} \; \{z \Rightarrow \overline{d}\}&\\
& | & e.m(e) &\\
& | & e.f &\\
& | & l &\\
&&\\
d & ::= & \texttt{val} \; f : T = p & declaration \\
  & |   & \texttt{def} \; m(x:T) = e : T &\\
  & |   & \texttt{type} \; L : T .. T&\\
&&\\
v & ::= & l & value \\
&&\\
\Gamma & :: = & \varnothing \; | \; \Gamma,\; x : T & environment \\
 \end{array}
& ~~~~~~
&
\begin{array}{lllr}
T & ::= & \{z \Rightarrow \overline{\sigma}\} & type \\
& | & p.L &\\
& | & \top & \\
& | & \bot & \\
&&\\
\sigma & ::= & \texttt{val} \; f:T & decl \; type\\
       & |   & \texttt{def} \; m:T \rightarrow T \\
		 & |   & \texttt{type} \; L : T .. T &\\
&&\\
p & ::= & x & paths \\
& | & l &\\
& | & p.f &\\
&&\\
\Sigma & :: = & \varnothing \; | \; \Sigma,\; l : T & store type \\
\end{array}
\end{array}
\]
\caption{Minimal Type Members in Wyvern}
\label{f:min_syntax}
\end{figure}

\subsection{Transitivity, Narrowing and Type Members}
\label{sec:trans_narrow}
Two properties that one might reasonably expect 
to occur naturally in a structurally typed language are 
\emph{Subtype Transitivity} and \emph{Environment Narrowing}
(Figure \ref{f:trans_narrowing}).
\begin{figure}[h]
\begin{mathpar}
\inferrule
	{\Gamma \vdash S <: T \\
	 \Gamma \vdash T <: U}
	{\Gamma \vdash S <: U}
  \quad (\textsc {Subtype Transitivity})
	\and
\inferrule
  {\Gamma, (x : U) \vdash T <: T'\\
   \Gamma \vdash S <: U}
  {\Gamma, (x : S) \vdash T <: T'}
  \quad (\textsc {Environment Narrowing})
\end{mathpar}
\caption{Subtype Transitivity and Environment Narrowing}
\label{f:trans_narrowing}
\end{figure}
Subtype transitivity is a familiar property, 
and environment narrowing simply expresses the 
expectation that we can treat a variable in an 
environment as having a more precise type without 
changing the type relationships within that environment.

Since transitivity is often used in proving subject 
reduction, it is a problem if a type system lacks this 
property. The issue with transitivity arises from the 
introduction of type member lower bounds, and their 
contra-variance. The following example demonstrates this (\hl{Julian: Better Example?}).
\begin{lstlisting}[mathescape, style=custom_lang]
A = {z $\Rightarrow$ type N : $\bot$ .. $\top$}

B = {z $\Rightarrow$ type N : $\bot$ .. $\top$
          def meth1(x : $\top$){return new{z $\Rightarrow$}}:$\top$}
         
S = {z $\Rightarrow$ type L : A .. $\top$
          val f : A}
         
T = {z $\Rightarrow$ type L : A .. $\top$
          val f : z.L}
         
U = {z $\Rightarrow$ type L : B .. $\top$
          val f : z.L}
\end{lstlisting}
Here \texttt{S} subtypes \texttt{T} and \texttt{T} subtypes 
\texttt{U}, but \texttt{S} does not subtype \texttt{U}. Because 
of the contra-variance of the lower bound of type member \texttt{L}, 
\texttt{A} subtypes \texttt{z.L} in \texttt{T} but not \texttt{U}. 
Amin et al. \cite{Amin 2014} attempt to reconcile this by narrowing 
the type of \texttt{z} in the larger types \texttt{T} and \texttt{U} 
by using the subtype rule in Figure \ref{f:sub_amin}.
\begin{figure}[h]
\begin{mathpar}
\inferrule
  {\Gamma, z : \{z \Rightarrow \overline{\sigma}_1\} \vdash \overline{\sigma}_1 <:\; \overline{\sigma}_2}
  {\Gamma \vdash \{z \Rightarrow \overline{\sigma}_1\}\; <:\; \{z \Rightarrow \overline{\sigma}_2\}}
  \quad (\textsc {Structural Subtyping})
\end{mathpar}
\caption{Structural Subtyping with Narrowing}
\label{f:sub_amin}
\end{figure}
Here we type check the declaration types of the larger type with a smaller receiver. While this allows for subtype transitivity, it introduces environment narrowing that proves unsound in a small step semantics \cite{Scala Stuff}(\hl{Julian: Example for Narrowing stuff?}). Amin et. al. were however able to reconcile both transitivity and narrowing within a big step semantics. Instead of confronting this problem we avoid it by removing narrowing, and as a result subtype transitivity. To achieve this we introduce an upcast to the expression syntax, $e \unlhd T$. Here we use $T$ to ignore the more precise type of $e$. Narrowing occurs in two places, during subtyping of structural types as we have already seen, and at field and method return. Two avoid narrowing during subtyping, we modify the structural subtyping rule and upcast the receiver of the second type. Our modified rule is given in Figure \ref{f:sub_wwyvern}.
\begin{figure}[h]
\begin{mathpar}
\inferrule
  {\Gamma, z : \{z \Rightarrow \overline{\sigma}_1\} \vdash \overline{\sigma}_1 <:\; [z \unlhd \{z \Rightarrow \overline{\sigma}_2\}/z]\overline{\sigma}_2}
  {\Gamma \vdash \{z \Rightarrow \overline{\sigma}_1\}\; <:\; \{z \Rightarrow \overline{\sigma}_2\}}
  \quad (\textsc {Structural Subtyping})
\end{mathpar}
\caption{Structural Subtyping without Narrowing}
\label{f:sub_wwyvern}
\end{figure}


\subsection{Path Equality}
To further complicate matters, well-formedness can be lost when reducing field accesses. 

\hl{Julian: Example?}


\section{Type System}

\subsection{Syntax} \label{s:syntax}
In this section we present the Generic Wyvern Syntax in Figure \ref{f:syntax}. 

\begin{figure}[h]
\[
\begin{array}{lll}
\begin{array}{lllr}
e & ::= & x & expression \\
& | & \texttt{new} \; \{z \Rightarrow \overline{d}\}&\\
& | & e.m(e) &\\
& | & e.f &\\
& | & e \unlhd T&\\
& | & l &\\
&&\\
p & ::= & x & paths \\
& | & l &\\
& | & p \unlhd T &\\
&&\\
v & ::= & l & value \\
& | & v \unlhd T &\\
&&\\
d & ::= & \texttt{val} \; f : T = p & declaration \\
  & |   & \texttt{def} \; m(x:T) = e : T &\\
  & |   & \texttt{type} \; L : T .. T&\\
&&\\
\Gamma & :: = & \varnothing \; | \; \Gamma,\; x : T & Environment \\
&&\\
A & :: = & \varnothing \; | \; A,\;  <: T & Assumption \; Context \\
 \end{array}
& ~~~~~~
&
\begin{array}{lllr}
T & ::= & \{z \Rightarrow \overline{\sigma}\} & type \\
& | & p.L &\\
& | & \top & \\
& | & \bot & \\
&&\\
\sigma & ::= & \texttt{val} \; f:T & decl \; type\\
       & |   & \texttt{def} \; m:T \rightarrow T \\
		 & |   & \texttt{type} \; L : T .. T &\\
&&\\
E & :: = & \bigcirc & eval \; context\\
       & | & E.m(e)\\
       & | & p.m(E)\\
       & | & E.f\\
       & | & E \unlhd T\\
&&\\
d_v & ::= & \texttt{val} \; f : T = v & declaration \; value \\
  & |   & \texttt{def} \; m(x:T) = e : T &\\
  & |   & \texttt{type} \; L : T .. T = T &\\
&&\\
\mu & :: = & \varnothing \; | \; \mu,\; l \mapsto \{z \Rightarrow \overline{d}\} & store \\
\Sigma & :: = & \varnothing \; | \; \Sigma,\; l : \{\texttt{z} \Rightarrow \overline{\sigma}\} & store \; type \\
\end{array}
\end{array}
\]
\caption{Syntax}
\label{f:syntax}
\end{figure}

\textbf{Expressions} ($e$): Expressions are variables ($x$), new expressions ($\texttt{new} \; \{z \Rightarrow \overline{d}\}$), method calls ($e.m(e)$), field accesses ($e.f$), expression upcasts ($e \unlhd T$) and locations ($l$) in the store. The only expression that differs from tradition is the explicit upcasts on expressions. We use upcasts to avoid the narrowing issues described in Section We employ a similar strategy with regard to the explicitly upcast expression $e \unlhd T$. Here $e$ may have a more precise type than $T$, but to avoid narrowing we maintain the type $T$.

\textbf{Types} ($T$): Types are restricted to structural types 
($\{z \Rightarrow \overline{\sigma}\}$), type member selections on 
paths ($p.L$), ($\top$) the top type at the top of the type lattice 
that represents the empty type and $\bot$ the bottom type at the 
bottom of the type lattice that represents the type containing 
al possible declaration labels with $\top$ in the contra-variant 
type position, and $\bot$ in the covariant type position.

\textbf{Paths} ($p$): Paths are expressions that type selections may be 
made on. We restrict these to variables ($x$), locations ($l$) and upcast paths ($p \unlhd T$).

\textbf{Values} ($v$): Values in our type system are locations ($l$) and upcast values ($v \unlhd T$). 

\textbf{Declarations} ($d$): Declarations may be fields (\texttt{val}), 
methods (\texttt{def}) or type members (\texttt{type}). These are all
standard.

\textbf{Declaration Types} ($\sigma$): Declaration types may be field 
(\texttt{val}), method (\texttt{def}) or type members (\texttt{type}). 

\textbf{Declaration Values} ($d_v$): Declaration values are similar to 
declarations, except we require field initializers to be values.

On top of these, we also include an evaluation context $E$, an environment 
$\Gamma$ that maps variables to types, a store $\mu$ that maps locations 
to objects, a store type $\Sigma$ used to type check the store and an 
assumption context $A$ that is used to type check recursive types (\cite{Amber Rules etc})
that consists of a list of type pairs.



\subsection{Semantics}
In this section we describe the Wyvern Type Members semantics.
\subsubsection{Path Functions}

\hl{Julian: I'll finish describing these functions later once they are finalized.}

\begin{figure}[h]
\begin{mathpar}
\inferrule
  {}
  {p \equiv p}
  \quad (\textsc{Eq-Refl})
  \and
\inferrule
  {p_1 \equiv p_2}
  {p_2 \equiv p_1}
  \quad (\textsc{Eq-Sym})
  \and
\inferrule
  {p_1 \equiv p_2 \\
   p_2 \equiv p_3}
  {p_1 \equiv p_3}
  \quad (\textsc{Eq-Trans})
  \and
\inferrule
  {p_1 \equiv p_2}
  {p_1 \equiv p_2 \unlhd T}
  \quad (\textsc{Eq-Path})
\end{mathpar}
\caption{Path Equivalence}
\label{f:path_equiv}
\end{figure}
In Figure \ref{f:path_equiv} we describe our path equivalence judgment. Two paths are equivalent if they wrap up the same variable or location. This is useful when trying to compare two types that are not technically the same types, but are derived from the same object.

%\begin{figure}[h]
%\hfill \fbox{$\mu; \Sigma \vdash v \leadsto l$}
%\begin{mathpar}
%\inferrule
%  {}
%  {\mu; \Sigma \vdash l \leadsto l }
%  \quad (\textsc {P-Loc})
%	\and
%\inferrule
%  {\mu; \Sigma \vdash v \leadsto v'}
%  {\mu; \Sigma \vdash v \unlhd T \leadsto v' \unlhd T}
%  \quad (\textsc {L-Type})
%	\and
%\inferrule
%  {\mu; \Sigma \vdash v \leadsto v' \\
%   \mu; \Sigma \vdash v' \leadsto_{f} v_f}
%  {\mu; \Sigma \vdash v.f \leadsto v_f}
%  \quad (\textsc {L-Path})
%\end{mathpar}
%\hfill \fbox{$\mu; \Sigma \vdash d_v \leadsto d$}
%\begin{mathpar}
%\inferrule
%  {\mu; \Sigma \vdash v \leadsto v'}
%  {\mu; \Sigma \vdash \texttt{val} \; f : T = v \leadsto \texttt{val} \; f : T = v'}
%  \quad (\textsc {L-Val})
%	\and
%\inferrule
%  {}
%  {\mu; \Sigma \vdash \texttt{def} \; m : S(x:T) = e \leadsto \texttt{def} \; m(x:S) = e : T}
%  \quad (\textsc {L-Def})
%	\and
%\inferrule
%  {}
%  {\mu; \Sigma \vdash \texttt{type} \; L : S .. U \leadsto \texttt{type} \; L : S .. U}
%  \quad (\textsc {L-Type})
%\end{mathpar}
%\caption{Path Leads-to Judgement}
%\label{f:path}
%\end{figure}
%
%\begin{figure}[h]
%\hfill \fbox{$\mu; \Sigma \vdash v \leadsto_{m} \sigma_0, \sigma_1, ..., \sigma_i$}
%\begin{mathpar}
%\inferrule
%  {\mu(l) = \{z \Rightarrow ..., \texttt{val} f : T_f = v_f, ...\}}
%  {\mu; \Sigma \vdash l \leadsto_{f} v_f \unlhd [l/z] T_f}
%  \quad (\textsc {L\textsubscript{$m$}-Loc})
%	\and
%\inferrule
%  {\mu; \Sigma \vdash v \leadsto_{m} v_f \\
%  	\varnothing; \Sigma; \varnothing \vdash v \unlhd T \ni \texttt{val} f : T_f}
%  {\mu; \Sigma \vdash v \unlhd T \leadsto_{m} v_f \unlhd T_f}
%  \quad (\textsc {L\textsubscript{$m$}-Type})
%\end{mathpar}
%\caption{Field Leadsto Judgement}
%\label{f:path_field}
%\end{figure}
%
%\begin{figure}[h]
%\hfill \fbox{$v \leadsto_{\unlhd} l$}
%\begin{mathpar}
%\inferrule
%  {}
%  {l \leadsto_{\unlhd} l}
%  \quad (\textsc {L\textsubscript{$\unlhd$}-Loc})
%	\and
%\inferrule
%  {v \leadsto_{\unlhd} l}
%  {v \unlhd T \leadsto_{\unlhd} l}
%  \quad (\textsc {L\textsubscript{$\unlhd$}-Type})
%\end{mathpar}
%\caption{Cast Leadsto Judgement}
%\label{f:path_cast}
%\end{figure}

\subsubsection{Subtyping}
Subtyping is given in Figure \ref{f:subtype}. We use a 
modified version of the subtyping relation 
described by Amin et. al. \cite{Scala stuff}. We remove the reflexivity rule, however this can be inferred from the other subtyping rules. We modify the \textsc{S-Struct} rule for structural subtyping by upcasting the self variable of the larger type to avoid narrowing. This can be seen in \textsc{S-Struct} in Figure \ref{f:subtype}.
%\begin{figure}[h]
%\begin{mathpar}
%\inferrule
%	{A; \Sigma; \Gamma, z : \{z \Rightarrow \overline{\sigma}_1\} \vdash \overline{\sigma}_1 <:\; [z \unlhd \{z \Rightarrow \overline{\sigma}_2\} / z]\overline{\sigma}_2}
%	{A; \Sigma; \Gamma \vdash \{z \Rightarrow \overline{\sigma}_1\}\; <:\; \{z \Rightarrow \overline{\sigma}_2\}}
%	\quad (\textsc {S-Struct})
%\end{mathpar}
%\caption{Structural Subtyping}
%\label{f:struct_sub}
%\end{figure}
If we didn't do this, we would have to type check the declaration types of the larger type with a smaller receiver, resulting in narrowing. We also need to be able subtype two path equivalent selection types(\textsc{S-Path} in Figure \ref{f:subtype}). Since we can have two paths that are equivalent in that they lead to the same location or variable while having different types, we need a way to subtype path equivalent type selections. We require that they have the appropriate contra-variant and covariant relationships between their lower and upper bounds respectively. To deal with recursive types, we use an assumption context $A$ which we use to type check the bounds of the selection type. For this we introduce the \textsc{S-Assume} rule. \textsc{S-Select-Upper} and \textsc{S-Select-Lower} are our rules for subtyping selection types. To supertype or subtype a selection type, a type needs to supertype or subtype its upper or lower bounds respectively. We then use \textsc{S-Top} and \textsc{S-Bottom} to type check the subtyping and supertyping of the top ($\top$) and bottom ($\bot$) types.

\begin{figure}[h]
\hfill \fbox{$A; \Sigma; \Gamma \vdash S <: T$}
\begin{mathpar}
\inferrule
  {(S <: T) \in A}
  {A; \Sigma; \Gamma \vdash S\; \texttt{<:}\; T}
  \quad (\textsc {S-Assume})
	\and
\inferrule
	{A; \Sigma; \Gamma, z : \{z \Rightarrow \overline{\sigma}_1\} \vdash \overline{\sigma}_1 <:\; [z \unlhd \{z \Rightarrow \overline{\sigma}_2\} / z]\overline{\sigma}_2}
	{A; \Sigma; \Gamma \vdash \{z \Rightarrow \overline{\sigma}_1\}\; <:\; \{z \Rightarrow \overline{\sigma}_2\}}
	\quad (\textsc {S-Struct})
	\and
\inferrule
	{p_1 \equiv p_2 \\
	 A; \Sigma; \Gamma \vdash p_1 \ni \texttt{type} \; L : S_1 .. U_1 \\
	 A; \Sigma; \Gamma \vdash p_2 \ni \texttt{type} \; L : S_2 .. U_2 \\
	 A, (p_1.L <: p_2.L); \Sigma; \Gamma \vdash S_2 <:\; S_1 \\
	 A, (p_1.L <: p_2.L); \Sigma; \Gamma \vdash U_1\; <:\; U_2}
	{A; \Sigma; \Gamma \vdash p_1.L\; <:\; p_2.L}
	\quad (\textsc {S-Path})
	\and
\inferrule
	{A; \Sigma; \Gamma \vdash p \ni \texttt{type} \; L : S .. U\\
	 A; \Sigma; \Gamma \vdash S <: U \\
	 A; \Sigma; \Gamma \vdash U <: T}
	{A; \Sigma; \Gamma \vdash p.L\; <:\; T}
	\quad (\textsc {S-Select-Upper})
	\and
\inferrule
	{A; \Sigma; \Gamma \vdash p \ni \texttt{type} \; L : S .. U \\
	 A; \Sigma; \Gamma \vdash S <: U \\
	 A; \Sigma; \Gamma \vdash T <: S}
	{A; \Sigma; \Gamma \vdash T \; <:\; p.L}
	\quad (\textsc {S-Select-Lower})
	\and
\inferrule
	{}
	{A; \Sigma; \Gamma \vdash T\; \texttt{<:}\; \top}
	\quad (\textsc {S-Top})
	\and
\inferrule
	{}
	{A; \Sigma; \Gamma \vdash \bot\; \texttt{<:}\; T}
	\quad (\textsc {S-Bottom})
\end{mathpar}
\hfill \fbox{$A; \Sigma; \Gamma \vdash \sigma <: \sigma'$}
\begin{mathpar}
\inferrule
	{}
	{A; \Sigma; \Gamma \vdash \texttt{val} \; f:T <: \texttt{val} \; f:T}
	\quad (\textsc {S-Decl-Val})
	\and
\inferrule
	{A; \Sigma; \Gamma \vdash S' <: S \\
	 A; \Sigma; \Gamma \vdash T <: T'}
	{A; \Sigma; \Gamma \vdash \texttt{def} \; m:S \rightarrow T <: \texttt{def} \; m:S' \rightarrow T'}
	\quad (\textsc {S-Decl-Def})
	\and
\inferrule
	{A; \Sigma; \Gamma \vdash S' <: S \\
	 A; \Sigma; \Gamma \vdash U <: U'}
	{A; \Sigma; \Gamma \vdash \texttt{type} \; L : S .. U \; <:\; \texttt{type} \; L : S' .. U'}
	\quad (\textsc {S-Decl-Type})
\end{mathpar}
\caption{Subtyping}
\label{f:subtype}
\end{figure}

\subsubsection{Well-Formedness}

\hl{Julian: Do we need to worry about this?}

\begin{figure}[h]
\hfill \fbox{$A; \Sigma; \Gamma \vdash T \;  \textbf{wf}$}
\begin{mathpar}
\inferrule
  {A; \Sigma; \Gamma \vdash p \ni \texttt{type} \; L : S .. U \\
  	A; \Sigma; \Gamma \vdash \texttt{type} \; L : S .. U \; \textbf{wf} }
  {A; \Sigma; \Gamma \vdash p.L \; \textbf{wf}}
  \quad (\textsc {WF-Sel})
	\and
\inferrule
  {A; \Sigma; \Gamma,z:\{z \Rightarrow \overline{\sigma}\} \vdash \overline{\sigma} \; \textbf{wf} \\
  	\forall j \neq i, \; dom(\sigma_j) \neq dom(\sigma_i)}
  {A; \Gamma; \Sigma \vdash \{z \Rightarrow \overline{\sigma}\} \; \textbf{wf}}
  \quad (\textsc {WF-Struct})
	\and
\inferrule
  {}
  {A; \Sigma; \Gamma \vdash \top \;  \textbf{wf}}
  \quad (\textsc {WF-Top})
	\and
\inferrule
  {}
  {A; \Sigma; \Gamma \vdash \bot \;  \textbf{wf}}
  \quad (\textsc {WF-Bot})
\end{mathpar}
\hfill \fbox{$A; \Sigma; \Gamma \vdash \sigma \;  \textbf{wf}$}
\begin{mathpar}
\inferrule
  {A; \Sigma; \Gamma \vdash T : \textbf{wf}}
  {A; \Sigma; \Gamma \vdash \texttt{val} \; f:T \;  \textbf{wf}}
  \quad (\textsc {WF-Val})
	\and
\inferrule
  {A; \Sigma; \Gamma \vdash T : \textbf{wf} \\
  	A; \Sigma; \Gamma \vdash S : \textbf{wf}}
  {A; \Sigma; \Gamma \vdash \texttt{def} \; m:S \rightarrow T \;  \textbf{wf}}
  \quad (\textsc {WF-Def})
	\and
\inferrule
  {A; \Sigma; \Gamma \vdash S : \textbf{wfe} \; \vee \; S = \bot\\
  	A; \Sigma; \Gamma \vdash U : \textbf{wfe} \\
  	A; \Sigma; \Gamma \vdash S <: U}
  {A; \Sigma; \Gamma \vdash \texttt{type} \; L : S .. U \; \textbf{wf}}
  \quad (\textsc {WF-Type})
\end{mathpar}
\hfill \fbox{$A; \Sigma \vdash \Gamma \;  \textbf{wf}$}
\begin{mathpar}
\inferrule
  {\forall x \in dom(\Gamma), A; \Sigma; \Gamma \vdash \Gamma(x) \; \textbf{wf}}
  {\Sigma \vdash \Gamma \;  \textbf{wf}}
  \quad (\textsc {WF-Environment})
\end{mathpar}
\hfill \fbox{$\Sigma \;  \textbf{wf}$}
\begin{mathpar}
\inferrule
  {\forall l \in dom(\Sigma), \varnothing; \Sigma; \varnothing \vdash \Sigma(l) \; \textbf{wf}}
  {\Sigma \;  \textbf{wf}}
  \quad (\textsc {WF-Store-Context})
\end{mathpar}
\begin{mathpar}
\inferrule
  {\forall l \in dom(\mu), \varnothing; \Sigma; \varnothing \vdash \mu(l) : \Sigma(l)}
  {\Sigma \vdash \mu \; \textbf{wf}}
  \quad (\textsc {WF-Store})
\end{mathpar}
\caption{Well-Formedness}
\label{f:wf}
\end{figure}

\begin{figure}[h]
\hfill \fbox{$A; \Sigma; \Gamma \vdash T \;  \textbf{wfe}$}
\begin{mathpar}
\inferrule
  {A; \Sigma; \Gamma \vdash T \; \textbf{wf} \\
  	A; \Sigma; \Gamma \vdash T \prec \overline{\sigma}}
  {A; \Sigma; \Gamma \vdash T \; \textbf{wfe}}
  \quad (\textsc {WFE})
\end{mathpar}
\caption{Well-Formed and Expanding Types}
\label{f:wfe}
\end{figure}

\subsubsection{Type Expansion and Membership}

We use a modified version of the type expansion (Figure \ref{f:exp}) definition and the same membership (Figure \ref{f:mem}) definition from Amin et al \cite{Scala stuff}. Expansion of types is used to extract the set of declaration types for a type. \textsc{E-Struct} is the expansion rule for structural types. Structural types simply expand to their defined types. The expansion of selection types (\textsc{E-Sel}) are slightly more complicated. Selection types expand to the expansion of the upper bound. The expansion of the upper bound is type checked with a less precise type (the upper bound) that the selection type. As with our subtyping rule for structural types, we need to prevent narrowing of types at type expansion. For this reason we upcast the self variable $z$ to the original type $U$. The top type expands to the empty set (\textsc{E-Top}). The membership judgement is used to determine the membership of a declaration type for an expression. We type check the expression, expand the expression's type and in the case of path expansion, substitute out the self variable.

\begin{figure}[h]
\hfill \fbox{$A; \Sigma; \Gamma \vdash T \prec \overline{\sigma}$}
\begin{mathpar}
\inferrule
  {}
  {A; \Sigma; \Gamma \vdash \{z \Rightarrow \overline{\sigma}\} \prec_z \overline{\sigma}}
  \quad (\textsc {E-Struct})
	\and
\inferrule
  {A; \Sigma; \Gamma \vdash p \ni \texttt{type} \; L : S..U \\
  	A; \Sigma; \Gamma \vdash U \prec_z \overline{\sigma}}
  {A; \Sigma; \Gamma \vdash p.L \prec_z [z \unlhd U/z]\overline{\sigma}}
  \quad (\textsc {E-Sel})
	\and
\inferrule
  {}
  {A; \Sigma; \Gamma \vdash \top \prec_z \varnothing}
  \quad (\textsc {E-Top})
\end{mathpar}
\caption{Expansion}
\label{f:exp}
\end{figure}

\begin{figure}[h]
\hfill \fbox{$A; \Sigma; \Gamma \vdash e \ni \sigma$}
\begin{mathpar}
\inferrule
  {A; \Sigma; \Gamma \vdash p : T \\
  	A; \Sigma; \Gamma \vdash T \prec_z \overline{\sigma}\\
  	A; \sigma_i \in \overline{\sigma}}
  {A; \Sigma; \Gamma \vdash p \ni [p/z]\sigma_i}
  \quad (\textsc {M-Path})
	\and
\inferrule
  {A; \Sigma; \Gamma \vdash e : T \\
  	A; \Sigma; \Gamma \vdash T \prec_z \overline{\sigma}\\
  	\sigma_i \in \overline{\sigma} \\
  	z \notin \sigma_i}
  {A; \Sigma; \Gamma \vdash e \ni \sigma_i}
  \quad (\textsc {M-Exp})
\end{mathpar}
\caption{Membership}
\label{f:mem}
\end{figure}

\subsubsection{Expression Typing}
Expression typing is fairly straight forward and is 
given in Figure \ref{f:e_typ}. Variables (\textsc{T-Var})
are typed with their types in the environment and 
locations (\textsc{T-Loc})in the store type. New 
expressions (\textsc{T-New}) are typed as a collection 
of declaration types that correspond to their declarations. 
Method calls (\textsc{T-Meth}) are typed as their 
return type provided the arguments 
subtype the method parameter types. Field accesses 
(\textsc{T-Field}) are typed as the field type for the receiver. 
Upcasts (\textsc{T-Type}) are typed as the upcast type if the 
upcast expression appropriately subtypes the upcast type.
\begin{figure}[h]
\hfill \fbox{$A; \Sigma; \Gamma \vdash e:T$}
\begin{mathpar}
\inferrule
  {x \in dom(\Gamma)}
  {	A; \Sigma; \Gamma \vdash x : \Gamma(x)}
  \quad (\textsc {T-Var})
	\and
\inferrule
  {	l \in dom(\Sigma)}
  {	A; \Sigma; \Gamma \vdash l : \Sigma(l)}
  \quad (\textsc {T-Loc})
	\and
\inferrule
  {A; \Sigma; \Gamma, z : \{z \Rightarrow \overline{\sigma}\} 
  \vdash \overline{d} : \overline{\sigma}}
  {A; \Sigma; \Gamma \vdash \texttt{new} \; \{z \Rightarrow \overline{d}\} : 
  \{z \Rightarrow \overline{\sigma}\}}
  \quad (\textsc {T-New})
	\and
\inferrule
  {A; \Sigma; \Gamma \vdash e_0 \ni \texttt{def} \; m:S \rightarrow T \\
  	A; \Sigma; \Gamma \vdash e_0 : T_0 \\
  	A; \Sigma; \Gamma \vdash e_1 : S' \\
  	A; \Sigma; \Gamma \vdash S' <: S}
  {A; 	\Sigma; \Gamma \vdash e_0.m(e_1) : T}
  \quad (\textsc {T-Meth})
	\and
\inferrule
  {	A; \Sigma; \Gamma \vdash e : S \\
  	A; \Sigma; \Gamma \vdash e \ni \texttt{val} \; f:T}
  {	A; \Sigma; \Gamma \vdash e.f : T}
  \quad (\textsc {T-Field})
	\and
\inferrule
  {A; \Sigma; \Gamma \vdash e : S \\
   A; \Sigma; \Gamma \vdash S <: T}
  {A; \Sigma; \Gamma \vdash e \unlhd T : T}
  \quad (\textsc {T-Type})
\end{mathpar}
\caption{Expression Typing}
\label{f:e_typ}
\end{figure}

\subsubsection{Reduction}

Reduction is given in Figure \ref{f:red}. Reduction is new (\textsc{R-New}), method (\textsc{R-meth}), field (\textsc{R-Field}) and context (\textsc{R-Context}) reduction. New expressions are reduced (\textsc{R-New}) once field initializers are reduced to values. Method calls on values with a value type parameter are reduced (\textsc{R-Meth}) using the \emph{Leadsto\textsubscript{m}} judgement. We use the \emph{Leadsto\textsubscript{m}} judgement to extract the method body from the store, and wrap it up in well-formed type layers. Field reduction (\textsc{R-Field}) is performed the same way, using the \emph{Leadsto\textsubscript{f}} judgement. Context reduction (\textsc{R-Context}) is standard.
\begin{figure}[h]
\hfill \fbox{$\mu \; | \; e \; \rightarrow \mu' \; | \; e'$}
\begin{mathpar}
\inferrule
  {l \notin dom(\mu) \\
  	\mu' = \mu, l \mapsto \{\texttt{z} \Rightarrow \overline{d_v}\}}
  {\mu \; | \; \texttt{new} \; \{\texttt{z} \Rightarrow \overline{d_v}\} \; \rightarrow \mu' \; | \; l}
  \quad (\textsc {R-New})
  \and
\inferrule
  {\mu : \Sigma \\
   \mu; \Sigma \vdash v_1 \leadsto_{m(v_2)} e}
  {\mu \; | \; v_1.m(v_2) \;\rightarrow \mu \; | \; e}
  \quad (\textsc {R-Meth})
  \and
\inferrule
  {\mu : \Sigma \\
   \mu; \Sigma \vdash v_1 \leadsto_{f} v_2}
  {\mu \; | \; v_1.f \;\rightarrow \mu \; | \; v_2}
  \quad (\textsc {R-Field})
  \and
\inferrule
  {	\mu \; | \; e \; \rightarrow \; \mu' \; | \; e'}
  {\mu \; | \; E[e] \; \rightarrow \mu' \; | \; E[e']}
  \quad (\textsc {R-Context})
\end{mathpar}
\caption{Reduction}
\label{f:red}
\end{figure}

\section{Proving Type Soundness}
Our type system makes several modifications based on previous work \cite{Scala stuff}. As discussed before, 

\section{Future Work}
As part of the development of Generic Wyvern, we simplified the concept of paths in the same way that Amin et. al. \cite{Amin 2014} did. This was due to the complexity of both the path equality and narrowing problems. In the future we hope to expand the notion of paths to include field accesses. We also intend to introduce \emph{Intersection} and \emph{Union} types. 

\section{Conclusion}
We have developed and proven sound a small-step semantics for type members in structurally type languages.


%%
%% Bibliography
%%

%% Either use bibtex (recommended), but commented out in this sample

%\bibliography{dummybib}

%% .. or use bibitems explicitely

\nocite{Simpson}

\begin{thebibliography}{50}
\bibitem{Simpson} Homer J. Simpson. \textsl{Mmmmm...donuts}. Evergreen Terrace Printing Co., Springfield, Somewhere, USA, 1998
\end{thebibliography}


\end{document}
