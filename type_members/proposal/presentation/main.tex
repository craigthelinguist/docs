% Copyright 2004 by Till Tantau <tantau@users.sourceforge.net>.
%
% In principle, this file can be redistributed and/or modified under
% the terms of the GNU Public License, version 2.
%
% However, this file is supposed to be a template to be modified
% for your own needs. For this reason, if you use this file as a
% template and not specifically distribute it as part of a another
% package/program, I grant the extra permission to freely copy and
% modify this file as you see fit and even to delete this copyright
% notice. 

\documentclass[dvipsnames]{beamer}

\usepackage{mathpartir}
\usepackage{listings}
%\usepackage[dvipsnames]{xcolor}
\usepackage{pgfgantt}

\lstdefinestyle{customlang}{
  language = java,
  xleftmargin=\parindent,
  showstringspaces=false,
  basicstyle=\ttfamily\tiny,
  otherkeywords={val, def, type, new},
  keywordstyle=\bfseries
%  keywordstyle=\color{blue},
}

%\lstset{%
%    backgroundcolor=\color{yellow!20},%
%    basicstyle=\small\ttfamily,%
%    numbers=left, numberstyle=\tiny, stepnumber=1, numbersep=5pt,%
%    }%

% Add your keywords here, and have this in a separate file
% and include it in your preamble
\lstset{emph={%  
    val, def, type, new%
    },emphstyle={\ttfamily\bf}
}%


%\lstset{%
%    language=[latex]tex,
%    breaklines=true}
    
\newsavebox{\tmExA}    
\newsavebox{\tmExAcont}
\newsavebox{\tmExPreservation}

% There are many different themes available for Beamer. A comprehensive
% list with examples is given here:
% http://deic.uab.es/~iblanes/beamer_gallery/index_by_theme.html
% You can uncomment the themes below if you would like to use a different
% one:
%\usetheme{AnnArbor}
%\usetheme{Antibes}
%\usetheme{Bergen}
%\usetheme{Berkeley}
%\usetheme{Berlin}
%\usetheme{Boadilla}
%\usetheme{boxes}
%\usetheme{CambridgeUS}
%\usetheme{Copenhagen}
%\usetheme{Darmstadt}
%\usetheme{default}
%\usetheme{Frankfurt}
%\usetheme{Goettingen}
%\usetheme{Hannover}
%\usetheme{Ilmenau}
%\usetheme{JuanLesPins}
%\usetheme{Luebeck}
\usetheme{Madrid}
%\usetheme{Malmoe}
%\usetheme{Marburg}
%\usetheme{Montpellier}
%\usetheme{PaloAlto}
%\usetheme{Pittsburgh}
%\usetheme{Rochester}
%\usetheme{Singapore}
%\usetheme{Szeged}
%\usetheme{Warsaw}

\title{Generic Wyvern}

% A subtitle is optional and this may be deleted
\subtitle{PhD Proposal}

\author{Julian Mackay}
% - Give the names in the same order as the appear in the paper.
% - Use the \inst{?} command only if the authors have different
%   affiliation.

\institute[VUW] % (optional, but mostly needed)
{
%  \inst{1}%
  School of Engineering and Computer Science\\
  Victoria University of Wellington
%  \and
%  \inst{2}%
%  Department of Theoretical Philosophy\\
%  University of Elsewhere
  }
% - Use the \inst command only if there are several affiliations.
% - Keep it simple, no one is interested in your street address.

\date{}
% - Either use conference name or its abbreviation.
% - Not really informative to the audience, more for people (including
%   yourself) who are reading the slides online

\subject{Theoretical Computer Science}
% This is only inserted into the PDF information catalog. Can be left
% out. 

% If you have a file called "university-logo-filename.xxx", where xxx
% is a graphic format that can be processed by latex or pdflatex,
% resp., then you can add a logo as follows:

% \pgfdeclareimage[height=0.5cm]{university-logo}{university-logo-filename}
% \logo{\pgfuseimage{university-logo}}

% Delete this, if you do not want the table of contents to pop up at
% the beginning of each subsection:
%\AtBeginSubsection[]
%{
%  \begin{frame}<beamer>{Outline}
%    \tableofcontents[currentsection,currentsubsection]
%  \end{frame}
%}

% Let's get started
\begin{document}


\begin{frame}
  \titlepage
\end{frame}

\begin{frame}{Outline}
  \tableofcontents[pausesections]
  % You might wish to add the option [pausesections]
\end{frame}

% Section and subsections will appear in the presentation overview
% and table of contents.
\section{Generic Wyvern}

\subsection{Wyvern}

\begin{frame}{Wyvern}
Wyvern is a new structurally typed object oriented language.
\end{frame}

\subsection{Generic Programming}

% You can reveal the parts of a slide one at a time
% with the \pause command:
\begin{frame}{Generic Type Parameters}
\end{frame}

\section{Type Members}

\subsection{What are Type members?}

\begin{frame}{Type Members}
\end{frame}

\begin{lrbox}{\tmExA}
\begin{lstlisting}[mathescape, style=customlang]
List = {z $\Rightarrow$	type E = $\top$
      val head : z.E
      val tail : List
      def add : z.E $\rightarrow$ bool}
      
Graph = {z1 $\Rightarrow$	type E = Edge
      type EdgeList = List{z2 $\Rightarrow$ type E = z1.E 
                                 val tail : EdgeList}
      type VertexList = List{z2 $\Rightarrow$ type E = z1.V
                                   val tail : VertexList}
      type V = Vertex}
		
Edge = {z $\Rightarrow$	type G = Graph
      type V = Vertex
      val source : V
      val target : V
      val graph : G}
		
Vertex = {z1 $\Rightarrow$	type G = Graph
      type E = Edge
      type EdgeList = List{z2 $\Rightarrow$ type E = z1.E
                                 val tail : EdgeList}
      type T = $\top$
      val edges : EdgeList
      val element : z1.T}
\end{lstlisting}
\end{lrbox}

\begin{lrbox}{\tmExAcont}
\begin{lstlisting}[mathescape, style=customlang]		      
IntegerGraph = Graph{z1 $\Rightarrow$	type E = IntegerEdge
      type V = IntegerVertex}
		
IntegerEdge = Edge{z $\Rightarrow$	type G = IntegerGraph
      type V = IntegerVertex}
		
IntegerVertex = Vertex{z1 $\Rightarrow$	type G = IntegerGraph
      type E = IntegerEdge
      type T = Integer}
\end{lstlisting}
\end{lrbox}

\begin{frame}{Type Members: Example}
\begin{example}
\usebox{\tmExA}
\end{example}
\end{frame}

\begin{frame}{Type Members: Example Continued}
\begin{example}
\usebox{\tmExAcont}
\end{example}
\end{frame}

\subsection{Developing Sound Type Members}

\begin{frame}{Soundness}
\begin{theorem}[Preservation]
For any well-formed expression and memory pair, if we step forward the result maintains the original well-formed type.
\begin{mathpar}
\inferrule
  {\Sigma; \Gamma \vdash e : T\\
   \mu|e \longrightarrow \mu'|e'\\
   \mu : \Sigma \\
   \mu' : \Sigma'}
  {\Sigma'; \Gamma \vdash e' : T}
\end{mathpar}
\end{theorem}
\begin{theorem}[Progress]
For any well-typed expression, either it is a value (a fully reduced expression), or it can make ``progress''.
\begin{mathpar}
\inferrule
  {\Sigma; \varnothing \vdash e : T \\
   \mu : \Sigma}
  {e = v \vee (\exists \; e' \; \mu': \; \mu|e \longrightarrow \mu'|e' \wedge \mu' \; extends \; \mu)}
\end{mathpar}
\end{theorem}
\end{frame}

\begin{lrbox}{\tmExPreservation}
\begin{lstlisting}[mathescape, style=customlang]
X = {z $\Rightarrow$ val f : $\top$}
Y = {z $\Rightarrow$ type L : $\top$ .. $\top$}
	  val f : z.L}
A = {z $\Rightarrow$ def meth : X}
...
var a = new A(def meth = {new Y(...)} : X)
a.meth.f
\end{lstlisting}
\end{lrbox}

\begin{frame}{Loss of Preservation}
\begin{example}
\usebox{\tmExPreservation}
\end{example}
\end{frame}

\subsection{Our Solution}

\begin{frame}{Avoiding Narrowing}
%\begin{block}{Block Title}
%You can also highlight sections of your presentation in a block, with it's own title
%\end{block}
%\begin{example}
%Here is an example of an example block.
%\end{example}
\end{frame}

\begin{frame}{Avoiding Transitivity}
%\begin{block}{Block Title}
%You can also highlight sections of your presentation in a block, with it's own title
%\end{block}
%\begin{example}
%Here is an example of an example block.
%\end{example}
\end{frame}

% Placing a * after \section means it will not show in the
% outline or table of contents.
\section{Extending Type Members}

\subsection{Path Dependent Types}

\begin{frame}{Path Dependent Types}
\end{frame}

\subsection{Decidable Subtyping}

\begin{frame}{Decidable Subtyping}
\end{frame}

\begin{frame}{Shapes, Materials and F-Bounded Polymorphism}
\end{frame}

\section{PhD Proposal}

\begin{frame}{Threats to Research}
\end{frame}

\begin{frame}{Timeline}

\begin{figure}
\noindent\resizebox{\textwidth}{!}{
\begin{ganttchart}[
y unit title=0.6cm,
y unit chart=0.8cm,
vgrid,
time slot format=isodate-yearmonth,
compress calendar,
title/.append style={draw=none, fill=RoyalBlue!50!black},
title label font=\sffamily\bfseries\color{white},
title label node/.append style={below=-1.6ex},
title left shift=.05,
title right shift=-.05,
title height=1,
bar/.append style={draw=none, fill=OliveGreen!75},
bar height=.6,
bar label font=\normalsize\color{black},
group right shift=0,
group top shift=.6,
group height=.3,
group peaks height=.2,
bar incomplete/.append style={fill=Maroon}
]{2015-03}{2018-02}
\gantttitlecalendar{year} \\
\ganttset{progress label text={}, link/.style={black, -to}}
\ganttgroup{Type Members}{2015-03}{2016-05} \\
\ganttbar[progress=100, name=T1A, 
bar progress label font=\small\color{OliveGreen!75},
bar progress label node/.append style={right=4pt},
bar label font=\normalsize\color{OliveGreen}]{Develop Type System}{2015-03}{2015-11} \\
\ganttbar[progress=100,
bar label font=\normalsize\color{OliveGreen}]{Proposal}{2015-12}{2016-02} \\
\ganttbar[progress=25]{SPLASH 2016}{2016-03}{2016-03} \\
\ganttbar[progress=0]{Intersection \& Union Types}{2016-04}{2016-05} \\
\ganttgroup{Path Dependent Types}{2016-06}{2016-012} \\
\ganttbar[progress=10]{Path Dependent Types}{2016-06}{2016-012} \\
\ganttgroup{Decidability}{2016-04}{2017-06} \\
\ganttbar[progress=2, name=T2A]{Decidability}{2016-04}{2017-06} \\
%\ganttgroup{Gradual Types}{2017-03}{2017-06} \\
%\ganttbar[progress=0, name=T2A]{Gradual Types}{2017-03}{2017-06} \\
\ganttgroup{Thesis}{2017-07}{2018-02} \\
\ganttbar[progress=0]{Thesis}{2017-07}{2018-02}
\ganttset{link/.style={OliveGreen}}
\end{ganttchart}}
%\caption{Proposed Time line for PhD}
\label{f:gantt}
\end{figure}
\end{frame}


% All of the following is optional and typically not needed. 
%\appendix
%\section<presentation>*{\appendixname}
%\subsection<presentation>*{For Further Reading}
%
%\begin{frame}[allowframebreaks]
%  \frametitle<presentation>{For Further Reading}
%    
%  \begin{thebibliography}{10}
%    
%  \beamertemplatebookbibitems
%  % Start with overview books.
%
%  \bibitem{Author1990}
%    A.~Author.
%    \newblock {\em Handbook of Everything}.
%    \newblock Some Press, 1990.
% 
%    
%  \beamertemplatearticlebibitems
%  % Followed by interesting articles. Keep the list short. 
%
%  \bibitem{Someone2000}
%    S.~Someone.
%    \newblock On this and that.
%    \newblock {\em Journal of This and That}, 2(1):50--100,
%    2000.
%  \end{thebibliography}
%\end{frame}

\end{document}


